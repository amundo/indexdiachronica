\documentclass[11pt]{article}
\usepackage{geometry}                % See geometry.pdf to learn the layout options. There are lots.
\geometry{letterpaper}                   % ... or a4paper or a5paper or ... 
%\geometry{landscape}                % Activate for for rotated page geometry
\usepackage[parfill]{parskip}    % Activate to begin paragraphs with an empty line rather than an indent
\usepackage{graphicx}
\usepackage{amssymb}
\usepackage{epstopdf}
\usepackage{textcomp}
\usepackage{tipa}
\usepackage{hyperref}
\usepackage{ulem}
\usepackage{semtrans}
\usepackage{underscore}
\usepackage{url}
\newcommand{\ipa}{\textipa}
\newcommand{\tab}{\hspace{25pt}}
\newcommand{\change}{\textrightarrow}
\setcounter{secnumdepth}{5}
\setcounter{tocdepth}{5}
\DeclareGraphicsRule{.tif}{png}{.png}{`convert #1 `dirname #1`/`basename #1 .tif`.png}

\title{Index Diachronica v.10.2}
%\author{Compiled by Pogostick Man}
%\date{}                                           % Activate to display a given date or no date

\begin{document}
\maketitle
\pagenumbering{roman}
\tableofcontents
\newpage
\pagenumbering{arabic}
\section{Preface}
\tab On September 18, 2003, jburke created a topic on the Zompist Bulletin Board with the aim of allowing conlangers to examine trends in sound changes within natlang families. It has since expanded to provide conlangers with a general gist of plausible sound changes in general. The thread, in its current iteration, is available here: \url{http://www.incatena.org/viewtopic.php?f=10&t=1533}. Many of the compilations of sound changes have either come from pages in the thread or from pages on the KneeQuickie Correspondence Library archives (available at \url{http://kneequickie.com/archive/The_Correspondence_Library}; the page at \url{http://www.kneequickie.com/kq/The_Correspondence_Library} has not yet been updated with subpages for sound changes); if an entry in this list has no known contributor listed, it is from KneeQuickie's Correspondence Library.

\tab The intended purpose of this document is to provide a tool in PDF form for conlangers interested in diachronic conlanging and linguistic change to be able to get a feel for what sorts of changes might plausibly occur. To that end, this document features a compilation of various historical series of diachronic sound changes (and on occasion some synchronic processes as well) that have occurred in natural languages. It is hoped that the changes featured within this document will be of use in these endeavors.

\tab No warrant is made that the entirety of the information herein is complete or correct. The ZBB was migrated over to a different setup some years back causing many special characters to disappear. Further, not all sources use IPA transcription, and may be unclear or missing information. Additionally, when listing sources, Wikipedia pages may be given with {\tt https:\textbackslash\textbackslash} instead of {\tt http:\textbackslash\textbackslash}, even though the page may have been accessed using {\tt http:\textbackslash\textbackslash} instead of {\tt https:\textbackslash\textbackslash}; this is for security, although doing so may in reality be pointless.

\tab Due to the limits of the \LaTeX\ software (and the skills of its user), full nesting is not possible. It is hoped that readers will understand and it is one of the goals of this project to provide correct nesting as far as is possible. Additionally, some overlap or multiple versions of changes may be present due to the nature of submitted sound-change lists.

\tab Finally, many thanks to all individuals who contributed to the Library. Without you, this document would not exist.

\section{Licensing and Legal Information}\tab This document is licensed under the Creative Commons Attribution-NonCommercial-\hspace{0pt}ShareAlike (CC BY-NC-SA) 3.0 license. Visit \url{http://creativecommons.org/licenses/by-nc-sa/3.0/} for further details.

\section{Contact Information}\tab Questions, comments, corrections, suggestions, missing authors for those changes taken from KneeQuickie, or other feedback may be sent to Pogostick Man at the Zompist Bulletin Board or the New Conlang Bulletin Board, Pan Pogostick at Polskie Forum J\textpolhook{e}zykowe, the CONLANG mailing list, or to \url{mailto:satorarepotenetoperarotas3@gmail.com}. Submitting corrections or lists of sound changes, preferably sourced, is encouraged.\clearpage

\section{Changelog}
\begin{itemize}
\item \textbf{v.1.0} (2013/11/26) -- Initial public release.
\item v.1.1 (2013/11/26) -- Small amendment to the preamble.
\item v.1.2 (2013/11/26) -- Small amendment to the preamble including a link to the KQ category where some of the changes were taken from.
\item v.1.3 (2013/11/26) -- Added omitted attribution on Wales I\~{n}upiaq changes; alteration to changelog layout and amendment to Contact Information.
\item v.1.4 (2013/11/26) -- Forgot to update the version information in the title.
\item v.1.5 (2013/12/01) -- Added changes from Proto-Slavic to Polish, which I forgot to include in the original release. Also, added a Proto-Salish phonology I also forgot to include in the original release.
\item v.1.6 (2013/12/03) -- Added the California Vowel Shift.
\item v.1.7 (2013/12/03) -- Fixed the California Vowel Shift.
\item v.1.8 (2013/12/06) -- Credited Macska for the Pre-Slavic Vowel Changes.
\item \textbf{v.2.0} (2014/03/11) -- Added Yokuts and Lakes Plain correspondences.
\item v.2.1 (2014/03/11) -- Forgot to change version number on the first page.
\item v.2.2 (2014/03/11) -- Moved Yokuts to the Penutian group; fixed some errors.
\item \textbf{v.3.0} (2014/03/27) -- Added Northwest Caucasian, fixed Yokuts nesting errors.
\item v.3.1 (2014/03/27) -- Notes per Tropylium's request and some formatting cleanup/explanation in the section on Northwest Caucasian.
\item v.3.2 (2014/04/03) -- Added considerable information on developments in West Germanic.
\item v.3.3 (2014/04/03) -- Cross-listed some of the West Germanic developments under the Vowel Shifts section; minor fixes.
\item \textbf{v.4.0} (2014/04/03) -- Replicated the ``Most wanted sound changes" article from KneeQuickie.
\item v.4.1 (2014/04/03) -- Minor fixes.
\item v.4.2 (2014/04/04) -- Minor correction.
\item \textbf{v.5.0} (2014/04/28) -- Added some Macro-Pama-Nyungan correspondences.
\item v.5.1 (2014/04/30) -- Correction of formatting errors; change of all remaining instances of ``Linguifex'' and ``Rorschach" to ``Pogostick Man"; amendment to contact info; other minor changes; addition of Proto-Tupar\'{i} consonantal reconstruction.
\item v.5.2 (2014/05/13) -- Added Qiangic, Proto-Lolo-Burmese final -\ipa{i}(C) and -\ipa{u}(C) sequences, Paiwan, Rukai, and changes to Old Proven\c{c}al.
\item v.5.3 (2014/05/13) -- Added changes to Rhaeto-Romance.
\item v.5.4 (2014/05/28) -- Added Austronesian, Turkic, and Italic.
\item {\bf v.6.0} (2014/06/21) -- Added Northeast Caucasian and Vasconic; clarified the situation with regards to sources without a known author (these are mostly from KneeQuickie); explained policy with regards to URLs when dealing with Wikipedia pages; corrected the description of the shorthand symbol $\langle$\%$\rangle$; added a note from Tropylium on the Finno-Ugric changes.
\item v.6.1 (2014/07/18) -- Added some Austronesian changes; corrected Avestan according to comments from Alex Fink on CONLANG-L.
\item \textbf{v.7.0} (2015/01/31) -- Added some changes to the Austronesian section (including Proto-Ongan); added some changes regarding the Bantu languages and related groups; added changes for Standard German and Gothic; added correspondences for Monde languages; added Trans-New Guinea section. Cleaned up the Key to Abbreviations.
\item v.7.1 (2015/01/31) -- Corrected an accidentally omitted attribution.
\item v.7.2 (2015/01/31) -- Corrected a typo in the changelog.
\item v.7.3 (2015/01/31) -- Corrected an oversight in formatting.
\item v.7.4 (2015/02/01) -- Formatting and attribution fixes. Added reconstructed Proto-Trans New Guinea phonology.
\item v.7.5 (2015/02/03) -- Correcting omissions, including an acknowledgment that a quote from Whimemsz is sourced from KneeQuickie.
\item v.7.6 (2015/02/25) -- Fixed misspelling of ``Falsican''; fixed an alphabetical ordering 
error in the Indo-European section.
\item v.7.7 (2015/06/02) -- Added missing section on the development of Cheyenne that I forgot to add to the document originally; added a lot of potential Bantu correspondences (see the note in the introduction to the Niger-Congo section regarding said correspondences); added Faroese vowel shift information; added correspondences for some Kalamian languages; added changes from Proto-Oceanic to Hiw.
\item {\bf v.8.0} (2015/11/16) -- Added Piscataway, Mi'kmaq, and Cree correspondences to the Algonquian section; added Kainantu-Goroka correspondences to the Trans-New Guinea section; added Vandalic correspondences; added some rGyalrongic and Tibetic correspondences to the Sino-Tibetan section; added Philippine and Oceanic correspondences to the Austronesian section; added Pai correspondences to the Yuman-Cochim\'{\i} section; added Chumashan correspondences; added Ofai\'{e}-J\^{e} correspondences; added Bakairi correspondences; added Lenmichian correspondences; added some Na-Dene correspondences and moved the Athabaskan section under Na-Dene. Fixed a formatting error in the changelog.
\item v.8.1 (2015/11/16) -- Corrected some formatting errors.
\item v.8.2 (2016/02/18) -- Added missing section on Cheyenne that was lost during the move from v.7.7 to v.8.1. Corrected some errors in the changeling.
\item {\bf v.9.0} (2016/02/22) -- Added changes involving Sanskrit, Hupa, Southern Athabaskan, Totozoquean, Mande, and Luangiua. Fixed some formatting errors and updated the version number on the title page. Corrected Mi'kmaq changes as per correspondence with Alex Fink.
\item {\bf v.10.0} (2016/02/29) -- Minor corrections for formatting and typos. Added Mi'kmaq corrections to v.9.0 changelog entry. Removed Proto-Norse sound changes due to inaccuracies per Elector Dark's request. Added an alternate set of changes to Scots and a set of changes from Scots to Falkirk Scots; added new changes for Proto-Norse, Old Norse, and Early Icelandic; added a section on the Muskogean languages; added changes to Orkney Norn and Shetland Norn; added correspondences from two Mono-Kawaiisu languages. Finished compiling sound changes from Proto-Nyulnyulan to Bardi; fixed the formatting in that section and added citations and a Proto-Nyulnyulan phonemic inventory. Created a section for Macro-Chibchan and moved Lenmichian under Macro-Chibchan. Added Chibchan correspondences. Replaced the original Tocharian sound changes with sound changes contributed by Nortaneous. Added rGyalrongic correspondences; added Tsouic correspondences. Added a section on the Tai-Kadai languages and added correspondences from Tai. Added Abenaki correspondences; replaced Sardinian correspondences with contributions from qwed117. Added Waray correspondences which I forgot to add to previous versions. Added some Gbe correspondences; added section on Macro-Panoan and added Tacanan correspondences. Added Cypriot Arabic correspondences.
\item v.10.1 (2016/03/07) -- Minor corrections.
\item v.10.2 (2016/03/31) -- Minor corrections and amendments.
\end{itemize}
\clearpage

\section{Key to Abbreviations}\tab Unless otherwise noted, the symbols below stand for: \\\\
" = Stress \\
! = Except when\textellipsis \\
(\textellipsis X) = For any number of X remaining \\
X$_0$ = The same/an identical X \\
X$^n$ = X with a given tone\\
X$_n$ = The \textit{n}th X of a sequence or series \\
X$_x$ = All X of a sequence or series\\
\d{X} = Retroflex/emphatic X \\
\# = Word boundary \\
\$ = Stem boundary \\
\% = Syllable boundary (or if X is one syllable away, or just representing a syllable in some changes from KneeQuickie or the ZBB) \\
\O\ = Nothing/Null/Zero \\
A = Affricate \\
B = Back vowel \\
C = Consonant \\
D = Voiced plosive \\
E = Front vowel \\
F = Fricative \\
H = Laryngeal \\
J = Approximant \\
K = Velar \\
\'{K} = Palatovelar \\
L = Liquid \\
M = Diphthong \\
N = Nasal \\
O = Obstruent \\
P = Labial/Bilabial \\
Q = Uvular consonant; click consonant (Khoisan) \\
R = Resonant/Sonorant \\
S = Plosive \\
T = Voiceless plosive \\
U = Syllable \\
V = Vowel \\
W = Semivowel \\
Z = Continuant \clearpage

\section{Afro-Asiatic}\tab For these Afro-Asiatic changes, s$_1$, s$_2$, s$_3$, h$_1$, and h$_2$ are consonants, believed to have most likely been fricatives, of indeterminate reconstruction. Dashes denote stem boundaries.

\tab The phonemic inventory of Proto-Afro-Asiatic has been reconstructed as follows:
%%%%% Possibly incorrect here�corroborate with somebody else?

\begin{center}\begin{tabular}{c | c c c c c c c c}
& Labial & Alveolar & Palatal & Velar & Pharyngeal & Glottal \\ \hline
Nasal & \ipa{m} & \ipa{n} & & & & \\
Plosive & \ipa{p p' b} & \ipa{t t' t\super l' d d\super l} & \ipa{c c' \textbardotlessj} & \ipa{k k\super w k'  k\super w' g g\super w} & & \textipa{P} \\
Fricative & \ipa{f} & \ipa{s s' z} & & \ipa{x x\super w G G\super w} & \textipa{\textcrh} \textipa{Q} & \ipa{h} \\
Lat. Fric. & & \textipa{\textbeltl} & & & & \\
Affricate & & \ipa{\t*{ts} \t*{dz}} & & & & \\
Trill & & \ipa{r} & & & & \\
Approximant & & \ipa{l} & \ipa{j} & \ipa{w} & & \\ \end{tabular}\end{center}

\begin{center}\begin{tabular}{c | c c c}
& Front & Central & Back \\ \hline
Close & \ipa{i} & & \ipa{u} \\
Open & & \ipa{a} & \end{tabular}\end{center}

\tab (From Fallon, Paul D. (2009), \textquotedblleft The Velar Ejective in Proto-Agaw". In \textit{Selected Proceedings of the 39th Annual Conference on African Linguistics}, Ojo, Akinloye and Lioba Moshi (Eds.), 10 -- 22. Sommerville, MA: Cascadilla Proceedings Project. \url{http://www.lingref.com}, document \#2182, citing Ehret, Christopher (1995), \textit{Reconstructing Proto-Afroasiatic (Proto-Afrasian): Vowels, Tone, Consonants, and Vocabulary (Voices from Asia)}; and from \url{http://starling.rinet.ru/cgi-bin/response.cgi?root=config\&morpho=0\&basename=\backslash data\backslash semham\backslash afaset\&first=1})

\subsection{Proto-Afro-Asiatic to Proto-Omotic}\textit{Mecislau}, from Ehret, Christopher (1995), \textit{Reconstructing Proto-Afroasiatic (Proto-Afrasian): Vowels, Tone, Consonants, and Vocabulary (Voices from Asia)}

\ipa{dz S tS} \change\ \ipa{Z} s$_1$ s$_2$\\
\ipa{dZ} \change\ \ipa{tS} \change\ \ipa{S}\\
\ipa{t} \change\ \O\ / _s\#\\
\ipa{\textbeltl} \change\ \ipa{l}\\
\ipa{f} \change\ \ipa{p}\\
\ipa{a(:)} \change\ \ipa{e(:)} / _\{\ipa{Q,q}\}\$\\
\ipa{q Q} \change\ \ipa{P h}\\
\ipa{a} \change\ \ipa{o} / \#Cw_\{\ipa{(d)l},s$_3$\}\\
\ipa{w} \change\ \O\ / \#C_V, except _\ipa{i(:)}\\
\ipa{S} \change\ s$_2$ / \{\ipa{i,j}\}_\\
VNC \change\ V\ipa{:}C[+voiced]

\subsubsection{Proto-Omotic to North Omotic}\textit{Mecislau}, from Ehret, Christopher (1995), \textit{Reconstructing Proto-Afroasiatic (Proto-Afrasian): Vowels, Tone, Consonants, and Vocabulary (Voices from Asia)}

\ipa{u o} \change\ \ipa{i e}\\
\ipa{e} \change\ \ipa{i} / \#N_C\\
\ipa{e} \change\ \ipa{i} / \#\ipa{l}_\{P,C[+voiced]\}\\
\ipa{e} \change\ \ipa{i} / \#\ipa{b}_\\
\ipa{e} \change\ \ipa{i} / \ipa{p}_\ipa{r}\\
\ipa{e} \change\ \ipa{i} / \#\{\ipa{s,S,ts'}\}_\{\ipa{k(w),P}\}\\
\ipa{e o} \change\ \ipa{i u} / \#C_P\\
\ipa{e o} \change\ \ipa{i u} / \#(\ipa{P})_C\\
\ipa{e o} \change\ \ipa{i u} / \#\{\ipa{k('),x}\}_\{\ipa{t('),ts'}\}\\
\ipa{e o} \change\ \ipa{i u} / \#(\ipa{P})_C\$\\
\ipa{e o} \change\ \ipa{i u} / \#P_\{\ipa{ts',tS'}\}\\
\ipa{a} \change\ \ipa{o} / \#\{\ipa{z,dZ}\}_P\\
\ipa{e(:)} \change\ \ipa{i(:)} / \#C[+sibilant]_\{\ipa{d,n,r}\}\\
Cw \change\ C\\
V\ipa{:} \change\ V / \#K[-voice]_C\\
\ipa{u} \change\ \ipa{u:} / \#S[+voice]_P[-voice]\\
V\ipa{:} \change\ V / \#C_C\$ + \$(V)C\$ suffix\\
N \change\ \O\ / V_\{C[+sibilant],\ipa{p}\}

\paragraph{North Omotic to Bench}\textit{Mecislau}, from Ehret, Christopher (1995), \textit{Reconstructing Proto-Afroasiatic (Proto-Afrasian): Vowels, Tone, Consonants, and Vocabulary (Voices from Asia)}

x$_1$ \change\ \ipa{k}\\
x$_2$ \change\ \ipa{k} / \#_\\
x$_2$ s$_3$ \change\ \O\ \ipa{S} / V_V\\
\ipa{tS} \change\ \ipa{ts}\\
s$_x$ \change\ \ipa{S}\\
\{\ipa{P},h$_x$\} \change\ \O\\
\ipa{l} \change\ \ipa{d} / \#_VC\\
\ipa{l} \change\ \ipa{n} / \#_VN\\
\ipa{d'} \change\ \ipa{t'}

\paragraph{North Omotic to Dizin}\textit{Mecislau}, from Ehret, Christopher (1995), \textit{Reconstructing Proto-Afroasiatic (Proto-Afrasian): Vowels, Tone, Consonants, and Vocabulary (Voices from Asia)}

\ipa{p'} \change\ \ipa{b}\\
\ipa{z} \change\ \ipa{d} / V\ipa{j}_\\
\ipa{ts'} \change\ \ipa{Z} / V_\\
x$_1$ \change\ \ipa{k}\\
x$_2$ \change\ \ipa{k} / _\#\\
x$_2$ \change\ \O\ / V_V\\
\ipa{Z} \change\ \{\ipa{tS,ts}\}\\
\ipa{ts} \change\ \ipa{tS} / _\ipa{i}\\
s$_x$ \change\ \ipa{tS}\\
s$_1$ \change\ \ipa{S}\\
\ipa{P} \change\ \O\\
\{h$_1$,h$_2$\} \change\ h\\
\ipa{d'} \change\ \ipa{t'}

\paragraph{North Omotic to Kafa}\textit{Mecislau}, from Ehret, Christopher (1995), \textit{Reconstructing Proto-Afroasiatic (Proto-Afrasian): Vowels, Tone, Consonants, and Vocabulary (Voices from Asia)}

\ipa{b} \change\ \ipa{w} / _\$\#\\
\ipa{p} \change\ \ipa{f} / V_\\
\ipa{z z:} \change\ \ipa{j dZ:}\\
\ipa{s} \change\ \ipa{S} / !V_\\
\ipa{ts'} \change\ \ipa{tS'}\\
x$_1$ \change\ \ipa{k}\\
x$_2$ \change\ \ipa{k} / \#_\\
x$_2$ \change\ \O\ / V_V\\
\{s$_3$,\ipa{Z}\} \change\ \ipa{S} / \#_\\
\{\ipa{ts,Z}\} \change\ \ipa{tS} / V_\\
s$_3$ \change\ \ipa{S} / V_V\\
s$_3$ \change\ \ipa{s} / V_\$\#\\
\ipa{ts'} \change\ \ipa{tS'} \\
\ipa{\textltailn} \change\ \ipa{n}\\
h$_2$ \change\ \ipa{w} / \#_\\
\ipa{l} \change\ \ipa{d} / \#_VC\\
\ipa{l} \change\ \ipa{n} / \#_V\ipa{b}\\
\ipa{d'} \change\ \ipa{t'}

\paragraph{North Omotic to Maale}\textit{Mecislau}, from Ehret, Christopher (1995), \textit{Reconstructing Proto-Afroasiatic (Proto-Afrasian): Vowels, Tone, Consonants, and Vocabulary (Voices from Asia)}

\ipa{b} \change\ \ipa{w} / V_V\\
\ipa{p} \change\ \ipa{f} / V_\\
\ipa{z} \change\ \ipa{d} / V_\\
\ipa{z} \change\ \ipa{ts} / V\ipa{j}_\\
x$_1$ \change\ \ipa{k}\\
x$_2$ \change\ \ipa{h} / \#_\\
x$_2$ \change\ \ipa{g} / V_V\\
\ipa{ts ts:} \change\ \ipa{s ts} / V_\\
s$_x$ \change\ \ipa{S} \\
\ipa{ts'} \change\ \ipa{tS'} / \#_\\
\ipa{ts'} \change\ \ipa{s} / V_\\
\ipa{\textltailn} \change\ \ipa{n}\\
h$_2$ \change\ \ipa{w} / \#_

\paragraph{North Omotic to Shekkacho}\textit{Mecislau}, from Ehret, Christopher (1995), \textit{Reconstructing Proto-Afroasiatic (Proto-Afrasian): Vowels, Tone, Consonants, and Vocabulary (Voices from Asia)}

\ipa{b} \change\ \ipa{w} / V_V\\
\ipa{p'} \change\ \ipa{p} / V_\\
\ipa{p'} \change\ \ipa{b}\\
\ipa{z} \change\ \ipa{j} / \{\#,V\}_\\
\ipa{z} \change\ \ipa{dZ:} / V\ipa{j}_\\
\ipa{z:} \change\ \ipa{dZ:}\\
\ipa{s} \change\ \ipa{S} / ! V_\\
\ipa{ts'} \change\ \ipa{tS'}\\
x$_1$ \change\ k\\
x$_2$ \change\ \O\ / V_V\\
\ipa{Z} \change\ \ipa{S} / \#_\\
\{s$_3$,\ipa{ts,Z}\} \change\ \ipa{s} / _\$\#\\
\ipa{ts} \change\ \ipa{S} / V_\\
s$_3$ \change\ \ipa{S} / \#_\\
s$_3$ \change\ \ipa{s} / V_\$\#\\
s$_2$ \change\ \ipa{S}\\
s$_2$ \change\ \{\ipa{s,tS:}\} / V_\\
h$_1$ \change\ \{\ipa{h},\O\} / \#_\\
h$_2$ \change\ \ipa{w} / \#_\\
\ipa{l} \change\ \ipa{d} / \#_VC\\
\ipa{l} \change\ \ipa{n} / \#_V\ipa{b}\\
\ipa{d'} \change\ \ipa{t'}

\paragraph{North Omotic to Wolaytta}\textit{Mecislau}, from Ehret, Christopher (1995), \textit{Reconstructing Proto-Afroasiatic (Proto-Afrasian): Vowels, Tone, Consonants, and Vocabulary (Voices from Asia)}

\ipa{b} \change\ \ipa{w} / V_V\\
\ipa{p} \change\ \ipa{f}\\
x$_1$ \change\ \ipa{k}\\
x$_2$ \change\ \O\ / V_V\\
x$_2$ s$_3$ \change\ \ipa{k S} / V_\$\#\\
\ipa{s} \change\ s$_3$ / V_(V)\\
s$_1$ \change\ \ipa{S} \\
s$_2$ \change\ \ipa{s} / V_\\
\ipa{\textltailn} \change\ \ipa{n}\\
\ipa{l} \change\ \ipa{n} / \#_VN\\
\ipa{d'} \change\ \ipa{t'} / \#_

\paragraph{North Omotic to Yemsa}\textit{Mecislau}, from Ehret, Christopher (1995), \textit{Reconstructing Proto-Afroasiatic (Proto-Afrasian): Vowels, Tone, Consonants, and Vocabulary (Voices from Asia)}

\ipa{b} \change\ \ipa{w} / V_V\\
\ipa{p} \change\ \ipa{f}\\
\ipa{p'} \change\ \ipa{b}\\
\ipa{z} \change\ \ipa{d} / V_\\
x$_1$ \change\ \ipa{k}\\
x$_2$ \change\ \ipa{k} / \#_\\
x$_2$ \change\ \O\ / V_V\\
\ipa{k'} \change\ \ipa{k} / \#\\
\ipa{tS Z} \change\ \ipa{Pj s}\\
\ipa{ts} \change\ \ipa{s} / \#_\\
s$_x$ \change\ \ipa{S} \\
\ipa{tS'} \change\ \ipa{tS}\\
\ipa{\textltailn} \change\ \ipa{n}\\
h$_1$ \change\ \{\ipa{h},\O\} / \#_\\
h$_2$ \change\ \ipa{w} / \#_\\
\ipa{l} \change\ \ipa{n} / \#_VC\\
\ipa{d'} \change\ \ipa{t}\\
\ipa{r} \change\ \{\ipa{r},\ipa{l:}\} / V_

\paragraph{North Omotic to Zayse-Zergulla}\textit{Mecislau}, from Ehret, Christopher (1995), \textit{Reconstructing Proto-Afroasiatic (Proto-Afrasian): Vowels, Tone, Consonants, and Vocabulary (Voices from Asia)}

\ipa{b} \change\ \ipa{w} / V_V\\
\ipa{p'} \change\ \ipa{Pp}\\
\ipa{z} \change\ \ipa{ts} / V\ipa{j}_\\
\ipa{ts'} \change\ \ipa{s'}\\
x$_1$ \change\ \ipa{k}\\
x$_2$ \change\ \ipa{h} / \#_\\
x$_2$ \change\ \O\ / V_V\\
x$_2$ \change\ \ipa{g} / \ipa{n}_\\
x$_3$ \change\ \ipa{g} / V_\#\\
\ipa{ts:} \change\ \ipa{ts} / V_\\
\{s$_1$,s$_3$\} \change\ \ipa{S} \\
s$_2$ \change\ \ipa{tS} / V_\\
\ipa{ts'} \change\ \{\ipa{tS',s}\}\\
\ipa{\textltailn} \change\ \ipa{n}\\
\ipa{l} \change\ \ipa{n} / \#_VN

\subsubsection{South Omotic}

\paragraph{South Omotic to Aari}\textit{Mecislau}, from Ehret, Christopher (1995), \textit{Reconstructing Proto-Afroasiatic (Proto-Afrasian): Vowels, Tone, Consonants, and Vocabulary (Voices from Asia)}

\ipa{p'} \change\ \{\ipa{b,p}\}\ipa{'}\\
\ipa{z} \change\ \{\ipa{d,z}\} / V_\\
\{x$_1$,x$_2$\} \change\ \ipa{g}\\
\ipa{k'} \change\ \ipa{q}\\
\ipa{tS} \change\ \ipa{ts}\\
s$_1$ s$_2$ s$_3$ \change\ \ipa{S z tS}\\
h$_1$ \change\ \O

\paragraph{South Omotic to Dime}\textit{Mecislau}, from Ehret, Christopher (1995), \textit{Reconstructing Proto-Afroasiatic (Proto-Afrasian): Vowels, Tone, Consonants, and Vocabulary (Voices from Asia)}

\ipa{p} \change\ \ipa{f}\\
\ipa{z} \change\ \{\ipa{d,z}\} / V_\\
\ipa{k'} \change\ \ipa{g'} / \#_\\
\ipa{tS} \change\ \ipa{ts}\\
\ipa{ts} \change\ \ipa{S} / _\ipa{i}\\
s$_1$ \change\ \ipa{S}\\
s$_2$ s$_3$ \change\ \ipa{tS: tS} / V_

\subsection{Proto-Afro-Asiatic to Proto-Erythrean}\textit{Mecislau}, from Ehret, Christopher (1995), \textit{Reconstructing Proto-Afroasiatic (Proto-Afrasian): Vowels, Tone, Consonants, and Vocabulary (Voices from Asia)}

\ipa{tS dZ} \change\ \ipa{ts dz}

\subsubsection{Proto-Erythrean to Proto-Cushitic}\textit{Mecislau}, from Ehret, Christopher (1995), \textit{Reconstructing Proto-Afroasiatic (Proto-Afrasian): Vowels, Tone, Consonants, and Vocabulary (Voices from Asia)}

\ipa{b} \change\ \ipa{m} / \#_V\ipa{n}\\
\ipa{g} \change\ \ipa{k} / \#\{\ipa{d,w}\}V_\\
\ipa{G} \change\ \ipa{g} / \#_V\ipa{x}\$

\paragraph{Agaw}

\subparagraph{Proto-Agaw to Awngi}{\it Pogostick Man}, from Fallon, Paul D. (2009), ``The Velar Ejective in Proto-Agaw". In {\it Selected Proceedings of the 39th Annual Conference on African Linguistics}, Ojo, Akinloye and Lioba Moshi (Eds.), 10 -- 22. Sommerville, MA: Cascadilla Proceedings Project. \textless\url{http://www.lingref.com}\textgreater, document \#2182, citing Appleyard, David L. (2006), {\it A comparative dictionary of the Agaw languages}. (Cushitic Language studies, 24.) Cologne: R\"{u}diger K\"{o}ppe Verlag.

\tab {\it NB: Does not include vowel developments.}

\{\ipa{x,\;G}\}(\ipa{\super w}) \change\ \O\ / at word boundaries\\
\ipa{z dz g} \change\ \ipa{g} \{\ipa{z,dz}\} \ipa{g(\super w)}\\
\{\ipa{x,\;G}\}(\ipa{\super w}) \change\ \ipa{G(\super w)}\\
\ipa{k' k\super w'} \change\ \{\ipa{G,q}\} \ipa{G\super w}\\
\ipa{P} \change\ \O

\subparagraph{Proto-Agaw to Blin}{\it Pogostick Man}, from Fallon, Paul D. (2009), ``The Velar Ejective in Proto-Agaw". In {\it Selected Proceedings of the 39th Annual Conference on African Linguistics}, Ojo, Akinloye and Lioba Moshi (Eds.), 10 -- 22. Sommerville, MA: Cascadilla Proceedings Project. \textless\url{http://www.lingref.com}\textgreater, document \#2182, citing Appleyard, David L. (2006), {\it A comparative dictionary of the Agaw languages}. (Cushitic Language studies, 24.) Cologne: R\"{u}diger K\"{o}ppe Verlag.

\tab {\it NB: Does not include vowel developments.}

\{\ipa{x,\;G}\}(\ipa{\super w}) \change\ \O\ / at word boundaries\\
\ipa{\;G(\super w)} \change\ \ipa{x(\super w)} / else\\
\{\ipa{ts,tS}\} \ipa{z dz} \change\ \ipa{S d dZ}\\
\ipa{t} \change\ \ipa{r} / medially

\subparagraph{Proto-Agaw to Kemantney}{\it Pogostick Man}, from Fallon, Paul D. (2009), ``The Velar Ejective in Proto-Agaw". In {\it Selected Proceedings of the 39th Annual Conference on African Linguistics}, Ojo, Akinloye and Lioba Moshi (Eds.), 10 -- 22. Sommerville, MA: Cascadilla Proceedings Project. \textless\url{http://www.lingref.com}\textgreater, document \#2182, citing Appleyard, David L. (2006), {\it A comparative dictionary of the Agaw languages}. (Cushitic Language studies, 24.) Cologne: R\"{u}diger K\"{o}ppe Verlag.

\tab {\it NB: Does not include vowel developments.}

\{\ipa{x,\;G}\}(\ipa{\super w}) \change\ \O\ / at word boundaries\\
\ipa{x} \change\ \O\\
\ipa{x\super w \;G\super w} \change\ \ipa{w G\super w}\\
\{\ipa{ts,tS}\} \ipa{dz} \change\ \ipa{S dZ}\\
\ipa{t} \change\ \ipa{j} / medially\\
\ipa{k'} \change\ \ipa{X\super w} / \#_\\
\ipa{k\super w'} \change\ \ipa{X\super w}\\
\ipa{P} \change\ \O

\subparagraph{Proto-Agaw to Xamtanga}{\it Pogostick Man}, from Fallon, Paul D. (2009), ``The Velar Ejective in Proto-Agaw". In {\it Selected Proceedings of the 39th Annual Conference on African Linguistics}, Ojo, Akinloye and Lioba Moshi (Eds.), 10 -- 22. Sommerville, MA: Cascadilla Proceedings Project. \textless\url{http://www.lingref.com}\textgreater, document \#2182, citing Appleyard, David L. (2006), {\it A comparative dictionary of the Agaw languages}. (Cushitic Language studies, 24.) Cologne: R\"{u}diger K\"{o}ppe Verlag.

\tab {\it NB: Does not include vowel developments.}

\{\ipa{x,\;G}\} \change\ \O\\
\{\ipa{x\super w,\;G\super w}\} \change\ \O\ / at word boundaries\\
\{\ipa{x\super w,\;G\super w}\} \change\ \ipa{w} / else\\
\ipa{ts tS dz} \change\ \ipa{s' tS' z}\\
\ipa{k} \change\ \{\ipa{k('),q}\}\\
\ipa{k'} \change\ \{\ipa{X\super w,q\super w}\} / \#_\\
\ipa{k'} \change\ \ipa{q} / else\\
\ipa{P} \change\ \O

\subsubsection{Proto-Erythrean to Proto-North Erythrean}\textit{Mecislau}, from Ehret, Christopher (1995), \textit{Reconstructing Proto-Afroasiatic (Proto-Afrasian): Vowels, Tone, Consonants, and Vocabulary (Voices from Asia)}

V\{\ipa{j,w}\} \change\ V\ipa{:} / C_C\\
\ipa{e: o:} \change\ \ipa{i u}\\
\{\ipa{e,o}\} \{\ipa{i,u}\} \change\ \ipa{a @}\\
\$VC\$ \change\ \$CV\$ \textit{``(This last rule turned all VC roots into CV)''}\\
\ipa{in} \change\ \ipa{N} / \#_C

\paragraph{Proto-North Erythrean to Proto-Chadic}\textit{Mecislau}, from Ehret, Christopher (1995), \textit{Reconstructing Proto-Afroasiatic (Proto-Afrasian): Vowels, Tone, Consonants, and Vocabulary (Voices from Asia)}

\ipa{a:} \change\ \ipa{a}\\
\ipa{\textcrh\ Q} \change\ \ipa{h P}\\
\ipa{ts dz} \{\ipa{t,ts}\}\ipa{' tS'} \change\ \ipa{s z s' S'}\\
\ipa{N} \change\ \O\ / V_\{\ipa{ts,q}\}

\subparagraph{Proto-North Erythrean to Proto-Boreafrasian}\textit{Mecislau}, from Ehret, Christopher (1995), \textit{Reconstructing Proto-Afroasiatic (Proto-Afrasian): Vowels, Tone, Consonants, and Vocabulary (Voices from Asia)}

\ipa{s'} \change\ \ipa{s}\\
\ipa{h} \change\ \ipa{\textcrh} / \#_V\ipa{s}\\
\ipa{z} \change\ \ipa{d} / {\it ``when another sibilant is in the word nearby''} and (word-finally?) when {\it ``noun-stem final''}\\
\{\ipa{\textltailn,Nw}\} \change\ \ipa{n}\\
V \change\ \O\ / _\# {\it ``in nominals''}\\
\ipa{N} \change\ \O\ / \#_CV

\subparagraph{Proto-Boreafrasian to Egypto-Berber}\textit{Mecislau}, from Ehret, Christopher (1995), \textit{Reconstructing Proto-Afroasiatic (Proto-Afrasian): Vowels, Tone, Consonants, and Vocabulary (Voices from Asia)}

\ipa{@} \change\ \ipa{i}\\
\ipa{h} \change\ \ipa{\textcrh} / _V\ipa{z}\\
\ipa{l} \change\ \O\ / \#\{\ipa{d,t'}\}_VC\\
\ipa{\textbeltl} \change\ \ipa{s} / \#_VC\\
\{\ipa{S,ts,z}\} \ipa{dz tS} \{\ipa{t',tS'}\} \ipa{dZ} \change\ \ipa{s z ts ts' dZ}\\
\ipa{f} \change\ \ipa{p} / \#_V\{Z,C[-voice],\ipa{r}\}\\	
\ipa{p'} \change\ \ipa{p}\\
\ipa{p} \change\ \ipa{b} / \#\ipa{dl}V_\\
\ipa{xw} \change\ \ipa{Gw} \change\ \ipa{\textcrh}\\
\ipa{k} \change\ \ipa{g} / _\{\ipa{w,j}\}\\
CV\ipa{Q} \change\ \ipa{\textcrh P} / ! C = \ipa{gw}\\
\ipa{gw}V\ipa{q} \change\ \ipa{Q}\\
\ipa{k(w)} \change\ \ipa{tS} / \#_V\ipa{t}\\
\ipa{g(w)} \change\ \ipa{dZ} / \#_V\ipa{d}\\
\ipa{x}V \change\ \ipa{k} / _\ipa{h}\\
K\super w \change\ K\\
\ipa{q} \change\ \O\ / _\ipa{i}\\
\ipa{q} \change\ \ipa{i} / \#_V\{Z,C[+dental]\}\\
\ipa{Q} \change\ \ipa{i} / \#_VR\\
\ipa{qu} \change\ \ipa{w} / _\{\ipa{f,s}\} (sporadic)\\
\ipa{P} \change\ \ipa{Q} / _V\{\ipa{n,r,g}\}\\
\{\ipa{h,\textcrh,q}\} \change\ \ipa{Q} / C[+voice]_V\\
\ipa{q} \change\ \ipa{P} / _C[+dental]\\
\{\ipa{h,\textcrh}\} \change\ \ipa{P} / KV_\\
\ipa{q} \change\ \ipa{P} / \ipa{h}_\\
\ipa{qh} \change\ \ipa{Q\textcrh}\\
\ipa{G} \change\ \ipa{Q} / \ipa{\textcrh}_\\
\ipa{tl'} \change\ \ipa{dl} / \#_V\ipa{\textcrh r}\\
O[+lateral] \change\ O[+palatal]\\
\ipa{r} \change\ \ipa{l} / \#_V(V)O[+labial]\\
\ipa{r} \change\ \ipa{P} / C_\{\ipa{t,w,j}\}\# ! C = \{\ipa{g,m,n,r,w,S,x}\}\\
\ipa{l} \change\ \ipa{j} / \#_\ipa{i}C ? \\ %check conditioning\\
\ipa{l} \change\ \ipa{r} / \#\ipa{n}V_C\\
\ipa{l} \change\ \ipa{n}

\subparagraph{Ancient Egyptian to Coptic}\textit{Mecislau}, from Ehret, Christopher (1995), \textit{Reconstructing Proto-Afroasiatic (Proto-Afrasian): Vowels, Tone, Consonants, and Vocabulary (Voices from Asia)}

\ipa{n} \change\ \ipa{l} / \#_V\ipa{b}\\
\ipa{n} \change\ \ipa{l} / \#_(V)\{\ipa{s,S,h}\}V\{\ipa{m,b}\}\#\\
\ipa{n} \change\ \ipa{l} / \#_V\{\ipa{m,b}\}\{\ipa{s,S,h}\}\\
\ipa{n} \change\ \ipa{l} / \#_V\ipa{k}\\
\ipa{n} \change\ \ipa{l} / \ipa{m}V_C\\
\ipa{n} \change\ \ipa{l} / CV_\ipa{m}\\
\ipa{r} \change\ \ipa{l} / \#(C)_\ipa{c}(C)\# ? \\ %check this
\ipa{r} \change\ \ipa{l} / \#\ipa{o}_\#

\subparagraph{Proto-Boreafrasian to Proto-Semitic}\textit{Mecislau}, from Ehret, Christopher (1995), \textit{Reconstructing Proto-Afroasiatic (Proto-Afrasian): Vowels, Tone, Consonants, and Vocabulary (Voices from Asia)}

\ipa{q} \change\ \ipa{Q}\\
\{\ipa{i,u}\} \change\ \ipa{@}\\
\ipa{tl'} \change\ \ipa{\textbeltl} / _C[+sibilant]\\
\ipa{G} \change\ \ipa{g} / \#_VCH\\
\ipa{G(w)} \change\ \ipa{g} / \#_V\ipa{x}\\
\ipa{k'(w)} \change\ \ipa{k} / \#\ipa{dl}V_\\
\ipa{w} \change\ \O\ / _C\\
\ipa{P} \change\ \ipa{Q} / \#K_\ipa{r}\#

\subparagraph{Proto-Semitic to Classical Arabic}{\it Khavaragh}

\ipa{p} \change\ \ipa{f}\\
\ipa{T\super Q k\super Q} \change\ \ipa{D\super Q q}\\
\ipa{g} \change\ \ipa{g\super j} \change\ \ipa{dZ}\\
\ipa{s} \change\ \{\ipa{S,h}\} / in ``anaphora and certain derivational prefixes. . .[t]his is common to many other Semitic languages as well''\\
\ipa{S} \change\ \ipa{s}\\
\ipa{\textbeltl} \change\ \ipa{S}\\
\ipa{\textbeltl\super Q} \change\ \ipa{d\textbeltl\super Q} \change\ \ipa{d\super Q}\\
\ipa{m} \change\ \ipa{n} / ``in certain contexts, notably in the nunation''\\
V\{\ipa{j,w}\}V \change\ \ipa{a:} / some sequences\\
``assimilation in some of the longer vowels''

\subparagraph{Classical Arabic to Cypriot Arabic}{\it Pogostick Man}, from Borg, Alexander (1985), {\it Cypriot Arabic}

\tab {\it NB: Changes may not be in chronological order.}

S[+ voice] \change\ S[- voice]\\
\ipa{q} \change\ \ipa{k}\\
S \change\ [+ voice] / \{V,R\}_V\\
S \change\ [+ voice] / V_R\\
\ipa{\{T,D\} \{f,v\} \{x,G\}} voicing neutralized ``in contact with other fricatives''\\
S \change\ F / _S\\
\ipa{f T} \change\ \ipa{p t} / F_\\
\ipa{k x} \change\ \ipa{c \c{c}} / _\{\ipa{j},E\}\\
\ipa{\{l,n\}j} \change\ \ipa{j:}\\
\ipa{j} \change\ \ipa{c} / \{O,\ipa{r}\}\\
\ipa{j} \change\ \O\ / C\ipa{k}_\$\\
\ipa{nx} \change\ \ipa{x:}\\
\O\ \change\ F / N_\{O,\ipa{r}\} ! \ipa{m}_\ipa{f}\\
\{\ipa{D\super Q,d\super Q}\} \change\ \ipa{D}\\
\ipa{t\super Q s\super Q} \change\ \ipa{s t}\\
\ipa{P h} \change\ \O\ \ipa{x}\\
\O\ \change\ \ipa{i} / \#al\$_\ipa{z}\\
\ipa{dZ} \change\ \ipa{z}\\
\ipa{G \textcrh} \change\ \ipa{Q x}\\
\ipa{w} \change\ \ipa{v} / _\%\\
\ipa{w:} \change\ \ipa{v}\\
\ipa{j(:)} \change\ \O\ / V_E\\
\ipa{u: i:} \change\ \ipa{o: e:} / _\ipa{Q}\\
\ipa{u: i:} \change\ \ipa{o: e:} / \ipa{Q}_\\
\ipa{i} \change\ \ipa{a} / C\ipa{\super Q}_\{\ipa{q,G,Q}\}\\
\ipa{i} \change\ \ipa{a} / \{\ipa{q,G,Q}\}_C\ipa{\super Q}\\
\ipa{a} \change\ \ipa{i} / _C(C), when stressed\\
\ipa{u} \change\ \ipa{o} / _\{\ipa{Q,G,x,r}\}\\
\ipa{u} \change\ \ipa{o} / \{\ipa{Q,G,x,r}\}_\\
\{\ipa{u,a,i}\} \change\ \O\ / _\%, when stressed (short only)\\
Epenthesis in medial CCC clusters, often so that the syllable break is between the second and third consonants\\
\ipa{u: i:} \change\ \ipa{u i}\\
\ipa{a} \change\ \ipa{a} / _C[+ dorsal]\\
\ipa{a} \change\ \ipa{e} / _(C)(C)\ipa{i(:)}\\
\ipa{a:} \change\ \ipa{a} / ! _\#\\
\ipa{a} \change\ \ipa{\{u,o\}} / P_\\
\ipa{a} \change\ \ipa{\{u,o\}} / _P\\
\ipa{a} \change\ \O\ / _\ipa{t}, in the feminine ending\\
\ipa{a:P} \change\ \ipa{e} / E(C)(C)_\#\\
\ipa{a:} \change\ \ipa{a} / \{C\ipa{\super Q,w}\}_\#

\subparagraph{Classical Arabic to Egyptian Arabic}{\it Pogostick Man}, from Brustad, Kristen, Mahmoud Al-Batal, and Abbas Al-Tonsi (2010), {\it Alif Baa: Introduction to Letters and Sounds}, 3rd. Ed.; \url{http://en.wikipedia.org/wiki/Egyptian_Arabic}; At-Tonsi, Abbas, Heba Salem, and Nevenka Korica Sullivan (2013), {\it Umm al-Dunya: Advanced Egyptian Colloquial Arabic}; and from correspondence with my own Arabic professor, who is a native speaker of this dialect

\ipa{T D} \change\ \ipa{t d} / ``usually in numbers or cases where a short vowel has been deleted and it's in contact with another stop, e.g. CA/MSA \textit{\ipa{ka"Ti:r}} \change\ EA \textit{\ipa{kti:r}}''\\
\ipa{T D} \change\ \ipa{s z}\\
\ipa{D\super Q} \change\ \ipa{z\super Q}, occasionally \ipa{t\super Q}\\
\ipa{d\super Q} \change\ \ipa{z\super Q} (seems to be a sporadic change only affecting a few words, e.g. CA/MSA \textit{\ipa{"d\super Qa:bit\super Q}} \change\ EA \textit{\ipa{"z\super Qa:bit\super Q}})\\
\ipa{dZ} \change\ \ipa{g}\\
\ipa{i u} \change\ \ipa{e o} / only when short, ! _\#\\
\ipa{u} \change\ \{\ipa{o,u}\} / short only, _\#\\
\ipa{aj aw} \change\ \ipa{e: o:} / in U[+closed]\\
V\ipa{:} \change\ V / C_C\{\ipa{:},C\}V\\
V \change\ V\ipa{:} / C_CV in U[-stress]\\
V \change\ V\ipa{:} / _\# + suffix\\
\{\ipa{i,u}\} \change\ \O\ / VC_CV when unstressed (short only)\\
Some other short-vowel deletions\\
\O\ \change\ \ipa{e} / CVCC_CVCV (applies across word boundaries)\\
Resyllabification across word boundaries to prevent vowel-initial syllables\\
\ipa{r} gains emphatic status except when next to \ipa{i}, and even then it's becoming more common in that environment\\
\ipa{a(:)} \change\ \ipa{A(:)} / near emphatics\\
\ipa{a(:)} \change\ \ipa{A(:)} / if \ipa{A(:)} is elsewhere in the word\\
\ipa{a(:)} \change\ \ipa{\ae(:)} / else (sometimes it seems more like \ipa{E(:)} to me)\\
\ipa{q} \change\ \ipa{P} / except in several words, two of which are \textit{al-Q\^{a}hira} and \textit{mus\^{i}q\^{a}}\\
Two consecutive consonants assimilate to the voicing of the second (obstruents only?)\\
\{\{\ipa{s,z}\}(\ipa{\super Q}),\ipa{Z}\}\ipa{S} \change\ \ipa{S:}\\
\ipa{Q} \change\ \{\ipa{\|`Q,\textcrh}\} / _\ipa{h}\\
Final short vowel loss\\
\ipa{h} \change\ \O\ / in coda

\subparagraph{Classical Arabic to Coastal Hadhrami Arabic}{\it Pogostick Man}, from Wikipedia Contributors (2013), ``Hadhrami Arabic''. {\it Wikipedia, the Free Encyclopedia}. \textless \url{http://en.wikipedia.org/w/index.php?title=Hadhrami_Arabic&oldid=580700095}\textgreater

\ipa{dZ} \change\ \ipa{j}, occasionally \ipa{\textbardotlessj} or \ipa{dZ} in educated speech\\
\ipa{T D D\super Q} \change\ \ipa{t d d\super Q}\\
\ipa{q} \change\ \ipa{g}\\
\ipa{a:} \change\ \ipa{e:} / in Form VI ({\it taf\={a}`ala}) verbs, though these apparently coexist with forms having the original vowel as well, with semantic distinctions\\
\ipa{a:} \change\ \ipa{\ae:} / when not near emphatics\\
Epenthesis (it seems \ipa{i} is preferred) breaking up final consonant clusters\\
V[-long] \change\ \O\ / \#C_C, in some words\\

\subparagraph{Classical Arabic to W\={a}d\={\i} Hadhrami Arabic}{\it Pogostick Man}, from Wikipedia Contributors (2013), ``Hadhrami Arabic''. {\it Wikipedia, the Free Encyclopedia}. \textless \url{http://en.wikipedia.org/w/index.php?title=Hadhrami_Arabic&oldid=580700095}\textgreater

\ipa{dZ} \change\ \ipa{j}, occasionally \ipa{\textbardotlessj} or \ipa{dZ} in educated speech\\
\ipa{T D D\super Q} \change\ \ipa{t d d\super Q}\\
\ipa{d\super Q q} \change\ \ipa{D\super Q g}\\
\ipa{a:} \change\ \ipa{e:} / in Form VI ({\it taf\={a}`ala}) verbs, though these apparently coexist with forms having the original vowel as well, with semantic distinctions\\
\ipa{a:} \change\ \ipa{\ae:} / when not near emphatics\\
Epenthesis (it seems \ipa{i} is preferred) breaking up final consonant clusters\\
V[-long] \change\ \O\ / \#C_C (sporadic?)

\subparagraph{Classical Arabic to Hass\={a}niyya Arabic}{\it Pogostick Man}, from \url{http://en.wikipedia.org/wiki/Hass%C4%81n%C4%ABya}

\tab {\it NB: Words borrowed directly from CA/MSA seem to be immune to these changes. Also, unless otherwise noted, changes also apply to geminate consonants.}

\ipa{d\super Q q} \change\ \ipa{D\super Q g}\\
\ipa{f T} \change\ \ipa{v \|[z} (the article isn't exactly clear on what this second phone is)\\ %]
\ipa{P} \change\ \{\O,\ipa{j,w}\} / depending on the environment; again, the article is unclear\\
\ipa{x} \change\ \ipa{X} (conjectured based upon the following but not outright stated in the article)\\
\ipa{G:} \change\ \ipa{K:} \change\ \ipa{q:}\\
\ipa{G} \change\ \{\ipa{K,q}\}\\
V[-long] \change\ \O\ / C_\{C,\#\} (except for the feminine marker)\\
\ipa{aj aw} \change\ \ipa{e:(\super j) o:(\super w)} (sometimes, the article is unclear)\\
The conditioning on these next two changes is conjectured based upon the source:\\
--- \ipa{j w} \change\ \ipa{i u} / \#_CV\\
--- \ipa{j w} \change\ \ipa{i: u:} / \#_CC

\subparagraph{Classical Arabic to Iraqi Arabic}{\it Pogostick Man}, from \url{http://en.wikipedia.org/wiki/Varieties_of_Arabic}

\ipa{k q} \change\ \ipa{tS} \{\ipa{g,q}\} (\ipa{g} is more common)\\
\ipa{g\super j} \change\ \ipa{j} / in southern regions\\
\ipa{Q} \change\ \ipa{P\super Q}\\
\ipa{aj aw} \change\ \ipa{e: o:}

\subparagraph{Classical Arabic to Eastern Libyan Arabic}{\it Pogostick Man}, from \url{http://en.wikipedia.org/wiki/Libyan_Arabic}

\ipa{d\super Q dZ q} \change\ \ipa{D\super Q Z g}\\
\ipa{aj aw} \change\ \ipa{e(:,j) o(:,w)}\\
\O\ \change\ \ipa{@} / C_CV(\ipa{:},V)CC

\subparagraph{Classical Arabic to Western Libyan Arabic}{\it Pogostick Man}, from \url{http://en.wikipedia.org/wiki/Libyan_Arabic}

\ipa{q dZ} \change\ \ipa{g Z}\\
\ipa{T D(\super Q)} \change\ \ipa{t d(\super Q)}\\
\ipa{aj aw} \change\ \ipa{e: o:}\\
\O\ \change\ \ipa{@} / CCV(\ipa{:},V)C_C

\subparagraph{Classical Arabic to Moroccan Arabic}{\it Pogostick Man}, from \url{http://en.wikipedia.org/wiki/Moroccan_Arabic}

\ipa{t} \change\ \ipa{\t*{ts}} / plain \ipa{t} only, distinguishable from the sequence \ipa{ts}\\
\{\ipa{a,i}\} \change\ \ipa{@} / short only; the change of short \ipa{a} blocked for some speakers before \ipa{\textcrh\ Q}\\
\ipa{u} \change\ \ipa{@} / short only, except near ``a labial or velar consonant''\\
C[+labial/+velar] \change\ \ipa{\super w} / adjacent to short \ipa{u}\\
\{\ipa{u,@}\} \change\ \O\ / ! C_C(C)\#\\
\ipa{@} \change\ \ipa{a} / near \ipa{\textcrh\ Q}\\
\ipa{@} \change\ \ipa{5} / near emphatics\\
\ipa{@} \change\ \ipa{I} / else\\
\ipa{u} \change\ \ipa{U} / short only\\
\ipa{a: i: u:} \change\ \ipa{A: e: o:} / near emphatics\\
\ipa{a:} \change\ \ipa{\ae:} / else\\
C$_1$\ipa{\super Q}C$_2$ \change\ C$_1$C$_2$\ipa{\super Q}\\
C\ipa{\super Q} \change\ C / \{\#,V\}_V\\
\ipa{q} \change\ \{\ipa{q,g}\}\\
\ipa{dZ} \change\ \{\ipa{d,g}\} / if \ipa{s} or \ipa{z} occur somewhere else in the word\\
\ipa{dZ} \change\ \ipa{Z} / else\\
\ipa{s} \change\ \ipa{S} / if \ipa{S} is somewhere in the stem after it\\
\ipa{z} \change\ \ipa{Z} / if \ipa{Z} is somewhere in the stem after it

\subparagraph{Classical Arabic to Sa`idi Arabic}{\it Pogostick Man}, from \url{http://en.wikipedia.org/wiki/Sa%27idi_Arabic}

\tab {\it NB: This is probably highly incomplete.}

\ipa{q x G} \change\ \ipa{g X K}

\subparagraph{Classical Arabic to Sudanese Arabic}{\it Pogostick Man}, from \url{http://en.wikipedia.org/wiki/Sudanese_Arabic}

\ipa{dZ q} \change\ \ipa{g\super j \;G}\\
\ipa{u(:)} \change\ \{\ipa{8,o}\}(\ipa{:})

\subparagraph{Classical Arabic to Tunisian Arabic}{\it Pogostick Man}, from \url{http://en.wikipedia.org/wiki/Tunisian_Arabic}

\ipa{a} \change\ \ipa{A} / near emphatics\\
\ipa{a} \change\ \ipa{E} (sometimes)\\
\ipa{d\super Q q} \change\ \ipa{D\super Q} \{\ipa{g,q}\}\\
\ipa{dZ x G} \change\ \ipa{Z X K}\\
\ipa{aj aw} \change\ \{\ipa{aj,e:,i:} \ipa{aw,o:,u:}\}\\
V\ipa{:} \change\ V[-long] / _\# (except as below)\\
V(\ipa{:}) \change\ V\ipa{:} / in accented or stressed monosyllables

\subparagraph{Proto-Semitic to Biblical Hebrew}{\it Maknas}, from \url{http://www.adath-shalom.ca/history_of_hebrew.htm} ``and other sources''

\tab {\it NB: \ipa{@} could be realized as an ultrashort [\ipa{a}], [\ipa{e}], or [\ipa{o}] depending on its surroundings.}

\ipa{T T\super Q D \textbeltl\ \textbeltl\super Q} \change\ \ipa{S S\super Q z s s\super Q}\\
\ipa{x G k\super Q} \change\ \ipa{\textcrh\ Q q}\\
Some mergers involving \ipa{j} and \ipa{w}\\
Frequent \ipa{h}-dropping

\hspace{25pt}Stressed-vowel correspondences:\\
--- \ipa{a:} \change\ \ipa{o:} / ! _\#\\
--- \ipa{i i: u u:} \change\ \ipa{e@ i:@ o@ u:@} / _R\\
--- \ipa{i:} \change\ \ipa{e:} / _\#\\
--- \ipa{a} \change\ \ipa{a:} / _\$\\
--- \ipa{a} \change\ \{\ipa{a,E}\} (not common)\\
--- \ipa{i u} \change\ \ipa{e a} / _R\{\$,\#\} (in verbs)\\
--- \ipa{i u} \change\ \ipa{e o} / _C\{\$,\#\} (in verbs)\\
--- \ipa{i} \change\ \ipa{e: o:} / else\\
--- \ipa{aw} \change\ \ipa{a:w}\\
--- \ipa{aj} \change\ \ipa{e:} / _\$\\
--- \ipa{aj} \change\ \ipa{E:} / _\#

\hspace{25pt}Unstressed-vowel correspondences:\\
--- \{\ipa{o,u}\}(\ipa{:}) \change\ \ipa{i:} / _\$\%\ipa{o:}\\
--- \ipa{o:} \change\ \ipa{u:}\\
--- \ipa{a} \change\ \O\ / _\#\\
--- \ipa{a} \change\ _\$\%\%(\textellipsis)"\\
--- \ipa{a} \change\ \ipa{@} / _R if \ipa{@} in an adjacent syllable\\
--- \ipa{a} \change\ \ipa{@} / R_ if \ipa{@} in an adjacent syllable\\
--- \ipa{i} \change\ \ipa{@} / _R if a frontal allophone of \ipa{@} in an adjacent syllable\\
--- \ipa{i} \change\ \ipa{@} / R_ if a frontal allophone of \ipa{@} in an adjacent syllable\\
--- \ipa{u} \change\ \ipa{@} / _R if a backed allophone of \ipa{@} in an adjacent syllable\\
--- \ipa{u} \change\ \ipa{@} / R_ if a backed allophone of \ipa{@} in an adjacent syllable\\
--- \ipa{i} \change\ \ipa{a} / _R\\
--- \ipa{i} \change\ \ipa{a} / R_\\
--- \ipa{a i} \change\ \ipa{a: e:} / _\%"\\
--- \ipa{u} \change\ \ipa{O} / _C\{\$,\#\}\\
--- \ipa{u} \change\ \O\ / ! _C\ipa{:}\\
--- \ipa{aj aw} \change\ \ipa{e: o:}

\ipa{p b t d k g} \change\ \ipa{b v T D x G} / non-intial singletons\\
\ipa{Q} \change\ \O\ / _\{\$,\#\}\\
\ipa{j} \change\ \O\ / E_ (not clear whether only short E or long also)\\
at \change\ \ipa{a:} / _\# (in feminine noun endings)

\subparagraph{Biblical Hebrew to Modern Israeli Hebrew}{\it Maknas}, from \url{http://www.adath-shalom.ca/history_of_hebrew.htm} ``and other sources''

\tab {\it NB: These aren't all true sound changes per se, since Modern Israeli Hebrew was artificially revived and is an amalgamation of dialects.}

\{\ipa{e(:),E}\} \change\ \ipa{E}\\
V\ipa{:} \change\ V[-long]\\
\ipa{@} \change\ \ipa{a} / near gutturals\\
\ipa{@} \change\ \ipa{E} / \#R_C or when breaking up what would otherwise be a three-consonant cluster; in the case of two schwas, only the first one is dropped\\
\ipa{@} \change\ \O\ / else\\
\ipa{w T D G} \change\ \ipa{v t d g} (sometimes)\\
\ipa{x Q} \change\ \ipa{X P}\\
\ipa{s\super Q t\super Q \textcrh\ q} \change\ \ipa{s t X k}\\
\ipa{h} \change\ \O\ / _\#\\
\ipa{P} \change\ \O\ / ! in onset of U[+stress] (colloquial)\\
\ipa{h} \change\ \O\ (colloquial)\\
C\ipa{:} \change\ C[-long]\\
\ipa{r} \change\ \ipa{K}

\clearpage

\section{Algonquian}\tab Proto-Algonquian is believed to have had the following phonology, as provided for by the Wikipedia:

\begin{center}\begin{tabular}{c | c c c c c}
& Labial & Alveolar & Palatal/Postalveolar & Velar & Glottal \\ \hline
Nasal & \ipa{m} & \ipa{n} & & & \\
Plosive & \ipa{p} & \ipa{t} & & \ipa{k} & \textipa{P} \\
Fricative & & \textipa{T s} & \textipa{S} & & \ipa{h} \\
Affricate & & & \textipa{\t*{tS}} & & \\
Rhotic & & \ipa{r} & & & \\
Approximant & \ipa{w} & & \ipa{j} & & \end{tabular}\end{center}

\begin{center}\begin{tabular}{c | c c c}
& Front & Central & Back \\ \hline
Close & \textipa{i i:} & & \\
Mid & \textipa{e e:} & & \textipa{o o:} \\
Open & & \textipa{a a:} & \end{tabular}\end{center}

\tab The phoneme denoted as /\textipa{T}/ may well have been actually /\ipa{\textbeltl}/ instead. Also, some debate exists as to whether or not /r/ was originally an /\ipa{l}/.

\tab (From Wikipedia contributors (2011), \textquotedblleft Proto-Algonquian language". \textit{Wikipedia, The Free Encyclopedia}. \textless\url{http://en.wikipedia.org/w/index.php?title=Proto-Algonquian_language&oldid=440788532}\textgreater)

\subsection{Proto-Algonquian to Kennebec River Abenaki}{\it Pogostick Man with acknowledgment to dhok}, from Warne, Janet Leila (1973), ``A Historical Phonology of Abenaki''. \textless\url{http://digitool.library.mcgill.ca/R/?func=dbin-jump-full&object_id=46078&local_base=GEN01-MCG02}\textgreater

\ipa{hl} \change\ \ipa{s:}\\
\ipa{P} \change\ \ipa{h} / _\ipa{l}\\
\ipa{l} \change\ \ipa{n} / \#_\\
\ipa{l} \change\ \ipa{r}\\
\ipa{nr} \change\ \ipa{r:}\\
N \change\ \O\ / _O\\
\ipa{a:} \change\ \ipa{a} / _OO\\
\ipa{a} \change\ \ipa{e} / \#C_OO\\
\ipa{a} \change\ \O\ / \#_OO\\
V \change\ \O\ / _\#\\
\ipa{iw} \change\ \ipa{o} / _\#\\
\ipa{w} \change\ \O\ / _\# ! \ipa{k(:)}_\\
\ipa{j} \change\ \O\ / _\# ! P_\\
\ipa{w} \change\ \O\ / C_ ! C = K\\
\ipa{j} \change\ \O\ / _C\\
\ipa{i} \change\ \O\ / \#\ipa{w}_\\
\ipa{T} \change\ \ipa{n} / \#_\\
\ipa{T} \change\ \ipa{s} / _\ipa{k}\\
\ipa{T} \change\ \ipa{r}\\
\ipa{S tS} \change\ \ipa{s ts}\\
\ipa{\{x,h\}}S \change\ S\ipa{:}\\
\ipa{sk} \change\ \ipa{k:} / ! _\ipa{a}\\
\ipa{Ps Pts} \change\ \ipa{s: ts:}\\
\ipa{o: a: e: i:} \change\ \ipa{o \~{O} a i}

\subsection{Proto-Algonquian to St. Francis Abenaki}{\it Pogostick Man with acknowledgment to dhok}, from Warne, Janet Leila (1973), ``A Historical Phonology of Abenaki''. \textless\url{http://digitool.library.mcgill.ca/R/?func=dbin-jump-full&object_id=46078&local_base=GEN01-MCG02}\textgreater

\ipa{nT nl} \change\ \ipa{s: \{s:,hl\}}\\
\ipa{P} \change\ \ipa{h} / _\ipa{l}\\
N \change\ \O\ / _RO\\
\ipa{a:} \change\ \ipa{a} / _OO\\
\ipa{a:} \change\ \ipa{\~{O}}\\
V[- high - long] \change\ \O\ / \#C_OO\\
\ipa{a} \change\ \O\ / \#_OO\\
V \change\ \O\ / _\#\\
\ipa{iw} \change\ \ipa{o} / _\#\\
\ipa{w} \change\ \O\ / _\# ! \ipa{k(:)}_\\
\ipa{j} \change\ \O\ / _\# ! \ipa{p}_\\
\ipa{w} \change\ \O\ / C_ ! C = K\\
\ipa{j} \change\ \O\ / C_\\
\{R,\ipa{h}\} \change\ \O\ / V_V (sporadic?)\\
\ipa{T} \change\ \ipa{n} / \#_\\
\ipa{T} \change\ \ipa{s} / _\ipa{k}\\
\ipa{T} \change\ \ipa{l}\\
\ipa{S tS} \change\ \ipa{s ts}\\
\ipa{nj} \change\ \ipa{i} / \#_\\
\ipa{\{x,h\}}S \change\ S\ipa{:}\\
\ipa{Ps Pts} \change\ \ipa{s: ts:}\\
\ipa{sk} \change\ \ipa{k:} / ! _\ipa{a}\\
\ipa{i} \change\ \ipa{e} / _R\\
\ipa{o: e: i:} \change\ \ipa{o a i}

\subsection{Proto-Algonquian to Proto-Arapaho-Atsina}{\it Whimemsz}, from Goddard, Ives 
(1974), ``An Outline of the Historical Phonology of Arapaho and Atsina". {\it International 
Journal of American Linguistics}, 40:102 -- 116

(W)V \change\ \O\ / _\#\\
\ipa{we} \change\ \ipa{o}\\
\ipa{o(:)} \change\ \ipa{i(:)}\\
W \change\ \O\ / C_\ipa{i(:)}\\
\ipa{e} \change\ \ipa{i} / \#_\\
\ipa{T} \change\ \ipa{S} / C_\\
\ipa{T h} \{\ipa{s,m,n,r}\} \change\ \ipa{S} \O\ \ipa{P} / _C\\
\ipa{tS} \change\ \ipa{S} / _\ipa{p}\\
W \change\ \ipa{j} / C_\\
W \change\ \ipa{n} / \{\#,V\}_\\
\ipa{p k} \change\ \ipa{k} \O\\
\ipa{s} \change\ \ipa{n} / \#_\\
\ipa{s} \change\ \ipa{h} / \{V,C\}_\\
\ipa{r} \change\ \ipa{n} / \{\#,V\}_\\
\ipa{r} \change\ \ipa{h} / C_\\
\ipa{tS} \change\ \ipa{T}\\
V\ipa{:} \change\ V[-long] / _CC\\
\ipa{a(:)} \change\ \ipa{o(:)}

\subsubsection{Proto-Arapaho-Atsina to Arapaho}{\it Whimemsz}, from Goddard, Ives 
(1974), ``An Outline of the Historical Phonology of Arapaho and Atsina". {\it International 
Journal of American Linguistics}, 40:102 -- 116

\ipa{hP} \change\ \ipa{Ph}\\
(\{C,\#\}V)\ipa{P} \change\ (\{C,\#\}V\ipa{:})\O\ / _C\\
\ipa{i(:)} \change\ \ipa{u(:)} / \ipa{o(:)}(C[-dental])(C[-dental])_\\
\O\ \change\ \ipa{P} / CV[-long]_\#\\
\ipa{S m} \change\ \ipa{x w} / _B\\
\ipa{S m} \change\ \ipa{x w} / B_\#\\
\ipa{S m} \change\ \ipa{x b} / \ipa{o(:)}_\ipa{e(:)}\\
\ipa{S k m} \change\ \ipa{s tS b} / _\{E,\ipa{j}\}\\
\ipa{S k m} \change\ \ipa{s tS b} / E_\#\\
(V[-long])N \change\ \O\ / _\#\\
\O\ \change\ \ipa{h} / \#_V\\
\ipa{e:} \change\ \ipa{ei} / \ipa{j}_\\
\ipa{o(:)} \change\ \ipa{e(:)} / C\ipa{j}_ (sporadic)\\
\ipa{n} \change\ \O\ / _\ipa{j}\\
\ipa{j} \change\ \O\ / C_\\
\ipa{h} \change\ \O\ / _\#

\subsubsection{Proto-Arapaho-Atsina to Gros Ventre}{\it Whimemsz}, from Goddard, Ives 
(1974), ``An Outline of the Historical Phonology of Arapaho and Atsina". {\it International 
Journal of American Linguistics}, 40:102 -- 116

\ipa{hP} \change\ \ipa{Ph}\\
(\{C,\#\}V[-long])\ipa{P} \change\ (\{C,\#\}V\ipa{:}[+falling tone])\O\ / _C\\
\ipa{j} \change\ \O\ / \{\ipa{S,T}\}\\
\ipa{i} \change\ \ipa{u} / \ipa{o(:)}_\\
\ipa{S T m} \change\ \ipa{T t w} / _\ipa{o(:)}\\
\ipa{S T m k} \change\ \ipa{T t b tS} / _\ipa{e(:)}\\
\ipa{S} \{\ipa{T,t}\} \ipa{m k} \change\ \ipa{s ts t\super j b\super j} / _\{\ipa{i(:),j,}\#\}\\
(V[-long])N \change\ \O\ / _\#\\
\O\ \change\ \ipa{P} / \#_V\\
\ipa{n} \change\ \O\ / _\ipa{j}

\subsection{Proto-Algonquian to Blackfoot}{\it Whimemsz}, from Proulx, Paul (1989), ``A Sketch of Blackfoot Historical Phonology''. {\it International Journal of American Linguistics}, 55:43 -- 82

\{\ipa{T,tS,S,r}\} \change\ \ipa{t} / unless adjacent to another consonant\\
\O\ \change\ \ipa{x} / _\ipa{s} ! _C\{C,\#\}\\
\ipa{j} \change\ \ipa{s} / ! C_\\
\ipa{h} \change\ \O\ / ! _C\\
\{\ipa{S,tS}\} \change\ \ipa{s} / \#_\\
\{\ipa{PT,Pr}\} \change\ \{\ipa{P,j,}\O\}\\
\ipa{nr} \change\ \ipa{s}\\
\ipa{h} \change\ \ipa{x} / _\{\ipa{p,k}\}\\
\ipa{hkw} \change\ \ipa{Pk}\\
\ipa{nT hs} \change\ \ipa{s:t s:}\\
\ipa{m} \change\ \ipa{P} / _\ipa{p}\\
\ipa{nkw} \change\ \ipa{P:}\\
\{\ipa{n,s}\} \change\ \ipa{x} / _\ipa{t}\\
\ipa{ntS} \change\ \ipa{Pt}\\
\ipa{ns sk} \change\ \{\ipa{x,s}\} \{\ipa{x,s:}\}\\
\ipa{Tp tSp Sp} \change\ \{\ipa{x,s:}\} \ipa{Pp s:p}\\
\ipa{S} \change\ \ipa{x} / _\ipa{k}\\
\ipa{x} \change\ \ipa{s:} / \{\ipa{i},\#\ipa{e,ja,ke}\}_\\
\ipa{x} \change\ \ipa{s:} / \ipa{e(:)}_\ipa{s}\\
\O\ \change\ \ipa{s} / \{\ipa{i(:),}\#\ipa{e}\}_\ipa{t}\\
\ipa{t} \change\ \ipa{ts} / _\{\ipa{i,e(:),a}\}\\
\O\ \change\ \ipa{s} / \ipa{k}_\ipa{i(:)}\\
\ipa{s::} \change\ \ipa{s:}\\
\ipa{e} \change\ \ipa{i} / \{\#,\ipa{k}\}_\\
\{\ipa{a,e,i}\} \change\ \ipa{o} / _\ipa{k\super w}\\
\O\ \change\ \ipa{j} / \{\ipa{o:w,i:j}\}_\ipa{i}\#\\
\ipa{w} \change\ \ipa{j} / \{\ipa{r,k}\}_\ipa{i}\#\\
\O\ \change\ \ipa{i} / \ipa{r}_\ipa{w}\\
\ipa{w} \change\ \O\ / C_\\
\{\ipa{ji:,ja,ahi}\} \{\ipa{owa:,awa,awe}\} \ipa{awi} \{\ipa{i:wa,e:wa,aji,aje,ani}\} \change\ \ipa{i o: o(ji) i:} / C_C\\
\ipa{hi} \change\ \O\ / \ipa{a:}_\\
\O\ \change\ \ipa{i} / \#_\ipa{j}C\\
\O\ \change\ \ipa{i} / C_\ipa{j}V\#\\
\ipa{a} \change\ \ipa{o} / _\ipa{w}\\
\ipa{e}L\ipa{wi} \change\ \ipa{i:}\\
\ipa{i(:)wi} \change\ \ipa{i:} / medially\\
\ipa{i(:)wi} \change\ \ipa{i} / _\#\\
\ipa{j} \change\ \ipa{s}\\
\ipa{w} \change\ \ipa{j} / _\ipa{i}\\
\{\ipa{i:,ij,j}\} \change\ \ipa{j} / C_B\\
\ipa{iji} \change\ \ipa{i:}\\
\ipa{w} \change\ \O\ / \{\ipa{a,o}\}_\ipa{i}C\\
\ipa{on} \change\ \ipa{u} / _\ipa{i}C\\
\ipa{tem} \{\ipa{k,p}\}\ipa{en} \change\ \ipa{m: n:}\\
\ipa{ket} \change\ \ipa{t:} (\change\ \ipa{s:}?)\\
\ipa{ke(h)} \change\ \ipa{t:} ?\\
\{\ipa{k(\super w)es,keT}\} \change\ \ipa{s:}\\
\ipa{e} \change\ \O\ / O_ in \#U (not universal)\\
\{\ipa{me,ne}\} \change\ \O\ / \#_O ``(followed by truncation of following x)''\\
\{\ipa{we,wi:}\} \change\ \ipa{o} / \#_\\
\ipa{tsi} \change\ \O\ / \$_OO ``(before a prefix; the first obstruent of the follow[ing] cluster then becomes \ipa{P})\\
\ipa{a:} \change\ \ipa{aa} / W_ ! when _\{C\{C,\ipa{:}\},\#\}\\
\ipa{a:} \change\ \ipa{a} / else\\
\ipa{o:} \change\ \ipa{o}\\
\ipa{a} \change\ \ipa{i} / ! at word boundaries\\
\ipa{e} \change\ \ipa{a} / _\#\\
\{\ipa{e:,i:}\} \change\ \ipa{i}

\subsection{Proto-Algonquian to Cheyenne}{\it jburke}, from ``Bloomfield and Leman"

\ipa{o a} \change\ \ipa{e o}\\
\ipa{e i} \change\ \ipa{a e}\\
\ipa{p t k} \change\ \{\ipa{hp},\O\} \ipa{ht} \{\ipa{hk},\O\}\\
\{\ipa{(t)l,T}\} \change\ \ipa{t}\\
\ipa{s} \change\ \ipa{h}\\
\ipa{S tS} \change\ \{\ipa{S,x}\} \ipa{s}\\
\ipa{w j} \change\ \{\ipa{v,o}\} \{\ipa{t,e}\}\\
\{\ipa{k}C,C\ipa{k}\} \change\ \ipa{P}\\
C[- nasal] \change\ \O\ / near nasals\\
\ipa{m} \change\ \O\ / near nasals\\
\ipa{p} \change\ \{\ipa{t},\O\} / near consonants\\
W \change\ \O\ / near nasals or \ipa{t}

\subsection{Proto-Algonquian to Northern East Cree}{\it Pogostick Man}, from \url{http://home.kpn.nl/cvkolmes/ojibwe/CorrCrOj.pdf} and Wikipedia contributors (2015), ``Cree language''. {\it Wikipedia, the Free Encyclopedia}. \textless\url{https://en.wikipedia.org/w/index.php?title=Cree_language&oldid=690521189}\textgreater

\ipa{we} \change\ \ipa{o}\\
\ipa{e e:} \change\ \ipa{i a:}\\
\ipa{ja} \change\ \ipa{a:} / C_\\
\ipa{Sje:} \change\ \ipa{se:}\\
\ipa{lwi} \change\ \ipa{jo}\\
\ipa{wi} \change\ \ipa{o} / C_\\
\{\ipa{n,q,h}\} \change\ \O\ / _\ipa{s}\\
\ipa{q} \change\ \ipa{h} / _\ipa{tS}\\
N \change\ \ipa{h} / _S\\
\ipa{(t)Sp} \change\ \ipa{sp}\\
\{\ipa{q,S}\} \change\ \ipa{s} / _\ipa{t}\\
\ipa{T} \change\ \ipa{s} / _\ipa{k}\\
\ipa{l} \change\ \ipa{h} / _\ipa{k}\\
\{\ipa{n,q,h}\}\ipa{S} \{\ipa{n,q,h}\}\ipa{l} \change\ \ipa{s} \{\ipa{h,j,hj}\}\\
\ipa{k} \change\ \ipa{tS} / _\ipa{i}\\
\ipa{a} \change\ \ipa{i} / in some unaccented syllables (short only)\\
\ipa{T} \change\ \ipa{t}\\
\ipa{l} \change\ \ipa{j}

\subsection{Proto-Algonquian to Southern East Cree}{\it Pogostick Man}, from \url{http://home.kpn.nl/cvkolmes/ojibwe/CorrCrOj.pdf} and Wikipedia contributors (2015), ``Cree language''. {\it Wikipedia, the Free Encyclopedia}. \textless\url{https://en.wikipedia.org/w/index.php?title=Cree_language&oldid=690521189}\textgreater

\ipa{we} \change\ \ipa{o}\\
\ipa{e} \change\ \ipa{i} (short only)\\
\ipa{ja} \change\ \ipa{a:} / C_\\
\ipa{Sje:} \change\ \ipa{Se:}\\
\ipa{lwi} \change\ \ipa{jo}\\
\ipa{wi} \change\ \ipa{o} / C_\\
\{\ipa{n,q,h}\} \change\ \O\ / _\ipa{s}\\
\ipa{q} \change\ \ipa{h} / _\ipa{tS}\\
N \change\ \ipa{h} / _S\\
\ipa{(t)Sp} \change\ \ipa{sp}\\
\{\ipa{q,S}\} \change\ \ipa{s} / _\ipa{t}\\
\ipa{T} \change\ \ipa{s} / _\ipa{k}\\
\ipa{l} \change\ \ipa{h} / _\ipa{k}\\
\{\ipa{n,q,h}\}\ipa{S} \{\ipa{n,q,h}\}\ipa{l} \change\ \ipa{S l}\\
\ipa{S} \change\ \{\ipa{S,s}\} / in inland varieties; remains /\ipa{S}/ in coastal varieties\\
\ipa{k} \change\ \ipa{tS} / _\ipa{i}\\
\ipa{tS} \change\ \ipa{ts}\\
\ipa{T} \change\ \ipa{t}\\
\ipa{l} \change\ \ipa{j}

\subsection{Proto-Algonquian to Plains Cree}{\it Pogostick Man}, from \url{http://home.kpn.nl/cvkolmes/ojibwe/CorrCrOj.pdf} and Wikipedia contributors (2015), ``Cree language''. {\it Wikipedia, the Free Encyclopedia}. \textless\url{https://en.wikipedia.org/w/index.php?title=Cree_language&oldid=690521189}\textgreater

\ipa{we} \change\ \ipa{o}\\
\ipa{e} \change\ \ipa{i} (short only in southern dialects, both short and long in northern dialects)\\
\ipa{ja} \change\ \ipa{a:} / C_\\
\ipa{Sje:} \change\ \ipa{se:}\\
\ipa{lwi} \change\ \ipa{jo}\\
\ipa{wi} \change\ \ipa{o} / C_\\
\{\ipa{n,q,h}\} \change\ \O\ / _\ipa{s}\\
\ipa{q} \change\ \ipa{h} / _\ipa{tS}\\
N \change\ \ipa{h} / _S\\
\ipa{(t)Sp} \change\ \ipa{sp}\\
\{\ipa{q,S}\} \change\ \ipa{s} / _\ipa{t}\\
\ipa{T} \change\ \ipa{s} / _\ipa{k}\\
\ipa{l} \change\ \ipa{h} / _\ipa{k}\\
\{\ipa{n,q,h}\}\ipa{S} \{\ipa{n,q,h}\}\ipa{l} \change\ \ipa{s} \{\ipa{h,j,hj}\}\\
\ipa{S tS} \change\ \ipa{s ts}\\
\ipa{T} \change\ \ipa{t}\\
\ipa{l} \change\ \ipa{j}

\subsection{Proto-Algonquian to Swampy Cree}{\it Pogostick Man}, from \url{http://home.kpn.nl/cvkolmes/ojibwe/CorrCrOj.pdf} and Wikipedia contributors (2015), ``Cree language''. {\it Wikipedia, the Free Encyclopedia}. \textless\url{https://en.wikipedia.org/w/index.php?title=Cree_language&oldid=690521189}\textgreater

\ipa{we} \change\ \ipa{o}\\
\ipa{e} \change\ \ipa{i} (short only)\\
\ipa{ja} \change\ \ipa{a:} / C_\\
\ipa{Sje:} \change\ \ipa{Se:}\\
\ipa{lwi} \change\ \ipa{jo}\\
\ipa{wi} \change\ \ipa{o} / C_\\
\{\ipa{n,q,h}\} \change\ \O\ / _\ipa{s}\\
\ipa{q} \change\ \ipa{h} / _\ipa{tS}\\
N \change\ \ipa{h} / _S\\
\ipa{(t)Sp} \change\ \ipa{sp}\\
\{\ipa{q,S}\} \change\ \ipa{s} / _\ipa{t}\\
\ipa{T} \change\ \ipa{s} / _\ipa{k}\\
\ipa{l} \change\ \ipa{h} / _\ipa{k}\\
\{\ipa{n,q,h}\}\ipa{S} \{\ipa{n,q,h}\}\ipa{l} \change\ \ipa{S l}\\
\ipa{S} \change\ \ipa{s} / in West Swampy Cree (remains /\ipa{S}/ in East Swampy Cree)\\
\ipa{tS} \change\ \ipa{ts}\\
\ipa{T} \change\ \ipa{t}\\
\ipa{l} \change\ \ipa{n}

\subsection{Proto-Algonquian to Woods Cree}{\it Pogostick Man}, from \url{http://home.kpn.nl/cvkolmes/ojibwe/CorrCrOj.pdf} and Wikipedia contributors (2015), ``Cree language''. {\it Wikipedia, the Free Encyclopedia}. \textless\url{https://en.wikipedia.org/w/index.php?title=Cree_language&oldid=690521189}\textgreater

\ipa{we} \change\ \ipa{o}\\
\ipa{e(:)} \change\ \ipa{i(:)}\\
\ipa{ja} \change\ \ipa{a:} / C_\\
\ipa{lwi} \change\ \ipa{jo}\\
\ipa{wi} \change\ \ipa{o} / C_\\
\{\ipa{n,q,h}\} \change\ \O\ / _\ipa{s}\\
\ipa{q} \change\ \ipa{h} / _\ipa{tS}\\
N \change\ \ipa{h} / _S\\
\ipa{(t)Sp} \change\ \ipa{sp}\\
\{\ipa{q,S}\} \change\ \ipa{s} / _\ipa{t}\\
\ipa{T} \change\ \ipa{s} / _\ipa{k}\\
\ipa{l} \change\ \ipa{s} / _\ipa{k}\\
\{\ipa{n,q,h}\}\ipa{S} \{\ipa{n,q,h}\}\ipa{l} \change\ \ipa{s} \{\ipa{h,j,hj}\}\\
\ipa{S tS} \change\ \ipa{s ts}\\
\ipa{T} \change\ \ipa{t}\\
\ipa{l} \change\ \{\ipa{r,D}\}

\subsection{Proto-Algonquian to Munsee Delaware}{\it Pogostick Man}, from Goddard, Ives (1982), ``The Historical Phonology of Munsee". {\it International Journal of American Linguistics}, 48:16 -- 48

\ipa{tS} \change\ \ipa{t} / in nouns\\
\ipa{t s} \change\ \ipa{tS S} / in diminutives\\
\{\ipa{T,l}\} \change\ \ipa{r} \change\ \ipa{l}\\
\{\ipa{T,S}\} \change\ \{\ipa{r,l}\}\\
\ipa{w} \change\ \O\ / m_C\\
\ipa{w} \change\ \O\ / \{\ipa{p,m}\}_\#\\
\ipa{w} \change\ \O\ / ! \{\ipa{k,p,m}\}_\\
C\ipa{\super w} \change\ C / _\ipa{@}\{(C)\{\ipa{p,k\super w}\},\ipa{m,w}\}\\
\ipa{kw pw mw} \change\ \ipa{k\super w p\super w m\super w}\\
\ipa{we} \change\ \ipa{w@} \change\ \ipa{o:} / ! adjacent to \{\ipa{p,m,k}\}\\
\ipa{j} \change\ \O\ / C_\\
\ipa{P} \change\ \ipa{h} / _C ! C = \ipa{l}, or when reduplicated\\
\ipa{h} \change\ \O\ / _\{\ipa{s,x}\}\\
\{\ipa{nT,nl}\} \change\ \ipa{hl}\\
\ipa{k} \change\ \O\ / \ipa{h}_ (sometimes restored via analogy, e.g., in verbs)\\
\ipa{T S x} \change\ \ipa{x s h} / _\{\ipa{p,k}\}\\
\{\ipa{tS,\c{c}}\} \change\ \ipa{h} / _\ipa{k}\\
\ipa{i o} \change\ \ipa{i: o:}\\
V\ipa{:} \change\ V[-long] / _\ipa{h}C\\
V \change\ \O\ / _\# ! some monosyllables and analogical developments, in the latter of which long vowels were shortened\\
\{\ipa{a,@}\} \change\ \O\ / _\{\ipa{x,h}\} ``in the odd-numbered of any sequence of one or more short-vowel open syllables''; such vowels are considered ``weak''\\
\ipa{@}[+weak] \change\ \O\ / \#_C\\
\ipa{@}[+weak] \change\ \O\ / _C[+voiced] (sporadic)\\
\ipa{a}[+weak] \ipa{@}[+weak] \change\ \ipa{@} \O\ / a_Z[+voiced]\\
NC sequences assimilate the nasal to the POA of the following consonant, which is then voiced

\tab Synchronic alterations:

\ipa{@} \change\ \ipa{o} / _\ipa{h}\{\ipa{p,k\super w,w,m}\}\\
\ipa{@} \change\ \ipa{i} / _\ipa{h}C\\
\ipa{@} \change\ \ipa{o} / _\ipa{x}\{\ipa{p,k\super w},V[+round]\}\\
\ipa{@} \change\ \ipa{o} / \{\ipa{p,m}\}_\ipa{x}\\
\ipa{x} \change\ \ipa{X\super w} / \ipa{o(:)}_\{V,\#\}\\
\ipa{@} \change\ \ipa{a} / _\ipa{x} ``[i]n a nonrounding environment''\\
\ipa{@} \change\ \ipa{o} / _\ipa{Nk\super w}\\
V[+high] \change\ \ipa{@} / _\ipa{j}\\
V[+high] \change\ \ipa{@} / _\ipa{w} (sporadic)\\

\subsection{Proto-Algonquian to Menominee}{\it Whimemsz}, from Hockett, C. F. (1981), ``The Phonological History of Menominee". {\it Anthropological Linguistics} 23(2): 51-87; and Miner, Kenneth L. (1979), ``Theoretical Implications of the Great Menominee Vowel Shift". {\it Kansas Working Papers in Linguistics} 4(1): 7-25.

\ipa{we je} \change\ \ipa{o i} / _C\\
\ipa{we} \change\ \ipa{o} / \#_\\
\ipa{T} \change\ \ipa{s} / _O\\
\ipa{T} \change\ \ipa{r}\\
V[-long] \change\ \O\ / _\# ``[does not apply in disyllabic words containing two short vowels]"\\
\O\ \change\ \ipa{h} / V[-long]_\#\\
H \change\ \O\ / _\ipa{m}\\
\{\ipa{s,r}\} \change\ \ipa{h} / _O\\
\ipa{w} \change\ \O\ / \ipa{h}_V\\
\ipa{a} \change\ \ipa{o} / \$\ipa{am}_\ipa{w}\\
V \change\ V\ipa{:} ``when V is the second vowel of a word and follows a short-vowel syllable. Does not apply in glottal words"\\
\ipa{e} \change\ \ipa{i} / V\ipa{:}\%_ ! _H\\
N \change\ \ipa{h} / _\{O,\ipa{r}\}\\
\ipa{e} \change\ \ipa{i} / \#(C)_ ! _H\\
\ipa{e} \change\ \ipa{i} / _\{\ipa{k,m}\} ``when in the second syllable of glottal words"\\
\{\ipa{w,j}\} \change\ \O\ / C_\#\\
C \change\ \O\ / C_\#\\
\ipa{wi(:)} \change\ \ipa{o(:)} / C_\ipa{w}\\
\ipa{S tS} \change\ \ipa{s ts}\\
V\ipa{:} \change\ V[-long] / CC(G)_C\{V,\#\} ``[i.e., when following a cluster but not followed by a cluster. Only applies `after the first long vowel of a nonglottal word, and everywhere in a glottal word']''\\
V \change\ V\ipa{:} / _CC in even syllables\\
V\ipa{:} \change\ V[-long] / _C\{V,\#\} in even syllables; ``does not apply in the second syllable of a non-glottal word"\\
\ipa{e(:) i} \change\ \ipa{\ae(:) e}\\
\ipa{i: o: oP} \change\ \ipa{e: u: uP} ``[blocked when \textbf{\ipa{i:}} or a C+G sequence follows anywhere in the word, but {\it does} apply if \textbf{\ipa{\ae(:)}} intervenes before any following \textbf{\ipa{i:}} or C+G]"\\
\{\ipa{wi:,ji:,we:,je:,w\ae:,j\ae:}\} \{\ipa{wi,ji,we,je,w\ae,j\ae}\} \change\ \ipa{i: i} / C_\\
\ipa{\ae} \change\ \ipa{e} / in odd syllables ! _\{\ipa{w,j},H\}\\
\ipa{r} \change\ \ipa{n}\\
\ipa{wa ja} \change\ \ipa{u\textsubarch{@} i\textsubarch{@}} / C_

\subsection{Proto-Algonquian to Miami-Illinois}{\it Pogostick Man}, from Costa, David J. (1991), ``The Historical Phonology of Miami-Illinois Consonants". {\it International Journal of American Linguistics}, 57:365 -- 393

\ipa{tS} \change\ \ipa{t} / in nominal suffixes\\
\ipa{t} \change\ \ipa{tS} / in diminutives\\
\ipa{s} \change\ \ipa{S} / _\ipa{i} (not universal)\\
\ipa{s} \change\ \ipa{S} / _\ipa{i}V\\
\{\ipa{T,l}\} \change\ \ipa{r} \change\ \ipa{l} / V_V\\
\{\ipa{T,l}\} \change\ \ipa{r} \change\ \ipa{n} / \#_ (and possibly in other places as well)\\
\{\ipa{T,l}\} \change\ \ipa{r} \change\ \ipa{l}\\
\ipa{m}V[-long] \change\ \O\ / \#_\{\ipa{\super h}C,\ipa{s,S}\} (allophonic, ``optional'')\\
\{\ipa{P,h}\}\{\ipa{\textbeltl,l}\} \change\ \ipa{hs}\\
\{\ipa{T,l}\} \change\ \ipa{t} / \ipa{n}_\\
\{\ipa{T,S,tS,\c{c},x,P}\} \change\ \ipa{h} / _C\\
C[-nas] \change\ \ipa{h} / _\ipa{k}\\
\ipa{h} \change\ \ipa{P} / _\{\ipa{s,S}\}\\
\ipa{hs hS} \change\ \ipa{s: S:} / sporadic, usually \{\#,V[+front]\}_\\
C[-voiced] \change\ C[+voiced] / N_\\
VN\ipa{s} VN\ipa{S} \change\ V[+nas]\ipa{z} V[+nas]\ipa{Z} / not universal?\\
S \change\ \ipa{\super n}S / \#NV_ (sporadic)\\
\ipa{s S} \change\ \ipa{\super ns \super nS} / U[-nas] ({\it highly} sporadic)\\
\{\ipa{h,P}\} \change\ \O\ / _\ipa{m}

\subsection{Proto-Algonquian to Mi'kmaq}{\it Pogostick Man with acknowledgment to dhok and Alex Fink}, the former citing Audrey Marie (1986), {\it The Fundamentals of Micmac Historical Morphology}, citing Hewson, John (1973), ``Proto-Algonkian Reflexes in Micmac'', and Hewson, John (1983), ``Some Micmac Etymologies'', and the latter citing Hewson, John (1973), ``Proto-Algonkian Reflexes in Micmac''

\ipa{tS} \change\ \ipa{S} / ! C_\\
\ipa{n\{T,l\} h\{T,S\}} \change\ \O\ \ipa{s}\\
\{\ipa{P,h,}N\} \change\ \O\ / _C\\
\ipa{P}\{\ipa{T,S}\} \ipa{Pl} \change\ \ipa{s} \O\\
\ipa{x} \change\ \O\ / _\{\ipa{p,k}\}\\
\ipa{S} \change\ \ipa{s}\\
\ipa{T} \change\ \ipa{l}\\
\ipa{k} \change\ \ipa{X} / _\ipa{(w)a(:)} ! \#_\\
\ipa{k} \change\ \ipa{X} / \ipa{a(:)}_\\
\ipa{e:k} \change\ \ipa{oX} / _\ipa{w}\\
\ipa{o(:) wa: e: i:} \change\ \ipa{u o e i}\\
\ipa{a:} \change\ \ipa{a}\\
\ipa{(aw)aha} \change\ \ipa{a:}\\
\{\ipa{awa,iwa,iwi}\} \change\ \ipa{u:}\\
\{\ipa{o,a}\}\ipa{wi} \change\ \ipa{o:}\\
\ipa{ehi} \change\ \ipa{e:}\\
\{\ipa{aja,iha,iji,ihi,ija}\} \change\ \ipa{i:}

\subsection{Proto-Algonquian to Ojibwe}{\it Whimemsz}, from his own work; \url{http://home.kpn.nl/cvkolmes/ojibwe/corrCrOj.htm}; Bloomfield, Leonard (1946), ``Algonquian"; and ``various asides and statements in dozens of different journal articles and conference papers dealing with Ojibwe or PA''

\tab {\it NB: For this sound-change set, H is ``either an */h/ or */\ipa{P}/, but we don't know which''.}

\ipa{we e} \change\ \ipa{o i}\\
\ipa{w} \change\ \O\ / \{\ipa{t,r}\}_\ipa{i}\\
\{\ipa{T,s,h,P}\} \change\ \O\ / _\{\ipa{p,t,tS,k}\}\\
\ipa{T} \change\ \ipa{r}\\
\{\ipa{P,h}\}\{\ipa{s,r}\} \change\ \ipa{s}\\
\{\ipa{P,h}\} \change\ \O\ / _\ipa{S}\\
\{\ipa{n,r}\} \change\ \O\ / _\ipa{r}\\
H \change\ \O\ / _\ipa{m}\\
\ipa{r} \change\ \ipa{s} / _\ipa{k}\\
\{\ipa{j,w}\}V[-long] \change\ \O\ / C_\# in disyllables with V\ipa{:} or in tri(-plus-)syllables\\
\{\ipa{w,j}\}V[-long] \change\ \O\ / V\ipa{:}_\# (Whimemsz is unsure if this change is across-the-board or not)\\
V[-long] \change\ \O\ / V[-long]\{\ipa{w,j}\}_\# (Whimemsz is unsure if this change is across-the-board or not)\\
\ipa{je:} \change\ \ipa{i:} / C_\\
\ipa{ja} \change\ \ipa{i:} / C_C\\
\ipa{j} \change\ \O\ / C_\\
\ipa{r} \change\ \ipa{n}

\subsection{Proto-Algonquian to Piscataway}{\it Pogostick Man}, from Mackie, Lisa (2006), ``Fragments of Piscataway: A Preliminary Description''

\tab{\it NB: This is very incomplete, partially because it seems that the only source we have on Piscataway is a single document in rather poor condition.}

*\#\ipa{we}- retained\\
\{\ipa{T,S}\} \change\ \ipa{\textbeltl} (conjectured based on $\langle$z$\rangle$ in the Piscataway source and on the lack of voicing in the original reconstructed sounds)\\
\ipa{k} \change\ \ipa{x}\\
\ipa{e} \change\ \ipa{o} / unclear conditioning\\
\ipa{P} \change\ \ipa{h} / _C

\subsection{Proto-Algonquian to Shawnee}{\it Whimemsz}, from bin Muzaffar, Towhid, {\it Computer Simulation of Shawnee Historical Phonology}, plus ``other corrections based on a few other papers plus my limited knowledge of comparative Algonquian"

\ipa{we} \change\ \ipa{o}\\
\ipa{T} \ipa{r} / ! _O\\
\ipa{r} \change\ \ipa{s} / H_\\
\ipa{r} \change\ \O\ / \ipa{n}_\\
N \change\ \O\ / _O\\
\{\ipa{h,s,tS,T}\} \change\ \ipa{P} / _O\\
\ipa{r} \change\ \ipa{S} / _O\\
\ipa{e} \change\ \ipa{i} / \#(C)_ ``(but remains e in a few cases?)"\\
\ipa{i:} \change\ \ipa{i} / _\ipa{j}\\
\ipa{j} \change\ \O\ / C_\ipa{i(:)}\\
\ipa{je} \change\ \ipa{i} / C_\\
\ipa{j} \change\ \O\ / \{\ipa{tS,S}\}_\ipa{e:}\\
\ipa{j} \change\ \O\ / \{\ipa{tS,S,w}\}_\ipa{a:}\\
\ipa{w} \change\ \O\ / \ipa{t}_\ipa{i}\\
\ipa{wa} \change\ \ipa{o} / \#_\\
V[-long] \change\ \O\ / _\{\ipa{Sp,Sk}\}\\
V[-long] \change\ \O\ / C_\ipa{P}C\\
V[-long] \change\ \O\ / _\ipa{h}V\\
V\ipa{:} \change\ V[-long] / _\#\\
V\ipa{:} \change\ V[-long] / _\{\ipa{P}C,\ipa{Sp,Sk,h}V\}\\
\ipa{P} \change\ \O\ / C\{\ipa{v,l,s}\}_C\\
\ipa{P} \change\ \O\ / _CC\\
\O\ \change\ \ipa{P} / C\{\ipa{v,d}\}_\{\ipa{Sp,Sk,h}V\}\\
\O\ \change\ \ipa{h} / \#_V\\
\ipa{s} \change\ \ipa{T}\\
\ipa{r} \change\ \ipa{l}\\
\O\ \change\ \ipa{i} / \#C_\ipa{j}V\ipa{:} ``(for some speakers)"\\
\ipa{S} \change\ \ipa{s} ``(for many speakers)"

\clearpage

\section{Altaic}\tab The Wikipedia gives the following reconstruction, slightly adapted, for a hypothetical Proto(-Macro)-Altaic language, citing Bla\v{z}ek (2006) citing Sarostin \textit{et al.} (2003) and porting over into IPA:

\begin{center}\begin{tabular}{c | c c c c c c}
& Bilabial & Alveolar/Dental & Alveolopalatal & Postalveolar & Paltal & Velar \\ \hline
Nasal & \ipa{m} & \ipa{n} & \ipa{n\super j} & & & \\
Plosive & \ipa{p p\super h b} & \ipa{t t\super h d} & & & & \ipa{k k\super h g} \\
Fricative & & \ipa{s z} & & \textipa{S} & & \\
Affricate & & & & \textipa{\t*{tS} \t*{tS}\super h \t*{dZ}} & & \\
Trill & & \ipa{r} & \ipa{r\super j} & & & \\
Approximant & & \ipa{l} & \ipa{l\super j} & & & \end{tabular}\end{center}

\begin{center}\begin{tabular}{c | c c c}
& Front & Central & Back \\ \hline
Close & \ipa{i y} & & \ipa{u} \\
Mid & \ipa{e \o} & & \ipa{o} \\
Near-Open & \ipa{\ae} & & \\
Open & & \ipa{a} & \end{tabular}\end{center}

\tab *\ipa{z} would only have ever existed word-initially; *\ipa{r} and *\ipa{j} would only have been medial. In addition, Proto-(Macro-)Altaic also is thought to have had a bitonal pitch-accent system, with the syllable carrying the tone.

\tab It is important to note that the Altaic grouping is highly controversial and is not accepted by many mainstream linguists.

\tab (From Wikipedia contributors (2011), \textquotedblleft Altaic languages". \textit{Wikipedia, the Free Encyclopedia.} \textless\url{http://en.wikipedia.org/w/index.php?title=Altaic_languages&oldid=453651228}\textgreater)

\subsection{Proto-Altaic to Proto-Japonic}{\it Pogostick Man}, from Wikipedia contributors (2011), ``Altaic languages''. {\it Wikipedia, the Free Encyclopedia}. \textless\url{http://en.wikipedia.org/w/index.php?title=Altaic_languages&oldid=453651228}\textgreater, citing Sarostin, Sergei A., Anna V. Dybo, and Oleg A. Mudrak (2003), {\it Etymological Dictionary of the Altaic Languages}. Leiden: Brill Academic Publishers

\tab {\it NB: Does not include clusters.}

\ipa{a} \change\ \ipa{@} / _C\ipa{e}\\
\ipa{a} \change\ \ipa{i} / _C\ipa{i}\\
\ipa{a} \change\ \ipa{u} / _C\ipa{u}\\
V \change\ \ipa{a} / _C\ipa{a}\\
\ipa{u} \change\ \ipa{a} / P_C\ipa{e}\\
\{\ipa{a,e,o,\ae}\} \ipa{i u y \o} \change\ \ipa{@ i ua} \{\ipa{u,@}\} \{\ipa{@,u}\} / _C\ipa{e}\\
\{\ipa{a,\ae,e,\o,i,y}\} \ipa{o} \change\ \ipa{i u} / _C\ipa{i}\\
\ipa{e i} \{\ipa{o,u}\} \ipa{\ae\ \o\ y} \change\ \{\ipa{@,a}\} \{\ipa{i,@}\} \ipa{@ a} \{\ipa{@,u}\} \{\ipa{u,@}\} / _C\ipa{o}\\
V \change\ \ipa{u} / _C\ipa{u}\\
\ipa{p\super h t\super h k\super h} \change\ \ipa{p t k}\\
\ipa{b} \change\ \ipa{p} / \#_\\
\ipa{b} \change\ \ipa{w} / ! _\{\ipa{a,@,}V\ipa{j}\}\\
\ipa{tS\super h} \change\ \ipa{t}\\
\ipa{tS dZ} \change\ \ipa{t d} / \#_\\
\ipa{tS} \change\ \ipa{s} / maybe ! _\#?\\
\ipa{dZ} \change\ \ipa{j}\\
\ipa{g} \change\ \O\ / \ipa{i}V_\\
\ipa{g} \change\ \ipa{k} / else\\
\{\ipa{S,z}\} \change\ \ipa{s} \\
\ipa{n} \change\ \ipa{m} / \#_\\
\ipa{N} \change\ \ipa{m} / \#_\{\ipa{\ae,\o,y}\}\\
\ipa{N} \change\ \{\O,\ipa{n}\} \#_ else\\
N \change\ \{\ipa{m,n}\}\\
\ipa{r} \change\ \ipa{t} / _\{\ipa{i,u}\}\\
\ipa{r\super j} \change\ \{\ipa{r,t}\}\\
\ipa{l(\super j)} \change\ \ipa{n} / \#_\\
\ipa{l l\super j} \change\ \ipa{r s} / else\\
\ipa{j} \change\ \{\ipa{j,}\O\}\\
U[+long] \change\ U[-long]

\subsubsection{Early Middle Japanese to Modern Japanese}{\it Zhen Lin}

\tab {\it NB: The ordering of these changes may be slightly anachronic.}

\ipa{p} \change\ \ipa{F}\\
\ipa{F} \change\ \ipa{w} / V_V\\
\ipa{(w)e} \change\ \ipa{je}\\
\O\ \change\ \ipa{w} / _\ipa{o}\\
\ipa{w} \change\ \O\ / ! _\{\ipa{a,o}\}\\
\ipa{au iu uu eu ou} \change\ \ipa{O: ju: u: jo: o:}\\
\ipa{j} \change\ \O\ / _\ipa{e}\\
\ipa{w} \change\ \O\ / _\ipa{o}\\
\ipa{w} \change\ \O\ / \ipa{k}_\ipa{a}\\
\ipa{F} \change\ \ipa{h} / ! _\ipa{u}\\
\ipa{O:} \change\ \ipa{o:}\\
``Affrication of /ti di/ probably happened very early. Denasalization of the prenasalized stops happened relatively later. Final /m/ merged with /n/ at some point, and [\ipa{dZ}] (from */dj/) and [\ipa{Z}] (\textless\ */zj/) also merged.''

\subsection{Proto-Altaic to Proto-Korean}{\it Pogostick Man}, from Wikipedia contributors (2011), ``Altaic languages''. {\it Wikipedia, the Free Encyclopedia}. \textless\url{http://en.wikipedia.org/w/index.php?title=Altaic_languages&oldid=453651228}\textgreater, citing Sarostin, Sergei A., Anna V. Dybo, and Oleg A. Mudrak (2003), {\it Etymological Dictionary of the Altaic Languages}. Leiden: Brill Academic Publishers

\tab {\it NB: Does not include clusters.}

\{\ipa{t\super h,d}\} \{\ipa{k,g}\} \change\ \ipa{r} \{\ipa{h},\O\} / \{C,V\}_\{C,V\}\\
\ipa{p\super h t\super h k\super h} \change\ \ipa{p t} \{\ipa{k,h}\}\\
\ipa{b} \change\ \ipa{p} / \#_\\
\ipa{d} \change\ \ipa{t}\\
\{\ipa{tS\super h,dZ}\} \change\ \ipa{tS}\\
\ipa{g} \change\ \ipa{k} / \#_\\
\{\ipa{S,z}\} \change\ \ipa{s}\\
\{\ipa{n\super j,N}\} \change\ \ipa{n} / \#_\\
\ipa{N} \change\ \{\ipa{N},\O\}\\
\ipa{r\super j} \change\ \ipa{r}\\
\ipa{l(\super j)} \change\ \ipa{n} / \#_\\
\ipa{l(\super j)} \change\ \ipa{r} / else\\
\ipa{j} \change\ \{\ipa{j},\O\}\\
U[+long] \change\ U[-long]\\
Syllable pitches reverse, basically, for whatever reason

\subsection{Proto-Altaic to Proto-Mongolic}{\it Pogostick Man}, from Wikipedia contributors (2011), ``Altaic languages''. {\it Wikipedia, the Free Encyclopedia}. \textless\url{http://en.wikipedia.org/w/index.php?title=Altaic_languages&oldid=453651228}\textgreater, citing Sarostin, Sergei A., Anna V. Dybo, and Oleg A. Mudrak (2003), {\it Etymological Dictionary of the Altaic Languages}. Leiden: Brill Academic Publishers

\tab {\it NB: Does not include clusters.}

\ipa{a} \change\ \{\ipa{a,i}\} / _C\ipa{e}\\
\ipa{a} \change\ \{\ipa{a,e}\} / _C\ipa{i}\\
\ipa{a} \change\ \{\ipa{a,i,e}\} / _C\ipa{o}\\
\ipa{a} \change\ \{\ipa{a,o,u}\} / _C\ipa{u}\\
\ipa{e o u \ae\ \o\ y} \change\ \{\ipa{a,e}\} \{\ipa{o,u}\} \{\ipa{a,o,u} \ipa{a} \{\ipa{a,o,u}\} \{\ipa{o,u,i}\} / _C\ipa{a}\\
\ipa{a e i o u \ae\ \o\ y} \change\ \{\ipa{a,i}\} \{\ipa{e,ja}\} \{\ipa{e,i}\} \{\ipa{\o,y,o}\} \{\ipa{o,u,y}\} \{\ipa{i,a,e}\} \{\ipa{e,\o}\} \{\ipa{\o,y,o,u}\} / _C\ipa{e}\\
\ipa{i} \change\ \ipa{e} / P_C\ipa{i}\\
\ipa{a e u \ae\ \o\ y} \change\ \{\ipa{a,e}\} \{\ipa{e,i}\} \{\ipa{y,\o}\} \{\ipa{i,e}\} \{\ipa{i,e,\o}\} \{\ipa{\o,y,o,u}\} / _C\ipa{i}\\
\ipa{e} \change\ \{\ipa{y,\o}\} / P_C\ipa{o}\\
\ipa{e} \change\ \{\ipa{y,\o}\} / C_P\ipa{o}\\
\ipa{e} \change\ \ipa{o} / P_C\ipa{u}\\
\ipa{e} \change\ \ipa{o} / C_P\ipa{u}\\
\ipa{a e o i \ae\ \o\ y} \change\ \{\ipa{a,i,e}\} \{\ipa{a,e}\} \ipa{u} \{\ipa{o,u}\} \ipa{e} \{\ipa{\o,y,o,u}\} \{\ipa{o,u}\} / _C\ipa{o}\\
\ipa{a e} \{\ipa{o,u}\} \ipa{\ae\ \o\ y} \change\ \{\ipa{a,o,u}\} \{\ipa{e,a}\} \{\ipa{o,u}\} \{\ipa{a,o,u}\} \{\ipa{e,i,u}\} \{\ipa{i,o,u,y,\o}\} / _C\ipa{u}\\
\ipa{b} \change\ \ipa{h} / medially, ! \{\ipa{r(\super j),l(\super j)}\}_ or _\ipa{g}\\
\ipa{p\super h} \change\ \{\ipa{h,j}\} / \#_\\
\ipa{p\super h} \change\ \{\ipa{b,h}\} / medially\\
\ipa{p\super h} \change\ \ipa{b} / \#_U[+high pitch]\\
\ipa{p} \change\ \ipa{h} (sporadic)\\
\ipa{p} \change\ \ipa{b}\\
\ipa{t\super h} \change\ \ipa{d} / _\#\\
\ipa{t(\super h) d} \change\ \ipa{tS dZ} / _\ipa{i}\\
\ipa{t\super h} \change\ \ipa{t} / else\\
\ipa{tS} \change\ \ipa{dZ} / \#_\ipa{i}\\
\ipa{tS} \change\ \ipa{d} / \#\\
\ipa{tS\super h} \change\ \ipa{tS}\\
\ipa{g} \change\ \ipa{h} / ! \{C,V\}_\ipa{h}\\
\ipa{k} \change\ \ipa{g} / ! \#_\\
\ipa{k\super h} \change\ \ipa{g} / \{C,V\}_\ipa{h}\\
\ipa{k\super h} \change\ \ipa{k} / else\\
\ipa{z} \change\ \ipa{s}\\
\ipa{S} \change\ \ipa{tS} / \#_\ipa{a}\\
\ipa{S} \change\ \ipa{s} / else\\
\ipa{n\super j} \change\ \ipa{dZ} / \#_\\
\ipa{n\super j} \change\ \{\ipa{j,n}\} / else\\
\ipa{N} \change\ \ipa{g} / \#_\ipa{u}\\
\ipa{N} \change\ \ipa{n} / \#_\{\ipa{a,o,e}\}\\
\ipa{N} \change\ \{\O,\ipa{j}\} / \#_\\
\ipa{N} \change\ \{\ipa{m,n,N,h}\}\\
\ipa{r\super j} \change\ \ipa{r}\\
\ipa{l} \change\ \{\ipa{n,l}\} / \#_\\
\ipa{l\super j} \change\ \ipa{dZ} / \#_\ipa{i}\\
\ipa{l\super j} \change\ \ipa{d} / \#_\\
\ipa{l\super j} \change\ \ipa{l}\\
\ipa{j} \change\ \{\ipa{j,h}\}\\
Loss of syllable pitch and length

\subsection{Proto-Altaic to Proto-Tungusic}{\it Pogostick Man}, from Wikipedia contributors (2011), ``Altaic languages''. {\it Wikipedia, the Free Encyclopedia}. \textless\url{http://en.wikipedia.org/w/index.php?title=Altaic_languages&oldid=453651228}\textgreater, citing Sarostin, Sergei A., Anna V. Dybo, and Oleg A. Mudrak (2003), {\it Etymological Dictionary of the Altaic Languages}. Leiden: Brill Academic Publishers

\tab {\it NB: Does not include clusters.}

\ipa{o} \change\ \{\ipa{o,u}\} / _CV\\
\ipa{\ae} \change\ \ipa{i} / \{\ipa{s,S,x}\}_C\ipa{a}\\
\{\ipa{u,\o,y}\} \ipa{\ae} \change\ \{\ipa{o,u}\} \ipa{ia} / _C\ipa{a}\\
\ipa{y} \change\ \ipa{u} / P_C\{\ipa{e,i}\}\\
\ipa{\ae\ \o} \change\ \ipa{i} \{\ipa{o,u}\} / _C\ipa{e}\\
\ipa{\ae} \change\ \ipa{i} / \{\ipa{s,S,x}\}_C\ipa{i}\\
\ipa{\ae\ \o\ y} \change\ \ipa{ia} \{\ipa{o,u}\} \ipa{i} / _C\ipa{i}\\
\{\ipa{u,\ae}\} \ipa{\o} \change\ \{\ipa{o,u}\} \ipa{i} / _C\ipa{o}\\
\ipa{\o} \change\ \ipa{i} / \{\ipa{s,S,x}\}_C\ipa{u}\\
\{\ipa{u,\ae,y}\} \ipa{\o} \change\ \{\ipa{o,u}\} \ipa{ia} / _C\ipa{u}\\
\ipa{p} \change\ \ipa{b} / medially\\
\ipa{p\super h} \change\ \ipa{p}\\
\ipa{t} \change\ \ipa{dZ} / \#_\{\ipa{\ae,\o,y}\}\\
\ipa{t} \change\ \ipa{d} / \#_\\
\ipa{t\super h tS\super h} \change\ \ipa{t tS}\\
\ipa{k} \change\ \{\ipa{k,g}\} / \#_\\
\ipa{k} \change\ \ipa{g}\\
\ipa{k\super h} \change\ \ipa{x} / \#_\\
\ipa{k\super h} \change\ \{\ipa{x,k}\}\\
\ipa{z} \change\ \ipa{s}\\
\ipa{r\super j l\super j} \change\ \ipa{r l}\\
U[-long +low pitch] U[+long -low pitch] \change\ U[+long] U[-long]

\subsection{Proto-Altaic to Proto-Turkic}{\it Pogostick Man}, from Wikipedia contributors (2011), ``Altaic languages''. {\it Wikipedia, the Free Encyclopedia}. \textless\url{http://en.wikipedia.org/w/index.php?title=Altaic_languages&oldid=453651228}\textgreater, citing Sarostin, Sergei A., Anna V. Dybo, and Oleg A. Mudrak (2003), {\it Etymological Dictionary of the Altaic Languages}. Leiden: Brill Academic Publishers

\tab {\it NB: Does not include clusters.}

\ipa{a \o} \change\ \{\ipa{a,2}\} \ipa{a} / P_C\ipa{a}\\
\ipa{a e i u \ae\ \o\ y} \change\ \ipa{a} \{\ipa{a,2,e}\} \{\ipa{W,i}\} \{\ipa{u,o}\} \{\ipa{ia,ja,E}\} \{\ipa{ia,ja}\} \ipa{W} / _C\ipa{a}\\
\ipa{y} \change\ \ipa{i} / \{\ipa{r(\super j),l(\super j)}\}_\ipa{e}\\
\ipa{e} \change\ \ipa{ja} / \#_C\{\ipa{e,i}\}\\
\ipa{\o} \change\ \ipa{2} / P_C\ipa{e}\\
\ipa{i} \change\ \ipa{e} / \{\ipa{r(\super j),l(\super j)}\}_\ipa{e}\\
\ipa{a} \{\ipa{e,i}\} \ipa{o u \ae\ \o\ y} \change\ \{\ipa{E,a}\} \ipa{E} \{\ipa{\o,o}\} \{\ipa{y,u}\} \{\ipa{ia,ja,E}\} \{\ipa{ia,ja}\} \{\ipa{y,\o}\} / _C\ipa{i}\\
\ipa{\ae} \change\ \ipa{a} / P_C\ipa{o}\\
\ipa{\ae} \change\ \ipa{2} / P_C\ipa{u}\\
\ipa{a e i \ae\ \o\ y} \change\ \{\ipa{o,ja,aj}\} \{\ipa{2,3}\} \ipa{W} \{\ipa{ia,ja}\} \{\ipa{o,u}\} \{\ipa{u,o}\} / _C\ipa{o}\\
\ipa{e i \ae\ \o\ y} \change\ \{\ipa{E,a,2}\} \{\ipa{W,i}\} \{\ipa{e,a}\} \{\ipa{u,o}\} \ipa{W} / _C\ipa{u}\\
\{\ipa{p\super h,N}\} \change\ \{\O,\ipa{j}\} / \#_\\
\ipa{p\super h} \change\ \ipa{p}\\
\ipa{t\super h} \change\ \ipa{d} / \#_(V)\{\ipa{l\super j,r(\super j)}\}\\
\ipa{t\super h} \change\ \ipa{t}\\
\{\ipa{t,tS}\} \change\ \ipa{d} / \#_\\
\ipa{k} \change\ \ipa{g} / _(V)\ipa{r}\\
\ipa{k\super h} \change\ \ipa{k} \\
\ipa{S} \change\ \ipa{tS} / \#_\ipa{a}\\
\ipa{S} \change\ \ipa{s} \\
\ipa{m n(\super j)} \change\ \ipa{b j} / \#_\\
Loss of syllable pitch

The wiki at Firespeaker.org gives the following alternate list of sound changes from Proto-Altaic to (Pre-)Proto-Turkic.

{\it Pogostick Man}, from Firespeaker.org wiki contributors (2014), ``Turkic sound changes". \textless\url{http://wiki.firespeaker.org/Turkic_sound_changes}\textgreater

\{\ipa{Z,dZ}\} \change\ \ipa{j} / \#_ (marked as to Pre-Proto-Turkic)\\
\{\ipa{d,n}\} \change\ \ipa{j} / \#_ (?) (marked as to Pre-Proto-Turkic)\\
\{N,\ipa{l,r,S,z}\} \change\ \O\ / \#_\\
\ipa{p} \change\ \ipa{F} \change\ \ipa{h} / \#_\\
\ipa{d g} \change\ \ipa{t k} (may have been part of a more sweeping merger; Firespeaker calls it ``lenis-fortis")\\
\{\ipa{d,n}\}\ipa{\super j s\super j} \change\ \ipa{j S} / \#_\\
\ipa{r\super j} \change\ \ipa{z}

\subsubsection{Proto-Turkic to Proto-Kypchak}{\it Pogostick Man}, from Firespeaker.org wiki contributors (2014), ``Turkic sound changes". \textless\url{http://wiki.firespeaker.org/Turkic_sound_changes}\textgreater

V[- long] \change\ \O\ (shared with Old Turkic)\\
\ipa{h} \change\ \O\ (shared with Old Turkic)\\
\ipa{n\super j} \change\ \ipa{j}\\
\ipa{b}\textellipsis\ipa{n} \change\ \ipa{m}\textellipsis\ipa{n}\\
\ipa{d G} \change\ \ipa{t x} / _\#\\
\ipa{d} \change\ \ipa{t} / \#_ (``kind of", something about evidence from borrowings)\\
V \change\ V[- round] / U_\\
\ipa{b} \change\ \ipa{v} / V_\\
\ipa{v} \change\ \ipa{w}\\
\ipa{gm rg} \change\ \ipa{mg gr} (this second one is listed as \change\ \ipa{rg} but it might be a typo)\\
\ipa{rd} \change\ \ipa{dr} (possibly sporadic and/or confined to Kazakh)\\
\ipa{G} \change\ \ipa{w} / \{\ipa{a,u,i,o}\}_\\
\{\ipa{e,\ae}\}\ipa{b ub} \change\ \ipa{ew uw}\\
\{\ipa{d,g}\} \change\ \ipa{j} / \ipa{\o}_\\
\ipa{d} \change\ \ipa{D} \change\ \ipa{j} / V_\\
\ipa{g} \change\ \ipa{w} / V_\\
\ipa{ew} (\change\ \ipa{\o j}) \change\ \ipa{yj}\\
\ipa{\ae} \change\ \ipa{e}\\
\ipa{s} \change\ \ipa{tS} / _V\ipa{tS}\\
\ipa{s} \change\ \ipa{\c{c}} / _V\ipa{\c{c}}\\
\ipa{a} \change\ \ipa{\ae} / ! _B\\
\ipa{f} \change\ \ipa{w} / _V\\
\ipa{f} \change\ \ipa{p} / else\\
\ipa{N} \change\ \ipa{g} / syllable-final

\paragraph{Proto-Kypchak to Kazakh}{\it Pogostick Man}, from Firespeaker.org wiki contributors (2014), ``Turkic sound changes". \textless\url{http://wiki.firespeaker.org/Turkic_sound_changes}\textgreater

\tab {\it NB: Most likely incomplete; all changes listed are stated as being ``[s]hared with Nogay and Karakalpak".}

\ipa{tS} \change\ \ipa{S}\\
\ipa{j} \change\ \ipa{dZ} / \#_ (did not occur in Qara Nogay)\\
\ipa{dZ} \change\ \ipa{Z} (did not occur in Qara Nogay or Central Nogay)\\
\ipa{w} \change\ \O\ / \ipa{W}_

\paragraph{Proto-Kypchak to Kyrgyz}{\it Pogostick Man}, from Firespeaker.org wiki contributors (2014), ``Turkic sound changes". \textless\url{http://wiki.firespeaker.org/Turkic_sound_changes}\textgreater

\ipa{j} \change\ \O\ / _\ipa{l} (sporadic?)\\
\ipa{b} \change\ \ipa{m} / V_V (sporadic?)\\
\{\ipa{u,W}\}\ipa{w} \{\ipa{i,y}\}\ipa{w} \ipa{aw} \{\ipa{\ae,e}\}\ipa{w} \change\ \ipa{u: y: o: \o:}\\
\ipa{G} \change\ \O\ / V_V\\
\ipa{\ae} V\ipa{h} \{\ipa{Q,h}\} \change\ \ipa{A:} V\ipa{:} \O\ (seems to have largely been confined to loanwords from Persian)\\
\ipa{j} \change\ \ipa{dZ} / \#_\\
\ipa{x} \change\ \ipa{q}\\
\ipa{nj} \change\ \ipa{jn}\\
\O\ \change\ U / \#_\{\ipa{l,r}\} (not sure what $\langle$U$\rangle$ represents here; maybe just some sort of back vowel?)\\
\ipa{e} \change\ \ipa{i} / _\ipa{g}\\
\ipa{e} \change\ \ipa{i} / \ipa{k}_\ipa{y} (maybe they mean \ipa{k}_\ipa{j}?)

\subsubsection{Proto-Turkic to Sakha}{\it Pogostick Man}, from Firespeaker.org wiki contributors (2014), ``Turkic sound changes". \textless\url{http://wiki.firespeaker.org/Turkic_sound_changes}\textgreater

\{\ipa{e,7}\}\ipa{:} \change\ \ipa{je} (the second one is conjectured based on my admittedly sparse knowledge of Turkish; I can only guess that $\langle$\.{e}$\rangle$ is some sort of back unrounded vowel)\\
\ipa{o og \o\ \o g ig} \change\ \ipa{wo 4\o\ o: \o: i:}\\
\ipa{a}\{\ipa{\v{g}(W),b}\} \{\ipa{o}\{\ipa{\v{g},b}\},\ipa{a\v{g}u}\} \ipa{u}\{\ipa{\v{g},b}\} \change\ \ipa{\textturnmrleg a wo u:}\\
\ipa{i\v{g}} \change\ \ipa{W:} \change\ \ipa{i:} (but original \ipa{W:} unaffected?)\\
\ipa{eg} \change\ \{\ipa{je,i:,ji}\}\\
\ipa{d s} \{\ipa{S,z}\} \change\ \ipa{t} \O\ \ipa{s} / V_V\\
\ipa{s} \change\ \O\ / \#_\\
\{\ipa{z,S}\} \change\ \ipa{h}\\
\ipa{j} \change\ \ipa{s} (possibly only initially?)

\clearpage

\section{Austroasiatic}

\subsection{Vietic}\tab Thompson reconstructs the following phonetic system for Proto-Viet-Muong:

\begin{center}\begin{tabular}{c | c c c c c}
& Bilabial & Alveolar & Palatal & Velar & Glottal\\ \hline
Nasal & \ipa{\r*{m} m} & \ipa{\r*{n} n} & \ipa{\r{\textltailn} \textltailn} & \ipa{\r{N N}}\\
Plosive & \ipa{p b} & \ipa{t t}* \ipa{d d}* & \ipa{c \textbardotlessj} & \ipa{k g} & \ipa{P}\\
Liquid & \ipa{\textturnw\ w} & \ipa{\r*{l} l \r*{r} r} & \ipa{\r{\j} j}\end{tabular}

\begin{tabular}{c | c c c}
& Front & Center & Back\\ \hline
High & \ipa{i i\textsubarch{@}} & \ipa{1 1\textsubarch{@}} & \ipa{u u\textsubarch{@}} \\
High-Mid & \ipa{e} & \ipa{@ @:} & \ipa{o}\\
Low-Mid & \ipa{E} & & \ipa{O}\\
Low & & \ipa{a a:}\end{tabular}

\end{center}

\tab Further, Thompson reconstructs Proto-Vietic as having had four tones, *A, *B, *C, and *D. In the development of Vietnamese, *B and *D merged.

\tab Thompson lists a few occasional alterations between Muong Khen and Vietnamese, but I'm not sure exactly which two languages were being compared, so I'm shunting the alterations here.

\begin{center}\begin{tabular}{r c l}
\ipa{-o} & : & \ipa{*-@w}\\
\ipa{-u} & : & \ipa{*-@w}\\
\ipa{-i} & : & \ipa{*-@j}\\
\ipa{-e} & : & \ipa{*-@j}\\
\ipa{a} & : & \ipa{1a}\end{tabular}\end{center}

\tab The \ipa{-e} : \ipa{*-@j} correspondence was listed as being rarer than the others.

\tab (From Thompson, Laurence C. (1976), ``Proto-Viet-Muong Phonology". {\it Oceanic Linguistics Special Publications} 13, Austroasiatic Studies II:1113 -- 1203; Wikipedia contributors (2012). ``Hanoi". {\it Wikipedia, the Free Encyclopedia}. \textless\url{http://en.wikipedia.org/w/index.php?title=Hanoi&oldid=509052974}\textgreater; Wikipedia contributors (2012), ``Vietnamese Language". {\it Wikipedia, the Free Encyclopedia}. \textless\url{http://en.wikipedia.org/w/index.php?title=Special:Cite&page=Vietnamese_language&id=509331797}\textgreater; Gage, William W. (1985), ``Glottal Stops and Vietnamese Tonogenesis". {\it Oceanic Linguistics Special Publications} 20:21 -- 36; and Thompson, Laurence C. (1979?), ``More on Viet-Muong Tonal Developments")

\subsubsection{Proto-Vietic to Muong Khen}{\it Pogostick Man}, from Thompson, Laurence C. (1976), ``Proto-Viet-Muong Phonology". {\it Oceanic Linguistics Special Publications} 13, Austroasiatic Studies II:1113 -- 1203; Wikipedia contributors (2012). ``Hanoi". {\it Wikipedia, the Free Encyclopedia}. \textless\url{http://en.wikipedia.org/w/index.php?title=Hanoi&oldid=509052974}\textgreater; Wikipedia contributors (2012), ``Vietnamese Language". {\it Wikipedia, the Free Encyclopedia}. \textless\url{http://en.wikipedia.org/w/index.php?title=Special:Cite&page=Vietnamese_language&id=509331797}\textgreater; Gage, William W. (1985), ``Glottal Stops and Vietnamese Tonogenesis". {\it Oceanic Linguistics Special Publications} 20:21 -- 36; and Thompson, Laurence C. (1979?), ``More on Viet-Muong Tonal Developments"

\textbf{Tonogenesis}
\begin{center}\begin{tabular}{| c | c | c | c | c |}\hline
Reg & A & B & C & d\\ \hline
1 & mid level & low rising$^1$ & high rising & high rising\\
2 & low falling & high-mid$^2$ & high-mid$^2$ & high-mid$^2$\\\hline\end{tabular}\end{center}
1. ``Constricted" (laryngealized?)\\
2. Terminates in a glottal stop if no final stop

Presyllables don't seem to have affected Muong much.

\textbf{Initials:}\\
\ipa{s} \change\ \ipa{h}\\
\ipa{c\super h} \change\ \ipa{s}\\
\ipa{t\super h} \change\ \ipa{h} (Only seems to have occurred with first-register tones)\\
\{\ipa{k\super h,g\super H}\} \change\ \ipa{x} (Presyllables don't seem to have affected this much)\\
\ipa{m n} \change\ \ipa{b d} (Only seems to have occurred with first-register tones)\\
\{\ipa{pj,bj}\} \{\ipa{tj,dj}\} \{\ipa{cj,\textbardotlessj j}\} \change\ \ipa{b d j}\\
\ipa{\!b \!d} \change\ \ipa{b d} (Only seems to have occurred with first-register tones)\\
\{\ipa{\r*{n}j,nj,\r{\textltailn}j,\textltailn j}\} \change\ \ipa{\textltailn\ j}\\
N[-voiced] W[-voiced] \change\ N[+voiced] W[+voiced]\\
\ipa{(h)@}\{\ipa{p,b}\} \change\ \ipa{t} / _\ipa{l}\\
\ipa{m} \change\ \O\ / _\ipa{l}\\
\ipa{t\*r} \change\ \ipa{t\super h}

\textbf{Miscellanea:}\\
\ipa{w} \change\ \O\ / \ipa{t\super h}V_\ipa{k} (conjectured)

\subsubsection{Proto-Vietic to Middle Vietnamese}{\it Pogostick Man}, from Thompson, Laurence C. (1976), ``Proto-Viet-Muong Phonology". {\it Oceanic Linguistics Special Publications} 13, Austroasiatic Studies II:1113 -- 1203; Wikipedia contributors (2012). ``Hanoi". {\it Wikipedia, the Free Encyclopedia}. \textless\url{http://en.wikipedia.org/w/index.php?title=Hanoi&oldid=509052974}\textgreater; Wikipedia contributors (2012), ``Vietnamese Language". {\it Wikipedia, the Free Encyclopedia}. \textless\url{http://en.wikipedia.org/w/index.php?title=Special:Cite&page=Vietnamese_language&id=509331797}\textgreater; Gage, William W. (1985), ``Glottal Stops and Vietnamese Tonogenesis". {\it Oceanic Linguistics Special Publications} 20:21 -- 36; and Thompson, Laurence C. (1979?), ``More on Viet-Muong Tonal Developments"

\textbf{Initials:}\\
\ipa{b\super H} \{\ipa{t\super h,d\super H}\} \{\ipa{t}*\ipa{\super h,d}*\ipa{\super H}\} \{\ipa{k\super h,g\super H}\} \change\ \ipa{p\super h t t\super h k\super h} (after *\ipa{k\super h} *\ipa{g\super H}, only first-register tones may occur)\\
\ipa{(h)@}\{\ipa{p,b}\} \ipa{(h)@}\{\ipa{t,d}\} \ipa{(h)@}\{\ipa{c,\textbardotlessj}\} \ipa{(h)@}\{\ipa{k,g}\} \change\ \ipa{B d\super j \textbardotlessj\ g}\\
\{\ipa{pj,bj}\} \{\ipa{tj,dj}\} \{\ipa{cj,\textbardotlessj j}\} \change\ \ipa{\{B,w\} d\super j \textbardotlessj}\\
\ipa{\!b \!d} \change\ \ipa{m n} (For some reason it seems that only first-register tones can occur in this environment)\\
\{\ipa{\r*{n}j,nj,\r{\textltailn}j,\textltailn j}\} \change\ \ipa{\textltailn} (Thompson appears to me to have hedged a bit on the last one; based on other evidence in the paper I'm sticking this one as a palatal nasal)\\
\ipa{tS} \change\ $\Omega$ (This is my own notation. I don't have a clue what the intermediate form was; became something else in different dialects)\\
N[-voiced] W[-voiced] \change\ N[+voiced] W[+voiced]\\
\ipa{((h)@)p d} \change\ \ipa{b t} / _\ipa{l}\\
\ipa{t} \change\ \O\ / _\ipa{\*r} (only first-register tones can occur in this environment)\\
\ipa{t}*\ipa{\super h d g} \change\ \ipa{t\super h t k} / _\ipa{w}\\
\ipa{s} \change\ \ipa{t(\super h?)}\\
\{\ipa{@k\super h,@gHj}\} \change\ \ipa{\textbardotlessj} (I think Thompson implied this was just a bit of a kludge)

\textbf{Finals:}\\
\ipa{l} \change\ \O\ / \{\ipa{i,e}\}_\\
\ipa{l} \change\ \ipa{j} / else\\
\ipa{c \textltailn} \change\ \ipa{t n} / ! E_ (apparently the precursor to Vietnamese short *a was treated as a short vowel here)

Thompson seems to list some changes as affecting Modern Vietnamese but I was unsure of where to put them so they'll go here:

\textbf{\ipa{a O}} \change\ \ipa{1@ u@}

In the original those first vowels were underlined.

\paragraph{Middle Vietnamese to Hanoi Vietnamese}{\it Pogostick Man}, from Thompson, Laurence C. (1976), ``Proto-Viet-Muong Phonology". {\it Oceanic Linguistics Special Publications} 13, Austroasiatic Studies II:1113 -- 1203; Wikipedia contributors (2012). ``Hanoi". {\it Wikipedia, the Free Encyclopedia}. \textless\url{http://en.wikipedia.org/w/index.php?title=Hanoi&oldid=509052974}\textgreater; Wikipedia contributors (2012), ``Vietnamese Language". {\it Wikipedia, the Free Encyclopedia}. \textless\url{http://en.wikipedia.org/w/index.php?title=Special:Cite&page=Vietnamese_language&id=509331797}\textgreater; Gage, William W. (1985), ``Glottal Stops and Vietnamese Tonogenesis". {\it Oceanic Linguistics Special Publications} 20:21 -- 36; and Thompson, Laurence C. (1979?), ``More on Viet-Muong Tonal Developments"

\textbf{Tonogenesis}
\begin{center}\begin{tabular}{| c | c | c | c |}\hline
Reg & A & B/D & C\\ \hline
1 & mid trailing & high rising & dipping\\
2 & low trailing & low dropping$^1$ & high rising$^2$\\\hline\end{tabular}\end{center}
1. Tense when _S\#; laryngealized elsewhere\\
2. Laryngealized

\textbf{Initials:}\\
\ipa{p\super h} \change\ \ipa{f}\\
\ipa{k\super h} \change\ \ipa{x} (only seems to have occurred with first-register tones)\\
\{\ipa{B,w}\} \{\ipa{d\super j,\textbardotlessj}\} \change\ \ipa{v z}\\
\ipa{\*r} \change\ \ipa{z} (only seems to have occurred with first-register tones)\\
\ipa{bl} \change\ \ipa{z}\\
\ipa{ml} \change\ \ipa{m\textltailn} \change\ \ipa{\textltailn} (Thompson seems to indicate that this may have become [l] as well; only seems to have occurred with second-register tones)\\
$\Omega$ \ipa{c\super h} \change\ \ipa{s tC}

\textbf{Vowels:}\\
\ipa{1} \change\ \ipa{i} / _\ipa{(@)w}\\
\ipa{E} \change\ \ipa{a} / _C[+palatal]

\textbf{Miscellanea:}\\
\ipa{w} \change\ \O\ / \ipa{t}V_\ipa{wk} (conjectured)

\paragraph{Middle Vietnamese to Saigon Vietnamese}{\it Pogostick Man}, from Thompson, Laurence C. (1976), ``Proto-Viet-Muong Phonology". {\it Oceanic Linguistics Special Publications} 13, Austroasiatic Studies II:1113 -- 1203; Wikipedia contributors (2012). ``Hanoi". {\it Wikipedia, the Free Encyclopedia}. \textless\url{http://en.wikipedia.org/w/index.php?title=Hanoi&oldid=509052974}\textgreater; Wikipedia contributors (2012), ``Vietnamese Language". {\it Wikipedia, the Free Encyclopedia}. \textless\url{http://en.wikipedia.org/w/index.php?title=Special:Cite&page=Vietnamese_language&id=509331797}\textgreater; Gage, William W. (1985), ``Glottal Stops and Vietnamese Tonogenesis". {\it Oceanic Linguistics Special Publications} 20:21 -- 36; and Thompson, Laurence C. (1979?), ``More on Viet-Muong Tonal Developments"\clearpage

\textbf{Tonogenesis}
\begin{center}\begin{tabular}{| c | c | c | c |}\hline
Reg & A & B/D & C\\ \hline
1 & mid trailing & high rising & mid rising\\
2 & low trailing & low$^1$ & high rising$^2$\\\hline\end{tabular}\end{center}
1. Level when _S\#; dipping otherwise\\
2. Laryngealized

\textbf{Initials:}\\
\ipa{p\super h} \change\ \ipa{f}\\
\ipa{k\super h} \change\ \ipa{x} (only seems to have occurred with first-register tones)\\
\{\ipa{B,w}\} \change\ \ipa{bj}\raisebox{-0.6ex}{\textasciitilde}\ipa{vj}\raisebox{-0.6ex}{\textasciitilde}\ipa{v}\\
\{\ipa{bl,tl}\} \change\ \ipa{\:t} (?)\\
\ipa{d\super j \textbardotlessj} \change\ \ipa{z j}\\
\ipa{m} \change\ \O\ / _\ipa{l}\\
$\Omega$ \ipa{c\super h} \change\ \ipa{\:s \:t\:s}\\
\ipa{\*r} \change\ \ipa{\:z} (sometimes?)

\textbf{Finals:}\\
\ipa{c \textltailn\ t n} \change\ \ipa{t n k N} / \ipa{a}_ (short /\ipa{a}/ only)\\
\ipa{c \textltailn} \change\ \ipa{t n} / \{\ipa{i,e}\}_\\
\{\ipa{c,\textltailn}\} \change\ \O\ / else

\textbf{Vowels:}\\
\ipa{@} \change\ \O\ / \{\ipa{i,1}\}_\{\ipa{p,m.w}\}\\
\ipa{@} \change\ \O\ / \ipa{1}_\ipa{j}\\
\ipa{@} \change\ \O\ / \ipa{u}_\{\ipa{m,j}\}\\
The contrast between short /\ipa{a}/ and short /\ipa{@}/ is neutralized when _\ipa{w}\{\ipa{k,N}\}\\
\ipa{a} \change\ \ipa{a:} / _\{\ipa{w,j}\}\\
\ipa{@(:) E} \change\ \ipa{1 E@} / _K\\
\ipa{E} \change\ \ipa{a} / _C[+palatal]

\textbf{Miscellanea:}\\
\ipa{w} \change\ \O\ / \ipa{t}V_\ipa{wk} (conjectured)

\clearpage

\section{Austronesian}\tab Wikipedia gives the following reconstruction of Proto-Austronesian created by Robert Blust:

\begin{center}\begin{tabular}{c | c c c c c c}
& Labial & Alveolar & Palatal & Retroflex & Velar & Glottal \\ \hline
Nasal & \ipa{m} & \ipa{n} & \ipa{\textltailn} & \textipa{\:n} & \textipa{N} & (\textipa{q,P}) \\
Plosive & \ipa{p b} & \ipa{t d} & & & \ipa{k g g\super j} & \\
Fricative & & \ipa{s} & \ipa{\c{c}} & & & \ipa{h} \\
Affricate & & \textipa{\t*{ts}} & \textipa{\t{c\c{c}} \t*{\textbardotlessj J}} & & & \\
Lateral & & \ipa{l} & \ipa{l\super j} & & & \\
Tap/Trill & & (\textipa{R,r,\;R}) & & & & \\
Approximant & \ipa{w} & & \ipa{j} & & & \\ \end{tabular}\end{center}

\begin{center}\begin{tabular}{c | c c c}
& Front & Central & Back \\ \hline
Close & \ipa{i} & & \ipa{u} \\
Mid & & \textipa{@} & \\
Open & & \ipa{a} & \\ \end{tabular}\end{center}

\begin{center}\begin{tabular}{c | c c c}
& Front & Central & Back \\ \hline
Close & \ipa{iw} & & \ipa{uj} \\
Open & & \ipa{aj aw} & \end{tabular}\end{center}

\tab Points of this phonology are in great dispute; Blust himself states this.

\tab (From Wikipedia contributors (2011), \textquotedblleft Proto-Austronesian language". \textit{Wikipedia, The Free Encyclopedia}. \textless\url{http://en.wikipedia.org/w/index.php?title=Proto-Austronesian_language&oldid=453318098}\textgreater)

\subsection{Proto-Austronesian to Proto-Malayo-Polynesian}{\it Pogostick Man}, from Wikipedia contributors (2011), ``Proto-Austronesian lanuage". {\it Wikipedia, the Free Encyclopedia}. \textless\url{http://en.wikipedia.org/w/index.php?title=Proto-Austronesian_language&oldid=453318098}\textgreater

\ipa{e} \change\ \ipa{a} / _\ipa{s}\\
\ipa{s ts l\super j} \change\ \ipa{h t n}

\subsubsection{Proto-Malayo-Polynesian to Proto-Bali-Sasak-Sumbawan}{\it TinyMusic}, from Adelaar, Alexander (2005), ``Malayo-Sumbawan". {\it Oceanic Linguistics} 44(2):357 -- 388

\ipa{j} \change\ \{\ipa{d,t}\} / \#_\\
\ipa{j z} \change\ \ipa{d j}\\
\ipa{w} \change\ \O\ / \#_\\
R \change\ \ipa{r}\\
\ipa{q} \change\ \ipa{h} / _\#\\
\{\ipa{q,h}\} \change\ \O\\
\ipa{iw uj} \change\ \{\ipa{i},?\} \ipa{i} / _\#\\
A:\\
--- \ipa{aj aw} \change\ \ipa{ej ow} / _\#\\
B:\\
--- \ipa{aj aw} \change\ \ipa{e ow} / _\#\\
C[+ voice] \change\ C[- voice] / _\#\\
H\ipa{@}S \change\ (\ipa{h})\ipa{@}(N)S / \#_

\paragraph{Proto-Bali-Sasak-Sumbawan to Balinese}{\it TinyMusic}, from Adelaar, Alexander (2005), ``Malayo-Sumbawan". {\it Oceanic Linguistics} 44(2):357 -- 388

\ipa{r} \change\ \{\ipa{r,h}\} \change\ \{\O,\ipa{h}\}\\
\ipa{h} \change\ \{\O,\ipa{h}\}\\
\ipa{w} \change\ \ipa{b} / \ipa{i}_\#\\
\ipa{ej ow} \change\ \ipa{i u}\\
``\ipa{@} assimilated to the following vowel after the loss of *-\ipa{r}-"\\
\ipa{a} \change\ \ipa{@}

\paragraph{Proto-Bali-Sasak-Sumbawan to Sasak}{\it TinyMusic}, from Adelaar, Alexander (2005), ``Malayo-Sumbawan". {\it Oceanic Linguistics} 44(2):357 -- 388

\ipa{h} \change\ \ipa{q} / _\# (might've been a retention?)\\
\ipa{h} \change\ \O\\
\ipa{iw ow ej} \change\ \ipa{i o e} / _\#\\
``*\ipa{i} and *\ipa{u} often become mid-vowels"\\
V(\ipa{h}) \change\ V(\ipa{q}) / _\# (again, might've been a retention?)\\
\ipa{a} \change \ipa{@} / _\# (Meno-mene and Mriak-mriku only)\\
\ipa{d} \change\ \ipa{r} / medial (Meno-mene and Mriak-mriku only)\\
\ipa{r} \change\ \ipa{h} / _\# (Meno-mene and Mriak-mriku only)

\paragraph{Proto-Bali-Sasak-Sumbawan to Sumbawan}{\it TinyMusic}, from Adelaar, Alexander (2005), ``Malayo-Sumbawan". {\it Oceanic Linguistics} 44(2):357 -- 388

\ipa{h} \change\ \ipa{q} / _\# (might've been a retention?)\\
\ipa{h} \change\ \O\\
\ipa{w} \change\ \O\ / \ipa{i}_\#\\
\ipa{ej ow} \change\ \ipa{e o} / _\#\\
\ipa{u i} \change\ \ipa{o e} / sometimes\\
V(\ipa{h}) \change\ V(\ipa{q}) / _\# (again, might've been a retention?)\\
S[+ voice] \change\ \O\ / _N\\
S[+ voice] \change\ \O\ / N_\\
\ipa{u} \change\ \ipa{i} / _\{\ipa{s,t,r,n,l}\} (blocked in Pusu)\\
``[C]ontraction of adjacent vowels" (not in Besar)\\
\ipa{b} \change\ \O\ / medial (sporadic)

\paragraph{Polynesian}

\paragraph{Proto-Polynesian to Luangiua}{\it thetha}, from Blust, Robert (2013), {\it The Austronesian Languages}, Revised Edition

\{\ipa{q,h}\} \change\ \O\\
\ipa{r} \change\ \ipa{l}\\
\ipa{f} \change\ \ipa{h}\\
\ipa{k t} \change\ \ipa{P k}\\
\ipa{n} \change\ \ipa{N}\\
\ipa{w} \change\ \ipa{v}

\subsubsection{Proto-Austronesian to Proto-Philippine}{\it Pogostick Man}, from Llamzon, Teodoro A. (1975), ``Proto-Philippine Phonology''. {\it Archipel} 9(1):29 -- 42; and from other changes and information from this document

*T \change\ \ipa{t}\\
\{*D,*Z,\ipa{z}\} \change\ \ipa{d} / \#_\\
*D \change\ \ipa{d} / _\#\\
*R \change\ \ipa{g} / \#_\\
*R \change\ \{\ipa{l,g}\} / _\#\\
\ipa{\textltailn} \change\ \ipa{n}\\
\ipa{c} \change\ \ipa{s}

\paragraph{Proto-Philippine to Bicol}{\it Pogostick Man}, from Llamzon, Teodoro A. (1975), ``Proto-Philippine Phonology''. {\it Archipel} 9(1):29 -- 42; and from other changes and information from this document

\ipa{@} \change\ \ipa{a} / V_V\\
\ipa{@} \change\ \ipa{u} / _\#\\
\ipa{*}j \{*D,\ipa{z}\} \change\ \ipa{r d} / V_V\\
\ipa{*}j \change\ \ipa{g} / \#_\\
\ipa{h} *j \change\ \O\ \ipa{d} / _\#\\
\ipa{*}R \change\ \ipa{g}\\
\ipa{q} \change\ \O\\
\ipa{iw} \change\ \ipa{uj}

\paragraph{Proto-Philippine to Cebuano}{\it Pogostick Man}, from Llamzon, Teodoro A. (1975), ``Proto-Philippine Phonology''. {\it Archipel} 9(1):29 -- 42; and from other changes and information from this document

\ipa{@} \change\ \ipa{u}\\
\ipa{*}D \ipa{d} \change\ \ipa{l r} / V_V\\
\ipa{*}j \change\ \ipa{d} / \#_\\
\{*j,*Z\} \ipa{z} \change\ \ipa{l r} / V_V\\
\ipa{h} *j \change\ \O\ \ipa{d} / _\#\\
\ipa{*}R \change\ \ipa{g}\\
\ipa{q} \change\ \O\\
\ipa{h} \change\ \O\ / _\#\\
\ipa{iw} \change\ \ipa{uj}

\paragraph{Proto-Philippine to Hiligaynon}{\it Pogostick Man}, from Llamzon, Teodoro A. (1975), ``Proto-Philippine Phonology''. {\it Archipel} 9(1):29 -- 42; and from other changes and information from this document

\ipa{@} \change\ \ipa{u}\\
\ipa{*}D \change\ \ipa{l}\\
\ipa{*}j \change\ \ipa{d} / \#_\\
\{*Z,*j\} \ipa{z} \change\ \ipa{l r} / V_V\\
\ipa{*}j \change\ \ipa{d} / _\#\\
\ipa{h} \change\ \O\ / _\#\\
\ipa{q} \change\ \O\\
\ipa{iw} \change\ \ipa{uj}

\paragraph{Proto-Philippine to Ibanag}{\it Pogostick Man}, from Llamzon, Teodoro A. (1975), ``Proto-Philippine Phonology''. {\it Archipel} 9(1):29 -- 42; and from other changes and information from this document

\ipa{@} \change\ \ipa{a}\\
Something happens to final voiceless stops but it isn't clear in the paper\\
\ipa{*}D \change\ \ipa{r}\\
\ipa{*}j \change\ \ipa{g} / possible exception in word-initial position?\\
\ipa{*}Z \change\ \ipa{r}\\
\ipa{z} \change\ \ipa{r} / V_V\\
\ipa{*}R \change\ \ipa{g}\\
\ipa{r} \change\ \ipa{d} / \#_ (?)\\
\{\ipa{s,c}\} \change\ \ipa{t}\\
\{\ipa{q,h}\} \change\ \O\\
\ipa{uj} \change\ \ipa{i}\\
\ipa{iw} \change\ \ipa{uj}

\paragraph{Proto-Philippine to Ifugao}{\it Pogostick Man}, from Llamzon, Teodoro A. (1975), ``Proto-Philippine Phonology''. {\it Archipel} 9(1):29 -- 42; and from other changes and information from this document

\ipa{@} \change\ \ipa{o}\\
*j \change\ \ipa{g} / _\#\\
\{\ipa{z},*Z,*D,*j\} \change\ \ipa{d}\\
\ipa{*}R seems to have had a few different reflexes, mainly one of /\ipa{l g j}/; if /\ipa{g j}/ occurred, /\ipa{g}/ was more probable at word boundaries and /\ipa{j}/ was more probable medially\\
\{\ipa{q,h}\} \change\ \O\\
\{\ipa{s,c}\} \change\ \ipa{h}\\
\O\ \change\ \ipa{j} / _\ipa{iw}\#\\
\ipa{ay} \change\ \ipa{e} / _\#

\paragraph{Proto-Philippine to Ilocano}{\it Pogostick Man}, from Llamzon, Teodoro A. (1975), ``Proto-Philippine Phonology''. {\it Archipel} 9(1):29 -- 42; and from other changes and information from this document

\ipa{*}D \change\ \ipa{d}\\
\{\ipa{z},*Z,*j\} \change\ \ipa{d} / V_V\\
\ipa{*}j \change\ \ipa{g}\\
R \change\ \{\ipa{g,r}\} / _\#\\
R \change\ \ipa{r}\\
\{\ipa{q,h}\} \change\ \O\\
\ipa{iw} \change\ \ipa{uj}

\paragraph{Proto-Philippine to Proto-Kalamian}{\it Pogostick Man}, from Himes, Ronald (2006), ``The Kalamian Microgroup of Philippine Languages''. Paper presented at Tenth International Conference on Austronesian Linguistics. 17� -- 20 January 2006. Puerto Princesa City, Palawan, Philippines. \textless\url{http://www.sil.org/asia/philippines/ical/papers.html}\textgreater

\{\ipa{h,P}\} \change\ \O\\
\{\ipa{z,j}\} \change\ \ipa{d}\\
\ipa{*}R \ipa{\textltailn} \change\ \ipa{l n}\\
\ipa{e} \change\ \ipa{u} / _C\ipa{u}\\
\ipa{e} \change\ \ipa{i} / _C\ipa{i}\\
\ipa{e} \change\ \ipa{u} / \ipa{u}C_\\
\ipa{e} \change\ \ipa{a} / _C[- voice]\#\\
\ipa{d} \change\ \ipa{r} / V_V\\
Contrastive stress lost

\subparagraph{Proto-Kalamian to Agutaynen}{\it Pogostick Man}, from Himes, Ronald (2006), ``The Kalamian Microgroup of Philippine Languages''. Paper presented at Tenth International Conference on Austronesian Linguistics. 17� -- 20 January 2006. Puerto Princesa City, Palawan, Philippines. \textless\url{http://www.sil.org/asia/philippines/ical/papers.html}\textgreater

O[- voice] \change\ \ipa{P} / _C\\
\ipa{k} \change\ \O\ / _\{V,\#\}\\
\ipa{q} \change\ \ipa{k}\\
\ipa{aI ai} \change\ \ipa{II ii} (not sure if there's a long vowel or hiatus here)\\
\ipa{t} \change\ \ipa{s} / _\ipa{i}\\
\ipa{s} \change\ \ipa{t} / _V ! _E\\
\ipa{s} \change\ \ipa{t} / _\#\\
\O\ \change\ \ipa{P} / \#_\\
\O\ \change\ \ipa{P} / V_\#

\subparagraph{Proto-Kalamian to Karamiananen}{\it Pogostick Man}, from Himes, Ronald (2006), ``The Kalamian Microgroup of Philippine Languages''. Paper presented at Tenth International Conference on Austronesian Linguistics. 17� -- 20 January 2006. Puerto Princesa City, Palawan, Philippines. \textless\url{http://www.sil.org/asia/philippines/ical/papers.html}\textgreater

\ipa{\{t,k\} q} \change\ \ipa{P k} / _C\\
\ipa{s} \change\ \ipa{P} / _C\\
\ipa{k} \change\ \O\ / _\{V,\#\}\\
\ipa{q} \change\ \ipa{k}\\
\ipa{aI ai} \change\ \ipa{II ii} (not sure if there's a long vowel or hiatus here)\\
\ipa{t} \change\ \ipa{s} / _\ipa{i}\\
\ipa{s} \change\ \ipa{t} / _V ! _E\\
\ipa{s} \change\ \ipa{t} / _\#\\
\ipa{s} \change\ \ipa{c}\& _ (the paper doesn't explain what this represents)\\
\ipa{b} \change\ \ipa{B} / V_V\\
\ipa{B} \change\ \ipa{w} / V[+ high]_\ipa{a}\\
\ipa{g} \change\ \ipa{h} / V_V\\
\O\ \change\ \ipa{P} / \#_\\
\O\ \change\ \ipa{P} / V_\#

\subparagraph{Proto-Kalamian to Kalamian Tagbanwa}{\it Pogostick Man}, from Himes, Ronald (2006), ``The Kalamian Microgroup of Philippine Languages''. Paper presented at Tenth International Conference on Austronesian Linguistics. 17� -- 20 January 2006. Puerto Princesa City, Palawan, Philippines. \textless\url{http://www.sil.org/asia/philippines/ical/papers.html}\textgreater

\ipa{\{t,k,q,s\}} \change\ \ipa{k} / _C\\
\ipa{k} \change\ \O\ / _\{V,\#\}\\
\ipa{q} \change\ \ipa{k}\\
\ipa{aI ai} \change\ \ipa{II ii} (not sure if there's a long vowel or hiatus here)\\
\ipa{b g} \change\ \ipa{B} V / V_V\\
\O\ \change\ \ipa{P} / \#_\\
\O\ \change\ \ipa{P} / V_\#

\paragraph{Proto-Philippine to Kankanay}{\it Pogostick Man}, from Llamzon, Teodoro A. (1975), ``Proto-Philippine Phonology''. {\it Archipel} 9(1):29 -- 42; and from other changes and information from this document

\{*D,*Z\} \change\ \ipa{d}\\
\{\ipa{z},*j\} \change\ \ipa{d} / V_V\\
\ipa{*}j \change\ \ipa{g} / _\#\\
\ipa{*}R seems to have had a few different reflexes, mainly one of /\ipa{l g j}/; if /\ipa{g j}/ occurred, /\ipa{g}/ was more probable at word boundaries and /\ipa{j}/ was more probable medially\\
\ipa{c} \change\ \ipa{s}\\
\{\ipa{h,q}\} \change\ \O\\
Something seems to have changed to \ipa{@w} finally but the paper may have an error here\\
\ipa{iw} \change\ \ipa{uj}

\paragraph{Proto-Philippine to Tagalog}{\it Pogostick Man}, from Llamzon, Teodoro A. (1975), ``Proto-Philippine Phonology''. {\it Archipel} 9(1):29 -- 42; and from other changes and information from this document

\ipa{@} \change\ \ipa{i}\\
\ipa{u} \change\ \ipa{o} / _\#\\
 \{*D,*j\} \ipa{\{d,z\}} \change\ \ipa{l r} / V_V\\
\ipa{*}j \change\ \ipa{d} / \#_\\
\ipa{h} *j \change\ \O\ \ipa{d} / _\#\\
\ipa{*}R \change\ \ipa{g}\\
\ipa{q} \change\ \O\ (not sure what happens word-finally to it)\\
\ipa{uj iw} \change\ \ipa{oj uj} / _\#

\paragraph{Proto-Philippine to Waray}{\it Pogostick Man}, from Llamzon, Teodoro A. (1975), ``Proto-Philippine Phonology''. {\it Archipel} 9(1):29 -- 42; and from other changes and information from this document

\ipa{h} \change\ \O\ / _\#\\
\ipa{@} \change\ \ipa{u}\\
\ipa{*}T \change\ \ipa{t}\\
\{*D,*Z\} \change\ \ipa{r} / V_V\\
\{*D,*Z,\ipa{z}\} \change\ \ipa{d}\\
*\ipa{j} \change\ \ipa{r} / V_V\\
\ipa{*j} \change\ \ipa{d}\\
\ipa{*}R \change\ \ipa{g}\\
\ipa{r} \change\ \ipa{l} / _\#\\
\ipa{*c} \change\ \ipa{s}\\
\ipa{q} \change\ \O\\
\ipa{iw} \change\ \ipa{uj}

\subsection{Proto-Austronesian to Proto-Batak}{\it TinyMusic}, from Adelaar, K.A. (1981), ``Reconstruction of Proto-Batak Phonology". In Blust, Robert (ed.), {\it Historical Linguistics in Indonesia} I:1 -- 20.

\tab {\it TinyMusic notes that this particular set of sound changes is with respect to the reconstruction of Proto-Austronesian by Dyen (1965), and that he had some trouble with *\ipa{j}.}

\ipa{w} \change\ \O\ / \ipa{i}_\#\\
\{\ipa{a,e}\} \change\ \ipa{o} / _\ipa{w}\#\\
\ipa{a} \change\ \ipa{e} / _\ipa{j}\#\\
\ipa{\{ts,\:t\} \{\t{\textbardotlessj J},\:d\} \{l\super j,\textltailn\} \{P,x,s,h\} \;R} \change\ \ipa{t d n \O\ r} (velar fricative is conjectured; changes \change\ \ipa{d} ``unsure")\\
\ipa{q} \change\ \O\ / \#_ (sometimes; ``represented by *\ipa{h} in PB")\\
\ipa{z} \change\ \ipa{j} (fricative changes to approximant)

\subsubsection{Proto-Malayo-Polynesian to Proto-Chamic}{\it TinyMusic}, from Adelaar, Alexander (2005), ``Malayo-Sumbawan". {\it Oceanic Linguistics} 44(2):357 -- 388

\ipa{z j} \change\ \ipa{j d}\\
\ipa{wa} \change\ \ipa{u} / \#_\\
\ipa{w} \change\ \O\ / \#_\\
R \ipa{q} \change\ \ipa{r h}\\
\ipa{i iw u} \change\ \ipa{Oj} ? \ipa{Ow} / _\#\\
C$_1$C$_2$ \change\ C$_2$\\
Nasal + stop clusters assimilate in POA\\
C[+ voice] \change\ C[- voice] / _\#\\
\ipa{l n} \change\ \ipa{r l} / \#_ (sporadic)\\
\ipa{d j} \change\ \ipa{r \:l} (sporadic)\\
``Sometimes a reduction of [the] penultimate vowel"\\
\ipa{a} \change\ \ipa{a:} / _C\# (sometimes)

\subsubsection{Proto-Malayo-Polynesian to Chamorro}{\it Whimemsz}, from Blust, Robert (2000), ``Chamorro Historical Phonology". {\it Oceanic Linguistics} 39(1):83 -- 122

\ipa{@} \change\ \ipa{u}\\
\ipa{@} \change\ \O\ / VC_CV\\
V \change\ \O\ / VC_CV (sporadic)\\
\ipa{i u} \change\ \ipa{e o} / _C\{C,\#\}\\
\ipa{i u} \change\ \ipa{e o} / CC\# (sporadic)\\
\ipa{a} \change\ \ipa{\ae} ``(in some forms; environment and conditioning unclear)''\\
\ipa{uj} \change\ \{\ipa{i,u}\}\\
\ipa{iw} \change\ \ipa{u}\\
\ipa{p c q} \change\ \ipa{f s P}\\
\ipa{k} \change\ \O\ / _\# (sporadic)\\
\ipa{k} \change\ \ipa{h} / ! _\#\\
\ipa{h} \change\ \O\\
V$_0$V$_0$ \change\ V$_0$\\
\ipa{b dz} \change\ \ipa{p ts}\\
\ipa{d} \change\ \O\ / _\#\\
\ipa{\textbardotlessj} \change\ \ipa{P}\\
\ipa{l} \change\ \ipa{d} / _\{C,\#\}\\
\ipa{R} \change\ \ipa{g}\\
O \change\ O[-voiced] / _\{C,\#\}\\
\O\ \change\ \ipa{j} / \ipa{i}_\ipa{a}\\
\O\ \change\ \ipa{w} / \ipa{u}_\ipa{a}\\
\O\ \change\ \ipa{w} / \ipa{a}_\ipa{u}\\
\O\ \change\ \ipa{w} / \#_V\\
\ipa{j w} \change\ \ipa{dz g\super w}\\
\ipa{g\super w} \change\ \ipa{g} / _V[+round]

\subsubsection{Proto-Malayo-Polynesian to Proto-Malayic}{\it TinyMusic}, from Adelaar, Alexander (2005), ``Malayo-Sumbawan". {\it Oceanic Linguistics} 44(2):357 -- 388

\ipa{j} \change\ \ipa{t} / _\#\\
\ipa{j} \change\ \ipa{d}\\
\ipa{z} \change\ \ipa{j}\\
\ipa{w} \change\ \O\ / \#_\\
R \change\ \ipa{r}\\
\ipa{h} \change\ \ipa{P} / _\# (sometimes)\\
\ipa{h} \change\ \O\ / else\\
\{\ipa{iw,uj}\} \change\ \ipa{i} / _\#\\
A:\\
--- \ipa{aj aw} \change\ \ipa{i u} / _\#\\
B:\\
--- \ipa{aj aw} \change\ \ipa{aj aw} / _\#\\
C$_1$C$_2$ \change\ C$_2$\\
C[+ A POA]C[+ B POA] \change\ C[+ B POA]C[+ B POA]\\
C[+ voice] \change\ C[- voice] / _\#\\
H \change\ \{\O,\ipa{h}\} / _\ipa{@}NS / \#_

\paragraph{Proto-Malayic to (Standard) Malay}{\it TinyMusic}, from Tryon, Darrell (1995), {\it Comparative Austronesian Dictionary}

V \change\ \ipa{@} / _(C\textellipsis)UU\#\\
\ipa{@} \change\ \ipa{a} / _(C\textellipsis)\#\\
\ipa{n} \change\ \ipa{\textltailn} / ``in the environment of {\it i} (sporadic)"\\
\ipa{h} \change\ \O\ / ! _\# (sporadic)

\subsubsection{Proto-Malayo-Polynesian to Proto-Malayo-Javanic}{\it TinyMusic}, from Adelaar, Alexander (2005), ``Malayo-Sumbawan". {\it Oceanic Linguistics} 44(2):357 -- 388

\ipa{j} \change\ \ipa{l}\\
\ipa{q h} \change\ \ipa{h} \O\\
A:\\
--- \ipa{aw aj} \change\ \ipa{@w @j} / _\#\\
B:\\
--- \ipa{aw aj} \change\ \ipa{aw aj} / _\#

\paragraph{Proto-Malayo-Javanic to Javanese}{\it TinyMusic}, from Adelaar, Alexander (2005), ``Malayo-Sumbawan". {\it Oceanic Linguistics} 44(2):357 -- 388

\{\ipa{l,d}\} \ipa{z} \change\ \ipa{r d}\\
\ipa{b} \change\ \ipa{w} / ! adjacent to another consonant\\
\ipa{@} \change\ \ipa{u} / _\ipa{h}\#\\
R \change\ \O\\
\ipa{h} \change\ \O\ / \#_\\
\ipa{h} \change\ \{\ipa{o,h,w}\} / V_V\\
\ipa{iw uj} \change\ \ipa{ju i} / _\#\\
A:\\
--- \ipa{@w @j} \change\ \ipa{i u} / _\#\\
B:\\
--- \ipa{aw aj} \change\ \ipa{e \'{o}} / _\#\\
C$_1$C$_2$ \change\ C$_2$\\
Nasal + stop clusters ``become homorganic"\\
H\ipa{@}S \change\ \ipa{(h)@}NS

\paragraph{Proto-Malayo-Polynesian to Madurese}{\it TinyMusic}, from Adelaar, Alexander (2005), ``Malayo-Sumbawan". {\it Oceanic Linguistics} 44(2):357 -- 388

\ipa{\:l} \change\ \ipa{lP} / _\#\\
\ipa{\:l} \change\ \ipa{l}\\
\ipa{z} \change\ \ipa{j\super h}\\
\ipa{w j} fortite when non-final\\
\ipa{b} \change\ \{\ipa{w},\O\} / \#_\\
R \change\ \ipa{P} / _\#\\
R \change\ \ipa{r}\\
\ipa{h} \change\ \ipa{P} / V$_0$V$_0$\\
\ipa{h} \change\ \O\\
``Aspiration of initial and intervocalic voiced stops and *\ipa{z}"\\
\{\ipa{p,t,k}\} \ipa{b d g} \change\ \ipa{P p t k} / _\#\\
\ipa{ij uw} \change\ \ipa{uj} \{\ipa{uj,\'{o}j}\} / _\#\\
V \change\ V\ipa{:} / \ipa{@}(C\textellipsis?)_\\
C[+ voice] \change\ C[- voice] / _\#\\
H\ipa{@}S \change\ \ipa{(h)@}NS / \#_

\paragraph{Proto-Malayo-Javanic to Sundanese}{\it TinyMusic}, from Adelaar, Alexander (2005), ``Malayo-Sumbawan". {\it Oceanic Linguistics} 44(2):357 -- 388

\{\ipa{l,j}\} \ipa{z} \change\ \ipa{r j}\\
\ipa{w} \change\ \{\O,\ipa{c}\} / \#_\\
R \change\ \{\O,\ipa{r,j}\}\\
\ipa{iw uj} \change\ \{\ipa{ju,i}\} \ipa{oj} / _\#\\
A:\\
--- \ipa{@w @j} \change\ \ipa{o e} / _\#\\
B:\\
--- \ipa{aw aj} \change\ \ipa{o aj} / _\#

\subsubsection{Proto-Malayo-Polynesian to Palauan}{\it Whimemsz}, from Blust, Robert (2009), ``Palauan Historical Phonology: Whence the Intrusive Velar Nasal?". {\it Oceanic Linguistics} 48(2):307 -- 336

\ipa{aj aw uj} \change\ \ipa{e o i}\\
\O\ \change\ \ipa{w} / \ipa{u}_V\\
\O\ \change\ \ipa{j} / \ipa{i}_V\\
\ipa{@} \change\ \O\ / \#_\\
\O\ \change\ \ipa{@} / C_C ``(for certain consonant combinations, which the paper doesn't specify)''\\
\ipa{h} \change\ \O\\
\ipa{@} \change\ \{\ipa{e,o}\} / stressed; ``(result of /e/ or /o/ unpredictable)''\\
\ipa{p} \change\ \ipa{w}\\
\ipa{wa} \change\ \ipa{o} / \#_ when unstressed\\
V \change\ \ipa{@} / unstressed\\
\{\ipa{aw,@w}\} \change\ \ipa{o} / _\#\\
\ipa{w@} \change\ \ipa{u} / \#_\\
\ipa{j l} \change\ \ipa{r j}\\
\ipa{@} \change\ \O\ / _\ipa{j}\\
\ipa{j} \change\ \O\ / C_\ipa{i}\\
\ipa{@} \change\ \O\ / _\#\\
\ipa{@} \change\ \O\ ``(sporadic)''\\
\ipa{t} \change\ \ipa{D} / ! adjacent to S\\
\ipa{s} \change\ \ipa{t}\\
\ipa{\;R} \change\ \ipa{r} / _C[+dental]\\
\ipa{\;R} \change\ \ipa{s}\\
\ipa{d} \change\ \ipa{r}\\
\ipa{\:d} \change\ \ipa{\:r} ``(only one example)''\\
\ipa{\textltailn} \change\ \ipa{n} (except possibly to \ipa{n} when \#_)\\
\ipa{n dz} \change\ \ipa{l r}\\
\ipa{rl} \change\ \ipa{l:}\\
\ipa{\textbardotlessj} \change\ \ipa{k} / _C\#\\
\ipa{\textbardotlessj} \change\ \ipa{s}\\
\ipa{N} \change\ \O\ / C_\#\\
\ipa{t} \change\ \{\ipa{s,D}\} / ``unpredictably, to eliminate sV(C)t and tV(C)s sequences"\\
\O\ \change\ \ipa{N} / \#_V\\
\ipa{q} \change\ \ipa{X} (\change\ \ipa{Q})

\subsubsection{Proto-Malayo-Polynesian to Proto-North Sarawak}{\it Whimemsz}, from Blust, Robert (2002), ``Kiput Historical Phonology". {\it Oceanic Linguistics} 41(2):384 -- 438; and Blust, Robert (2007), ``\`{O}ma L\'{o}ngh Historical Phonology". {\it Oceanic Linguistics} 46(1):1 -- 53

\ipa{q} \change\ \ipa{P}\\
\{\ipa{P,h}\} \change\ \O\ / \#_\\
\{\ipa{P,h}\} \change\ \O\ / V$_1$[+high]_V$_2$\\
\ipa{h} \change\ \O\ / _\#\\
\ipa{h} \change\ \ipa{P} / V$_0$_V$_0$\\
\ipa{h} \change\ \ipa{P} / \ipa{a}_\{\ipa{i,u}\}\\
\ipa{@} \change\ \O\ / adjacent to a vowel\\
\ipa{a} \change\ \ipa{@} / _UU(U\textellipsis)\#\\
\ipa{@} \change\ \O\ / _V\\
\ipa{@} \change\ \O\ VC_CV\\
Nasal assimilation to following stops in some words; in other words it results in a geminate stop\\
Postvocalic obstruents with different POAs become geminates of the second when ``in reduplicated monosyllabic roots" and ``in non-reduplicated bases which had undergone the change of schwa syncope in medial syllables"\\
C \change\ C\ipa{:} / \ipa{@}_V (?)\\
\ipa{\textbardotlessj(:)} \change\ \ipa{d(:)}\\
\ipa{b: d: dz: g:} \change\ \ipa{b\super H d\super H dz\super H g\super H} (Whimemsz says these become ``voiced stops with voiceless releases. . .treated as unit phonemes, not clusters)"\\
\ipa{@} \change\ \O\ / \#_UU(U\textellipsis)\# ``(i.e., in word-initial position in prepenultimate syllables)\\
\ipa{p: t: c: k:} \change\ \ipa{p t c k}\\
\ipa{c} \change\ \ipa{s}

\paragraph{Proto-North Sarawak to Kiput}{\it Whimemsz}, from Blust, Robert (2002), ``Kiput Historical Phonology". {\it Oceanic Linguistics} 41(2):384 -- 438

Stress reassignment to the final syllable\\
\ipa{P} \change\ \O\ / V_V\\
\ipa{k} \change\ \O\ / V_V ``(in some forms)''\\
\ipa{ai au} \change\ \ipa{a\textsubarch{i} a\textsubarch{u}} / _\#\\
\ipa{@} \change\ \ipa{a} / _\ipa{P}\#\\
\ipa{ai au} \change\ \ipa{E: O:} / _\textellipsis\#\\
\ipa{i u} \change\ \ipa{@\textsubarch{i} @\textsubarch{u}} / _\#\\
\O\ \change\ \ipa{h} / \ipa{a}_\#\\
\ipa{s} \change\ \O\ / V_V (sporadic)\\
V$_0$V$_0$ \change\ V$_0$\\
\ipa{@} \change\ \O\ / adjacent to a vowel\\
V[+stress] \change\ V\ipa{:} / _C\# ! V = \ipa{@} and/or C = \ipa{h} ``(applies to diphthongal nuclei as well a[s] monophthongs)"\\
\ipa{\;R} \change\ \{\ipa{l,R}\} / ! _\# (the latter is more common)\\
\ipa{l} \change\ \O\ / ! _\# (irregular)\\
\ipa{u} \change\ \ipa{@w} / _V ``(also cases of (C)u \change\ w /__V)"\\
\ipa{i} \change\ \ipa{@j} / _V ``(also cases of (C)i \change\ j /__V)"\\
\ipa{w j} \change\ \ipa{v \textbardotlessj}\\
\ipa{i\textsubarch{u}} \change\ \ipa{u\textsubarch{i}}\\
\ipa{s} \change\ \O\ / _\#\\
\ipa{i u} \change\ \ipa{E O} / _C\# ! _P (sporadic)\\
\ipa{\;R} \change\ \ipa{P} / _\# ``(in a handful of forms)"\\
\ipa{\;R} \change\ \ipa{R}\\
\ipa{a} \change\ \ipa{i} / O[+voiced]\textellipsis_(C)\# ``(blocked if there was an intervening nasal, and sometimes if there was an intervening voiceless stop or liquid)''\\
\ipa{i u} \change\ \ipa{@\textsubarch{i} @\textsubarch{u}} / _(\ipa{P})\#\\
\ipa{@\textsubarch{i} @\textsubarch{u}} \change\ \ipa{a\textsubarch{i} a\textsubarch{u}} / ! O[+voiced] earlier in the word\\
\ipa{b\super H} \{\ipa{d\super H,dz\super H}\} \ipa{g\super H} \change\ \ipa{f s k}\\
\ipa{f} \change\ \ipa{s}\\
\ipa{v g \textbardotlessj} \change\ \ipa{f k c} / V_V\\
\ipa{v \textbardotlessj} \change\ \ipa{f c} / \#_\\
\{\ipa{i,E}\} \{\ipa{u,O}\} \change\ \ipa{i\textsubarch{@} u\textsubarch{@}} / _\{\ipa{k,N}\}\# ``(and also sporadically before final *t and *n and some other consonants)\\
\ipa{k N} \change\ \ipa{P} \O\ / V\ipa{\textsubarch{@}}_\#\\
\ipa{@} \change\ \{\ipa{@,a}\} / _C\# ``(free variants)''\\
NS[-voice] \change\ S\ipa{:}\\
``Numerous different possible reflexes of N\-\-S[+voice] clusters''\\
(C)V \change\ \O\ / \#_C\textellipsis\ ``(irregular)''\\
\ipa{n} \change\ \ipa{l} / \#_\\
\ipa{@} \change\ \O\ / \#_\\
\{\ipa{l,R}\} \change\ \ipa{n} / _\#\\
\ipa{dz} \change\ \ipa{d} / \#_ (though sometimes \change\ \{\ipa{s,\textbardotlessj}\})\\
\{\ipa{s,c}\} \change\ \ipa{t} / _V\{\ipa{s,c}\}V\\
\ipa{b d} \change\ \ipa{p t} / _\#

\paragraph{Proto-North Sarawak to Proto-Kenyah}{\it Whimemsz}, from Blust, Robert (2007), ``\`{O}ma L\'{o}ngh Historical Phonology". {\it Oceanic Linguistics} 46(1):1 -- 53

\ipa{d} \change\ \ipa{l} / \#_ (sporadic)\\
\ipa{\;R} \change \ipa{h} / V_V(C)\#\\
\ipa{\;R} \change\ \ipa{h} / _\#\\
\ipa{\:R} \change\ \O\\
\ipa{s} \change\ \ipa{h} / _\#\\
\ipa{i u} \change\ \ipa{e o} / _\ipa{h}\#\\
\ipa{h} \change\ \O\ / _\#\\
S \change\ S[-voice] / _\#\\
\ipa{l} \change\ \ipa{n} / _\#\\
\ipa{s} \change\ \ipa{t} / _V\ipa{s}VC\\
CV \change\ \O\ / _NCVC ``(in reduplications)''\\
Word-initial nasals assimilate to the POA of a following consonant

\subparagraph{Proto-Kenyah to \`{O}ma L\'{o}ngh}{\it Whimemsz}, from Blust, Robert (2007), ``\`{O}ma L\'{o}ngh Historical Phonology". {\it Oceanic Linguistics} 46(1):1 -- 53

\ipa{b\super H d\super H dz\super H g\super H} \change\ \ipa{p t c k}\\
\ipa{i} \change\ \ipa{e} / _\ipa{k}\#\\
\ipa{i} \change\ \ipa{i@} / _\ipa{N}\#\\
\ipa{p} \change\ \ipa{k} / \ipa{u}_\#\\
\ipa{u} \change\ \ipa{o} / _\ipa{k}\#\\
\ipa{u} \change\ \ipa{o} / _\ipa{N}\# ``(sporadically failed to occur)''\\
\ipa{u} \change\ \ipa{W} / _(C)\# ! _\ipa{P}\#\\
\ipa{a} \change\ \ipa{o} / _\# ``(not in all forms)''\\
\ipa{P} \change\ \O\ / _\# ``(but a\ipa{P} \change\ \ipa{@P} in some forms)''\\
\ipa{k} \change\ \ipa{P} / _\#\\
\ipa{a} \change\ \ipa{E} / \{\ipa{t,n}\}_\#\\
\{\ipa{p,t}\} \ipa{n} \change\ \ipa{c\textcorner\ \textltailn} / \{\ipa{i,E}\}_\#\\
\ipa{m} \change\ \ipa{\textltailn} / \ipa{i}_\#\\
\ipa{a\textsubarch{i} a\textsubarch{u}} \change\ \ipa{E O}\\
\{\ipa{u\textsubarch{i},i\textsubarch{u}}\} \change\ \ipa{e}\\
\ipa{i u} \change\ \ipa{e o} / _CV[+close-mid](C)\# ``(i\ipa{@} is treated as close mid for this change)''\\
\ipa{i u} \change\ \ipa{E O} / _CV[+open-mid](C)\#\\
\ipa{i u} \change\ \ipa{e o} / _C\ipa{W}\#\\
\ipa{N} \change\ \ipa{\r{N}} / _\#\\
\{\ipa{p,t}\} \{\ipa{m,n}\} \change\ \ipa{k N} / \ipa{W}_\#\\
\ipa{i u} \change\ \ipa{@j @w} / _V(C)\#\\
\ipa{j w} \change\ \ipa{z v}\\
\ipa{p} \change\ \ipa{f} / \#_\\
\ipa{@} \change\ \O\ / \#_\\
\ipa{p k} \change\ \ipa{f G} / V_V ! "\ipa{@}_V\\
\ipa{d} \change\ \ipa{r} / V_V ``(irregular)''\\
\ipa{dz} \change\ \ipa{\textbardotlessj} / V_V\\
\ipa{b d dz g} \change\ \ipa{p t c k} / N_\\
N \change\ \O\ / _S ``(sporadic)''\\
\ipa{h} \change\ \O\ / V_V

\paragraph{Proto-Malayic to Minangkabau}{\it TinyMusic}, from Tryon, Darrell (1995), {\it Comparative Austronesian Dictionary}

\ipa{n} \change\ \ipa{\textltailn} / ``in the environment of {\it i}"\\
\ipa{h} \change\ \O\ / ! _\# (some exceptions)\\
\ipa{a} \change\ \ipa{o} / _(C\textellipsis)\#\\
\ipa{u i} \change\ \ipa{o e} (sporadic)\\
Chronologically-ordered changes:\\
--- \ipa{@} \change\ \ipa{a} / _(C\textellipsis)\# (eventually spread to everywhere)\\
--- \ipa{m p} \change\ \ipa{n t} / \{\ipa{u,i}\}_\\
--- \ipa{a u} \change\ \ipa{e uj} / _\{\ipa{t,s}\}\#\\
--- \ipa{a} \change\ \ipa{o} / _\ipa{p}\#\\
--- \ipa{u i} \change\ \ipa{u@ i@} / _\ipa{\{k,N,h,l,r\}}\#\\
--- \ipa{\{p,t,k\} s} \change\ \ipa{P h} / _\#\\
--- \ipa{\{l,r\}} \change\ \O\ / _\# (retained across morpheme boundaries)

\subsection{Proto-Austronesian to Proto-Oceanic}{\it Pogostick Man}, from Wikipedia contributors (2011), ``Proto-Austronesian language". {\it Wikipedia, the Free Encyclopedia}. \textless\url{http://en.wikipedia.org/w/index.php?title=Proto-Austronesian_language&oldid=453318098}\textgreater

\ipa{mb} \change\ \ipa{p}\\
\{\ipa{nts,ns,nz,ng\super j}\} \{\ipa{ts,z,g\super j}\} \change\ \ipa{g\super j s}\\
\{\ipa{Nk,Ng}\} \ipa{g} \change\ \ipa{g k}\\
\ipa{d} \change\ \ipa{r}\\
\ipa{e} \{\ipa{uj,iw}\} \change\ \ipa{o i}\\
\ipa{aw} \change\ \ipa{o} / _\#\\

\subsubsection{Proto-Oceanic to Hawai'ian}{\it Chris Zoller}, from Trask, R.L. (1996), {\it Historical Linguistics}

\tab {\it NB: Zoller states that these changes are ``[s]implified".}\\

\{\ipa{h,P}\} \change\ \O\\
\{\ipa{s,f}\} \change\ \ipa{h}\\
\ipa{k t} \change\ \ipa{P k}\\
\ipa{N r v} \change\ \ipa{n l w}

\subsubsection{Proto-Oceanic to Hiw}{\it thetha}, from Fran\c{c}ois, Alexander (2005), ``Unraveling the history of the vowels of seventeen north Vanuatu languages'', and Fran\c{c}ois, Alexander (2010), ``Phonotactics and the prestopped velar lateral of Hiw''

\ipa{p p\super w \{c},*j\} \ipa{k q} \change\ \ipa{B B\super w s G} \O\\
\ipa{B b m} \change\ \ipa{B\super w b\super w m\super w} / typically near *\ipa{u}\\
\ipa{dr} *r \change\ \ipa{d r}\\
C \change\ \O\ / _\#\\
\ipa{\textltailn} \change\ \ipa{n}\\
\ipa{b d g} \change\ \ipa{p t k}\\
\ipa{B\super w b\super w m\super w} \change\ \ipa{w k\super w N\super w}\\
\ipa{l} \change\ \ipa{j}\\
\ipa{s} \change\ \ipa{h} \change\ \O\ (sporadic)\\
\ipa{r} \change\ \ipa{g\;L}\\
"V[+ high(er)] \change\ \O\ / _CV\\
"V \change\ \ipa{@} / _CV\\
"\ipa{@} sometimes assimilates to a following vowel\\
\ipa{a} \change\ \ipa{e} / _CV[+ mid] (sporadic)\\
\ipa{a} \change\ \ipa{e} / _C\ipa{i} (sporadic)\\
\ipa{a}(C)V[+ high] \ipa{a}(C)V[+ mid] \ipa{a}C\ipa{a} \change\ \ipa{O}(C) \ipa{a}(C) \{\ipa{O,a}\}(C)\ipa{@}\\
\ipa{e}(C)\{V[- low]\} \ipa{e}(C)\ipa{a} \change\ \ipa{e}(C) \ipa{e}(C)\ipa{@}\\
\ipa{i}(C)V[+ high] \ipa{i}(C)V[+ mid] \ipa{i}(C)\ipa{a} \change\ \ipa{i}(C) \ipa{i}(C)\ipa{@} \{\ipa{e,i}\}(C)\ipa{@}\\
\ipa{o}(C)V[+ high] \ipa{o}(C)V[+ mid] \ipa{o}(C)\ipa{a} \change\ \ipa{8}(C) \ipa{o}(C) \ipa{O}(C)\ipa{@}\\
\ipa{u}(C)V[+ high] \ipa{u}(C)\ipa{e u}(C)\ipa{o u}(C)\ipa{a} \change\ \{\ipa{u,i}\}(C) \ipa{u}(C)\ipa{@ e}(C)\ipa{@} \{\ipa{u,8}\}(C)\ipa{@}\\
\ipa{u} \change\ \ipa{0} / ! C\ipa{w}_\\
\{\ipa{e,i}\} \change\ \ipa{I} (sporadic)\\
V$_0$V$_0$ \change\ V$_0$\\
``[W]hen pretonic u was lost, its labialness was usually absorbed onto the previous consonant''

\subsubsection{Proto-Oceanic to Lemerig}{\it thetha}, from Fran\c{c}ois, Alexander (2005), ``Unraveling the history of the vowels of seventeen north Vanuatu languages''

\ipa{p p\super w k q} \change\ \ipa{B w G} \O\\
VV \change\ V\\
\ipa{ndr} *R \change\ \ipa{d r}\\
\{\ipa{c},*j\} \ipa{\textltailn} \change\ \ipa{s n}\\
\ipa{t} \change\ \ipa{P} ``often''\\
\ipa{b b\super w d g} \change\ \ipa{p kp\super w t k}\\
\ipa{m\super w} \change\ \ipa{Nm\super w}\\
thetha says ``intervening consonants sometimes optional in the [following] sound changes'':\\
--- \ipa{i}CV[- high] \change\ \ipa{a}C\\
--- \ipa{e}CV[+ mid] \change\ \ipa{E}C\\
--- \ipa{e}CV[+ low] \change\ \ipa{a}C\\
--- \ipa{a}CV[+ high] \change\ \{\ipa{E,\oe}\}C\\
--- \ipa{a}C\ipa{a} \change\ \{\ipa{9,a}\}C\\
--- \ipa{o}CV[+ high] \change\ \ipa{\o}C\\
--- \ipa{o}C\ipa{o} \change\ \ipa{\oe}C (sporadic)\\
--- \ipa{o}CV[- high] \change\ \ipa{O}C\\
--- \ipa{u}CV[- high] \change\ \ipa{o}C\\
\ipa{o e} \change\ \ipa{U I}\\
\ipa{ia} \change\ \ipa{I} ``(only somoetimes?)''\\
V \change\ \O\ / \#_C"V\\
V \change\ \O\ / CVC_C"V\\
CV$_1$C"V$_2$ \change\ CV$_2$C"V$_2$

\subsubsection{Proto-Oceanic to Mwotlap}{\it thetha}, from Fran\c{c}ois, Alexander (2005), ``Unraveling the history of the vowels of seventeen north Vanuatu languages''

\ipa{q} \change\ \O\\
V$_0$V$_0$ \change\ V$_0$\\
\ipa{ndr} \change\ \ipa{d}\\
R \change\ \ipa{r}\\
\ipa{d} \change\ \ipa{r} (sporadic)\\
\ipa{p p\super w b\super w k g} \change\ \ipa{B w kp\super w G k}\\
\ipa{m\super w \textltailn} \change\ \ipa{Nm\super w n}\\
\{\ipa{c},*j\} \change\ \ipa{s}\\
\ipa{s} \change\ \ipa{h} ``(often)''\\
\ipa{r} \change\ \ipa{j}\\
\ipa{o e i} \change\ \ipa{O E I} / C_V[- high]\\
V[- high] \change\ \O\ / \{\ipa{O,E,I}\}C_\\
\ipa{o}CV[+ high] / \ipa{I}C (sporadic)\\
\ipa{u}C\ipa{i} \change\ \ipa{i}C (sporadic)\\
\ipa{u a} \change\ \ipa{U E} / _CV[+ high]\\
V[+ high] / \{\ipa{U,E}\}C_\\
\ipa{o e} \change\ \ipa{U I}\\
\ipa{a}CV[+ high] \change\ \ipa{I} / when stressed unless primarily stressed\\
(C)V$_1$C"V$_2$ \change\ (C)V$_2$V$_2$\\
V$_1$ \change\ \O\ / _"V$_2$\\
V \change\ \O\ / (C)V_VC"V\\
\ipa{b B d} \change\ \ipa{m p n} / _\{C,\#\}\\
\ipa{kp\super w} \change\ \ipa{k} / _C (sporadic)\\
*\ipa{u} *\ipa{o} ``sometimes offload their labialization onto the previous labial consonant'' when they change to something else

\subsubsection{Proto-Oceanic to Proto-New Caledonia}{\it thetha}, from Ozanne-Rivierre, Fran\c{c}oise (1992), ``The Proto-Oceanic Consonantal System and the Languages of New Caledonia''. {\it Oceanic Linguistics} 31(2):191 -- 207; and Ozanne-Rivierre, Fran\c{c}oise (1995), ``Structural Changes in the Languages of Northern New Caledonia''. {\it Oceanic Linguistics} 34(1):44 -- 72

\ipa{c} \change\ \ipa{s}\\
\{\ipa{l,\textltailn}\} \change\ \ipa{n}\\
\ipa{*}R \change\ \O\\
\ipa{r} \change\ \ipa{\:t}\\
V \change\ \O\ / _C"V\\
NS \change\ \ipa{\super n}S\\
CC \change\ C\ipa{:} (fortis)

\paragraph{Proto-New Caledonia to Caa\`{a}c}{\it thetha}, from Ozanne-Rivierre, Fran\c{c}oise (1992), ``The Proto-Oceanic Consonantal System and the Languages of New Caledonia''. {\it Oceanic Linguistics} 31(2):191 -- 207; and Ozanne-Rivierre, Fran\c{c}oise (1995), ``Structural Changes in the Languages of Northern New Caledonia''. {\it Oceanic Linguistics} 34(1):44 -- 72

\ipa{p: pw: t: \:t: q: k:} \change\ \ipa{p\super h pw\super h c\super h t\super h h j\super h}\\
\ipa{q \{k,t\} \{s,\:t\}} \change\ \ipa{k c t}\\
\ipa{k} \change\ \O\ / _\{\ipa{o,a}\}\\
N \change\ \O\ / _\#\\
V \change\ \O\ / \#(C)V(C)(C)_\#\\
\ipa{j\super h} \change\ \ipa{h} / _\ipa{i}\\
\{\ipa{p,pw,k}\} \change\ \O\ / V_V\\
\ipa{t c} \change\ \ipa{l j} / V_V\\
V\ipa{n}V \change\ \~{V}\ipa{l}\~{V}\\
\ipa{\super n}S \change\ N / _\#\\
\ipa{u} \change\ \ipa{i} (typical)\\
\ipa{u i} \change\ \ipa{o e} (not always)

\paragraph{Proto-New Caledonia to Jaw\'{e}}{\it thetha}, from Ozanne-Rivierre, Fran\c{c}oise (1992), ``The Proto-Oceanic Consonantal System and the Languages of New Caledonia''. {\it Oceanic Linguistics} 31(2):191 -- 207; and Ozanne-Rivierre, Fran\c{c}oise (1995), ``Structural Changes in the Languages of Northern New Caledonia''. {\it Oceanic Linguistics} 34(1):44 -- 72

\ipa{q k \{\:t,s\}} \change\ \ipa{k c \|[t}\\
\ipa{p: pw: \|[t: t: k: c:} \change\ \ipa{p\super h h\super w \|[t\super h t\super h h j\super h}\\
\ipa{\|[t} \change\ \ipa{l} / V_V\\
C\ipa{:} \change\ C\ipa{\super h}\\
\ipa{\|[t \|[t\super h t t\super h} \change\ \ipa{t t\super h c s}\\
\ipa{j\super h} \change\ \ipa{h} / _\ipa{i}

\paragraph{Proto-New Caledonia to Nemi-Pije-Fwai}{\it thetha}, from Ozanne-Rivierre, Fran\c{c}oise (1992), ``The Proto-Oceanic Consonantal System and the Languages of New Caledonia''. {\it Oceanic Linguistics} 31(2):191 -- 207; and Ozanne-Rivierre, Fran\c{c}oise (1995), ``Structural Changes in the Languages of Northern New Caledonia''. {\it Oceanic Linguistics} 34(1):44 -- 72

\ipa{q k \{\:t,s\}} \change\ \ipa{k c \|[t}\\
\ipa{p: pw: t: \|[t: k: c:} \change\ \ipa{f h\super w \|[t\super h t\super h h j\super h}\\
\ipa{\|[t} \change\ \ipa{l} / V_V\\
C\ipa{:} \change\ C\ipa{\super h}\\
\ipa{t t\super h} \change\ \ipa{c h} / _E\\
\ipa{\|[t \|[t\super h} \change\ \ipa{t t\super h}\\
\ipa{j\super h} \change\ \ipa{h} / _\ipa{i}\\
\ipa{bw mw} \change\ \ipa{g N}\\
\ipa{n \r*n} \change\ \ipa{\textltailn\ \r{\textltailn}} / _E\\
ONV \change\ S\ipa{\super h}\~{V} / Pije and Fwai\\
\ipa{f} \change\ \ipa{F} / Pije and Fwai

\paragraph{Proto-New Caledonia to Proto-Northern}{\it thetha}, from Ozanne-Rivierre, Fran\c{c}oise (1992), ``The Proto-Oceanic Consonantal System and the Languages of New Caledonia''. {\it Oceanic Linguistics} 31(2):191 -- 207; and Ozanne-Rivierre, Fran\c{c}oise (1995), ``Structural Changes in the Languages of Northern New Caledonia''. {\it Oceanic Linguistics} 34(1):44 -- 72

\ipa{q q: k k: s s:} \change\ \ipa{k k: c c: \|[t \|[t:}\\
C\ipa{:} \change\ C\ipa{\super h}\\
Velars were in the process of palatalizing\\
C \change\ \O\ / _\$(possessive suffix)\#\\
\O\ \change\ \ipa{j} / \#_\ipa{a}

\subparagraph{Proto-Northern to Nixumwak-N\^{e}l\^{e}mwa}{\it thetha}, from Ozanne-Rivierre, Fran\c{c}oise (1992), ``The Proto-Oceanic Consonantal System and the Languages of New Caledonia''. {\it Oceanic Linguistics} 31(2):191 -- 207; and Ozanne-Rivierre, Fran\c{c}oise (1995), ``Structural Changes in the Languages of Northern New Caledonia''. {\it Oceanic Linguistics} 34(1):44 -- 72

V \change\ \O\ / _\#, often\\
\ipa{k} \change\ \ipa{c} / V_V\\
\ipa{k} \change\ \ipa{c} / _\#\\
\ipa{t} \change\ \ipa{k}\\
\ipa{\|[t} \change\ \ipa{t}\\
\ipa{k} \change\ \O\ / _\{\ipa{o,a}\}\\
\ipa{c\super h} \change\ \{\ipa{S,j\super h}\}\\
\ipa{k\super h} \change\ \ipa{h} / _\ipa{a}\\
\ipa{pw p t \:t k c} \change\ \ipa{(v)w v r l G j} / V_V\\
\ipa{pw} \change\ \ipa{w}\\
\ipa{\:t} \change\ \ipa{t}\\
V\ipa{n}V \change\ \~{V}\ipa{l}\~{V}\\
\ipa{\super n}S \change\ N / _\#\\
SN \change\ N[- voice]\\
\ipa{pw\super h p\super h t\super h k\super h} \change\ \ipa{fw f r\super h x} / in Nelemwa\\
\ipa{u}C\ipa{u} \change\ \ipa{i}C\ipa{i}\\
V[+ mid] \change\ \ipa{a} / near nasals?\\
\ipa{u i} \change\ \ipa{o e} / ``in monosyllabic forms almost always''\\
"V \change\ V\ipa{:} (usually)\\
/\ipa{1 @}/ gained\\
\ipa{N} \change\ \ipa{n}

\paragraph{Proto-New Caledonia to Nyel\^{a}yu}{\it thetha}, from Ozanne-Rivierre, Fran\c{c}oise (1992), ``The Proto-Oceanic Consonantal System and the Languages of New Caledonia''. {\it Oceanic Linguistics} 31(2):191 -- 207; and Ozanne-Rivierre, Fran\c{c}oise (1995), ``Structural Changes in the Languages of Northern New Caledonia''. {\it Oceanic Linguistics} 34(1):44 -- 72

V \change\ \O\ / \#(C)V(C)(C)_\#\\
C \change\ \O\ / _\# ``sometimes''\\
\ipa{k} \change\ \ipa{c}\\
\ipa{k t s} \change\ \ipa{j c \|[t}\\
\ipa{p pw \|[t \:t c} \change\ \ipa{v (v)w r l j} / V_V\\
\ipa{\|[t \:t} \change\ \ipa{r l} / _\#\\
\ipa{pw} \change\ \ipa{w} (sporadic)\\
\ipa{\{\|[t,\:t\}} \change\ \ipa{t}\\
\ipa{\;Nq} \change\ \ipa{Nk}\\
\ipa{p: pw: t: q: c:} \change\ \ipa{p\super h pw\super h t\super h h c\super h}\\
\ipa{q} \change\ \O\\
\ipa{j\super h j} \change\ \ipa{h} \O\ / _\ipa{i}\\
\ipa{w} \change\ \ipa{G} (sporadic, conditioning unknown)\\
\ipa{\super n}S \change\ N / _\#\\
SN \change\ N[- voice]\\
V\ipa{n}V \change\ \~{V}\ipa{l}\~{V} \\
V \change\ \~{V} / _N\\
\ipa{\super n}S \change\ N / _\~{V}\\
\ipa{u} \change\ \ipa{i} ``often''\\
\ipa{u i} \change\ \ipa{o e} (not always)\\
\ipa{N} \change\ \ipa{n}

\paragraph{Proto-New Caledonia to Pwaamei}{\it thetha}, from Ozanne-Rivierre, Fran\c{c}oise (1992), ``The Proto-Oceanic Consonantal System and the Languages of New Caledonia''. {\it Oceanic Linguistics} 31(2):191 -- 207; and Ozanne-Rivierre, Fran\c{c}oise (1995), ``Structural Changes in the Languages of Northern New Caledonia''. {\it Oceanic Linguistics} 34(1):44 -- 72

V \change\ \O\ / _\# (sporadic?)\\
\ipa{q k t s} \change\ \ipa{k j c t}\\
\ipa{p: pw: t: \:t k: c:} \change\ \ipa{f h\super w t\super h l\super h h s}\\
\ipa{\:t} \change\ \ipa{l} / V_V\\
\ipa{k c} \change\ \O\ \{\ipa{j},\O\} / V_V\\
C\ipa{:} \change\ C\ipa{\super h}\\
\ipa{j\super h j} \change\ \ipa{s z}\\
\ipa{s} \change\ \ipa{h} / _\ipa{i}\\
\ipa{bw mw} \change\ \ipa{g N}

\paragraph{Proto-New Caledonia to Pwapw\^{a}}{\it thetha}, from Ozanne-Rivierre, Fran\c{c}oise (1992), ``The Proto-Oceanic Consonantal System and the Languages of New Caledonia''. {\it Oceanic Linguistics} 31(2):191 -- 207; and Ozanne-Rivierre, Fran\c{c}oise (1995), ``Structural Changes in the Languages of Northern New Caledonia''. {\it Oceanic Linguistics} 34(1):44 -- 72

V \change\ \O\ / _\# (sporadic?)\\
\ipa{q k \{\:t,s\}} \change\ \ipa{k c \|[t}\\
\ipa{p: pw: \|[t: t: k: c:} \change\ \ipa{p\super h x\super w \|[t\super h t\super h x s}\\
\ipa{\|[t(\super h) t(\super h)} \change\ \ipa{c\super h t\super h}\\
C\ipa{:} \change\ C\ipa{\super h}\\
\ipa{p pw t \{k,c\}} \change\ \{\ipa{v},\O\} \ipa{w l} \O\ / V_V\\
\ipa{j} \change\ \ipa{z}\\
\ipa{bw mw} \change\ \ipa{gw Nw} (\change\ \ipa{g N} / _V[+ rounded])

\paragraph{Proto-New Caledonia to Proto-Yunaga}{\it thetha}, from Ozanne-Rivierre, Fran\c{c}oise (1992), ``The Proto-Oceanic Consonantal System and the Languages of New Caledonia''. {\it Oceanic Linguistics} 31(2):191 -- 207; and Ozanne-Rivierre, Fran\c{c}oise (1995), ``Structural Changes in the Languages of Northern New Caledonia''. {\it Oceanic Linguistics} 34(1):44 -- 72

\ipa{q k s} \change\ \ipa{k c \|[t}\\
\ipa{p: pw: \|[t: t: \:t: c: k:} \change\ \ipa{p\super h pw\super h \|[t\super h t\super h \:t\super h j\super h h}\\
C\ipa{:} \change\ C\ipa{\super h}\\
\ipa{k} \change\ \O\ / _\{\ipa{o,a}\}\\
\ipa{t(\super h)} \change\ \ipa{k(\super h)}\\
\ipa{j\super h} \change\ \ipa{h} / _\ipa{i}\\
\ipa{p pw \|[t \:t k c} \change\ \ipa{v w D l} \O\ \ipa{j} / V_V\\
V \change\ \O\ / _\#\\
SN \change\ N[- voice]\\
\ipa{\super n}S \change\ N / _\#\\
\ipa{u i} \change\ \ipa{o e} / in monosyllables\\
\ipa{au ai} \change\ \ipa{O E}\\
\ipa{o} \change\ \ipa{O} ``sometimes''\\
\ipa{a} \change\ \{\ipa{E,e}\} ``in some words"\\
\ipa{N} \change\ \ipa{n}

\subparagraph{Proto-Yunaga to Yunaga 1}{\it thetha}, from Ozanne-Rivierre, Fran\c{c}oise (1992), ``The Proto-Oceanic Consonantal System and the Languages of New Caledonia''. {\it Oceanic Linguistics} 31(2):191 -- 207; and Ozanne-Rivierre, Fran\c{c}oise (1995), ``Structural Changes in the Languages of Northern New Caledonia''. {\it Oceanic Linguistics} 34(1):44 -- 72

V \change\ \~{V} / _N\#\\
C \change\ \O\ / _\#\\
\ipa{j\super h} \change\ \ipa{T}\\
\ipa{j} \change\ \{\ipa{D,z}\} ?

\subparagraph{Proto-Yunaga to Yunaga 2}{\it thetha}, from Ozanne-Rivierre, Fran\c{c}oise (1992), ``The Proto-Oceanic Consonantal System and the Languages of New Caledonia''. {\it Oceanic Linguistics} 31(2):191 -- 207; and Ozanne-Rivierre, Fran\c{c}oise (1995), ``Structural Changes in the Languages of Northern New Caledonia''. {\it Oceanic Linguistics} 34(1):44 -- 72

\{\ipa{\|[t,\:t}\} \change\ \ipa{t}\\
\ipa{D} \change\ \ipa{l}

\subsubsection{Proto-Oceanic to Proto-Reefs/Santa Cruz}{\it Pogostick Man}, from Ross, Malcolm, and \AA shlid N\ae ss (2007), ``An Oceanic Origin for \"{A}iwoo, the Language of the Reef Islands?". {\it Oceanic Linguistics} 46(II):456 -- 498

\ipa{p} \change\ \O\ / _B\\
\ipa{p} \change\ \ipa{v}\\
\ipa{r} \change\ \O\ / \ipa{d}_\\
\ipa{r} \change\ \ipa{l}\\
C \change\ \O\ / _\#

\paragraph{Proto-Reefs/Santa Cruz to \"{A}iwoo}{\it Pogostick Man}, from Ross, Malcolm, and \AA shlid N\ae ss (2007), ``An Oceanic Origin for \"{A}iwoo, the Language of the Reef Islands?". {\it Oceanic Linguistics} 46(II):456 -- 498

\{\ipa{t,k}\} \change\ \O\ / V_V\\
\ipa{k} \change\ \{\ipa{k},\O\} / \#_\\
\ipa{q} \change\ \{\ipa{k},\O\}\\
\ipa{*}R \change\ \ipa{l}

\paragraph{Proto-Reefs/Santa Cruz to Nagu}{\it Pogostick Man}, from Ross, Malcolm, and \AA shlid N\ae ss (2007), ``An Oceanic Origin for \"{A}iwoo, the Language of the Reef Islands?". {\it Oceanic Linguistics} 46(II):456 -- 498

\ipa{m\super w} \change\ \ipa{m}\\
\ipa{t} \change\ \ipa{l} / V_V\\
\ipa{k} \change\ \{\ipa{k},\O\} / \#_\\
\ipa{N} \change\ \ipa{n} / _\ipa{i}\\
\ipa{q} *R \change\ \O\ \{\ipa{l},\O\}

\paragraph{Proto-Reefs/Santa Cruz to Nat\"{u}gu}{\it Pogostick Man}, from Ross, Malcolm, and \AA shlid N\ae ss (2007), ``An Oceanic Origin for \"{A}iwoo, the Language of the Reef Islands?". {\it Oceanic Linguistics} 46(II):456 -- 498

\ipa{m\super w} \change\ \ipa{m}\\
\ipa{t} \change\ \{\ipa{t,l}\} / \#_\\
\ipa{t} \change\ \ipa{l} / _\{\ipa{u,i}\}\\
\ipa{t k} \change\ \ipa{l} \O\ / V_V\\
\ipa{r} \change\ \ipa{l} / _\{\ipa{u},\#\}\\
\ipa{N} \change\ \ipa{n} / _\ipa{i}\\
\ipa{q} *R \change\ \O\ \{\ipa{l},\O\}

\subsubsection{Proto-Oceanic to Shark Bay}{\it thetha}, from Guy, Jacques (1978), ``Proto-North New Hebridean Reconstructions''

C \change\ \O\ / _\#\\
\ipa{q} \change\ \O\\
\ipa{*}R \change\ \{\O,\ipa{r}\}\\
\ipa{\textltailn\ c} *j \change\ \ipa{n s z}\\
\ipa{p p\super w k} \change\ \ipa{v v\super w G}\\
\ipa{b b\super w g} \change\ \ipa{p p\super w k}\\
V[+ high] \change\ \O\ / _\# ! \{\ipa{p,z,d(r)}\}_\\
\ipa{v t l r} \change\ \ipa{p dr n w} / _\#\\
\ipa{t} \change\ \ipa{ts} / _V[+ high]\\
"\ipa{a} \change\ \ipa{i} / _CV[- high]\\
"\ipa{a} \change\ \ipa{e} / _CV[+ high]\\
\ipa{p N} \change\ \ipa{f} \O\ / "V_V\\
\ipa{G s d} \change\ \O\ \{\ipa{j dr}\} / _\#\\
\ipa{G s d} \change\ \O\ \{\ipa{j dr}\} / ``before a post-tonic vowel''\\
V \change\ \ipa{e} / C_\# ! C = \ipa{j}\\
V \change\ \O\ / \{"V,\ipa{j}\}_\\
\ipa{p v m} \change\ \ipa{t T n} / _\{\ipa{a,e,i}\}\\
\ipa{z} \change\ \ipa{s}\\
\O\ \change\ \ipa{h} / \#_V (``it isn't clear if this happens unconditionally'')\\
An /\ipa{o O}/ distinction is gained somehow

\subsubsection{Proto-Oceanic to Tolomako}{\it thetha}, from Lynch, John (2005), ``The Apicolabial Shift in Nese''. {\it Oceanic Linguistics} 44(2):389 -- 403; and \url{http://language.psy.auckland.ac.nz/austronesian/}

C \change\ \O\ / _\#\\
\ipa{q} \change\ \O\\
\ipa{*}R \change\ \{\O,\ipa{r}\} (the former seems more common)\\
\ipa{\textltailn\ c} *j \change\ \ipa{n s z}\\
\ipa{m b} \change\ \ipa{\|m{n} \|m{t}} \change\ \ipa{n t} / _\{\ipa{a,e,i}\}\\
\ipa{p(\super w) k} \change\ \ipa{v(\super w) G}\\
\ipa{m\super w b(\super w) v\super w} \change\ \ipa{m p b}\\
\ipa{d g} \change\ \ipa{r k}\\
\{\ipa{z,dr}\} \change\ \ipa{ts}\\
\ipa{u} \change\ \ipa{i} (``sporadic'')\\
\ipa{a} \change\ \ipa{e} (rare?)

\subsubsection{Proto-Oceanic to Proto-Utupua}{\it Pogostick Man}, from Ross, Malcolm, and \AA shlid N\ae ss (2007), ``An Oceanic Origin for \"{A}iwoo, the Language of the Reef Islands?". {\it Oceanic Linguistics} 46(II):456 -- 498

\ipa{p q} \change\ \ipa{v} \O\ (in general, seems like there was something going on with conditioning in the case of *\ipa{p}?)\\
\ipa{w} \change\ \O\ (? Tanibili [\ipa{w}] may just be phonetically determined)\\
C \change\ \O\ / _\# (except for *\ipa{k}?)

\paragraph{Proto-Utupua to Asuboa}{\it Pogostick Man}, from Ross, Malcolm, and \AA shlid N\ae ss (2007), ``An Oceanic Origin for \"{A}iwoo, the Language of the Reef Islands?". {\it Oceanic Linguistics} 46(II):456 -- 498

PU *\ipa{p} had occasional reflexes of \ipa{p} or \O\\
\ipa{p} \change\ \ipa{w} / _B\\
\ipa{p\super w m\super w} \change\ \ipa{w m}\\
\ipa{dr s l} \change\ \ipa{\{d,\textbardotlessj\}} \{\O,\ipa{s}\} \{\ipa{n},\O\}\\
\ipa{c \textltailn} \change\ \O\ \{\ipa{\textltailn,j}\}\\
\ipa{t r l} \change\ \{\ipa{j,s}\} \{\ipa{j},\O\} \ipa{j} / _\ipa{u}\\
\ipa{t} \change\ \ipa{s} / _\ipa{i}\\
\ipa{k} \change\ \{\ipa{k},\O\} / \#_\\
\ipa{k} \change\ \{\O,\ipa{s}\} / _\#\\
\ipa{r} \change\ \{\ipa{l,n,}\O\}\\
*R \change\ \{\ipa{l},\O\}

\paragraph{Proto-Utupua to Nebao}{\it Pogostick Man}, from Ross, Malcolm, and \AA shlid N\ae ss (2007), ``An Oceanic Origin for \"{A}iwoo, the Language of the Reef Islands?". {\it Oceanic Linguistics} 46(II):456 -- 498

PU *\ipa{p} had occasional \ipa{h} or \O\ reflexes\\
\ipa{p\super w} \change\ \ipa{v\super w}\\
\ipa{t k} \change\ \ipa{r} \O\ / \#_ (though *\ipa{t} seems to have occasionally survived?)\\
\ipa{t} \change\ \{\ipa{r,t}\} / _B\\
\ipa{t} \change\ \{\ipa{r,t}\} / V_V\\
\ipa{r} \change\ \{\ipa{l},\O\} / _\ipa{u}\\
\ipa{r} *R \change\ \ipa{l} \O\\
\ipa{l} \change\ \O\ (occasionally?)\\
\ipa{c \textltailn} \change\ \O\ \ipa{n}\\
\ipa{N} \change\ \ipa{n} / _\ipa{i}

\paragraph{Proto-Utupua to Tanibili}{\it Pogostick Man}, from Ross, Malcolm, and \AA shlid N\ae ss (2007), ``An Oceanic Origin for \"{A}iwoo, the Language of the Reef Islands?". {\it Oceanic Linguistics} 46(II):456 -- 498

PU *\ipa{p} seems to have remained; PU *\ipa{w} is listed as having both \O\ and \ipa{w} as reflexes although the latter may just be an epenthetic glide between vowels of unlike rounding\\
\{\ipa{s},*R\} \change\ \O\\
\ipa{p t \{r,l\}} \change\ \O\ \ipa{s j} / _\ipa{u}\\
\ipa{p\super w bw} \change\ \ipa{p b}\\
\ipa{t} \change\ \{\ipa{t,r,k\super w}\} / \#_ (I'm not kidding. That's what's listed as the reflexes.)\\
\ipa{k} \change\ \{\ipa{k},\O\} / \#_\\
\ipa{t k} \change\ \{\ipa{t,r,k\super w},\O\} \O\ / V_V\\
\ipa{k} \change\ \{\O,\ipa{j}\} / _\#\\
\ipa{dr} \change\ \ipa{\textbardotlessj} / _\ipa{i}\\
\ipa{d c \textltailn} \change\ \ipa{\textbardotlessj} \{\ipa{s},\O\} \ipa{n}\\
\ipa{\{r,l\}} \change\ \ipa{l} (occasionally \change\ \O?)

\subsubsection{Proto-Oceanic to Proto-Vanikoro}{\it Pogostick Man}, from Ross, Malcolm, and \AA shlid N\ae ss (2007), ``An Oceanic Origin for \"{A}iwoo, the Language of the Reef Islands?". {\it Oceanic Linguistics} 46(II):456 -- 498

\ipa{p} \change\ \O\ / _\ipa{u}\\
\ipa{p q} \change\ \{\ipa{v,p}\} \O\\
\ipa{r} \change\ \O\ / \ipa{d}_\\
\ipa{k} \change\ \O\ / V_V\\
\{\ipa{s},*R\} \change\ \ipa{r} / _\#

\paragraph{Proto-Vanikoro to Buma}{\it Pogostick Man}, from Ross, Malcolm, and \AA shlid N\ae ss (2007), ``An Oceanic Origin for \"{A}iwoo, the Language of the Reef Islands?". {\it Oceanic Linguistics} 46(II):456 -- 498

\ipa{p t} \change\ \O\ \{\ipa{s,k}\} / _\ipa{u}\\
\ipa{p} \change\ \O\ / _\#\\
\ipa{k} \change\ \O\ / \#_\\
\ipa{r} \change\ \ipa{l} / ! _\#\\
\ipa{p\super w bw m\super w} \change\ \ipa{p b m}\\
\ipa{c} *R \change\ \O\ \{\ipa{l},\O\}\\
\ipa{N} \change\ \{\ipa{N,g}\} (\ipa{N} remains when _\ipa{i})

\paragraph{Proto-Vanikoro to Tanema}{\it Pogostick Man}, from Ross, Malcolm, and \AA shlid N\ae ss (2007), ``An Oceanic Origin for \"{A}iwoo, the Language of the Reef Islands?". {\it Oceanic Linguistics} 46(II):456 -- 498

\ipa{p} \change\ \ipa{v} / _\#\\
\ipa{p\super w w} \change\ \ipa{b} \O\\
\ipa{t} \change\ \{\O,\ipa{t}\} / \#_\\
\ipa{t} \change\ \ipa{s} / _\ipa{u}\\
\ipa{r} \change\ \ipa{l} / ! _\#\\
\ipa{c} *R \change\ \O\ \{\ipa{l},\O\}\\
\ipa{s} \change\ \{\ipa{s,d,c}\}

\paragraph{Proto-Vanikoro to Vano}{\it Pogostick Man}, from Ross, Malcolm, and \AA shlid N\ae ss (2007), ``An Oceanic Origin for \"{A}iwoo, the Language of the Reef Islands?". {\it Oceanic Linguistics} 46(II):456 -- 498

\ipa{w} \change\ \{\O,\ipa{w}\}\\
\ipa{p\super w m\super w} \change\ \ipa{p m}\\
\ipa{t} \change\ \ipa{l} / \#_, in nouns\\
\ipa{t} \change\ \ipa{s} / _\{\ipa{u,i}\}\\
\ipa{t} \change\ \ipa{l} / V_V\\
\ipa{r} \change\ \ipa{l} / ! _\{\ipa{u},\#\}\\
\ipa{s} \change\ \ipa{r} / _\#\\
\ipa{c} *R \change\ \{\ipa{j},\O\} \ipa{l}

\subsubsection{Proto-Oceanic to Proto-Southern Vanuatu}{\it thetha}, from Lynch, John (2001), {\it The Linguistic History of Southern Vanuatu}

\ipa{m b} \change\ \ipa{m\super w b\super w} / _\ipa{u}\\
\ipa{p} \change\ \ipa{b(\super w)} (sporadic)\\
\ipa{p} \change\ \ipa{v\super w} / _\ipa{u} (a change thetha reconstructs in order to account for phenomena in later posts about this group of languages)\\
\ipa{p} \change\ \ipa{v}\\
\ipa{k} *R \change\ \ipa{G r} ``(frequently)''\\
\ipa{*}R \change\ \O\\
\ipa{dr} \change\ \{\ipa{d,r}\}\\
\ipa{\textltailn} \change\ \ipa{j}\\
\ipa{n} \change\ \ipa{N} / \ipa{q}V[- stress]_\\
\ipa{n} \change\ \ipa{N} / _V[- stress]\ipa{q}\\
\ipa{c} \change\ \ipa{s}\\
\ipa{t} \change\ \ipa{c} / _E\\
\ipa{q} (\change\ \ipa{kw} ?) \change\ \ipa{v} (rare)\\
\ipa{a} \change\ \ipa{e} / _(C)\ipa{i}\\
\ipa{a} \change\ \ipa{@} / _C\ipa{a}

\paragraph{Proto-Southern Vanuatu to Anejom}{\it thetha}, from Lynch, John (2001), {\it The Linguistic History of Southern Vanuatu}

C \change\ \O\ / _\# ! C = \ipa{t}\\
\ipa{t} \change\ \ipa{s} / _\#\\
\ipa{v(\super w)} \change\ \ipa{h}\\
\ipa{k} \change\ \O\ / V_V (sporadic?)\\
\ipa{b(\super w) g} \change\ \ipa{p(\super w) k}\\
\ipa{s} \change\ \ipa{h} / ``rarely''\\
\ipa{s} \change\ \ipa{T} / ! _\ipa{i}, occasionally\\
\ipa{d} \{\ipa{c},*j\} \change\ \ipa{tS s}\\
\ipa{\{n,N\}} \change\ \ipa{\textltailn} / _E\\
\ipa{w} \change\ \ipa{v}\\
\ipa{l} \change\ \ipa{tS} / _\{\ipa{o},E\}\\
\ipa{q} \change\ \O\\
V \change\ \O\ / _\# (with very few exceptions)\\
\{\ipa{r,h}\} \change\ \O\ / _\#\\
``a lot of word medial vowels get elided, sometimes even when they should be stressed''\\
\{\ipa{i,o}\} \change\ \ipa{e}\\
\ipa{u} \change\ \ipa{o}\\
\ipa{i} \change\ \ipa{o} / \{\ipa{u,w}\}_\\
\ipa{u} \change\ \ipa{e} / \{\ipa{T,G}\}_\\
\ipa{u} \change\ \ipa{e} / _\ipa{T}\\
\ipa{ai} \change\ \ipa{i} / _C\\
\ipa{ei} \change\ \ipa{i}\\
\ipa{ua} \change\ \ipa{ou}\\
\ipa{au} \change\ \{\ipa{u,o}\} ``sometimes''\\
\ipa{e} \change\ \ipa{i} / \'{K}_ ``[tendency]''\\
\ipa{e} \change\ \ipa{i} / _\'{K} ``[tendency]''\\
\ipa{a} \change\ \ipa{o} / P_ ``[tendency]''\\
\ipa{a} \change\ \ipa{o} / _P ``[tendency]''

\paragraph{Proto-Southern Vanuatu to Proto-Erromango}{\it thetha}, from Lynch, John (2001), {\it The Linguistic History of Southern Vanuatu}

\ipa{m\super w p\super w b\super w v\super w} \change\ \ipa{m p b v}\\
\ipa{v} \change\ \ipa{p} / \#_\\
\ipa{v} \change\ \ipa{f} / C[+ sibilant]\%\\
\ipa{v} \change\ \ipa{f} / \%C[+ sibilant]\\
\ipa{r} \change\ *L (some sort of lateral?) / occasionally\\
\ipa{s} \{\ipa{c},*j\} \change\ \ipa{h s}\\
\ipa{o} \change\ \ipa{a}\\
\ipa{u i} \change\ \ipa{o e} (sporadic)\\
\ipa{a} \change\ \ipa{i} / _CV[+ high]\\
``many word medial vowels lost''

\subparagraph{Proto-Erromango to Sye}{\it thetha}, from Lynch, John (2001), {\it The Linguistic History of Southern Vanuatu}

\ipa{*}L \change\ \ipa{r}\\
\ipa{b d g} \change\ \ipa{p t k} / \{\#,C\}_\\
\ipa{b d g} \change\ \ipa{m n N} / _\#\\
\ipa{b d g} \change\ \ipa{mp nt Nk}\\
\ipa{f} \change\ \ipa{p} / \#_\\
\ipa{f} \change\ \ipa{v} / V_V\\
\ipa{k} \change\ \ipa{G}\\
\ipa{G} \change\ \ipa{k} / _\ipa{i}\\
\ipa{s} \change\ \ipa{h} ``often''\\
\ipa{s} \change\ \O\ / _C (occasionally blocked)\\
\ipa{i} \change\ \ipa{e} / O[+ labial]_\\
\ipa{i} \change\ \ipa{e} / _O[+ labial]\\
\ipa{e} \change\ \ipa{o} / K_\\
\ipa{e} \change\ \ipa{o} / _K\\
\ipa{a} \change\ \ipa{o} / \{\ipa{w,m,N}\}_\\
\ipa{a} \change\ \ipa{e} / _\#\\
\ipa{@} \change\ \{\ipa{o,e}\}

\subparagraph{Proto-Erromango to Ura}{\it thetha}, from Lynch, John (2001), {\it The Linguistic History of Southern Vanuatu}

\ipa{h} \change\ \O\\
\ipa{*}L \change\ \ipa{l}\\
\ipa{nr} \change\ \ipa{d}\\
\ipa{b d g} \change\ \ipa{m n N} / _C\\
\ipa{b d g} \change\ \ipa{p t k} / _\#\\
\ipa{p} \change\ \ipa{b} / V_V\\
\ipa{u} \change\ \ipa{e} / \ipa{G}_\# (? this change is a bit unclear)\\
\ipa{G} \change\ \O\ / _\#\\
\ipa{k} \change\ \O\ (perhaps doesn't always happen but happens often)\\
\{\ipa{s,t}\} \change\ \ipa{h} / _\{\ipa{n,l,r}\}\\
\ipa{t} \change\ \ipa{r} / ! at word boundaries\\
\ipa{@} \change\ \ipa{i}

\paragraph{Proto-Southern Vanuatu to Proto-Tanna}{\it thetha}, from Lynch, John (2001), {\it The Linguistic History of Southern Vanuatu}

\{\ipa{v\super w,w}\} \change\ \ipa{k\super w}\\
\ipa{s c} *J \change\ \{\ipa{h,z}\} \{\ipa{s,z}\} \ipa{z}\\
\ipa{g q} \change\ \ipa{k} \O\\
\ipa{l} \change\ \ipa{r}\\
\ipa{o e} \change\ \ipa{\{u,@\} i}\\
\ipa{a} \change\ \ipa{o} / _\{P,C\ipa{u}\}\\
\ipa{a} \change\ \ipa{o} / P_\\
\ipa{a} \change\ \ipa{e} / _C\ipa{i}\\
\ipa{a} \change\ \ipa{@} / _C\ipa{a}\\
``vowels tend to lower near h''

\subparagraph{Proto-Tanna to Kwamera}{\it thetha}, from Lynch, John (2001), {\it The Linguistic History of Southern Vanuatu}

\ipa{t} \change\ \ipa{r}\\
\ipa{b(\super w) d} \change\ \ipa{p(\super w) t}\\
\ipa{G} \change\ \O\\
\ipa{s} \change\ \ipa{h} ``irregularly''\\
\{\ipa{p(\super w),v}\}V\ipa{h} \change\ \ipa{f}V\\
/\ipa{f\super w}/ gained\\
\ipa{u} \change\ \{\ipa{e,i}\} / _C\ipa{u}\\
\ipa{@} \change\ \ipa{a} / in U\#\\
\ipa{@} \change\ \ipa{e} / else

\subparagraph{Proto-Tanna to Lenakel}{\it thetha}, from Lynch, John (2001), {\it The Linguistic History of Southern Vanuatu}

\ipa{r} \change\ \{\ipa{l,i}\}\\
\ipa{t} \change\ \ipa{r}\\
\ipa{b(\super w) d} \change\ \ipa{p(\super w) t}\\
\ipa{k\super w} \change\ \{\ipa{w,u}\}\\
\ipa{G} \change\ \O\ / E_\\
\ipa{G} \change\ \O\ / _E\\
\ipa{G} \change\ \ipa{k}\\
\ipa{r} \change\ \ipa{l} / _V\ipa{l}\\
\ipa{c} *j \change\ \ipa{s \{z,s\}}\\
\ipa{z} \change\ \ipa{t}\\
\ipa{s} \change\ \ipa{h} ``irregularly''\\
\{\ipa{p(\super w),v}\}V\ipa{h} \change\ \ipa{f}V

\subparagraph{Proto-Tanna to North Tanna}{\it thetha}, from Lynch, John (2001), {\it The Linguistic History of Southern Vanuatu}

\ipa{k\super w} \change\ \ipa{p} / _\#\\
\ipa{k\super w} \change\ \O\ / _\ipa{u}\\
\ipa{k\super w} \change\ \O\ / _\ipa{a} (rare)\\
\ipa{k\super w} \change\ \{\ipa{w,u}\}\\
\ipa{v} \change\ \O\ / _\ipa{i}\\
\ipa{v} \change\ \{\ipa{w,u}\} (``sporadically'')\\
\ipa{G} \change\ \O\ / \#_\\
\ipa{G} \change\ \O\ / _E\\
\ipa{G} \change\ \ipa{N}\\
\ipa{r} \change\ \ipa{l} / _\{\ipa{o},E\}\\
\ipa{r} \change\ \ipa{i}\\
\ipa{d} \change\ \ipa{t} (often)\\
\ipa{d} \change\ \ipa{k} / _\ipa{N}\\
\ipa{z} \change\ \ipa{r}\\
\{\ipa{s,c}\} \change\ \{\ipa{h,s}\}

\subparagraph{Proto-Tanna to Southwest Tanna}{\it thetha}, from Lynch, John (2001), {\it The Linguistic History of Southern Vanuatu}

\ipa{t} \change\ \ipa{r}\\
\ipa{b(\super w) d} \change\ \ipa{p(\super w) t}\\
\ipa{G} \change\ \O\ / \#_\\
\ipa{G} \change\ \ipa{k}\\
\ipa{r} \change\ \ipa{l}\\
\{\ipa{c},*j\} \change\ \ipa{s}\\
\ipa{s} \change\ \ipa{h} ``irregularly''\\
\{\ipa{p(\super w),v}\}V\ipa{h} \change\ \ipa{f}V\\
\ipa{u} \change\ \{\ipa{e,i}\} / _C\ipa{u}\\
\ipa{@} \change\ \ipa{a} / in U\#

\subparagraph{Proto-Tanna to Whitesands}{\it thetha}, from Lynch, John (2001), {\it The Linguistic History of Southern Vanuatu}

\ipa{r} \change\ \{\ipa{l,i}\}\\
\ipa{d} \change\ \ipa{r} / _\ipa{N} (occasionally elsewhere as well)\\
\ipa{b(\super w) d} \change\ \ipa{p(\super w) t}\\
\ipa{k\super w} \change\ \O\ / \ipa{u}_\\
\ipa{k\super w} \change\ \O\ / _\ipa{u}\\
\ipa{k\super w} / \{\ipa{w,u}\}\\
\ipa{G} \change\ \O\ / \{\#,E\}_\\
\ipa{G} \change\ \O\ / _E\\
\ipa{G} \change\ \ipa{N}\\
\ipa{c} *j \change\ \ipa{s \{z,s\}}\\
\ipa{s} \change\ \ipa{h} ``often''\\
\ipa{z} \change\ \ipa{r}

\subsubsection{Proto-Oceanic to Vera'a}{\it thetha}, from Fran\c{c}ois, Alexander (2005), ``Unraveling the history of the vowels of seventeen north Vanuatu languages''

\ipa{p p\super w k q} \change\ \ipa{B w G} \O\\
VV \change\ V\\
\ipa{ndr} *R \change\ \ipa{d r}\\
\{\ipa{c},*j\} \ipa{\textltailn} \change\ \ipa{s n}\\
\ipa{t} \change\ \ipa{P} ``often''\\
\ipa{B} \change\ \ipa{f} / \#_ (usually)\\
\ipa{B} \change\ \ipa{f} / else (rarely)\\
\ipa{b\super w g} \change\ \ipa{kp\super w k}\\
\ipa{m\super w} \change\ \ipa{Nm\super w}\\
\ipa{i}(C)V[+ high] \change\ \ipa{i}(C)\\
\ipa{i}(C)V[- high] \change\ \ipa{i}(C)\ipa{I}\\
\ipa{e}(C)V[- low] \change\ \ipa{e}(C)\\
\ipa{e}(C)V[+ low] \change\ \ipa{E}C\ipa{E}\\
\ipa{a}(C)\ipa{i a}(C)\ipa{u} \change\ \{\ipa{a,E}\}(C) \{\ipa{O,a,E}\}(C)\\
\ipa{a}(C)\{\ipa{o,e}\} \change\ \ipa{a}(C)\\
\ipa{o}C\ipa{a} \change\ \ipa{O}C\ipa{O}\\
\ipa{u}CV[+ high] \change\ \ipa{i}C ``sometimes''\\
\ipa{u}CV[- high] \change\ \ipa{u}C\ipa{U}\\
\ipa{oa \{ae,ea\}} \change\ \ipa{uO iE}\\
\ipa{o e} \change\ \ipa{U I}\\
V \change\ \O\ / \#_C"V\\
V \change\ \O\ / CVC_C"V\\
CV$_1$C"V$_2$ \change\ CV$_2$C"V$_2$\\
\ipa{b d} \change\ \ipa{m n} / _\{C,\#\}

\subsection{Micronesian}

\subsubsection{Proto-Micronesian to Marshallese}{\it Ketsuban}, from Hale, Mark, {\it Historical Linguistics: Theory and Method}

\tab {\it NB: ``. . .the precise contrast between *s and *S, and *t and *T is unknown, as is the precise phonetic nature of *c and *Z." Blust (}v.s.{\it) rejected *T and *D; according to his reconstruction. . .*s and *S were /\ipa{\c{c} s}/, respectively. The substitution of /\ipa{ts}/ for *T is inferred from the Wikipedia article but may be incorrect.}

K \change\ K\super w / _\{C[+round],V[+round]\}\\
V[+high] \change\ \ipa{9} / _C[-high]\\
V \change\ \O\ / _\#\\
V[+mid] \change\ \ipa{9} / _C[+high] when stressed\\
V \change\ \O\ / "VC_\\
V \change\ \O\ / ``in CV reduplications"\\
\ipa{a} \change\ \ipa{3} / _\ipa{wo}\\
\ipa{f} \change\ \O\ / \#_\{C[-low],V[-low]\}\\
\ipa{f} \change\ \ipa{\textturnmrleg} / \#_\ipa{a}C\ipa{o}\\
\ipa{f} \change\ \ipa{j} / else\\
\O\ \change\ \ipa{\textturnmrleg} / \#_\ipa{a}C[-low]\\
\O\ \change\ \ipa{j} / \#_\{\ipa{a}C[+low],V[-back],C[-back]\}\\
\O\ \change\ \ipa{w} / \#_\{C[+round],V[+round]\}\\
\{\ipa{i,u}\} \{\ipa{e,o}\} \change\ \ipa{1 3}\\
\ipa{p} \{\ipa{t,ts}\} \ipa{c} \change\ \ipa{p\super j t\super j r\super j}\\
\{\ipa{\c{c},s}\} \ipa{x} \change\ \ipa{t\super W} \O\\
\ipa{m \textltailn} \change\ \ipa{m\super j n\super j}\\
``The author does not elaborate on the complex development of vowels without an onset consonant, other than to say that a glide is inserted (*Saa \textgreater\ t\ipa{\super W}a\ipa{\textturnmrleg}), nor does he go into more detail than to say that l and n generally develop into l\super j and n\super j before Proto-Micronesian front vowels, and l, r, and n turn into l\ipa{\super M}, r\ipa{\super M}, and n\ipa{\super M} before a and l\super w, r\super w, and n\super w before Proto-Micronesian round vowels, but the author does not elaborate."

\subsection{Proto-Austronesian to Proto-Ongan}{\it Pogostick Man}, from Blevins, Juliette (2007), ``A Long Lost Sister of Proto-Austronesian? Proto-Ongan, Mother of Jarawa and Onge of the Andaman Islands". {\it Oceanic Linguistics} 46(I):154 -- 198

\tab {\it NB: Blevins floats the idea that Proto-Ongan was a sister of rather than a daughter of Proto-Austronesian, but for reasons of simplicity in editing this document it is placed here.}

\ipa{b} \change\ \O\ / \#_\{\ipa{u,i}\}\\
\ipa{q} \change\ \O\ / \#_V\\
\ipa{q} \change\ \ipa{k}\\
\{\ipa{q\super w,ku,qu}\} \change\ \ipa{k\super w} (note that PAn might have had *\ipa{q\super w} *\ipa{k\super w} \change\ \ipa{q \{k,w\}} instead; may be a change from POn-PAn, if it existed)\\
\{\ipa{c},*C,\ipa{s},*S\} \change\ \ipa{c} (again, possibly a change from POn-PAn, if it existed)\\
S \change\ \O\ / _\#\\
\ipa{u a @} \change\ \ipa{\{u,o\} \{a,e\} e}\\
\ipa{\textbardotlessj\ g} *N *R \change\ \ipa{\{\textbardotlessj,j\} \{\textbardotlessj,g\} \{l,j\} \{l,r\}}\\
\ipa{z} \change\ \ipa{c} (again, possibly evidence of a change from POn-PAn, if it existed)\\
\ipa{h} \change\ \ipa{\{h,j,}\O\} (Blevins has marked what apparently is *\ipa{\textbardotlessj} but I'm assuming it's an error)\\
\ipa{e} \change\ \ipa{@} / _N when unstressed ! \'{K}_ (?; included here based on a comment earlier in the paper, but not listed on the correspondence list)\\
\{\ipa{m,\textltailn}\} \ipa{n} \change\ \{\ipa{\textltailn},\O\} \{\ipa{N},\O\} / _\# (first change marked ``in progress?")\\
\ipa{aj} \change\ \ipa{e}

\subsubsection{Proto-Ongan to Jarawa}{\it Pogostick Man}, from Blevins, Juliette (2007), ``A Long Lost Sister of Proto-Austronesian? Proto-Ongan, Mother of Jarawa and Onge of the Andaman Islands". {\it Oceanic Linguistics} 46(I):154 -- 198

\ipa{e} \change\ \ipa{@} / _N, when unstressed (?)\\
\ipa{n} \change\ \ipa{N} / _\# (?)\\
\ipa{k(\super w)} \change\ \ipa{h(\super w)}\\
\O\ \change\ \ipa{a} / \ipa{h}\#_ (that's not a typo; this happens across the word boundary)\\
\ipa{g} \change\ \ipa{j}\\
\ipa{e} \change\ \ipa{\{e,@,o\}} / _V\\
\ipa{e} \change\ \O\ / _\# (?)\\
\ipa{p} \change\ \ipa{b} / \#_ (change seems to be ongoing)\\
/\ipa{a e}/ reduce when unstressed (change seems to be ongoing?)

\subsubsection{Proto-Ongan to Onge}{\it Pogostick Man}, from Blevins, Juliette (2007), ``A Long Lost Sister of Proto-Austronesian? Proto-Ongan, Mother of Jarawa and Onge of the Andaman Islands". {\it Oceanic Linguistics} 46(I):154 -- 198

\ipa{e} \change\ \ipa{@} / _N, when unstressed (?)\\
\ipa{n} \change\ \ipa{N} / _\# (?)\\
\ipa{d} \change\ \ipa{r} / V_\{V,\#\}\\
\{\ipa{w,r}\} \change\ \O\ / \{\ipa{a,e}\}_\#\\
\O\ \change\ \ipa{e} / C_\#\\
\ipa{p} \change\ \ipa{b}\\
\ipa{aw} \change\ \ipa{o}\\
\ipa{e} \change\ \{\ipa{e,@,o}\} / _V\\
N \change\ \ipa{n} / _\{\ipa{d,l}\}\\
\ipa{gd gl} \change\ \ipa{d: l:}

\subsection{Proto-Austronesian to Proto-Paiwan}{\it Pogostick Man}, from Chen, Chun-Mei (2006), {\it A Comparative Study on Formosan Phonology: Paiwan and Budai Rukai} 313 -- 320

t$_1$ d$_1$ d$_3$ Z \change\ \ipa{t d \:d \textbardotlessj}\\
l *L \change\ \ipa{\:l L}\\
b d$_2$ \change\ \{\ipa{v,b}\} \ipa{z}\\
S$_1$ s c \change\ \ipa{s t ts}\\
V\ipa{:} \change\ V[- long]

\subsubsection{Proto-Paiwan to Northern Paiwan}{\it Pogostick Man}, from Chen, Chun-Mei (2006), {\it A Comparative Study on Formosan Phonology: Paiwan and Budai Rukai} 313 -- 320

\ipa{c \textbardotlessj\ q \:l} \change\ \ipa{t d P l}\\
Something about final stress and preceding /\ipa{@}/

\subsubsection{Proto-Paiwan to Central Paiwan}{\it Pogostick Man}, from Chen, Chun-Mei (2006), {\it A Comparative Study on Formosan Phonology: Paiwan and Budai Rukai} 313 -- 320

\ipa{w} \change\ \ipa{v} / _\#\\
Something about final stress and preceding /\ipa{@}/

\subsubsection{Proto-Paiwan to Southern Paiwan}{\it Pogostick Man}, from Chen, Chun-Mei (2006), {\it A Comparative Study on Formosan Phonology: Paiwan and Budai Rukai} 313 -- 320

\ipa{k r} \change\ \ipa{P G}\\
Something about final stress and preceding /\ipa{@}/

\subsection{Proto-Austronesian to Proto-Rukai}{\it Pogostick Man}, from Chen, Chun-Mei (2006), {\it A Comparative Study on Formosan Phonology: Paiwan and Budai Rukai} 313 -- 320

\{t$_1$,c\} \{d$_1$,z\} d$_3$ \change\ \ipa{t d \:d}\\
R l L \change\ \{\ipa{r,P}\} \ipa{\:l l}\\
S$_1$ s d$_2$ *C \change\ \ipa{s T D ts} (not sure what *C stands for here)\\
Something about echo-vowel epenthesis and stress that isn't really clear from skimming it\\

\subsubsection{Proto-Rukai to Budai Rukai}{\it Pogostick Man}, from Chen, Chun-Mei (2006), {\it A Comparative Study on Formosan Phonology: Paiwan and Budai Rukai} 313 -- 320

\{\ipa{v,P}\} \ipa{D} \change\ \O\ \ipa{j}\\
Long vowels acquire a high-low contour, but it looks like this is more prosodic than anything

\subsection{Proto-Austronesian to Proto-Tsouic}{\it Pogostick Man}, from Wikipedia contributors (2016), ``Tsouic languages''. {\it Wikipedia, the Free Encyclopedia}. \textless\url{https://en.wikipedia.org/w/index.php?title=Tsouic_languages&oldid=602917078}\textgreater

\{*C,\ipa{d}\} \ipa{j} *R \change\ \ipa{c z r}

\clearpage

\section{Northeast Caucasian}\tab Proto-Northeast Caucasian is reconstructed as having had the following consonant inventory. Phonemes in parentheses or braces are so marked on User:Petusek's page. Due to the inventory, the usual table format is modified.

\begin{center}\begin{tabular}{c | c c c c c c}
& Nasal & Plosive & Fricative & Affricate & Cluster & Resonant\\ \hline
Bilabial & \ipa{m} & \ipa{(p) b}\\
Alveolar & \ipa{n} & \ipa{t t' (d)} & \ipa{s (s:)} & \ipa{ts ts: ts' ts:' dz} & \ipa{st st:} & \ipa{r}\\
Lateral & & & \ipa{\textbeltl\ \textbeltl:} & \ipa{t\textbeltl\ t\textbeltl: t\textbeltl' t\textbeltl:' (d\textlyoghlig)} & & \ipa{l}\\
Postalveolar & & & \ipa{S S:} & \ipa{tS tS: tS' tS:' dZ}\\
Velar & & & \ipa{(x) (x:)} & \ipa{k (k:) k' (k':) g}\\
Uvular & & & & \ipa{q q: (q') q:' (\;G)}\\
Pharyngeal & & & & & \ipa{\{Q\}}\\
Glottal & & \ipa{\{P\}}
\end{tabular}\end{center}

\tab My guess is that what I've transcribed here as length (it's represented by doubled consonants in the source) is probably supposed to represent some sort of fortis-lenis distinction, given that in other places I think I've seen these doubled consonants in initial position, although I might be wrong, as I'm not very familiar with the morphology of the languages in question. Consonants such as *\ipa{ts:} are written $\langle$tts$\rangle$ in the source; unless it's the actual fricative that is geminate, the stop is the doubled consonant.

\tab The citation format for Nichols (2003) is modified from that found in Wikipedia contributors (2014), ``Northeast Caucasian languages". {\it Wikipedia, the Free Encyclopedia}. \textless\url{https://en.wikipedia.org/w/index.php?title=Northeast_Caucasian_languages&oldid=610673712}\textgreater, and is assumed to be the same article. The publication date for User:Petusek's page is taken from the revision history at \url{https://en.wikipedia.org/w/index.php?title=User:Petusek/Drafts/Northeast_Caucasian&oldid=351133322}.

\tab (From User:Petusek (2010), ``User:Petusek/Drafts/Northeast Caucasian". {\it Wikipedia, the Free Encyclopedia}. \textless\url{https://en.wikipedia.org/w/index.php?title=User:Petusek/Drafts/Northeast_Caucasian&oldid=351133322}\textgreater, apparently citing Nichols, Johanna (2003), ``The Nakh-Daghestanian Consonant Correspondences", in Tuite, Kevin, and Dee Ann Holisky, {\it Current Trends in Caucasian, East European, and Inner Asian Linguistics: Papers in Honor of Howard I. Aronson} 207 -- 251)

\subsection{Proto-Northeast Caucasian to Proto-Avar-Andic}{\it Pogostick Man}, from User:Petusek (2010), ``User:Petusek/Drafts/Northeast Caucasian". {\it Wikipedia, the Free Encyclopedia}. \textless\url{https://en.wikipedia.org/w/index.php?title=User:Petusek/Drafts/Northeast_Caucasian&oldid=351133322}\textgreater, apparently citing Nichols, Johanna (2003), ``The Nakh-Daghestanian Consonant Correspondences", in Tuite, Kevin, and Dee Ann Holisky, {\it Current Trends in Caucasian, East European, and Inner Asian Linguistics: Papers in Honor of Howard I. Aronson} 207 -- 251

\ipa{l} \change\ \{\ipa{l,r}\}\\
*\ipa{b} is ``[p]rone to change to *\ipa{m}"

\subsubsection{Proto-Avar-Andic to Akhvakh}{\it Pogostick Man}, from User:Petusek (2010), ``User:Petusek/Drafts/Northeast Caucasian". {\it Wikipedia, the Free Encyclopedia}. \textless\url{https://en.wikipedia.org/w/index.php?title=User:Petusek/Drafts/Northeast_Caucasian&oldid=351133322}\textgreater, apparently citing Nichols, Johanna (2003), ``The Nakh-Daghestanian Consonant Correspondences", in Tuite, Kevin, and Dee Ann Holisky, {\it Current Trends in Caucasian, East European, and Inner Asian Linguistics: Papers in Honor of Howard I. Aronson} 207 -- 251

\ipa{p} \change\ \ipa{h}\\
\ipa{dz dZ} \change\ \{\ipa{ts:',z}\} \{\ipa{ts:',dZ}\}\\
\{\ipa{ts,st}\} \change\ \ipa{tS}\\
\ipa{t\textbeltl: d\textlyoghlig} \change\ \ipa{t\textbeltl(:) t\textbeltl:}\\
\ipa{k: \;G} \change\ \ipa{x\super j} \{\ipa{q:',G}\}\\
\{\ipa{l,r}\} \change\ \O\ (sometimes, only from original *\ipa{l})

\subsubsection{Proto-Avar-Andic to Andi}{\it Pogostick Man}, from User:Petusek (2010), ``User:Petusek/Drafts/Northeast Caucasian". {\it Wikipedia, the Free Encyclopedia}. \textless\url{https://en.wikipedia.org/w/index.php?title=User:Petusek/Drafts/Northeast_Caucasian&oldid=351133322}\textgreater, apparently citing Nichols, Johanna (2003), ``The Nakh-Daghestanian Consonant Correspondences", in Tuite, Kevin, and Dee Ann Holisky, {\it Current Trends in Caucasian, East European, and Inner Asian Linguistics: Papers in Honor of Howard I. Aronson} 207 -- 251

\{\ipa{ts,st}\} \ipa{dz} \change\ \ipa{s} \{\ipa{ts:',z}\}\\
\ipa{st:} \change\ \ipa{s:}\\
\ipa{dZ} \change\ \{\ipa{tS:',dZ}\}\\
\ipa{t\textbeltl\ t\textbeltl(:) t\textbeltl(:)(') dZ} \change\ \ipa{\textbeltl\ t\textbeltl(:) t\textbeltl(:)(') t\textbeltl:}\\
\ipa{q \;G} \change\ \ipa{x} \{\ipa{q:',G}\}\\

\subsubsection{Proto-Avar-Andic to Avar}{\it Pogostick Man}, from User:Petusek (2010), ``User:Petusek/Drafts/Northeast Caucasian". {\it Wikipedia, the Free Encyclopedia}. \textless\url{https://en.wikipedia.org/w/index.php?title=User:Petusek/Drafts/Northeast_Caucasian&oldid=351133322}\textgreater, apparently citing Nichols, Johanna (2003), ``The Nakh-Daghestanian Consonant Correspondences", in Tuite, Kevin, and Dee Ann Holisky, {\it Current Trends in Caucasian, East European, and Inner Asian Linguistics: Papers in Honor of Howard I. Aronson} 207 -- 251

\ipa{ts ts: ts' ts:' dz} \change\ \ipa{sh ts tS' ts(:)' ts'}\\
\ipa{st(:)} \change\ \ipa{ts}\\
\ipa{tS tS: tS:' dZ} \change\ \ipa{ts(?) tS tS(:)' ts'}\\
\ipa{t\textbeltl\ t\textbeltl' t\textbeltl: t\textbeltl:'} \change\ \ipa{\textbeltl\ t' t\textbeltl\ t\textbeltl(:)'}\\
\ipa{q q:' \;G} \change\ \ipa{x} \{\ipa{q(:)',gh}\} \ipa{q'} (not sure if that last $\langle$gh$\rangle$ should be \ipa{G})\\
\ipa{s s: \textbeltl: x x:} \change\ \{\ipa{s,S}\} \ipa{x \textbeltl} \{\ipa{x,h}\} \ipa{x}\\
\ipa{m} \change\ \{\ipa{m,n}\}

\subsection{Proto-Northeast Caucasian to Dargi}{\it Pogostick Man}, from User:Petusek (2010), ``User:Petusek/Drafts/Northeast Caucasian". {\it Wikipedia, the Free Encyclopedia}. \textless\url{https://en.wikipedia.org/w/index.php?title=User:Petusek/Drafts/Northeast_Caucasian&oldid=351133322}\textgreater, apparently citing Nichols, Johanna (2003), ``The Nakh-Daghestanian Consonant Correspondences", in Tuite, Kevin, and Dee Ann Holisky, {\it Current Trends in Caucasian, East European, and Inner Asian Linguistics: Papers in Honor of Howard I. Aronson} 207 -- 251

*\ipa{b} is ``[p]rone to change to *\ipa{m}"\\
\{\ipa{ts:,st}\} \ipa{ts:' dz st:} \change\ \ipa{ts z ts: s}\\
\ipa{tS tS: dZ} \change\ \{\ipa{ts',tS'}\} \ipa{Z ts}\\
\ipa{t\textbeltl\ t\textbeltl: t\textbeltl' t\textbeltl:'} \change\ \ipa{k} \{\ipa{x\super j:,k:}\} \ipa{k\super h} \{\ipa{g,q}\}(?)\\
\ipa{q:'} \change\ \ipa{Q}\\
\ipa{S S: \textbeltl\ \textbeltl:} \change\ \{\ipa{s,S}\} \ipa{S x\super j} \{\ipa{x\super j,S}\}

\subsection{Proto-Northeast Caucasian to Khinalug}{\it Pogostick Man}, from User:Petusek (2010), ``User:Petusek/Drafts/Northeast Caucasian". {\it Wikipedia, the Free Encyclopedia}. \textless\url{https://en.wikipedia.org/w/index.php?title=User:Petusek/Drafts/Northeast_Caucasian&oldid=351133322}\textgreater, apparently citing Nichols, Johanna (2003), ``The Nakh-Daghestanian Consonant Correspondences", in Tuite, Kevin, and Dee Ann Holisky, {\it Current Trends in Caucasian, East European, and Inner Asian Linguistics: Papers in Honor of Howard I. Aronson} 207 -- 251

\ipa{b d} \change\ \{\ipa{b,v}\} \ipa{z}\\
The development of *\ipa{ts'} is unclear; in the user page there's a slash but it might be a typo for an apostrophe\\
\{\ipa{ts:,st}\} \{\ipa{ts:',dz}\} \change\ \ipa{ts ts'}\\
\{\ipa{tS(:),dZ}\} \change\ \ipa{tS'}\\
\ipa{t\textbeltl\ t\textbeltl: t\textbeltl' t\textbeltl:' d\textlyoghlig} \change\ \ipa{k} \{\ipa{k,x\super j}\} \{\ipa{k',g}\} \{\ipa{k',k:}\} \ipa{k'}\\
\ipa{k: k' g} \change\ \ipa{k} \{\ipa{k',g}\} \{\ipa{k',k:}\}\\
\{\ipa{q:',\;G}\} \change\ \ipa{q'}\\
\ipa{s: \textbeltl(:) x:} \change\ \ipa{h x\super j x}\\
\ipa{r} \change\ \ipa{n} / _C\\
\ipa{r} \change\ \{\ipa{r},\O\}

\subsection{Proto-Northeast Caucasian to Lak}{\it Pogostick Man}, from User:Petusek (2010), ``User:Petusek/Drafts/Northeast Caucasian". {\it Wikipedia, the Free Encyclopedia}. \textless\url{https://en.wikipedia.org/w/index.php?title=User:Petusek/Drafts/Northeast_Caucasian&oldid=351133322}\textgreater, apparently citing Nichols, Johanna (2003), ``The Nakh-Daghestanian Consonant Correspondences", in Tuite, Kevin, and Dee Ann Holisky, {\it Current Trends in Caucasian, East European, and Inner Asian Linguistics: Papers in Honor of Howard I. Aronson} 207 -- 251

\ipa{b d} \change\ \{\ipa{p:,b}\} \{\ipa{t:,d}\} (but *\ipa{b} is ``[p]rone to change to *\ipa{m}")\\
\{\ipa{ts:,st:}\} \ipa{dz st} \change\ \ipa{s:} \{\ipa{ts:,z}\} \ipa{ts}\\
\ipa{tS tS: dZ} \change\ \{\ipa{ts,tS}\} \{\ipa{ts',tS'}\} \ipa{tS(:)'}\\
\ipa{t\textbeltl\ t\textbeltl: t\textbeltl(:)' d\textlyoghlig} \change\ \ipa{x\super j x: k'} \{\ipa{k:,l}\}\\
\ipa{g} \change\ \ipa{k:}\\
\ipa{q' q': \:g} \change\ \{\ipa{q',j}\} \ipa{q'} \{\ipa{q:,G}\}\\
\ipa{S \textbeltl\ \textbeltl:} \change\ \ipa{s} \{\ipa{x\super j,S}\} \ipa{x:}\\
\ipa{m l} \change\ \{\ipa{m,n}\} \{\ipa{l},\O\}

\subsection{Proto-Northeast Caucasian to Proto-Lezgic}{\it Pogostick Man}, from User:Petusek (2010), ``User:Petusek/Drafts/Northeast Caucasian". {\it Wikipedia, the Free Encyclopedia}. \textless\url{https://en.wikipedia.org/w/index.php?title=User:Petusek/Drafts/Northeast_Caucasian&oldid=351133322}\textgreater, apparently citing Nichols, Johanna (2003), ``The Nakh-Daghestanian Consonant Correspondences", in Tuite, Kevin, and Dee Ann Holisky, {\it Current Trends in Caucasian, East European, and Inner Asian Linguistics: Papers in Honor of Howard I. Aronson} 207 -- 251

\tab {\it NB: These changes here probably aren't ``proper'' sound changes, whatever that's supposed to mean, but there doesn't seem to be any particular sound change or set of sound changes that defines this family, so I've elected to go with something that seems to nearly work and note the exceptions.}

\ipa{q \;G} \change\ \{\ipa{x,q}\} \ipa{G}

\subsubsection{Proto-Lezgic to Agul}{\it Pogostick Man}, from User:Petusek (2010), ``User:Petusek/Drafts/Northeast Caucasian". {\it Wikipedia, the Free Encyclopedia}. \textless\url{https://en.wikipedia.org/w/index.php?title=User:Petusek/Drafts/Northeast_Caucasian&oldid=351133322}\textgreater, apparently citing Nichols, Johanna (2003), ``The Nakh-Daghestanian Consonant Correspondences", in Tuite, Kevin, and Dee Ann Holisky, {\it Current Trends in Caucasian, East European, and Inner Asian Linguistics: Papers in Honor of Howard I. Aronson} 207 -- 251

\ipa{b d} \change\ \{\ipa{b,w}\} \{\ipa{d,z}\}\\
\{\ipa{ts,st}\} \ipa{ts: ts:' st: dz} \change\ \{\ipa{s,ts}\} \{\ipa{ts,tS}\} \ipa{t: s z}\\
\ipa{tS:'} \change\ \ipa{tS:}\\
\ipa{t\textbeltl\ t\textbeltl: t\textbeltl' t\textbeltl:' d\textlyoghlig} \change\ \ipa{x\super j x k' k:} \{\ipa{,j,x\super j}\}\\
\ipa{q:'} \change\ \ipa{q:}\\
\ipa{\textbeltl(:) x:} \change\ \ipa{x\super j x}\\
\ipa{m} \change\ \{\ipa{m,b}\}

\subsubsection{Proto-Lezgic to Archi}{\it Pogostick Man}, from User:Petusek (2010), ``User:Petusek/Drafts/Northeast Caucasian". {\it Wikipedia, the Free Encyclopedia}. \textless\url{https://en.wikipedia.org/w/index.php?title=User:Petusek/Drafts/Northeast_Caucasian&oldid=351133322}\textgreater, apparently citing Nichols, Johanna (2003), ``The Nakh-Daghestanian Consonant Correspondences", in Tuite, Kevin, and Dee Ann Holisky, {\it Current Trends in Caucasian, East European, and Inner Asian Linguistics: Papers in Honor of Howard I. Aronson} 207 -- 251

\ipa{d} \change\ \ipa{r} / _\#\\
\ipa{d} \change\ \{\ipa{d,t:}\}\\
\{\ipa{ts,st}\} \ipa{ts: ts:' st: dz} \change\ \ipa{s} \{\ipa{s,S}\} \ipa{ts' s: ts}\\
\ipa{tS(:) dZ} \change\ \ipa{S tS}\\
\ipa{t\textbeltl\ t\textbeltl: t\textbeltl' t\textbeltl:' d\textlyoghlig} \change\ \ipa{\textbeltl\ \textbeltl(:) k' t\textbeltl' t\textbeltl:}\\
\ipa{k: g} \change\ \ipa{x} \{\ipa{k:,g}\}\\
\ipa{G} \change\ \ipa{q} (more likely, *\ipa{\;G} \change\ \ipa{q} instead of \change\ \ipa{G})\\
\ipa{q q:'} \change\ \ipa{x q(:)'}\\
\ipa{s: x:} \change\ \ipa{\{s:,h\} x}

\subsubsection{Proto-Lezgic to Lezghi}{\it Pogostick Man}, from User:Petusek (2010), ``User:Petusek/Drafts/Northeast Caucasian". {\it Wikipedia, the Free Encyclopedia}. \textless\url{https://en.wikipedia.org/w/index.php?title=User:Petusek/Drafts/Northeast_Caucasian&oldid=351133322}\textgreater, apparently citing Nichols, Johanna (2003), ``The Nakh-Daghestanian Consonant Correspondences", in Tuite, Kevin, and Dee Ann Holisky, {\it Current Trends in Caucasian, East European, and Inner Asian Linguistics: Papers in Honor of Howard I. Aronson} 207 -- 251

\ipa{b d} \change\ \ipa{\{p:,b\} \{t:,d\}}\\
\ipa{\{ts,st\} ts: ts:' dz} \change\ \ipa{\{S,tS\} tS(') \{tS:,dZ\}}\\
\ipa{t\textbeltl\ t\textbeltl: t\textbeltl' t\textbeltl:' d\textlyoghlig} \change\ \ipa{x\super j \{G\super j,g\} q' k(') \{k:,G\super j\}}\\
\ipa{k: g} \change\ \ipa{G k:}\\
\ipa{G} \change\ \ipa{\{q:,G\}} (again, probably a difference in the development of *\ipa{\;G} than this strict sound change)\\
\ipa{q:'} \change\ \ipa{q(')}\\
\ipa{S: \textbeltl\ \textbeltl: x x:} \change\ \ipa{G\super j j S G x}

\subsubsection{Proto-Lezgic to Rutul}{\it Pogostick Man}, from User:Petusek (2010), ``User:Petusek/Drafts/Northeast Caucasian". {\it Wikipedia, the Free Encyclopedia}. \textless\url{https://en.wikipedia.org/w/index.php?title=User:Petusek/Drafts/Northeast_Caucasian&oldid=351133322}\textgreater, apparently citing Nichols, Johanna (2003), ``The Nakh-Daghestanian Consonant Correspondences", in Tuite, Kevin, and Dee Ann Holisky, {\it Current Trends in Caucasian, East European, and Inner Asian Linguistics: Papers in Honor of Howard I. Aronson} 207 -- 251

\ipa{b d} \change\ \ipa{\{b,w\} \{d,z\}}\\
\{\ipa{ts,st}\} \ipa{ts:' dz} \change\ \{\ipa{s,ts}\} \ipa{\{d,t\} z}\\
\ipa{tS tS: tS:'} \change\ \ipa{\{S,tS\} S tS}\\
\ipa{t\textbeltl: t\textbeltl' t\textbeltl:' d\textlyoghlig} \change\ \ipa{x\super j q' q(') \{w,x\super j,j\}}\\
\ipa{k:} \change\ \ipa{x}\\
\ipa{q:'} \change\ \ipa{q(')}\\
\ipa{s: \textbeltl(:) x:} \change\ \ipa{\{\textcrh,x\super j\} \{s:,h\} x}\\
\ipa{m} \change\ \ipa{\{m,b\}}

\subsubsection{Proto-Lezgic to Tabassaran}{\it Pogostick Man}, from User:Petusek (2010), ``User:Petusek/Drafts/Northeast Caucasian". {\it Wikipedia, the Free Encyclopedia}. \textless\url{https://en.wikipedia.org/w/index.php?title=User:Petusek/Drafts/Northeast_Caucasian&oldid=351133322}\textgreater, apparently citing Nichols, Johanna (2003), ``The Nakh-Daghestanian Consonant Correspondences", in Tuite, Kevin, and Dee Ann Holisky, {\it Current Trends in Caucasian, East European, and Inner Asian Linguistics: Papers in Honor of Howard I. Aronson} 207 -- 251

\ipa{b d} \change\ \ipa{\{b,w\} \{d,z\}}\\
\ipa{\{ts,st\} ts: ts:'} \change\ \ipa{\{s,ts\} \{ts,tS\} ts:}\\
\ipa{tS tS: tS:'} \change\ \ipa{\{S,tS\} \{tS,dZ\} \{tS:,tS'\}}(?)\\
\ipa{t\textbeltl\ t\textbeltl: t\textbeltl' t\textbeltl:' d\textlyoghlig} \change\ \ipa{x\super j \{G\super j,g\} k k: G\super j}\\
\ipa{k:} \change\ \ipa{q}\\
\ipa{\textbeltl\ \{S:,\textbeltl:\} x:} \change\ \ipa{x\super j S x}\\
\ipa{m} \change\ \ipa{\{m,b\}}

\subsubsection{Proto-Lezgic to Udi}{\it Pogostick Man}, from User:Petusek (2010), ``User:Petusek/Drafts/Northeast Caucasian". {\it Wikipedia, the Free Encyclopedia}. \textless\url{https://en.wikipedia.org/w/index.php?title=User:Petusek/Drafts/Northeast_Caucasian&oldid=351133322}\textgreater, apparently citing Nichols, Johanna (2003), ``The Nakh-Daghestanian Consonant Correspondences", in Tuite, Kevin, and Dee Ann Holisky, {\it Current Trends in Caucasian, East European, and Inner Asian Linguistics: Papers in Honor of Howard I. Aronson} 207 -- 251

\ipa{ts ts: ts' ts:' dz} \change\ \{\O,\ipa{s}\} \ipa{tS} \{\O,\ipa{ts'}\} \{\ipa{ts',tS'}\} \ipa{z}\\
\ipa{t} \change\ \O\ / \ipa{s}_\\
\ipa{tS tS: tS' tS:' dZ} \change\ \{\O,\ipa{S}\} \ipa{tS} \O\ \ipa{tS'} \{\ipa{dZ,tS}\}\\
\ipa{t\textbeltl\ t\textbeltl: t\textbeltl' t\textbeltl:' d\textlyoghlig} \change\ \{\O,\ipa{x}\} \ipa{q} \{\O,\ipa{q'}\} \ipa{q'} \{\ipa{G,l}\}\\
\ipa{k:} \change\ \ipa{q}\\
\ipa{q q' q:' \;G} \change\ \O(?) \O\ \ipa{q' G}\\
\{\ipa{\textbeltl(:),x:}\} \change\ \ipa{x}

\subsection{Proto-Northeast Caucasian to Nakh}{\it Pogostick Man}, from User:Petusek (2010), ``User:Petusek/Drafts/Northeast Caucasian". {\it Wikipedia, the Free Encyclopedia}. \textless\url{https://en.wikipedia.org/w/index.php?title=User:Petusek/Drafts/Northeast_Caucasian&oldid=351133322}\textgreater, apparently citing Nichols, Johanna (2003), ``The Nakh-Daghestanian Consonant Correspondences", in Tuite, Kevin, and Dee Ann Holisky, {\it Current Trends in Caucasian, East European, and Inner Asian Linguistics: Papers in Honor of Howard I. Aronson} 207 -- 251

\ipa{st st:} \change\ \ipa{st(') st}\\
\ipa{ts: tS:} \change\ \ipa{ts tS} / \#_\\
\ipa{ts:'} \change\ \ipa{ts:}\\
\ipa{\{ts:,tS:,dz,dZ\}} \change\ \ipa{t:} / _\#\\
\ipa{t\textbeltl(:) t\textbeltl' t\textbeltl:'} \change\ \ipa{x q' Q}\\
\ipa{k: q:'} \change\ \ipa{xk(?) q'}\\
\ipa{S: \textbeltl: x:} \change\ \ipa{S \textbeltl\ x}\\
\ipa{l} \change\ \ipa{r} / _\#\\
\ipa{r} \change\ \ipa{d} / \#_

\subsection{Proto-Northeast Caucasian to Proto-Tsezic}{\it Pogostick Man}, from User:Petusek (2010), ``User:Petusek/Drafts/Northeast Caucasian". {\it Wikipedia, the Free Encyclopedia}. \textless\url{https://en.wikipedia.org/w/index.php?title=User:Petusek/Drafts/Northeast_Caucasian&oldid=351133322}\textgreater, apparently citing Nichols, Johanna (2003), ``The Nakh-Daghestanian Consonant Correspondences", in Tuite, Kevin, and Dee Ann Holisky, {\it Current Trends in Caucasian, East European, and Inner Asian Linguistics: Papers in Honor of Howard I. Aronson} 207 -- 251

*\ipa{b} is ``[p]rone to change to *\ipa{m}"\\
\ipa{\{ts,st\} \{ts:',dz\}} \change\ \ipa{s ts}\\
\ipa{tS \{tS:,dZ\}} \change\ \ipa{S tS} (the change of *\ipa{tS:'} is conjectured for Bezhta, as the change is only listed in Tsez, but given the development of *\ipa{ts:'} I don't find it unreasonable to put it here)\\
\ipa{t\textbeltl\ t\textbeltl:'} \change\ \ipa{\textbeltl\ t\textbeltl}\\
\ipa{\;G} \change\ \ipa{q}\\
\ipa{s x:} \change\ \ipa{z x}\\
\ipa{l} \change\ \ipa{\{l,r\}}

\subsubsection{Proto-Tsezic to Bezhta}{\it Pogostick Man}, from User:Petusek (2010), ``User:Petusek/Drafts/Northeast Caucasian". {\it Wikipedia, the Free Encyclopedia}. \textless\url{https://en.wikipedia.org/w/index.php?title=User:Petusek/Drafts/Northeast_Caucasian&oldid=351133322}\textgreater, apparently citing Nichols, Johanna (2003), ``The Nakh-Daghestanian Consonant Correspondences", in Tuite, Kevin, and Dee Ann Holisky, {\it Current Trends in Caucasian, East European, and Inner Asian Linguistics: Papers in Honor of Howard I. Aronson} 207 -- 251

*\ipa{\textbeltl} may have remained \ipa{\textbeltl}\\
*\ipa{r} remained when intervocalic

\subsubsection{Proto-Tsezic to Tsez}{\it Pogostick Man}, from User:Petusek (2010), ``User:Petusek/Drafts/Northeast Caucasian". {\it Wikipedia, the Free Encyclopedia}. \textless\url{https://en.wikipedia.org/w/index.php?title=User:Petusek/Drafts/Northeast_Caucasian&oldid=351133322}\textgreater, apparently citing Nichols, Johanna (2003), ``The Nakh-Daghestanian Consonant Correspondences", in Tuite, Kevin, and Dee Ann Holisky, {\it Current Trends in Caucasian, East European, and Inner Asian Linguistics: Papers in Honor of Howard I. Aronson} 207 -- 251

\ipa{ts: t\textbeltl:} \change\ \ipa{z l} / V_V\\
\ipa{ts: t\textbeltl:} \change\ \ipa{s \textbeltl}\\
\ipa{k:' g} \change\ \ipa{k k'}\\
\ipa{q q:'} \change\ \ipa{x(?) q}\\
\ipa{s: S S: \textbeltl\ \textbeltl: x} \change\ \ipa{s Z S l \textbeltl\ G}\\
\ipa{r} \change\ \{\ipa{r,l},\O\}

\clearpage

\section{Northwest Caucasian}\tab Wikipedia contributors (2014) give the following reconstruction for Proto-Northwest Caucasian consonants; due to the size of the reconstructed inventory, the usual table format has been modified:

\begin{tabular}{c | c c c c c}
& Stop & Fricative & Affricate & Resonant\\ \hline
Plain Labial & \ipa{p p: p' b} & \ipa{f} & & \ipa{m\super Q}\\
Palatalized Labial & \ipa{p\super j p\super j: p\super j b\super j}\\
Labialized Labial & \ipa{p\super w b\super w}\\
Palatalized Labialized Labial & \ipa{p\super w\super j p\super w\super j' b\super w\super j}\\
Pharyngealized Labial & \ipa{p\super Q p\super Q: p\super Q' b\super Q}\\
Plain Coronal & \ipa{t t: t' d} & \ipa{s z} & \ipa{ts ts: ts' dz} & \ipa{r n}\\
Palatalized Coronal & \ipa{t\super j} & \ipa{s\super j z\super j} & \ipa{ts\super j ts\super j: ts\super j dz\super j} & \ipa{r\super j}\\
Labialized Coronal & \ipa{t\super w t\super w: t\super w' d\super w} & \ipa{ts\super w ts\super w' dz\super w}\\
Palatalized Labialized Coronal & \ipa{t\super j\super w t\super w\super j'} & \ipa{ts\super j\super w ts\super j\super w dz\super j\super w}\\
Plain Lateral & & \ipa{\textbeltl\ \textbeltl:} & \ipa{t\textbeltl\ t\textbeltl: t\textbeltl' d\textlyoghlig} & \ipa{l}\\
Palatalized Lateral & &\ipa{\textbeltl\super j}(\raisebox{-0.6ex}{\textasciitilde}\ipa{\textbeltl\super j':}) \ipa{\textlyoghlig} & \ipa{t\textbeltl\super j t\textbeltl}\ipa{\super j' d}\ipa{\textlyoghlig\super j} & \ipa{l\super j}\\
Labialized Lateral & & \ipa{\textbeltl\super w \textbeltl\super w:} & \ipa{t\textbeltl\super w t\textbeltl\super w: t\textbeltl\super w' d\textlyoghlig\super w}\\
Palatalized Labialized Lateral & & \ipa{\textbeltl\super j\super w}(\raisebox{-0.6ex}{\textasciitilde}\ipa{\textbeltl\super w:}) \ipa{\textlyoghlig\super j\super w} & \ipa{t\textbeltl\super j\super w t\textbeltl\super j\super w: t\textbeltl\super j\super w' d\textlyoghlig\super j\super w}\\
Plain Postalveolar & & \ipa{S}(\raisebox{-0.6ex}{\textasciitilde}\ipa{S:}) \ipa{Z} & \ipa{tS tS' dZ}\\
Labialized Postalveolar & & \ipa{S\super w S\super w: Z\super w} & \ipa{tS\super w tS\super w: tS\super w' dZ\super w}\\
Plain Palatal & & \ipa{C C: \textctz} & \ipa{tC tC: tC' d\textctz} & \ipa{j}\\
Labialized Palatal & & \ipa{C\super w C\super w: \textctz\super w} & \ipa{tC\super w tC\super w: tC\super w'}\\
Plain Velar & \ipa{k k' g} & \ipa{x G}\\
Palatalized Velar & \ipa{k\super j g\super j} & \ipa{x\super j G\super j}\\
Labialized Velar & \ipa{k\super w k\super w: k\super w' g\super w} & \ipa{x\super w}\\
Palatalized Labialized Velar & \ipa{k\super j\super w' g\super j\super w} & \ipa{x\super j\super w G\super j\super w}(?)\\
Plain Uvular & \ipa{q q: q' \;G} & \ipa{X K}\\
Palatalized Uvular & \ipa{q\super j: q\super j \;G\super j} & \ipa{X\super j K\super j}\\
Labialized Uvular & \ipa{q\super w q\super w: q\super w' \;G\super w} & \ipa{X\super w K\super w}\\
Labialized Palatalized Uvular & \ipa{q\super j\super w q\super j\super w: q\super j\super w' \;G\super j\super w} & \ipa{X\super j\super w K\super j\super w}\\
Pharyngealized Uvular & \ipa{q\super Q: q\super Q'} & \ipa{X\super Q K\super Q}\\
Pharyngealized Palatalized Uvular & \ipa{q\super Q\super j q\super Q\super j: q\super Q\super j'} & \ipa{X\super Q\super j R\super Q\super j}\\
Pharyngealized Labialized Uvular & \ipa{q\super Q\super w q\super Q\super w: q\super Q\super w'} & \ipa{X\super Q\super w K\super Q\super w}\\
Pharyngeal Labialized Palatal Uvular & \ipa{q\super Q\super j\super w q\super Q\super j\super w: q\super Q\super j\super w'} & \ipa{K\super Q\super j\super w}
\end{tabular}

\tab (From Wikipedia contributors (2014), ``Proto-Northwest Caucasian language''. {\it Wikipedia, the Free Encyclopedia}. \textless\url{http://en.wikipedia.org/w/index.php?title=Proto-Northwest_Caucasian_language&oldid=596995618}\textgreater, presumably citing Starostin, Sergei A. and Sergei L. Nikolayev (1994), {\it A North Caucasian Etymological Dictionary})

\subsection{Proto-Northwest Caucasian to Proto-Abazgi}{\it Pogostick Man}, from Wikipedia contributors (2014), ``Proto-Northwest Caucasian language''. {\it Wikipedia, the Free Encyclopedia}. \textless\url{http://en.wikipedia.org/w/index.php?title=Proto-Northwest_Caucasian_language&oldid=596995618}\textgreater, presumably citing Starostin, Sergei A. and Sergei L. Nikolayev (1994), {\it A North Caucasian Etymological Dictionary}\\

\{\ipa{p\super w,p\super j}\} \{\ipa{p(\super j):,b\super Q}\} \ipa{b\super w} \ipa{p\super w\super j} \ipa{b\super w\super j} \textrightarrow\ \ipa{p b f ts\super j dz\super j}\\
\ipa{m\super Q} \change\ \ipa{m}\\
\ipa{r\super j} \change\ \ipa{r}\\
\ipa{t\super w: t\super w\super j(')} \change\ \ipa{d(w) t\super w(')}\\
\ipa{ts\super j:} \change\ \ipa{dz\super j} (marked as dubious in the source)\\
\ipa{s\super w\super j z\super j} \change\ \ipa{s\super j z}\\
\ipa{ts\super w: ts\super w\super j} \change\ \ipa{ts\super w (tS)}\\
\ipa{z\super w\super j ts\super w\super j' dz\super w\super j} \change\ \ipa{dz\super j}\raisebox{-0.6ex}{\textasciitilde}\ipa{z\super j} \ipa{tS' dZ}\raisebox{-0.6ex}{\textasciitilde}\ipa{Z}\\
\ipa{tS(') dZ} \change\ \ipa{ts\super j(') dz\super j}\\
\ipa{S}(\raisebox{-0.6ex}{\textasciitilde}\ipa{S:}) \ipa{Z} \change\ \ipa{s\super j s\super j}\\
\ipa{tC:} \change\ \ipa{dz\super j}(\raisebox{-0.6ex}{\textasciitilde}\ipa{tC})\\
\ipa{C(:) \textctz} \change\ \ipa{S Z} (the change of singleton *\ipa{C} to \ipa{S} is marked as dubious)\\
\ipa{tS\super w(') tS\super w: dZ\super w} \change\ \ipa{tS(') z\super w dz\super w}\raisebox{-0.6ex}{\textasciitilde}\ipa{z\super w}\\
\ipa{S\super w S\super w: Z\super w} \change\ \ipa{s\super w S z\super j} (this final change is marked as dubious)\\
\ipa{tC\super w(') tC\super w:} \change\ \ipa{tS(') dZ}\raisebox{-0.6ex}{\textasciitilde}\ipa{Z}\\
\ipa{C\super w C\super w: \textctz\super w} \change\ \ipa{S\super w S Z\super w}\\
\ipa{t}\ipa{\textbeltl} \ipa{t}\ipa{\textbeltl:} \ipa{t}\ipa{\textbeltl'} \ipa{d}\ipa{\textlyoghlig} \change\ \ipa{x ts \{x,ts'\} l}\\
\ipa{\textbeltl:} \change\ \ipa{x}\\
\ipa{t\textbeltl\super j t\textbeltl\super j' d\textlyoghlig\super j} \change\ \ipa{x\super j C G\super j}\\
\ipa{\textbeltl\super j}(\raisebox{-0.6ex}{\textasciitilde}\ipa{\textbeltl\super j\super w:}) \ipa{\textlyoghlig\super j} \change\ \ipa{C \textctz}\\
\ipa{l\super j} \change\ \ipa{r} / \#_\\
\ipa{l\super j} \change\ \ipa{l}\raisebox{-0.6ex}{\textasciitilde}\ipa{G\super j}\\
\ipa{t\textbeltl\super w(:) t\textbeltl' d\textlyoghlig\super w} \change\ \ipa{ts\super w ts'(\super w) l}\\
\ipa{\textbeltl\super w(:)} \change\ \ipa{S}\\
\ipa{t\textbeltl\super w\super j t\textbeltl\super w\super j: t\textbeltl\super w\super j' d\textlyoghlig\super w\super j} \change\ \ipa{tS\super w Z\super w tS\super w' dZ\super w}\\
\ipa{\textbeltl\super w\super j}(\raisebox{-0.6ex}{\textasciitilde}\ipa{\textbeltl\super w:}) \change\ \ipa{S}\\
\ipa{\textlyoghlig\super w\super j} \change\ \ipa{Z}\\
\ipa{k\super w:} \change\ \ipa{g\super w}\\
\ipa{k\super w\super j'} \change\ \ipa{k\super w'}\\
\ipa{x\super w\super j G\super w\super j} \change\ \ipa{s\super w z\super w} (*\ipa{G\super w\super j} is marked as dubious)\\
\ipa{q q: \;G} \change\ (\ipa{\textcrh}) \ipa{q G} (*\ipa{\;G} is marked as *G in the document)\\
\ipa{X(\super j) K(\super j)} \change\ \ipa{\textcrh\ Q}\\
\ipa{q\super j: \;G\super j} \change\ \ipa{q G\super j}\\
\ipa{q\super w q\super w: \;G\super w} \change\ \ipa{\textcrh\super w q\super w G\super w}\\
\ipa{X\super w K\super w} \change\ \ipa{\textcrh\super w Q\super w}\\
\ipa{q\super w\super j \;G\super w\super h} \change\ \ipa{\textcrh(\super w) G(\super w)}\\
\ipa{q\super w\super j'} \change\ \ipa{Q\super w} (\ipa{q\super w'}?)\\
\ipa{X\super w\super j K\super w\super j} \change\ \ipa{\textcrh(\super w) Q(\super w)}\\
\ipa{q\super Q: q\super Q'} \change\ \ipa{Q \textcrh} (this latter is marked as dubious)\\
\ipa{KQ} \change\ \ipa{Q}\\
\ipa{q\super Q\super j q\super Q\super j: q\super Q\super j'} \change\ \ipa{q P \textcrh}(?)\\
\ipa{X\super Q\super j K\super Q\super j} \change\ \ipa{\textcrh\ Q}\\
\ipa{q\super Q\super w q\super Q\super w: q\super Q\super w' X\super Q\super w K\super Q\super w} \change\ \ipa{\textcrh\super w Q\super w q\super w(') \textcrh\super w} (\ipa{\textcrh\super w}?)\\
\ipa{q\super Q\super w\super j q\super Q\super w\super j q\super Q\super w\super j' K\super Q\super w\super j} \change\ \ipa{q\super w Q\super w \textcrh\super w Q\super w}

\subsubsection{Proto-Abazgi to Ashkharywa Abaza}{\it Nortaneous}, from Chirikba, Viacheslav A. (2003), ``Abkhaz". {\it Languages of the World/Materials} 119.

\ipa{tS\super w(') tC(') dZ\super w d\textctz} \change\ \ipa{f(') ts(') v dz}\\
\ipa{C \textctz} \change\ \ipa{s z}\\
"V\ipa{Q} \ipa{Q}"V \change\ "\ipa{aa a}"\ipa{a} (but stays /\ipa{Q}/ sometimes?)\\
\ipa{Q\super w} \change\ \ipa{4}\\
\ipa{t\super w(') d\super w} \change\ \{\ipa{t\super w('),p(')}\} \{\ipa{d\super w,b}\}

\subsubsection{Proto-Abazgi to Tapanta Abaza}{\it Nortaneous}, from Chirikba, Viacheslav A. (2003), ``Abkhaz". {\it Languages of the World/Materials} 119.

\{\ipa{tS\super w('),tC}\} \{\ipa{dZ\super w,d\textctz}\} \change\ \ipa{ts(') dz}\\
\ipa{tC\super w(') d\textctz\super w} \change\ \{\ipa{tC(\super w)('),tS\super w(')}\} \{\ipa{d\textctz(\super w),dZ\super w}\}\\
\ipa{C \textctz} \change\ \ipa{s z}\\
\ipa{S\super w Z\super w C\super w \textctz\super w} \change\ \{\ipa{C(\super w),S(\super w)}\} \{\ipa{\textctz(\super w),Z(\super w)}\} \{\ipa{C(\super w),S\super w}\} \{\ipa{\textctz(\super w),Z\super w}\}\\
\{\ipa{t\super w('),d\super w}\} \change\ \{\ipa{tC(\super w)('),tS(\super w)(')}\} \{\ipa{dZ(\super w),d\textctz(\super w)}\}

\subsubsection{Proto-Abazgi to Ahchypsy Abkhaz}{\it Nortaneous}, from Chirikba, Viacheslav A. (2003), ``Abkhaz". {\it Languages of the World/Materials} 119.

\ipa{tS\super w tS\super w' dZ\super w tC(') d\textctz} \change\ \ipa{f p' ts(') v dz}\\
\ipa{C \textctz} \change\ \ipa{s z}\\
"V\ipa{Q} \ipa{Q}"V \change\ "\ipa{aa a}"\ipa{a}\\
\ipa{Q Q\super w} \change\ \ipa{a: 4}\\
\ipa{q q\super w} \change\ \ipa{X\super Q X\super Q\super w}

\subsubsection{Proto-Abazgi to Bzyp Abkhaz}{\it Nortaneous}, from Chirikba, Viacheslav A. (2003), ``Abkhaz". {\it Languages of the World/Materials} 119.

\ipa{tS\super w tS\super w' dZ\super w} \change\ \ipa{p' f v}\\
"V\ipa{Q} \ipa{Q}"V \change\ "\ipa{aa a}"\ipa{a}\\
\ipa{Q\super w} \change\ \ipa{4}\\
\ipa{q q\super w} \change\ \ipa{X\super Q X\super Q\super w}

\subsubsection{Proto-Abazgi to Abzhywa Proper}{\it Nortaneous}, from Chirikba, Viacheslav A. (2003), ``Abkhaz". {\it Languages of the World/Materials} 119.

\ipa{tS\super w(') tC(') tC\super w(') dZ\super w d\textctz\ d\textctz\super w} \change\ \ipa{f(') ts(') tC\super w(') v dz d\textctz}\\
\ipa{C C\super w \textctz\ \textctz\super w} \change\ \ipa{s S\super w z Z\super w}\\
"V\ipa{Q} \ipa{Q}"V \change\ "\ipa{aa a}"\ipa{a}\\
\ipa{Q Q\super w} \change\ \ipa{a: 4}\\
\ipa{q q\super w} \change\ \ipa{X X\super w}

\subsubsection{Proto-Abazgi to Tsabal Abzhywa}{\it Nortaneous}, from Chirikba, Viacheslav A. (2003), ``Abkhaz". {\it Languages of the World/Materials} 119.

\ipa{tS\super w tS\super w' tC(') tC\super w(') dZ\super w d\textctz\ d\textctz\super w} \change\ \ipa{f p' ts(') tC\super w(') v dz d\textctz}\\
\ipa{C C\super w \textctz\ \textctz\super w} \change\ \ipa{s S\super w z Z\super w}\\
"V\ipa{Q} \ipa{Q}"V \change\ "\ipa{aa a}"\ipa{a}\\
\ipa{Q Q\super w} \change\ \ipa{a: 4}\\
\ipa{q q\super w} \change\ \ipa{X\super Q X\super Q\super w}

\subsubsection{Proto-Abazgi to Khaltsys Sadz}{\it Nortaneous}, from Chirikba, Viacheslav A. (2003), ``Abkhaz". {\it Languages of the World/Materials} 119.

\ipa{tS\super w(') tC(') tC\super w(') dZ\super w d\textctz\ d\textctz\super w} \change\ \ipa{f(') ts(') tC\super w(') v dz d\textctz\super w}\\
\ipa{C C\super w \textctz\ \textctz\super w} \change\ \ipa{s} \{\ipa{S\super w,C\super w}\} \ipa{z} \{\ipa{Z\super w,\textctz\super w}\}\\
"V\ipa{Q} \ipa{Q}"V \change\ "\ipa{aa a}"\ipa{a}\\
\ipa{Q\super w} \change\ \ipa{4}\\
\ipa{q q\super w} \change\ \ipa{X X\super w}

\subsubsection{Proto-Abazgi to Tswydzhy Sadz}{\it Nortaneous}, from Chirikba, Viacheslav A. (2003), ``Abkhaz". {\it Languages of the World/Materials} 119.

\ipa{tS\super w(') tC(') tC\super w(') dZ\super w d\textctz\ d\textctz\super w} \change\ \ipa{f(') ts(') tC\super w(') v dz d\textctz\super w}\\
\ipa{C \textctz} \change\ \ipa{s z}\\
"V\ipa{Q} \ipa{Q}"V \change\ "\ipa{aa a}"\ipa{a}\\
\ipa{Q\super w} \change\ \ipa{4}\\
\ipa{q q\super w} \change\ \ipa{X\super Q X\super w}

\subsection{Proto-Northwest Caucasian to Proto-Circassian}{\it Pogostick Man}, from Wikipedia contributors (2014), ``Proto-Northwest Caucasian language''. {\it Wikipedia, the Free Encyclopedia}. \textless\url{http://en.wikipedia.org/w/index.php?title=Proto-Northwest_Caucasian_language&oldid=596995618}\textgreater, presumably citing Starostin, Sergei A. and Sergei L. Nikolayev (1994), {\it A North Caucasian Etymological Dictionary}\\

\ipa{f} \change\ \ipa{x\super w}\\
\ipa{p\super j(:) p\super j' b\super j} \change\ \ipa{t(:) t' d}\\
\ipa{p\super w p\super w\super j p\super w\super j' b\super w b\super w\super j} \change\ \ipa{p t\super w t\super w' b d}\\
\ipa{p\super Q(:) p\super Q b\super Q} \change\ \ipa{p(:) p' b}\\
\ipa{m\super Q} \change\ \ipa{m}\\
\ipa{r l} \change\ \ipa{t: t\textcrh} / \#_\\
\ipa{l} \change\ \ipa{\textlyoghlig}\\
\ipa{t\super w(:) t\super w' d\super w} \change\ \ipa{t(:) t' d}\\
\ipa{t\super w\super j t\super w\super j'} \change\ \ipa{ts ts'}\\
\ipa{ts ts: dz} \change\ \{\ipa{s,c}\} \ipa{ts: dz}\raisebox{-0.6ex}{\textasciitilde}\ipa{z}\\
\ipa{ts\super j ts\super j: ts\super j' dz\super j} \change\ \ipa{(s) ts: ts' dz}\raisebox{-0.6ex}{\textasciitilde}\ipa{z}\\
\ipa{s\super j z\super j} \change\ \ipa{s z}\\
\ipa{ts\super w l\super j} \{\ipa{q\super j\super w,q\super Q\super w}\} \change\ \ipa{s\super w d q\super w} / \#_ (data not given for non-initial forms)\\
\ipa{ts\super j\super w ts\super j\super w dz\super j\super w} \change\ \ipa{ts\super j ts\super j' dz\super j}\\
\ipa{s\super j\super w z\super j\super w} \change\ \ipa{s\super j z\super j}\\
\ipa{tS tS' dZ d\textctz} \change\ \ipa{s ts\super j'(?) dz}\raisebox{-0.6ex}{\textasciitilde}\ipa{z d\textctz}\raisebox{-0.6ex}{\textasciitilde}\ipa{\textctz}\\
\ipa{S}(\raisebox{-0.6ex}{\textasciitilde}\ipa{S:}) \change\ \ipa{s}\\
\ipa{C(:) \textctz} \change\ \ipa{S(:) Z}\\
\ipa{tS\super w(:) tS\super w' dZ\super w} \change\ \ipa{tC(:) tC' d\textctz}\raisebox{-0.6ex}{\textasciitilde}\ipa{\textctz}\\
\ipa{S\super w(:) Z\super w} \change\ \ipa{S(:) Z}\\
\ipa{tc\super w(:) tC\super w'} \change\ \ipa{tS(:) tS'}\\
\ipa{C\super w C\super w \textctz\super w} \change\ \ipa{s\super j S: z\super j}\\
\ipa{\textbeltl(:) t\textbeltl(:) t\textbeltl' d\textlyoghlig} \change\ \ipa{C(:) tC(:) tC' t\textcrh}\\
\ipa{t\textbeltl\super j t\textbeltl\super j' d\textlyoghlig\super j} \change\ \ipa{tC t\textbeltl' G}\\
\ipa{t\textbeltl\super w(:) t\textbeltl\super w' d\textlyoghlig\super w} \change\ \ipa{tS(:) tS' \textcrh}\\
\ipa{\textbeltl\super w \textbeltl\super w:} \change\ \ipa{x(\super w) C:}\\
\ipa{t\textbeltl\super j\super w t\textbeltl\super j\super w t\textbeltl\super j\super w d\textlyoghlig\super j\super w} \change\ \ipa{x tC: tC' \textlyoghlig}\\
\ipa{\textbeltl\super j\super w}(\raisebox{-0.6ex}{\textasciitilde}\ipa{\textbeltl\super w:}) \ipa{\textlyoghlig\super j\super w} \change\ \ipa{x(\super w) G\super j}\\
\ipa{k k' g} \change\ \ipa{k\super j k\super j' g\super j}\\
\ipa{x\super j G\super j} \change\ \ipa{C \textctz}\\
\ipa{x\super w} \change\ \ipa{x(\super w)}\\
\ipa{g\super j\super w x\super j\super w G\super j\super w}(?) \change\ \ipa{g\super w x\super w K\super w}\\
\ipa{\;G} \change\ \ipa{K}\\
\ipa{q\super j' K\super j} \change\ \ipa{P K}\\
\ipa{q\super w} \change\ \ipa{q\super w:} / !_\\
\ipa{q\super w' \;G\super w} \change\ \ipa{q\super w: K\super w}\\
\ipa{q\super j\super w: q\super j\super w} \ipa{\;G\super j\super w,K\super j\super w}\} \ipa{X\super j\super w} \change\ \ipa{q\super w: P\super w K\super w X\super w}\\
\{\ipa{q\super Q:,q\super Q'}\} \ipa{X\super Q K\super Q} \change\ \ipa{q: X K}\\
\{\ipa{q\super Q\super j,X\super Q\super j}\} \ipa{q\super Q\super j K\super Q\super j} \{\ipa{q\super Q\super j: q\super Q\super j'}\} \change\ \ipa{\textcrh\ P j}\\
\{\ipa{q\super Q\super w:,q\super Q\super w'}\} \ipa{X\super Q\super w K\super Q\super w} \change\ \ipa{q\super w: X\super w K\super w}\\
\ipa{q\super Q\super j\super w} \{\ipa{q\super Q\super j\super w:,q\super Q\super j\super w}\} \ipa{K\super Q\super j\super w} \change\ \ipa{\textcrh\ P\super w w}\raisebox{-0.6ex}{\textasciitilde}\ipa{K\super w}

\subsubsection{Proto-Circassian to Adyghe}{\it Pogostick Man}, from Wikipedia contributors (2014), ``Proto-Circassian language". {\it Wikipedia, the Free Encyclopedia}, \textless\url{http://en.wikipedia.org/w/index.php?title=Proto-Circassian_language&oldid=591849172}\textgreater; and Wikipedia contributors (2014), ``Adyghe language". {\it Wikipedia, the Free Encyclopedia}. \textless\url{http://en.wikipedia.org/w/index.php?title=Adyghe_language&oldid=593857358}\textgreater

--- Stress changes:
\begin{tabular}{l c l}
"C\ipa{a}.C\ipa{a}& \change &C\ipa{a:}C\\
"C\ipa{a}.C\ipa{@}& \change &C\ipa{a}C\\
"C\ipa{@}.C\ipa{a}& \change &C\ipa{@}C\\
"C\ipa{@}.C\ipa{@}& \change &C\ipa{@}C\\
C\ipa{a}."C\ipa{a}& \change &C\ipa{a:}.C\ipa{a}\\
C\ipa{a}."C\ipa{@}& \change &C\ipa{a}.C\ipa{@}\\
C\ipa{@}."C\ipa{a}& \change &C\ipa{@}.C\ipa{a}\\
C\ipa{@}."C\ipa{@}& \change &C\ipa{@}.C\ipa{@}\end{tabular}

--- Consonant correspondences:\\
\ipa{ts\super j} \change\ \ipa{tC}\\
\ipa{ts\super w} \change\ \ipa{ts\super j\super w}\\
\ipa{tS tC} \change\ \ipa{S \:s}\\
\ipa{P(\super w)}\raisebox{-0.6ex}{\textasciitilde}\ipa{q'(\super w)} \change\ \ipa{P(\super w)}\\
\ipa{d\textlyoghlig} \change\ \ipa{G}\\
\ipa{dz\super j dz\super w} \change\ \ipa{d\textctz\ \textctz\super w}\\
\ipa{ts\super j'} \change\ \ipa{C'}\raisebox{-0.6ex}{\textasciitilde}\ipa{S'}\\
\ipa{s\super w C} \change\ \ipa{C\super w}\raisebox{-0.6ex}{\textasciitilde}\ipa{S\super w C}\raisebox{-0.6ex}{\textasciitilde}\ipa{S}\\
\ipa{x\super w X\super j} \change\ \ipa{f}\raisebox{-0.6ex}{\textasciitilde}\ipa{F}? \ipa{\textcrh}\\
\ipa{\textlyoghlig} \change\ \ipa{l}\\
\ipa{z\super w} \change\ \ipa{\textctz\super w}\raisebox{-0.6ex}{\textasciitilde}\ipa{Z\super w}\\

\paragraph{Adyghe to Abadzekh Adyghe}{\it Pogostick Man}, from Wikipedia contributors (2014), ``Proto-Circassian language". {\it Wikipedia, the Free Encyclopedia}, \textless\url{http://en.wikipedia.org/w/index.php?title=Proto-Circassian_language&oldid=591849172}\textgreater; and Wikipedia contributors (2014), ``Adyghe language". {\it Wikipedia, the Free Encyclopedia}. \textless\url{http://en.wikipedia.org/w/index.php?title=Adyghe_language&oldid=593857358}\textgreater

\ipa{ts\super j(:)} \change\ \ipa{tC}\\
\ipa{ts\super w} \change\ \ipa{tS\super w}\\
\ipa{p: t: ts: ts\super w S: tS: tC: k\super j: k\super w: q:}\raisebox{-0.6ex}{\textasciitilde}\ipa{q\;X q\super w:}\raisebox{-0.6ex}{\textasciitilde}\ipa{q\;X\super w} \change\ \ipa{p t ts tS\super w S C tC tS k\super w q: q\super w}\\
\ipa{tS' t\textbeltl'} \change\ \ipa{Paj}\raisebox{-0.6ex}{\textasciitilde}\ipa{P \textbeltl'}\\
\ipa{tS\super w'}\raisebox{-0.6ex}{\textasciitilde}\ipa{S\super w'} \change\ \ipa{C\super w'}\raisebox{-0.6ex}{\textasciitilde}\ipa{S\super w'}\\
\ipa{k\super j'} \change\ \ipa{tS'}\\
\ipa{\:s: S:} \change\ \ipa{\:s S}\\
\ipa{s' S'}\raisebox{-0.6ex}{\textasciitilde}\ipa{C'} \change\ \ipa{ts' S'}

\paragraph{Adyghe to Bzhedug Adyghe}{\it Pogostick Man}, from Wikipedia contributors (2014), ``Proto-Circassian language". {\it Wikipedia, the Free Encyclopedia}, \textless\url{http://en.wikipedia.org/w/index.php?title=Proto-Circassian_language&oldid=591849172}\textgreater; and Wikipedia contributors (2014), ``Adyghe language". {\it Wikipedia, the Free Encyclopedia}. \textless\url{http://en.wikipedia.org/w/index.php?title=Adyghe_language&oldid=593857358}\textgreater

\ipa{ts\super j ts\super j:} \change\ \ipa{tC tC:}\\
\ipa{k\super j(:) k\super j' g\super j} \change\ \ipa{tS(:) tS' d\textctz}\\
\ipa{ts\super w:} \change\ \ipa{ts\super j\super w:}\\
\ipa{q:}\raisebox{-0.6ex}{\textasciitilde}\ipa{\;X q\super w:}\raisebox{-0.6ex}{\textasciitilde}\ipa{q\;X\super w} \change\ \ipa{q: q\super w:}\\
\ipa{ts\super w'}\raisebox{-0.6ex}{\textasciitilde}\ipa{S\super w'} \change\ \ipa{C\super w'}\raisebox{-0.6ex}{\textasciitilde}\ipa{S\super w'}\\
\ipa{t\textbeltl'} \change\ \ipa{\textbeltl'}\\
\ipa{s' S'}\raisebox{-0.6ex}{\textasciitilde}\ipa{C'} \change\ \ipa{ts' S'}

\paragraph{Adyghe to Shapsug Adyghe}{\it Pogostick Man}, from Wikipedia contributors (2014), ``Proto-Circassian language". {\it Wikipedia, the Free Encyclopedia}, \textless\url{http://en.wikipedia.org/w/index.php?title=Proto-Circassian_language&oldid=591849172}\textgreater; and Wikipedia contributors (2014), ``Adyghe language". {\it Wikipedia, the Free Encyclopedia}. \textless\url{http://en.wikipedia.org/w/index.php?title=Adyghe_language&oldid=593857358}\textgreater

\ipa{ts\super j(:) ts\super w tS tC} \change\ \ipa{tC tS\super w S \:s}\\
\ipa{p: t: ts: ts\super w: S: ts: tc: k\super j: k\super w: q:}\raisebox{-0.6ex}{\textasciitilde}\ipa{qX q\super w:}\raisebox{-0.6ex}{\textasciitilde}\ipa{qX\super w} \change\ \ipa{p t ts tS\super w S tS tC k\super j k\super w X}\raisebox{-0.6ex}{\textasciitilde}\ipa{q X\super w}\raisebox{-0.6ex}{\textasciitilde}\ipa{q\super w}\\
\ipa{p(\super w)' t(\super w)' ts' ts\super w'}\raisebox{-0.6ex}{\textasciitilde}\ipa{S\super w'} \change\ \ipa{p\super Q t\super Q ts\super Q \:s\super w}\\
\ipa{t\textbeltl'} \change\ \ipa{\textbeltl\super Q}\\
\ipa{\:s: S:} \change\ \ipa{\:s S}\\
\ipa{s' S'}\raisebox{-0.6ex}{\textasciitilde}\ipa{C'} \change\ \ipa{s\super Q \:s}

\paragraph{Adyghe to Temirgoy Adyghe}{\it Pogostick Man}, from Wikipedia contributors (2014), ``Proto-Circassian language". {\it Wikipedia, the Free Encyclopedia}, \textless\url{http://en.wikipedia.org/w/index.php?title=Proto-Circassian_language&oldid=591849172}\textgreater; and Wikipedia contributors (2014), ``Adyghe language". {\it Wikipedia, the Free Encyclopedia}. \textless\url{http://en.wikipedia.org/w/index.php?title=Adyghe_language&oldid=593857358}\textgreater

\ipa{ts\super j(:) ts\super w} \change\ \ipa{tC ts\super j\super w}\\
\ipa{k\super j(:) k\super j' k\super w: g\super j} \change\ \ipa{tS tS' k\super w d\textctz}\\
\ipa{q q\super w} \change\ \ipa{q: q\super w:} / ! \#_\\
\ipa{p: t: ts: ts\super w: S: tS: tC:} \change\ \ipa{p t ts ts\super w\super j S tS tC}\\
\ipa{q:}\raisebox{-0.6ex}{\textasciitilde}\ipa{q\;X q\super w:}\raisebox{-0.6ex}{\textasciitilde}\ipa{q\;X\super w} \change\ \ipa{q: q\super w:}\\
\ipa{ts\super j' ts\super w'}\raisebox{-0.6ex}{\textasciitilde}\ipa{S\super w'} \change\ \ipa{C'}\raisebox{-0.6ex}{\textasciitilde}\ipa{S' C\super w'}\raisebox{-0.6ex}{\textasciitilde}\ipa{S\super w'}\\
\ipa{t\textbeltl'} \change\ \ipa{\textbeltl'}\\
\ipa{\:s: S:} \change\ \ipa{\:s S}\\
\ipa{G} \change\ \ipa{G}\raisebox{-0.6ex}{\textasciitilde}\ipa{g}\\
\ipa{s' S'}\raisebox{-0.6ex}{\textasciitilde}\ipa{C'} \change\ \ipa{ts' S'}

\subsubsection{Proto-Circassian to Kabardian}{\it Pogostick Man}, from Wikipedia contributors (2014), ``Proto-Circassian language". {\it Wikipedia, the Free Encyclopedia}, \textless\url{http://en.wikipedia.org/w/index.php?title=Proto-Circassian_language&oldid=591849172}\textgreater

--- Stress changes:
\begin{tabular}{l c l}
"C\ipa{a}.C\ipa{a}& \change &C\ipa{a:}.C\ipa{a}\\
"C\ipa{a}.C\ipa{@}& \change &C\ipa{a}C\\
"C\ipa{@}.C\ipa{a}& \change &C\ipa{@}.C\ipa{a}\\
"C\ipa{@}.C\ipa{@}& \change &C\ipa{@}C\\
C\ipa{a}."C\ipa{a}& \change &C\ipa{a:}.C\ipa{a}\\
Ca."C\ipa{@}& \change &C\ipa{a}C\\
C\ipa{@}."C\ipa{a}& \change &C\ipa{@}.C\ipa{a}\\
C\ipa{@}."C\ipa{@}& \change &C\ipa{@}C\end{tabular}

--- Consonant correspondences:\\
\ipa{ts\super j(:) ts\super w} \{\ipa{tS,tC}\} \change\ \ipa{C f S}\\
\ipa{k\super j} \change\ \ipa{tS}\\
\ipa{P}\raisebox{-0.6ex}{\textasciitilde}\ipa{q'} \change\ \ipa{P\super w}\\
\ipa{p: t: ts: ts\super w: S:} \{\ipa{tS:,tC:}\} \ipa{k\super w: q:}\raisebox{-0.6ex}{\textasciitilde}\ipa{qX q\super w:}\raisebox{-0.6ex}{\textasciitilde}\ipa{qX\super w} \change\ \ipa{b d dz v C Z dZ g\super w q'}\raisebox{-0.6ex}{\textasciitilde}\ipa{qX q\super w'}\raisebox{-0.6ex}{\textasciitilde}\ipa{qX\super w}\\
\ipa{d\textlyoghlig\ dz\super j dz\super w d\textctz\ g\super j} \change\ \ipa{Z \textctz} \{\ipa{v,w}\} \ipa{Z dZ}\\
\ipa{ts\super j' tS\super w'}\raisebox{-0.6ex}{\textasciitilde}\ipa{S\super w' tS' tC' t\textbeltl' k\super j'} \change\ \ipa{C' f' C'} \{\ipa{C',c\c{c}'}\} \ipa{\textbeltl' tS'}\\
\ipa{s\super w z\super w} \{\ipa{S,\:s}\} \{\ipa{\textctz,\:z,G\super j}\} \ipa{\textctz\ X\super j} \change\ \ipa{f v C \textctz\ \textctz}\raisebox{-0.6ex}{\textasciitilde}\ipa{Z X}\\
\ipa{\:s: S:} \change\ \ipa{C Z}\\
\ipa{s'(?) S'}\raisebox{-0.6ex}{\textasciitilde}\ipa{C'} \change\ \ipa{ts' C'}

\subsection{Proto-Northwest Caucasian to Ubykh}{\it Pogostick Man}, from Wikipedia contributors (2014), ``Proto-Northwest Caucasian language''. {\it Wikipedia, the Free Encyclopedia}. \textless\url{http://en.wikipedia.org/w/index.php?title=Proto-Northwest_Caucasian_language&oldid=596995618}\textgreater, presumably citing Starostin, Sergei A. and Sergei L. Nikolayev (1994), {\it A North Caucasian Etymological Dictionary}\\

\ipa{p\super j(:) b\super j} \change\ \ipa{t(:) d}\\
\ipa{p\super j'} \change\ \ipa{t\super w'}\\
\{\ipa{p\super w,b\super w}\} \change\ \ipa{f}\\
\ipa{p\super j:} \change\ \ipa{t\super w}\raisebox{-0.6ex}{\textasciitilde}\ipa{d\super w}\\
\ipa{p\super w\super j b\super w\super j} \change\ \ipa{t\super w d\super w}\\
\ipa{p\super w\super j' p\super Q p\super Q: p\super Q' b\super Q} \change\ \ipa{t\super w' v\super Q b\super Q p\super Q' b\super Q}\\
\ipa{t(\super w): t\super w\super j'} \change\ \ipa{t(\super w) t\super w'}\\
\{\ipa{r,l}\} \ipa{l\super j} \change\ \ipa{d r} / \#_\\
\ipa{l l\super j} \change\ \O\raisebox{-0.6ex}{\textasciitilde}\ipa{j l}\raisebox{-0.6ex}{\textasciitilde}\ipa{G\super j}\\
\ipa{r r\super j} \change\ \ipa{r}\raisebox{-0.6ex}{\textasciitilde}\ipa{K \textlyoghlig}\\
\ipa{ts(\super j): ts\super w: dz\super j} \change\ \ipa{ts ts\super w dz}\\
\ipa{z(\super j) z\super w} \change\ \ipa{dz(\super j)}\raisebox{-0.6ex}{\textasciitilde}\ipa{z(\super j) dz\super w}\raisebox{-0.6ex}{\textasciitilde}\ipa{z\super w}\\
\ipa{s\super w\super j z\super w\super j ts\super w\super j' dz\super w\super j} \change\ \ipa{tS\super w Z\super w tS\super j' dZ\super j}\\
\ipa{tS(') dZ} \change\ \ipa{ts(') dz}\\
\ipa{S}(\raisebox{-0.6ex}{\textasciitilde}\ipa{S:}) \ipa{Z} \change\ \ipa{s z}\\
\ipa{C: tC:} \change\ \ipa{C tC}\\
\ipa{S\super w(:) Z\super w tS\super w(') dZ} \change\ \ipa{S Z tS(') dZ}\\
\ipa{tC\super w(:) tC\super w'} \change\ \ipa{tC tC'}\\
\ipa{C\super w C\super w: \textctz\super w} \change\ \ipa{S\super w s\super w Z\super w}\\
\ipa{t\textbeltl\ t\textbeltl: t\textbeltl' d\textlyoghlig} \change\ \ipa{C (s\super j) ts\super j' \textlyoghlig}\\
\ipa{\textbeltl(:)} \change\ \ipa{s\super j}\\
\ipa{ t\textbeltl\super j t\textbeltl\super j' d\textlyoghlig\super j} \change\ \ipa{C t\textbeltl' C K}(\raisebox{-0.6ex}{\textasciitilde}\ipa{z\super j})\\
\ipa{\textbeltl\super j}(\raisebox{-0.6ex}{\textasciitilde}\ipa{\textbeltl\super j':}) \change\ \ipa{\textbeltl}\\
\ipa{\textlyoghlig} \change\ \ipa{\textctz}\\
\ipa{t\textbeltl'} \change\ \ipa{ts\super j'}\\
\{\ipa{\textlyoghlig\super j,l\super j}\} \change\ \ipa{\textlyoghlig}\\
\ipa{t\textbeltl\super w(:) t\textbeltl\super w' d\textlyoghlig\super w} \change\ \ipa{ts\super w ts(\super w)' w}\\
\ipa{\textbeltl\super w \textbeltl\super w:} \change\ \ipa{s\super w s(\super w)}\\
\ipa{t\textbeltl\super w\super j t\textbeltl\super w\super j: t\textbeltl\super w\super j' d\textlyoghlig\super w\super j} \change\ \ipa{f d\textctz\ ts' dZ}\\
\ipa{\textbeltl\super w\super j}(\raisebox{-0.6ex}{\textasciitilde}\ipa{\textbeltl\super w:}) \ipa{\textlyoghlig\super w\super j} \change\ \ipa{S\super w Z\super w}\\
\ipa{k k' g x G} \change\ \ipa{k\super j k\super j' g\super j C G}\raisebox{-0.6ex}{\textasciitilde}\ipa{K}\\
\ipa{x\super j G\super j} \change\ \ipa{s\super j z\super j}\\
\ipa{k\super w: x\super w} \change\ \ipa{g\super w x}\\
\ipa{k\super j\super w: x\super j\super w G\super j\super w}(?) \change\ \ipa{g\super j k\super j' x\super j K\super j}\\
\ipa{\:G} \change\ \ipa{K}\\
\ipa{q\super j:(') \;G\super j X\super j} \change\ \ipa{q\super j(') K\super j x\super j}\\
\ipa{q\super w: \;G\super w} \change\ \ipa{q\super w K\super w}\\
\ipa{q\super j\super w q\super j\super w: q\super j\super w' \;G\super j\super w X\super j\super w K\super j\super w} \change\ \ipa{x\super j q\super j q\super j' K\super j X\super j K\super j}\\
\ipa{q\super Q\super j} \change\ \ipa{q(\super Q)}\\
\ipa{q\super Q\super j\super w} \{\ipa{q\super Q\super j\super w:,q\super Q\super j\super w}\} \ipa{K\super Q\super j\super w} \change\ \ipa{X\super w q\super w' w}

\clearpage

\section{Chumashan}Klar (1977) reconstructs the following phonemic inventory for Proto-Chumashan:

\begin{center}\begin{tabular}{c | c c c c c c}
& Bilabial & Dental & Palatoalveolar & Velar & Uvular & Glottal\\ \hline
Nasal & \ipa{m \super Pm} & \ipa{n \super Pn}\\
Stop & \ipa{p p'} & \ipa{t t'} & & \ipa{k k'} & \ipa{q q'} & \ipa{P}\\
Affricate & & \ipa{ts ts'} & \ipa{tS tS'}\\
Fricative & & \ipa{s (s')} & \ipa{S (S')} & & & \ipa{h}\\
Approximant & \ipa{w \super Pw} & \ipa{l \super Pl} & \ipa{j \super Pj}
\end{tabular}

\begin{tabular}{c | c c c}
& Front & Central & Back\\ \hline
High & \ipa{i} & \ipa{1} & \ipa{u}\\
Mid & \ipa{e} & & \ipa{o}\\
Low & & \ipa{a}\end{tabular}\end{center}

Ablaut and vowel harmony appear to have been productive in the proto-language; it is possible that consonant harmony affecting sibilants was also productive. *\ipa{1} may have been a loan phoneme.

({\it CatDoom}, from Klar, Kathryn (1977), {\it Topics in Historical Chumash Grammar}. \textless\url{http://linguistics.berkeley.edu/~survey/documents/dissertations/klar-1977.pdf}\textgreater)

\subsection{Proto-Chumash to Barbare\~{n}o}{\it CatDoom}, from Klar, Kathryn (1977), {\it Topics in Historical Chumash Grammar}. \textless\url{http://linguistics.berkeley.edu/~survey/documents/dissertations/klar-1977.pdf}\textgreater

R[- glottalized]V\ipa{\super P}R \change\ \ipa{\super P}RVR[- glottalized] / _\$\\
R[- glottalized]VO\ipa{'} \change\ \ipa{\super P}RVO[- ejective] / _\$

\subsection{Proto-Chumash to Cruze\~{n}o}{\it CatDoom}, from Klar, Kathryn (1977), {\it Topics in Historical Chumash Grammar}. \textless\url{http://linguistics.berkeley.edu/~survey/documents/dissertations/klar-1977.pdf}\textgreater

\ipa{k} \change\ \ipa{tS} ``(`in certain cases')''

\subsection{Proto-Chumash to Inse\~{n}o}{\it CatDoom}, from Klar, Kathryn (1977), {\it Topics in Historical Chumash Grammar}. \textless\url{http://linguistics.berkeley.edu/~survey/documents/dissertations/klar-1977.pdf}\textgreater

\ipa{t' q'} \change\ \ipa{t q}\\
\ipa{\super P}N \ipa{\super Pw} \change\ N \ipa{w}\\

\subsection{Proto-Chumash to Obispe\~{n}o}{\it CatDoom}, from Klar, Kathryn (1977), {\it Topics in Historical Chumash Grammar}. \textless\url{http://linguistics.berkeley.edu/~survey/documents/dissertations/klar-1977.pdf}\textgreater

S\ipa{'} \change\ \ipa{P}\\
\ipa{q k} \change\ \{\ipa{q,k}\} \{\ipa{k(S),t\super j}\} (allophonic)\\
\{\ipa{\super Pm,\super Pn}\} \change\ \{\O,\ipa{P}\} (the former is more likely)\\
\ipa{\super Pw} \change\ \ipa{w} (may have remained glottalized)\\
\ipa{\super Pj} \change\ \O

\subsection{Proto-Chumash to Purisime\~{n}o}{\it CatDoom}, from Klar, Kathryn (1977), {\it Topics in Historical Chumash Grammar}. \textless\url{http://linguistics.berkeley.edu/~survey/documents/dissertations/klar-1977.pdf}\textgreater

\ipa{\super Pj} \change\ \O\\
\ipa{q'} \change\ \ipa{q}

\subsection{Proto-Chumash to Venture\~{n}o}{\it CatDoom}, from Klar, Kathryn (1977), {\it Topics in Historical Chumash Grammar}. \textless\url{http://linguistics.berkeley.edu/~survey/documents/dissertations/klar-1977.pdf}\textgreater

\ipa{P} \change\ \O\ / _\#\\
\ipa{p' k' q'} \change\ \ipa{p k q}\\
\ipa{\super Pm \super Pn \super Pl \super Pj} \change\ \ipa{m n l j}

\clearpage

\section{Elamo-Dravidian}\tab McAlpin (1974) reconstructs Proto-Elamo-Dravidian as having the following phonemic inventory; the following table is slightly modified for reasons to be explained.

\begin{center}
\begin{tabular}{c | c c c c}
& Bilabial & Alveolar & Palatal & Velar\\ \hline
Nasal & \ipa{m m:} & \ipa{n n:}\\
Plosive & \ipa{p} & \ipa{t t:} & \ipa{c c:} & \ipa{k k:}\\
Fricative & \ipa{v} (?) & \ipa{s}\\
Liquid & & \ipa{\`{r} \'{r} l l:} & \ipa{j} & \ipa{w}
\end{tabular}

\begin{tabular}{c | c c c}
& Front & Center & Back\\ \hline
High & \ipa{i} & & \ipa{u}\\
Mid & \ipa{e} & & \ipa{o}\\
Low & & \ipa{a}\end{tabular}
\end{center}

\tab What here is denoted *\ipa{s} the author has *\v{s} for, but no other sibilant is readily identifiable in his paper. He makes mention of language written in cuneiform which may have influenced this convention. The phonemes *\`{r} and *\'{r} seem to have been contrastive rhotics. In *NS clusters, the nasal appears to have assimilated to the following stop.

\tab (From McAlpin, David W. (1974), ``Toward Proto-Elamo-Dravidian". {\it Language} 50(1):89 -- 101)

\subsection{Proto-Elamo-Dravidian to Proto-Dravidian}{\it Pogostick Man}, from McAlpin, David W. (1974), ``Toward Proto-Elamo-Dravidian". {\it Language} 50(1):89 -- 101

\ipa{w} \change\ \ipa{v} / \#_\{\ipa{i,e}\}\ipa{l}V\\
\ipa{w} \change\ \ipa{v} / V_\\
\ipa{k Sk} \change\ \ipa{k}* \ipa{k:} / V_V (the asterisk-marked \ipa{k} is what McAlpin terms ``weak {\it k}", which tends to drop out in morphology)\\
\ipa{t} \change\ \O\ / \#_V\ipa{r}C\\
\ipa{t} \change\ \{\ipa{t,\:t}\} / V_V\\
\ipa{rt} \change\ \ipa{\:t} / V_V\\
\ipa{p} \change\ \ipa{v} / V_V\\
\ipa{s} \change\ \ipa{t} / \#_VLV\\
\ipa{s} \change\ \ipa{j} / V_\{V,\#\}\\
\ipa{s} \change\ \O\ / \#V_\{\ipa{\`{r},l}\}\\
\ipa{s} \change\ \O\ / C_V\\
\ipa{s} \change\ \O\ / V_C\\
\ipa{\`{r}} \change\ \ipa{\=*r}\\
\ipa{\'{r}} \change\ \ipa{r} / V_V\\
\ipa{n n: rn} \change\ \{\ipa{\=*n,r}\} \ipa{\=*n(:) \:n} / V_V\\
\ipa{n:} \change\ \ipa{\=*n(:)}\\
\ipa{n\`{r}} \change\ \ipa{\=*n\=*r}\\
\ipa{l l:} \change\ \{\ipa{l,\:l}\} \ipa{\:l(:)} / V_V\\
\ipa{l} \change\ \ipa{\:l} / V_\#\\
Proto-Dravidian retained long vowels, possibly from the simplification of consonant clusters and/or deletion of intervocalic consonants with compensatory lengthening and/or the resulting vowels in hiatus merging

\subsection{Proto-Elamo-Dravidian to Achaemanid Elamite}{\it Pogostick Man}, from McAlpin, David W. (1974), ``Toward Proto-Elamo-Dravidian". {\it Language} 50(1):89 -- 101

\{\ipa{i,e,u}\} \change\ \O\ / \#_\{\ipa{t,n}\}\ipa{a}\\
\ipa{e} \change\ \{\ipa{e,i}\} / \#C_C\\
\ipa{w} \change\ \ipa{\'{u}} / V_ (McAlpin uses the accented-vowel notation due to some apparent height-contrast neutralizations before /\ipa{a}/)\\
\ipa{k Nk Nk:} \change\ \O\ \ipa{k k:} / V_V\\
\ipa{mp} \change\ \ipa{p(:)} / V_V\\
\ipa{c} \change\ \ipa{s} / \#_\{\ipa{a,u}\}\\
\ipa{\textltailn c} \change\ \ipa{ns} / V_V\\
\ipa{\`{r}} \change\ \ipa{r} / V_\{V,C\}\\
\ipa{\'{r}} \change\ \ipa{r:} / V_V\\
\ipa{n\`{r}} \change\ \ipa{nr}\\
\ipa{l} \change\ \ipa{n} / V_\#\\
\ipa{v} \change\ \ipa{m} / \#_V (?)

\subsection{Tamil}

\subsubsection{Standard Tamil to Colloquial Tamil}{\it schwatever}, from Shiffman, Harold F. {\it A Reference Grammar on Spoken Tamil}

\ipa{aj} \change\ \ipa{e:} ``(exception: never finally in monosyllables, never initially in multisyllabic words)"\\
\ipa{avu aji} \change\ \ipa{aw aj}\\
\ipa{i u} \change\ \ipa{e o} / _C\ipa{a}\\
\{\ipa{k,v}\} \change\ \O\ / V_V\\
\ipa{a: e: i: o: u:} \change\ \ipa{a E i o u} / _\#\\
\ipa{am an} \{\ipa{a:m,a:n}\} \change\ \ipa{\~{o} \~{\ae} \~{a}} / _\#\\
\{\ipa{om,on}\} \{\ipa{em,en}\} \{\ipa{o:m,o:n}\} \{\ipa{e:m,e:n}\} \change\ \ipa{\~{O} \~{E} \~{o} \~{e}}\\
\ipa{um} \change\ \ipa{\~{u}} / _\#\\
\O\ \change\ \ipa{1} / _N\#\\
\{\ipa{\:l,\:r}\} \change\ \O\ (sporadic, the latter very much so and contributing some compensatory lengthening)\\
\ipa{l \:l} \change\ \ipa{l:0 \:l:0} / _\#(C)V[-long]\\
\O\ \change\ \ipa{0} / \{\ipa{l,\:l}\}_\# if \{M,V\ipa{:}\} previously in the lexeme\\
\ipa{r} \change\ \ipa{R} ``in most dialects''\\
\ipa{\:r} \change\ \ipa{\:l}\\
\{\ipa{R,l,\:l}\} \change\ \O\ / V_S\\
\ipa{i u} \change\ \ipa{1 0} / short only when unstressed ! in \#U\\
\ipa{1 0} \change\ \O\ / ! _\#\\
\O\ \change\ \{\ipa{1,0}\} / to break up clusters\\
\ipa{n} \change\ \ipa{N} / _\{\ipa{k,g}\}\\
\ipa{i(:) e(:)} \change\ \ipa{u(:) o(:)} / \{\ipa{m,v,p}\}_\d{C}\\
\ipa{j} \change\ \O\ / V[-front]_\#\\
\ipa{j} \change\ \ipa{j:i} / E_\#\\
\ipa{t: nt} \change\ \ipa{c: \textltailn c} / \{\ipa{i,j}\}_\\
\ipa{\:tk} \change\ \ipa{k:}\\
\ipa{n t:} \change\ \ipa{\|[n \|[t:} \\ %]]
\ipa{\:n} \change\ \ipa{n} ``(sporadic and dialect development)''\\
\ipa{\:l} \change\ \ipa{l} ``(again, sporadic)"\\
\ipa{c} \change\ \ipa{s} / _\{\ipa{a,o,u,e}\}\\
\ipa{c:} \change\ \ipa{tS:} ``(most dialects)"\\
\ipa{o e} \change\ \ipa{u i} / _C\{\ipa{u,i}\} ``(highly sporadic)"\\
``There's also only a few changes necessary to turn this into the British dialect (which didn't merge retroflexes with alveolars):"\\\
--- \ipa{i(:) e(:)} \change\ \ipa{u(:) o(:)} / _\ipa{\:l}\\
--- \ipa{e}C\ipa{@ o}C\ipa{@} \change\ C\ipa{e:} C\ipa{o:} / \#_

\clearpage

\section{Eskimo-Aleut}\tab The following phonological reconstruction of Proto-Eskimo-Aleut is adapted from Wikipedia.
\begin{center}
\begin{tabular}{c | c c c c c}
& Labial & Alveolar & Palatal & Velar & Uvular \\ \hline
Nasal & \ipa{m} & \ipa{n (n\super j)} & & \ipa{N} & \\
Plosive & \ipa{p} & \ipa{t t\super j} & & \ipa{k} & \ipa{q}\\
Fricative/Affricate & \ipa{v} & \ipa{D c s\super j} & & \ipa{G} & \ipa{K}\\
Lateral Fricative & & (\ipa{\textbeltl}) & & & \\
Approximant & & \ipa{l} & \ipa{j} & &\end{tabular}

\begin{tabular}{c | c c c}
& Front & Central & Back \\ \hline
High & \ipa{i} & & \ipa{u}\\
Mid & & \ipa{@} & \\
Low & & \ipa{a} \end{tabular}\end{center}

\tab It is noted that *\ipa{n} and *\ipa{n\super j} may not have been distinct phonemes; the article cites Fortescue mentioning that Sirilenski Eskimo has instances of initial /\ipa{j}/ whereas others have /\ipa{n}/; that *\ipa{c} *\ipa{s\super j} may have been either fricatives (*\ipa{s} *\ipa{s\super j}) or affricates (*\ipa{ts} *\ipa{ts\super j}), the source being unclear; and that *\ipa{\textbeltl} may have actually arisen from *\ipa{l} + plosive combinations.

\tab (From Wikipedia contributors (2013), ``Proto-Eskimo--Aleut language''. {\it Wikipedia, the Free Encyclopedia}. \textless\url{http://en.wikipedia.org/w/index.php?title=Proto-Eskimo%E2%80%93Aleut_language&oldid=573345407}\textgreater)


\subsection{Proto-Eskimo-Aleut to Proto-Aleut}{\it Pogostick Man}, from Marsh, Gordon and Morris Swadesh (1951), \textquotedblleft Kleinschmidt Centennial V: Eskimo Aleut Correspondences", \textit{International Journal of American Linguistics}, Vol. 17, No. 4 (Oct., 1951), pp. 209 -- 216

\ipa{a} \change\ \ipa{i} / \ipa{i}_ \\
\ipa{u} \textrightarrow\hspace{0pt} \ipa{a} / \ipa{a}_ \\
\ipa{p} \textrightarrow\hspace{0pt} \ipa{h} / \#_ \\
\ipa{v} \textrightarrow\hspace{0pt} \ipa{m} / medial \\
\ipa{v} \textrightarrow\hspace{0pt} \ipa{w} / \ipa{a}_\ipa{a} (in eastern dialects) \\
\{\textipa{t,D}\} \textrightarrow\hspace{0pt} \ipa{n} / _\# \\
\textipa{D} \textrightarrow\hspace{0pt} \ipa{t} / else \\
\O\ \change\ \ipa{t} / \#_\ipa{s} \\
\ipa{z} \textrightarrow\hspace{0pt} \ipa{s} / \#_ \\
\ipa{z} \textrightarrow\hspace{0pt} \textipa{D} / medial \\
\textipa{\r*l} \textrightarrow\hspace{0pt} \ipa{l} \\
\ipa{m} \textrightarrow\hspace{0pt} \ipa{w} / \#_ \\
\ipa{n} \textrightarrow\hspace{0pt} \ipa{t} / \#_ (except, maybe, \textquotedblleft in exclamations") \\
\textipa{dZ} \textrightarrow\hspace{0pt} \textipa{D} / \ipa{i}_ (in eastern and central dialects) \\
\textipa{dZ} \textrightarrow\hspace{0pt} \textipa{D} / \ipa{u}_\ipa{a} (in eastern dialects) \\
\ipa{i} \change \O\ / \#_\{\textipa{z,dZ}\} \\
\textipa{@} \textrightarrow\hspace{0pt} \O\ / \#_, \textquotedblleft under certain conditions not yet discovered" \\
Deletion of medial vowels as per stress rules, \textquotedblleft mostly affecting vowels before the accented syllable" \\
\ipa{n}V$_1$\ipa{n}V$_2$ \textrightarrow\ \ipa{n}V$_2$\ipa{n}V$_2$

\subsection{Proto-Eskimo-Aleut to Proto-Eskimo}{\it Pogostick Man}, from Marsh, Gordon and Morris Swadesh (1951), \textquotedblleft Kleinschmidt Centennial V: Eskimo Aleut Correspondences", \textit{International Journal of American Linguistics}, Vol. 17, No. 4 (Oct., 1951), pp. 209 -- 216

\textipa{D z} \textrightarrow\hspace{0pt} \ipa{t s} \\
\textipa{G K} \textrightarrow\hspace{0pt} \ipa{k q} / \#_ \\
\textipa{@} \textrightarrow\hspace{0pt} \O\ / t_, \textquotedblleft in certain positions"

\subsubsection{Proto-Eskimo to Barrow I\~{n}upiaq}{\it Pogostick Man}, from Swadesh, Morris (1952), \textquotedblleft Unaaliq and Proto Eskimo IV: Diachronic Notes", \textit{International Journal of American Linguistics}, Vol. 18, No. 3 (Jul., 1952), pp. 166--171

\textipa{\r*l} \textrightarrow\hspace{0pt} l / medial \\
t \textrightarrow\hspace{0pt} s / i_ \\
\textipa{@} \textrightarrow\hspace{0pt} i / at word boundaries \\
\textipa{@} \textrightarrow\hspace{0pt} u / u_ \\
\textipa{@} \textrightarrow\hspace{0pt} a / a_ \\
\textipa{@} \textrightarrow\hspace{0pt} \O\hspace{0pt} / else \\
\textipa{G K} \textrightarrow\hspace{0pt} k q / _\# \\
C$_0$VC$_0$ \textrightarrow\hspace{0pt} C$_0$\textipa{:} \\
Regressive MOA assimilation and progressive voicing assimilation in consonant clusters (at least, when C$_2$ is either /l/ or /\textipa{\r*l}/) \\
m n \textipa{N} \textrightarrow\hspace{0pt} v t \textipa{G} / _C[-nasal] \\
\textipa{\r*l} \textrightarrow\hspace{0pt} t / _C \\
v \textrightarrow\hspace{0pt} p / _s \\
v \textipa{K} \textrightarrow\hspace{0pt} p q / S_ \\
v \textipa{G K} \textrightarrow\hspace{0pt} p k q / _C (unless C = one of /l d\textipa{Z} m n \textipa{N}) \\
\{p,v\} t \{k,\textipa{G}\} \textrightarrow\hspace{0pt} m n \textipa{N} / _N \\
v \textrightarrow\hspace{0pt} \O\hspace{0pt} / u_i \\
d\textipa{Z} \textrightarrow\hspace{0pt} \textipa{K} / i_u \\
\textipa{@dZ} \textrightarrow\hspace{0pt} i / _\{a,u\} (except in \#U) \\
ad\textipa{Z} \textrightarrow\hspace{0pt} i / _a (except in \#U?)

\subsubsection{Proto-Eskimo to Greenlandic I\~{n}upiaq}{\it Pogostick Man}, from Swadesh, Morris (1952), \textquotedblleft Unaaliq and Proto Eskimo IV: Diachronic Notes", \textit{International Journal of American Linguistics}, Vol. 18, No. 3 (Jul., 1952), pp. 166--171

\textipa{@} \textrightarrow\hspace{0pt} u / u_ \\
\textipa{@} \textrightarrow\hspace{0pt} a / a_ \\
C\textipa{:} \textrightarrow\hspace{0pt} C / except when CV_V in U$_1$U$_2$ \\
t \textrightarrow\hspace{0pt} s / i_ \\ % check to see if this happens before @ --> i, causing "weak /i/" vs. "strong /i/"
\textipa{@} \textrightarrow\hspace{0pt} i / else \\
d\textipa{Z} \textrightarrow\hspace{0pt} t\textipa{S} \textrightarrow\hspace{0pt} s / \textquotedblleft in certain positions" (except for Thule Greenlandic, where d\textipa{Z} \textrightarrow\hspace{0pt} t\textipa{S} and stayed there, apparently) \\
m n \textipa{N} t \{\{\textipa{G,K}\} \textrightarrow\hspace{0pt} \{k,q\}\} \textrightarrow\hspace{0pt} p t k n \textipa{N} / _\# \\
m n \textipa{N} \textrightarrow\hspace{0pt} v t \textipa{G} / _C[-nasal] \\
\textipa{\r*l} \textrightarrow\hspace{0pt} \textipa{K} / _C \\
v \textipa{K} \textrightarrow\hspace{0pt} p q / S_ \\
v \textipa{G K} \textrightarrow\hspace{0pt} p k q / _C (except where C = /l d\textipa{Z} m n \textipa{N}/) \\
\{p,v\} t \{k,\textipa{G}\} \textrightarrow\hspace{0pt} m n \textipa{N} / _N \\
S$_1$S$_2$ \textrightarrow\hspace{0pt} F$_1$F$_2$\\
Some metathesis in consonant clusters, the conditions of which are not elaborated upon; the given example cited within the text is l\textipa{K} \textrightarrow\hspace{0pt} \textipa{K}l \\
v \textrightarrow\hspace{0pt} \O\hspace{0pt} / u_a \\
iv \textrightarrow\hspace{0pt} uj / _u \\
d\textipa{Z} \textrightarrow\hspace{0pt} t\textipa{S} / i_\{u,i\} \\
\textipa{@dZ} \textrightarrow\hspace{0pt} i / _\{a,u\} (except in \#U) \\
ad\textipa{Z} \textrightarrow\hspace{0pt} i / _a (except in \#U?)

\subsubsection{Proto-Eskimo to Mackenzie I\~{n}upiaq}{\it Pogostick Man}, from Swadesh, Morris (1952), \textquotedblleft Unaaliq and Proto Eskimo IV: Diachronic Notes", \textit{International Journal of American Linguistics}, Vol. 18, No. 3 (Jul., 1952), pp. 166--171

\textipa{@} \textrightarrow\hspace{0pt} u / u_ \\
\textipa{@} \textrightarrow\hspace{0pt} a / a_ \\
\textipa{@} \textrightarrow\hspace{0pt} i / else \\
\textipa{G K} \textrightarrow\hspace{0pt} k q / _\# \\
C$_0$VC$_0$ \textrightarrow\hspace{0pt} C\textipa{:} \\
Regressive MOA assimilation and progressive voicing assimilation in consonant clusters, at least when C$_2$ is either /l/ or /\textipa{\r*l}/ \\
m n \textipa{N} \textrightarrow\hspace{0pt} v t \textipa{G} / _C[-nasal] \\
\textipa{\r*l} \textrightarrow\hspace{0pt} t / _C \\
v \textrightarrow\hspace{0pt} p / _s \\
v \textipa{K} \textrightarrow\hspace{0pt} p q / S_ \\
\{p,v\} t \{k,\textipa{G}\} \textrightarrow\hspace{0pt} m n \textipa{N} / _N \\
v \textrightarrow\hspace{0pt} \O\hspace{0pt} / u_i \\
\textipa{@dZ} \textrightarrow\hspace{0pt} i / _\{a,u\} (except in \#U) \\
ad\textipa{Z} \textrightarrow\hspace{0pt} i / _a (except in \#U?)
\subsubsection{Proto-Eskimo to Wales I\~{n}upiaq}{\it Pogostick Man}, from Swadesh, Morris (1952), \textquotedblleft Unaaliq and Proto Eskimo IV: Diachronic Notes", \textit{International Journal of American Linguistics}, Vol. 18, No. 3 (Jul., 1952), pp. 166--171

\textipa{@} \textrightarrow\hspace{0pt} u / u_ \\
\textipa{@} \textrightarrow\hspace{0pt} a / a_ \\
\textipa{@} \textrightarrow\hspace{0pt} i / else \\
v \textrightarrow\hspace{0pt} u\\
\textipa{G} \textrightarrow\hspace{0pt} u / \textquotedblleft in some positions" \\
p k q s \textrightarrow\hspace{0pt} v \textipa{G K} z / V_V \\
\textipa{G K} \textrightarrow\hspace{0pt} k q / _\# \\
Regressive MOA assimilation and progressive voicing assimilation in consonant clusters, at least where C$_2$ is either /l/ or /\textipa{\r*l}/ \\
m n \textipa{N} \textrightarrow\hspace{0pt} v t \textipa{G} / _C[+nasal] \\
\textipa{\r*l} \textrightarrow\hspace{0pt} t / C_ \\
v \textrightarrow\hspace{0pt} p / _s \\
v \textipa{K} \textrightarrow\hspace{0pt} p q / S_ \\
v \textipa{G K} \textrightarrow\hspace{0pt} p k q / _C (except if C = /l d\textipa{Z m n N}/) \\
\{p,v\} t \{k,\textipa{G}\} \textrightarrow\hspace{0pt} m n \textipa{N} / _N \\
v \textrightarrow\hspace{0pt} u / V_V \\
v \textrightarrow\hspace{0pt} \O\hspace{0pt} / u_V \\
v \textrightarrow\hspace{0pt} \O\hspace{0pt} / V_u \\
d\textipa{Z} \textrightarrow\hspace{0pt} \textipa{K} / i_u \\
\textipa{@dZ} \textrightarrow\hspace{0pt} i / _\{a,u\} (except in \#U) \\
ad\textipa{Z} \textrightarrow\hspace{0pt} i / _a (except in \#U?) \\
\textipa{G} \textrightarrow\hspace{0pt} \O\hspace{0pt} / V_u \\
\textipa{G} \textrightarrow\hspace{0pt} \O\hspace{0pt} / u_V \\
\textipa{G} \textrightarrow\hspace{0pt} u / \{i,\textipa{@}\}_V

\subsubsection{Proto-Eskimo to Kuskokwim Yup'ik}{\it Pogostick Man}, from Swadesh, Morris (1952), \textquotedblleft Unaaliq and Proto Eskimo IV: Diachronic Notes", \textit{International Journal of American Linguistics}, Vol. 18, No. 3 (Jul., 1952), pp. 166--171

C\textipa{:} \textrightarrow\hspace{0pt} C \\
C \textrightarrow\hspace{0pt} C\textipa{:} / _V(\textellipsis V) except in \#U \\
S \textrightarrow\hspace{0pt} \O\hspace{0pt} / \#_F \\
s \textrightarrow\hspace{0pt} ts / in certain situations? \\
C[+voice] \textrightarrow\hspace{0pt} C[-voice] / adjacent to \{S,s,\textipa{\r*l}\} \\
\textipa{G K} \textrightarrow\hspace{0pt} k q / _\# \\
\textipa{@} \textrightarrow\hspace{0pt} a / _\# \\
\textipa{@} \textrightarrow\hspace{0pt} \O\hspace{0pt} \\
i \textrightarrow\hspace{0pt} \O\hspace{0pt} / \#C[+dental]_C[+dental]V \\
F[+voice] \textrightarrow\hspace{0pt} F[-voice] / adjacent to \{S,ts\} \\
F[+voice] \textrightarrow\hspace{0pt} S[+same POA] / \textipa{\r*l}_ \\
t \textrightarrow\hspace{0pt} s / _\{k,q\} \\
i a u \textrightarrow\hspace{0pt} ii aa uu / C_ in U[+open -initial -final] such that U[+open]_ \\
\textipa{@} \textrightarrow\hspace{0pt} i / u_ \\
v \textrightarrow\hspace{0pt} \O\hspace{0pt} / u[+short]_V[+short] \\
v \textrightarrow\hspace{0pt} \O\hspace{0pt} / V[+short]_u[+short] \\
u \textrightarrow\hspace{0pt} \O\hspace{0pt} / \#_vV \\
iv \textrightarrow\hspace{0pt} j / \#_u \\
s \textrightarrow\hspace{0pt} d\textipa{Z} / \{i,u\}_V \\
d\textipa{Z} \textrightarrow\hspace{0pt} \O\hspace{0pt} / i_i \\
\textipa{@} \textrightarrow\hspace{0pt} \O\hspace{0pt} / _d\textipa{Z}\{a,u\}, except in \#U \\
a \textrightarrow\hspace{0pt} \O\hspace{0pt} / _d\textipa{Z}a, except in \#U \\
in \textrightarrow\hspace{0pt} d\textipa{Z} / _u (possibly only word-initially?)

\subsubsection{Proto-Eskimo to Nunivak Yup'ik}{\it Pogostick Man}, from Swadesh, Morris (1952), \textquotedblleft Unaaliq and Proto Eskimo IV: Diachronic Notes", \textit{International Journal of American Linguistics}, Vol. 18, No. 3 (Jul., 1952), pp. 166--171

C\textipa{:} \textrightarrow\hspace{0pt} C \\
C \textrightarrow\hspace{0pt} C\textipa{:} / _V(\textellipsis V) except in \#U \\
S \textrightarrow\hspace{0pt} \O\hspace{0pt} / \#_F \\
s \textrightarrow\hspace{0pt} ts / in certain situations? \\
C[+voice] \textrightarrow\hspace{0pt} C[-voice] / adjacent to \{S,s,\textipa{\r*l}\} \\
\textipa{G K} \textrightarrow\hspace{0pt} x \textipa{X} / _\# \\
\textipa{@} \textrightarrow\hspace{0pt} a / _\# \\
\textipa{@} \textrightarrow\hspace{0pt} \O\hspace{0pt} \\
i \textrightarrow\hspace{0pt} \O\hspace{0pt} / \#C[+dental]_C[+dental]V \\
a \textrightarrow\hspace{0pt} \O\hspace{0pt} / C[+velar]_C[+velar] \\
Regressive MOA and voicing assimilation in consonant clusters, at least when C$_2$ is either /l/ or /\textipa{\r*l}/ \\
v \textipa{K} \textrightarrow\hspace{0pt} f \textipa{X} / S_ \\
F[+voice] \textrightarrow\hspace{0pt} F[-voice] / adjacent to \{S,ts\} \\
F[+voice] \textrightarrow\hspace{0pt} S / \textipa{\r*l}_ \\
t \textrightarrow\hspace{0pt} s / _\{k,q\} \\
i a u \textrightarrow\hspace{0pt} ii aa uu / C_ in U[+open -initial -final] such that U[+open]_ \\
\textipa{@} \textrightarrow\hspace{0pt} i / {u,a}_ (though a\textipa{@} seems to have become i in some circumstances) \\
v \textrightarrow\hspace{0pt} \O\hspace{0pt} / u[+short]_V[+short] \\
v \textrightarrow\hspace{0pt} \O\hspace{0pt} / V[+short]_u[+short] \\
u \textrightarrow\hspace{0pt} \O\hspace{0pt} / \#_vV \\
iv \textrightarrow\hspace{0pt} j / \#_u \\
s \textrightarrow\hspace{0pt} d\textipa{Z} / \{i,u\}_V \\
d\textipa{Z} \textrightarrow\hspace{0pt} \O\hspace{0pt} / i_i \\
\textipa{@} \textrightarrow\hspace{0pt} \O\hspace{0pt} / _d\textipa{Z\{a,u\}} except in \#U \\
a \textrightarrow\hspace{0pt} \O\hspace{0pt} / _d\textipa{Z}a (except in \#U?) \\
in \textrightarrow\hspace{0pt} d\textipa{Z} / _u (possibly only word-initially?)

\subsubsection{Proto-Eskimo to Siberian Yup'ik}{\it Pogostick Man}, from Swadesh, Morris (1952), \textquotedblleft Unaaliq and Proto Eskimo IV: Diachronic Notes", \textit{International Journal of American Linguistics}, Vol. 18, No. 3 (Jul., 1952), pp. 166--171

C\textipa{:} \textrightarrow\hspace{0pt} C \\
S \textrightarrow\hspace{0pt} \O\hspace{0pt} / \#_F \\
s \textrightarrow\hspace{0pt} ts \textrightarrow\hspace{0pt} t\textipa{S} / in some dialects? \\
C[+voice] \textrightarrow\hspace{0pt} C[-voice] / next to \{S,s,\textipa{\r*l}\} \\
\textipa{G K} \textrightarrow\hspace{0pt} k q / _\# \\
\textipa{@} \textrightarrow\hspace{0pt} a / \\
\textipa{@} \textrightarrow\hspace{0pt} \O\hspace{0pt} / \#_ \\
i \textrightarrow\hspace{0pt} \O\hspace{0pt} / \#C[+dental]_C[+dental]V \\
a \textrightarrow\hspace{0pt} \O\hspace{0pt} / C[+velar]_C[+velar] \\
F[+voice] \textrightarrow\hspace{0pt} F[-voice] / adjacent to \{S,ts\} \\
F[+voice] \textrightarrow\hspace{0pt} S / \textipa{\r*l}_ \\
\O \textrightarrow\hspace{0pt} n / \#_iN (This one is sort of a guess, given a singular example in the text that isn't really commented upon) \\
t \textrightarrow\hspace{0pt} s / _\{k,q\} \\
u \textrightarrow\hspace{0pt} a / a_ \\
\textipa{@} \textrightarrow\hspace{0pt} i / \{u,a\}_ \\
v \textrightarrow\hspace{0pt} \O\hspace{0pt} / u[+short]_V[+short] \\
v \textrightarrow\hspace{0pt} \O\hspace{0pt} / V[+short]_u[+short] \\
u \textrightarrow\hspace{0pt} \O\hspace{0pt} / \#_vV \\
iv \textrightarrow\hspace{0pt} j / \#_u \\
s \textrightarrow\hspace{0pt} d\textipa{Z} / \{i,u\}_V \\
d\textipa{Z} \textrightarrow\hspace{0pt} \O\hspace{0pt} / i_i \\
\textipa{@} \textrightarrow\hspace{0pt} \O\hspace{0pt} / _d\textipa{Z}\{a,u\} except in \#U \\
a \textrightarrow\hspace{0pt} \O\hspace{0pt} / _d\textipa{Z}a (except in \#U?) \\
in \textrightarrow\hspace{0pt} d\textipa{Z} / _u (possibly only word-initially?)

\subsubsection{Proto-Eskimo to Unaaliq Yup'ik}{\it Pogostick Man}, from Swadesh, Morris (1952), \textquotedblleft Unaaliq and Proto Eskimo IV: Diachronic Notes", \textit{International Journal of American Linguistics}, Vol. 18, No. 3 (Jul., 1952), pp. 166--171

C\textipa{:} \textrightarrow\hspace{0pt} C \\
C \textrightarrow\hspace{0pt} C\textipa{:} / _V(\textellipsis V), after \#U \\
S \textrightarrow\hspace{0pt} \O\hspace{0pt} / \#_F \\
s \textrightarrow\hspace{0pt} ts / in certain situations? \\
C[+voice] \textrightarrow\hspace{0pt} C[-voice] / next to \{S,s,\textipa{\r*l}\} \\
\textipa{G K} \textrightarrow\hspace{0pt} k q / _\# \\
\textipa{@} \textrightarrow\hspace{0pt} a / _\# \\
\textipa{@} \textrightarrow\hspace{0pt} \O\hspace{0pt} / \#_ \\
i \textrightarrow\hspace{0pt} \O\hspace{0pt} / \#C[+dental]_C[+dental]V \\
a \textrightarrow\hspace{0pt} \O\hspace{0pt} / C[+velar]_C[+velar] \\
v \textrightarrow\hspace{0pt} ft / _s \\
F[+voice] \textrightarrow\hspace{0pt} F[-voice] / adjacent to \{S,ts\} \\
F[+voice] \textrightarrow\hspace{0pt} S / \textipa{\r*l}_ \\
t \textrightarrow\hspace{0pt} s / _\{k,q\} \\
i a u \textrightarrow\hspace{0pt} ii aa uu / C_ in U[+open -initial -final] such that U[+open]_ \\
\textipa{@} \textrightarrow\hspace{0pt} i / \{u,a\}_ (though a\textipa{@} seems to have become i in some circumstances) \\
v \textrightarrow\hspace{0pt} \O\hspace{0pt} / u[+short]_V[+short] \\
v \textrightarrow\hspace{0pt} \O\hspace{0pt} / V[+short]_u[+short] \\
u \textrightarrow\hspace{0pt} \O\hspace{0pt} / \#_vV \\
iv \textrightarrow\hspace{0pt} j / \#_u \\
d\textipa{Z} \textrightarrow\hspace{0pt} \O\hspace{0pt} / i_i \\
\textipa{@} \textrightarrow\hspace{0pt} \O\hspace{0pt} / _d\textipa{Z}\{a,u\} except in \#U \\
a \textrightarrow\hspace{0pt} \O\hspace{0pt} / _d\textipa{Z}a (except in \#U?) \\
s \textrightarrow\hspace{0pt} d\textipa{Z} / \{i,u\}_V \\
in \textrightarrow\hspace{0pt} d\textipa{Z} / _u (possibly only word-initially?)

\clearpage

\section{Extended West Papuan}

\subsection{Tabla-Sentani}\tab Proto-Tabla-Sentani is reconstructed by Gregerson and Hartzler (1987) as having had the following phonology:

\begin{center}\begin{tabular}{c | c c c c}
& Bilabial & Alveolar & Palatal & Velar \\ \hline
Nasal & \ipa{m} & \ipa{n} & & \\
Plosive & \ipa{p b} & \ipa{t d} & & \ipa{k} \\
Approximant & & & \ipa{j} & \ipa{w} \end{tabular}\end{center}

\begin{center}\begin{tabular}{c | c c c}
& Front & Central & Back \\ \hline
Close & \ipa{i} & & \ipa{u} \\
Mid & \ipa{e} & \textipa{@} & \ipa{o} \\
Open & & \ipa{a} & \end{tabular}\end{center}

\tab (From Gregerson, Kenneth, and Margaret Hartzler (1987), \textquotedblleft Towards a Reconstruction of Proto-Tabla-Sentani Phonology". \textit{Oceanic Linguistics}, Vol. 26, No. 1/2 (Summer -- Winter, 1987), 1 -- 29.)


\subsubsection{Proto-Tabla-Sentani to Nafri}{\it Pogostick Man}, from Gregerson, Kenneth, and Margaret Hartzler (1987), ``Towards a Reconstruction of Proto-Tabla-Sentani Phonology". {\it Oceanic Linguistics}, Vol. 26, No. 1/2 (Summer -- Winter, 1987), 1 -- 29.

\ipa{p k} \change\ \ipa{b g} / V_V\\
\ipa{p t} \change\ \ipa{f} \{\ipa{s,h}\} / \#_\\
N \change\ \ipa{N} / _\#\\
\ipa{e} \change\ \ipa{i} / _(C)\ipa{i}\\
\ipa{e} \change\ \ipa{\ae} / \{P,K\}_\\
\ipa{e} \change\ \ipa{\ae} / _P\\
\ipa{e} \change\ \ipa{@} / _C\{\ipa{a,@,u}\} (seems to have become \ipa{a} in a few instances)\\
\ipa{@} \change\ \ipa{o} / _(C)\ipa{o}\\
\ipa{@} \change\ \ipa{e} / _\{C[+palatal],E\}\\
\ipa{@} \change\ \ipa{\ae} / ! _\{B,K,H\}\\
\ipa{a} \change\ \ipa{\ae} / _(C)\ipa{e}\\
\ipa{a} \change\ \ipa{\ae} / \ipa{i}(C)_\\
\ipa{o} \change\ \ipa{e} (sporadic, highly unusual)

\subsubsection{Proto-Tabla-Sentani to Central Sentani}{\it Pogostick Man}, from Gregerson, Kenneth, and Margaret Hartzler (1987), ``Towards a Reconstruction of Proto-Tabla-Sentani Phonology". {\it Oceanic Linguistics}, Vol. 26, No. 1/2 (Summer -- Winter, 1987), 1 -- 29.

\ipa{p k} \change\ \ipa{b g} / V_V\\
\ipa{p b t} \change\ \ipa{f p} \{\ipa{s,h}\} / \#_\\
\ipa{d} \change\ \ipa{l} / medially\\
N \change\ \ipa{m} / _\#\\
\ipa{e} \change\ \ipa{i} / _(C)\ipa{i}\\
\ipa{e} \change\ \ipa{\ae} / \{P,K\}_\\
\ipa{e} \change\ \ipa{\ae} / _P\\
\ipa{e} \change\ \ipa{@} / _C\{\ipa{a,@,u}\} (seems to have become \ipa{a} in a few instances)\\
\ipa{@} \change\ \ipa{o} / _(C)\ipa{o}\\
\ipa{@} \change\ \ipa{e} / _\{C[+palatal],E\}\\
\ipa{@} \change\ \ipa{\ae} / ! _\{B,K,H\}\\
\ipa{a} \change\ \ipa{\ae} / _(C)\ipa{e}\\
\ipa{a} \change\ \ipa{\ae} / \ipa{i}(C)_\\
\ipa{o} \change\ \ipa{e} (sporadic, highly unusual)

\subsubsection{Proto-Tabla-Sentani to Eastern Sentani}{\it Pogostick Man}, from Gregerson, Kenneth, and Margaret Hartzler (1987), ``Towards a Reconstruction of Proto-Tabla-Sentani Phonology". {\it Oceanic Linguistics}, Vol. 26, No. 1/2 (Summer -- Winter, 1987), 1 -- 29.

\ipa{p k} \change\ \ipa{b g} / V_V\\
\ipa{p t} \change\ \ipa{f} \{\ipa{s,h}\} / \#_\\
\ipa{d} \change\ \ipa{l} / medially\\
N \change\ \ipa{m} / _\#\\
\ipa{e} \change\ \ipa{i} / _(C)\ipa{i}\\
\ipa{e} \change\ \ipa{\ae} / \{P,K\}_\\
\ipa{e} \change\ \ipa{\ae} / _P\\
\ipa{e} \change\ \ipa{@} / _C\{\ipa{a,@,u}\} (seems to have become \ipa{a} in a few instances)\\
\ipa{@} \change\ \ipa{o} / _(C)\ipa{o}\\
\ipa{@} \change\ \ipa{e} / _\{C[+palatal],E\}\\
\ipa{@} \change\ \ipa{\ae} / ! _\{B,K,H\}\\
\ipa{a} \change\ \ipa{\ae} / _(C)\ipa{e}\\
\ipa{a} \change\ \ipa{\ae} / \ipa{i}(C)_\\
\ipa{o} \change\ \ipa{e} (sporadic, highly unusual)

\subsubsection{Proto-Tabla-Sentani to Western Sentani}{\it Pogostick Man}, from Gregerson, Kenneth, and Margaret Hartzler (1987), ``Towards a Reconstruction of Proto-Tabla-Sentani Phonology". {\it Oceanic Linguistics}, Vol. 26, No. 1/2 (Summer -- Winter, 1987), 1 -- 29.

\ipa{p k} \change\ \ipa{b g} / V_V\\
\ipa{s} \change\ \ipa{t} / \#_\\
N \change\ \ipa{N} / _\#\\
\ipa{e} \change\ \ipa{i} / _(C)\ipa{i}\\
\ipa{e} \change\ \ipa{\ae} / \{P,K\}_\\
\ipa{e} \change\ \ipa{\ae} / _P\\
\ipa{e} \change\ \ipa{@} / _C\{\ipa{a,@,u}\} (seems to have become \ipa{a} in a few instances)\\
\ipa{@} \change\ \ipa{o} / _(C)\ipa{o}\\
\ipa{@} \change\ \ipa{e} / _\{C[+palatal],E\}\\
\ipa{@} \change\ \ipa{\ae} / ! _\{B,K,H\}\\
\ipa{a} \change\ \ipa{\ae} / _(C)\ipa{e}\\
\ipa{a} \change\ \ipa{\ae} / \ipa{i}(C)_\\
\ipa{o} \change\ \ipa{e} (sporadic, highly unusual)

\subsubsection{Proto-Tabla-Sentani to Tabla}{\it Pogostick Man}, from Gregerson, Kenneth, and Margaret Hartzler (1987), ``Towards a Reconstruction of Proto-Tabla-Sentani Phonology". {\it Oceanic Linguistics}, Vol. 26, No. 1/2 (Summer -- Winter, 1987), 1 -- 29.

\ipa{d} \change\ \ipa{r} / medially\\
N \change\ \ipa{N} / _\#\\
\ipa{i} \change\ \O\ / V_ (with a few exceptions)\\
Some allophony triggered where \ipa{p}\raisebox{-0.7ex}{\textasciitilde}\ipa{F}, and probably some others\\
\ipa{e} \change\ \ipa{@} / unstressed (possibly only in disyllables?)\\
\ipa{oi} \change\ \ipa{oe}\\
Some vowel assimilations, mostly dealing with central vowels\\
\ipa{o} \change\ \ipa{e} (sporadic, highly unusual)

\clearpage

\section{Indo-European}\tab Wikipedia gives the following phonological reconstruction for Proto-Indo-European, reproduced here with some slight adjustments for presentation's sake:

\begin{center}\begin{tabular}{c | c c c c c c}
& Labial & Coronal & Palatovelar & Plain Velar & Labiovelar & Laryngeal \\ \hline
Nasal & m & n & & & & \\
Plosive & \textipa{p b b\super H} & \textipa{t d d\super H} & \textipa{\'{k} \'{g} \'{g}\super H} & \textipa{k g g\super H} & \ipa{k\super w g\super w g\super w\super H} & \\
Fricative & & \ipa{s} & & & & h$_1$ h$_2$ h$_3$ \\
Approximant & & \textipa{\*r} & \ipa{j} & & \ipa{w} & \\
Lat. Approx. & & \ipa{l} & & & & \\ \end{tabular}\end{center}

\tab There is some debate as to whether the voiced and voiced aspirate stops were actually glottalized and plain voiced, respectively; the status of the palatovelars, plain velars, and labiovelars as possible plain velar, uvular, and labialized uvular consonants, or as to whether the labiovelars existed at all, are also subjects of much contention.

\begin{center}\begin{tabular}{c | c c c}
& Front & Central & Back \\ \hline
Mid & e e\textipa{:} & & o o\textipa{:} \\
Back & & (a) (a\textipa{:}) & \\ \end{tabular}\end{center}

\tab It is noted in the source that the nasals, approximants, and potentially laryngeals could also act as vowels; such allophones of /j w/ would then be [i u]. There is some debate over the presence of /a \textipa{a:}/ in the language, although the Wikipedia does mention that if Stang's law holds, /\textipa{a:}/ at least must have been phonemic.

\tab The accentual system was apparently a sort of free pitch accent, heavily related to ablaut and the vestiges of which may be best seen in Vedic Sanskrit, Ancient Greek, and Lithuanian and some West South Slavic tongues.

\tab (From Wikipedia contributors (2011), \textquotedblleft Pitch accent". \textit{Wikipedia, The Free Encyclopedia}. \textless\url{http://en.wikipedia.org/w/index.php?title=Pitch_accent&oldid=451210103}\textgreater; and Wikipedia contributors (2011), \textquotedblleft Proto-Indo-European language". \textit{Wikipedia, The Free Encyclopedia}. \textless\url{http://en.wikipedia.org/w/index.php?title=Proto-Indo-European_language&oldid=455124616}\textgreater)

\subsection{Albanian}

\subsubsection{Proto-Indo-European to Gheg Albanian}{\it Pogostick Man}, from Wikipedia contributors (2013), ``Albanian language''. {\it Wikipedia, the Free Encyclopedia}. \textless\url{http://en.wikipedia.org/w/index.php?title=Albanian_language&oldid=582390175}\textgreater

\ipa{d d\super H} \change\ \ipa{D} / V_V\\
\ipa{d d\super H} \change\ \ipa{D} / \ipa{r}_\\
\ipa{\'{k}} \change\ \ipa{s} / _\{\ipa{\textsubarch{u},u,\textsubarch{i},i}\}\\
\ipa{\'{k}} \change\ \ipa{k} / _R\\
\ipa{\'{k}} \change\ \{\ipa{ts,tS}\} (``[a]rchaic relic'')\\
\ipa{\'{k}} \change\ \ipa{T}\\
\ipa{k\super w} \change\ \ipa{s} / _"E? \\
\ipa{k\super w} \change\ \ipa{c} / _B?\\
\ipa{k\super w} \change\ \ipa{k} / else?\\
\ipa{\'{g}(\super H)} \change\ \ipa{d} / \%_C[+sibilant]\\
\ipa{g\super w(\super H)} \change\ \{\ipa{g,z}\} \\
\ipa{b\super H d\super H \'{g}(\super H)} \change\ \ipa{b d} dh\\
\ipa{s} \change\ \ipa{\textbardotlessj} / \#_\\
\ipa{s} \change\ \ipa{S} / V\{\ipa{\textsubarch{i},\textsubarch{u},r,k}\}_V\\
\ipa{sd} \change\ \ipa{T} / medial\\
\ipa{s\'{k}} \change\ \ipa{h} / medial\\
\ipa{sp} \change\ \ipa{f} / medial\\
\ipa{st} \change\ \ipa{St} / medial\\
\ipa{s} \change\ \ipa{T} / sometimes, involving ``[d]issimilation with following vowel''\\
\ipa{s} \change\ \ipa{h} / V_V\\
\ipa{\textsubarch{i}} \change\ \ipa{\textbardotlessj} / \{\ipa{a,e,i}\}_\\
\ipa{\textsubarch{i}} \change\ \ipa{j} / _B\\
\ipa{\textsubarch{i}} \change\ \O\ / E_\\
\ipa{\textsubarch{i}} \change\ \ipa{h} / V_\\
\ipa{\textsubarch{u}} \change\ \ipa{v}\\
V\ipa{n} \change\ V[+nasal] / _C?\\
\ipa{n} \change\ \ipa{\textltailn} (sometimes?)\\
\ipa{l r} \change\ \ipa{l(:) r(:)}\\
\{\ipa{\s{m},\s{n}}\} \ipa{\s{l} \s{r}} \change\ \ipa{e uj} \{\ipa{ri,ir}\}\\
Loss of laryngeals, with the possible exception of h$_4$, if it existed; h$_3$ and h$_4$ seem to have possibly fronted a following back vowel\\
\ipa{e: i: o: u:} \change\ \ipa{o i e} \{\ipa{y,i}\}\\
\ipa{a e i o} \change\ \{\ipa{a,e}\} \ipa{(j)e} \{\ipa{e,i}\} \ipa{a}\\
Gheg seems to have maintained or innovated vowel length whereas Tosk has not\\
\ipa{@} \change\ \O\ / _\#\\
\ipa{c \textbardotlessj} \change\ \ipa{tS dZ} (for most speakers)


\subsubsection{Proto-Indo-European to Tosk Albanian}{\it Pogostick Man}, from Wikipedia contributors (2013), ``Albanian language''. {\it Wikipedia, the Free Encyclopedia}. \textless\url{http://en.wikipedia.org/w/index.php?title=Albanian_language&oldid=582390175}\textgreater

\ipa{d d\super H} \change\ \ipa{D} / V_V\\
\ipa{d d\super H} \change\ \ipa{D} / \ipa{r}_\\
\ipa{\'{k}} \change\ \ipa{s} / _\{\ipa{\textsubarch{u},u,\textsubarch{i},i}\}\\
\ipa{\'{k}} \change\ \ipa{k} / _R\\
\ipa{\'{k}} \change\ \{\ipa{ts,tS}\} (``[a]rchaic relic'')\\
\ipa{\'{k}} \change\ \ipa{T}\\
\ipa{k\super w} \change\ \ipa{s} / _"E? \\
\ipa{k\super w} \change\ \ipa{c} / _B?\\
\ipa{k\super w} \change\ \ipa{k} / else?\\
\ipa{\'{g}(\super H)} \change\ \ipa{d} / \%_C[+sibilant]\\
\ipa{g\super w(\super H)} \change\ \{\ipa{g,z}\} \\
\ipa{b\super H d\super H \'{g}(\super H)} \change\ \ipa{b d} dh\\
\ipa{s} \change\ \ipa{\textbardotlessj} / \#_\\
\ipa{s} \change\ \ipa{S} / V\{\ipa{\textsubarch{i},\textsubarch{u},r,k}\}_V\\
\ipa{sd} \change\ \ipa{T} / medial\\
\ipa{s\'{k}} \change\ \ipa{h} / medial\\
\ipa{sp} \change\ \ipa{f} / medial\\
\ipa{st} \change\ \ipa{St} / medial\\
\ipa{s} \change\ \ipa{T} / sometimes, involving ``[d]issimilation with following vowel''\\
\ipa{s} \change\ \ipa{h} / V_V\\
\ipa{\textsubarch{i}} \change\ \ipa{\textbardotlessj} / \{\ipa{a,e,i}\}_\\
\ipa{\textsubarch{i}} \change\ \ipa{j} / _B\\
\ipa{\textsubarch{i}} \change\ \O\ / E_\\
\ipa{\textsubarch{i}} \change\ \ipa{h} / V_\\
\ipa{\textsubarch{u}} \change\ \ipa{v}\\
\ipa{n} \change\ \O\ / V_C?\\
\ipa{n} \change\ \ipa{\textltailn} (sometimes?)\\
\ipa{n} \change\ \ipa{R}\\
\ipa{l r} \change\ \{\ipa{l,\textsuperimposetilde{l}}\} \{\ipa{R,r}\}\\
\{\ipa{\s{m},\s{n}}\} \ipa{\s{l} \s{r}} \change\ \ipa{e uj} \{\ipa{ri,ir}\}\\
Loss of laryngeals, with the possible exception of h$_4$, if it existed; h$_3$ and h$_4$ seem to have possibly fronted a following back vowel\\
\ipa{e: i: o: u:} \change\ \ipa{o i e} \{\ipa{y,i}\}\\
\ipa{a e i o} \change\ \{\ipa{a,e}\} \ipa{(j)e} \{\ipa{e,i}\} \ipa{a}\\
\ipa{c \textbardotlessj} \change\ \ipa{tS dZ} (much less widespread than in Gheg)

\subsection{Proto-Indo-European to Common Anatolian}{\it Alces}, from \url{http://www.unc.edu/~melchert/anathistphon.pdf} {\it (link is dead)}

\ipa{e}h$_2$ \change\ \ipa{\ae:} / ``tautosyllabic''\\
\ipa{ei eu} \change\ \ipa{E: u:}\\
D\ipa{\super H} \change\ D\\
H \change\ R / VR_V\\
h$_1$ \change\ \O\\
h$_3$ \change\ \O\ / ``medially''\\
T h$_2$ \change\ D h$_3$ / _\#\\
T h$_2$ \change\ D h$_3$ / V[-stress]_V[-stress]\\
T h$_2$ \change\ D h$_3$ / "V\ipa{:}\\
T h$_2$ \change\ D h$_3$ / "W\\
\ipa{t} \change\ \ipa{z} / \ipa{j} ``(allophonic)''\\
\ipa{r} \change\ \O\ / \#_ (unclear)\\
\ipa{j} \change\ \O\ / \#_\ipa{e} (not widely attested)\\
\{\{h$_1$,h$_3$\}\ipa{s,s}\{h$_1$,h$_3$\}\} \change\ \ipa{s:} (contested)

\subsubsection{Common Anatolian to Hittite}{\it Alces}, from \url{http://www.unc.edu/~melchert/anathistphon.pdf} {\it (link is dead)}

D T \change\ T T[+lenis] / \#_\\
T D \change\ T[+fortis] T[+lenis]\\
h$_2$ h$_3$ \change\ \ipa{hh h}\\
\'{K} \change\ K\\
V \change\ V\ipa{:} / in "U[+open]\\
\ipa{e o} \change\ \ipa{e: o:} / in "U[+stress]\\
\ipa{o(:) \ae:} \change\ \ipa{a(:) e:}\\
\ipa{e} \change\ \ipa{i} / _\{\ipa{m,N}\} when posttonic in U[+closed] or when pretonic\\
\ipa{e} \change\ \ipa{a} / _\ipa{n} in U[+open +posttonic]\\
\ipa{e} \change\ \ipa{a} / _\{\ipa{r,l}\} (sporadic)\\
\ipa{t} \change\ \ipa{ts} / _\ipa{i} ! \ipa{s}_\\
\ipa{d} \change\ \ipa{s} / \#_\{\ipa{i,j}\}\\
\ipa{w} \change\ \ipa{m} / _\ipa{u}\\
\ipa{w} \change\ \ipa{m} / \ipa{u}_\\
\ipa{j} \change\ \O\ / V_V\\
\ipa{aj aw} \change\ \ipa{E: u:} / !_\{\ipa{s,n,r,l}\}

\subsubsection{Common Anatolian to Luwian}{\it Alces}, from \url{http://www.unc.edu/~melchert/anathistphon.pdf} {\it (link is dead)}

D T \change\ T T[+lenis] / \#_\\
T D \change\ T[+fortis] T[+lenis]\\
h$_2$ h$_3$ \change\ \ipa{hh h}\\
\ipa{\'{k}:} \change\ \ipa{z}\\
\ipa{k\super w} \change\ \ipa{w}\\
\{\ipa{\'{k},k}\} \change\ \ipa{j} / _\ipa{e(:)}\\
\{\ipa{\'{k},k}\} \change\ \O\ / _\ipa{i(:)}\\
\ipa{k} \change\ \O\ / V_V\\
\ipa{k} \change\ \O\ / _N\\
\ipa{g} \change\ \ipa{dZ} (sporadic)\\
\ipa{e} \change\ \ipa{i} / \ipa{j}_\\
\ipa{e} \change\ \ipa{a}\\
V \change\ V\ipa{:} / in "U[+open]\\
V \change\ V\ipa{:} / in \#"U\\
\ipa{o(:)} \change\ \ipa{a(:)}\\
\{\ipa{d,l}\} \change\ \ipa{r} ``in Hieroglyphic Luwian, occasionally''\\
\ipa{j} \change\ \O\ / \ipa{z}_\\
\ipa{hh h} \change\ \ipa{h} \O\ / "V\ipa{:}_\ipa{u}\\
\ipa{hh h} \change\ \ipa{h} \O\ / \ipa{u}_"V\ipa{:}\\
\ipa{h} \change\ \O\ / _\ipa{w} ! at word boundaries\\
\ipa{hh} \change\ \O\ / _\{\ipa{w,m,n,r,l}\} ``medially, and sporadically"\\
D R \change\ D\ipa{:} R\ipa{:} / "\ipa{e}_ in U[+open]\\
\{\ipa{E,e}\}\ipa{: \ae:} \change\ \ipa{i: a:}

\subsubsection{Common Anatolian to Lycian}{\it Alces}, from \url{http://www.unc.edu/~melchert/anathistphon.pdf} {\it (link is dead)}

D \change\ T / \#_\\
D \change\ F[+voice]\\
N\{F[+voice],T\} \change\ \ipa{n}D\\
\ipa{d} \change\ \ipa{k} / _\ipa{w}\\
\ipa{k\super w} \change\ \ipa{t} / _E\\
\ipa{k\super w} \change\ \ipa{k} / _E, in Milyan\\
\{\ipa{\'{g},g}\} \change\ \ipa{j} / _\ipa{e(:)}\\
\{\ipa{\'{g},g}\} \change\ \O\ / _\ipa{i(:)}\\
\ipa{g} \change\ \O\ / V_V\\
\ipa{j} \change\ \O\ / \ipa{ts}_\\
\ipa{s} \change\ \ipa{z} / _\{R,\ipa{j,w}\} ``(in Milyan, this happened after the change of \'{k} to s)''\\
\ipa{s} \change\ \ipa{h}\\
\ipa{\'{k}} \change\ \ipa{s} \\
h$_3$ \change\ \ipa{g} / _B\\
h$_2$ \change\ \ipa{k} / E_E (probably a palatal stop)\\
h$_2$ \change\ \ipa{q} / _E (possibly plain velar stop)\\
h$_2$ \change\ \ipa{x} / else (possibly a uvular stop)\\
\ipa{w} \change\ \ipa{b} / C_\\
\ipa{g} \change\ \ipa{dZ} (sporadic)\\
\ipa{e} \change\ \ipa{i} / \ipa{j}_\\
\{\ipa{E,e}\}\ipa{: o \ae:} \change\ \ipa{i: e a:}\\
V\ipa{:} \change\ V[-long]\\
\ipa{e} \change\ \ipa{a} / _U[+\ipa{u,a}]\\
\ipa{a} \change\ \ipa{e} / _U[+\ipa{e,i}]\\
``[P]lus lots of syncope which he doesn't elaborate on''

\subsubsection{Common Anatolian to Lydian}{\it Alces}, from \url{http://www.unc.edu/~melchert/anathistphon.pdf} {\it (link is dead)}

\ipa{d} \change\ \ipa{tS} / _\{\ipa{i,u}\}\\
\ipa{d} \change\ \ipa{t} / \{\#,N\}_\\
\ipa{p d} D \change\ \ipa{f D} T\\
T \change\ D / N_\\
\'{K} \change\ K\\
K\super w \change\ K / _V[+round]\\
\ipa{S} \change\ \ipa{s}\\
\ipa{s} \change\ \ipa{S} / _\{\ipa{i,e}\}\\
\ipa{s} \change\ \ipa{S} / \ipa{i}_\\
\{h$_2$,h$_3$\} \change\ \O\\
\{\ipa{m,n}\} \change\ \ipa{V} / _\# ``(that's a Greek nu, I'm not sure what it's supposed to represent)''\\
\ipa{l} \change\ \ipa{L} / _\{\ipa{i,j}\}\\
\ipa{w} (\change\ \ipa{v}?) \change\ \ipa{f} / \ipa{s}_\\
\ipa{j} \change\ \O\ / C_\\
\ipa{j u} \change\ \ipa{D w} / \#_\\
\ipa{j} \change\ \ipa{D} / V_V\\
\ipa{e} \change\ \ipa{i} / \ipa{j}_\\
\{\ipa{e,a,o}\} \change\ \={e} / _N in "U[+closed]\\
\{\ipa{e,a,o}\} \change\ \={a} / _N in "U[+open]\\
\{\ipa{e,o}\} \change\ \ipa{a} / in U[-stress]\\
\ipa{n} \change\ \O\ / _P ``(leaves nasalization on the previous vowel)''\\
\ipa{o(:) e: \ae:} \change\ \ipa{a(:) i: a:}\\
V\ipa{:} \change\ V[-long]\\
``[P]lus lots of syncope which he doesn't elaborate on''

\subsubsection{Common Anatolian to Palaic}{\it Alces}, from \url{http://www.unc.edu/~melchert/anathistphon.pdf} {\it (link is dead)}

D T \change\ T T[+lenis] / \#_\\
T D \change\ T[+fortis] T[+lenis]\\
h$_2$ h$_3$ \change\ \ipa{hh h}\\
\'{K} \change\ K\\
V \change\ V\ipa{:} / "U[+open]\\
\ipa{e} \change\ \ipa{i} / pretonic\\
\ipa{e} \change\ \ipa{a} / posttonic in U[+open]\\
\ipa{o(:)} \change\ \ipa{a(:)}\\
\ipa{a e} \change\ \ipa{a: e:} / in "U[+closed]\\
\ipa{\ae:} \change\ \ipa{e:}\\
\ipa{g\super w} \change\ \ipa{h\super w} / medially\\
hhy \change\ ``something like /\ipa{Z}/''\\
\ipa{h} \change\ \O\ / "V\ipa{:}_\ipa{u}\\
\ipa{h} \change\ \O\ / \ipa{u}_"V\ipa{:}\\
\ipa{w} \change\ \ipa{j} / ``in *d\'{i}wots \textgreater\ Tiyaz `sun-god'; conditioning unknown''\\
\ipa{E:} \change\ \ipa{i:}

\subsection{Armenian}

\subsubsection{Proto-Indo-European to Artsakh Armenian}{\it Mecislau \& Pogostick Man}, the latter citing Wikipedia contributors (2013), ``Armenian Language''. {\it Wikipedia, the Free Encyclopedia}. \textless\url{http://en.wikipedia.org/w/index.php?title=Armenian_language&oldid=582063933}\textgreater

\tab {\it NB: The changes in plosives are the most contentious; the Wikipedia article gives differences between the seven dialects in initial position for only the alveolar series by way of comparison, so take plosive changes with a huge grain of salt.}

\{\ipa{e,i}\}\ipa{:} \{\ipa{u,o}\}\ipa{:} \change\ \ipa{i u}\\
\{\ipa{e,o}\}\ipa{j Ew} \change\ \ipa{Ej ow}\\
\{\ipa{e,o}\} \change\ \ipa{a} (rare)\\
\ipa{a:} \change\ \ipa{a}\\
\ipa{e} \change\ \ipa{E}\\
\ipa{E o} \change\ \ipa{i u} / _N\\
\ipa{ej ia} \change\ \ipa{e Ea}\\
\{\ipa{i,u}\} \change\ \ipa{@} / in some unstressed syllables\\
\ipa{e oj Ea} \change\ \ipa{i u E} / when unstressed\\
\ipa{p t} \change\ \ipa{h t\super h} / \#_ (?)\\
\ipa{t} \change\ \ipa{t\super h} / \{\ipa{aw,ow}\}_\\
\ipa{k\super w} \change\ \ipa{tS\super h} / _\{\ipa{e,i}\}\\
\ipa{t k(\super w)} \change\ \ipa{d g} / \{N,L\}_\\
\ipa{p \'{k} k(\super w)} \change\ \{\ipa{w,v}\} \ipa{s k\super h}\\
\ipa{\'{g}} \change\ \ipa{ts} (?)\\
\ipa{b d g(\super w)} \change\ \ipa{p t k}\\
\ipa{b\super H d\super H \'{g}\super H g\super w\super H} \change\ \ipa{p t j k} / \#_\\
\ipa{b\super H d\super H \'{g}\super H g\super H g\super w\super H} \change\ \{\ipa{w,v}\} \ipa{d z g Z}\\
\ipa{j} \change\ \ipa{w} / _\ipa{o}\\
\ipa{j} \change\ ?\\
\{\ipa{sk,ks}\} \ipa{kj} \change\ \ipa{ts\super h tS\super h} (?)\\
\{\ipa{sr,rs}\} \change\ \ipa{r:}\\
\ipa{r} \change\ \ipa{r:} / _N\\
\ipa{l} \change\ \ipa{\textsuperimposetilde{l}} / \{C,\ipa{l}V\}_\\
\ipa{l} \change\ \ipa{\textsuperimposetilde{l}} / V_V\\
N\ipa{s s}N \change\ \ipa{s} N\\
N \change\ \ipa{w} / S_S\\
VN \change\ V[+nasal] \change\ V \change\ (?) / _\#, in polysyllables\\
N \change\ \ipa{n} / _\#, in monosyllables\\
\s{N} \change\ \ipa{n} / _\#\\
\ipa{\s{m} \s{n} \s{r} \s{l}} \change\ \ipa{am an ar a\textsuperimposetilde{l}} \\
V \change\ (?) / _(C)\#

\subsubsection{Proto-Indo-European to Erevan Armenian}{\it Mecislau \& Pogostick Man}, the latter citing Wikipedia contributors (2013), ``Armenian Language''. {\it Wikipedia, the Free Encyclopedia}. \textless\url{http://en.wikipedia.org/w/index.php?title=Armenian_language&oldid=582063933}\textgreater

\tab {\it NB: The changes in plosives are the most contentious; the Wikipedia article gives differences between the seven dialects in initial position for only the alveolar series by way of comparison, so take plosive changes with a huge grain of salt.}

\{\ipa{e,i}\}\ipa{:} \{\ipa{u,o}\}\ipa{:} \change\ \ipa{i u}\\
\{\ipa{e,o}\}\ipa{j Ew} \change\ \ipa{Ej ow}\\
\{\ipa{e,o}\} \change\ \ipa{a} (rare)\\
\ipa{a:} \change\ \ipa{a}\\
\ipa{e} \change\ \ipa{E}\\
\ipa{E o} \change\ \ipa{i u} / _N\\
\ipa{ej ia} \change\ \ipa{e Ea}\\
\{\ipa{i,u}\} \change\ \ipa{@} / in some unstressed syllables\\
\ipa{e oj Ea} \change\ \ipa{i u E} / when unstressed\\
\ipa{p t} \change\ \ipa{h t\super h} / \#_ (?)\\
\ipa{t} \change\ \ipa{t\super h} / \{\ipa{aw,ow}\}_\\
\ipa{k\super w} \change\ \ipa{tS\super h} / _\{\ipa{e,i}\}\\
\ipa{t k(\super w)} \change\ \ipa{d g} / \{N,L\}_\\
\ipa{p \'{k} k(\super w)} \change\ \{\ipa{w,v}\} \ipa{s k\super h}\\
\ipa{\'{g}} \change\ \ipa{ts} (?)\\
\ipa{b d g(\super w)} \change\ \ipa{p t k}\\
\ipa{g\super w\super H} \change\ \ipa{g\super H} / \#_\\
\ipa{b\super H d\super H \'{g}\super H g\super H g\super w\super H} \change\ \{\ipa{w,v}\} \ipa{d z g Z}\\
\ipa{j} \change\ \ipa{w} / _\ipa{o}\\
\ipa{j} \change\ ?\\
\{\ipa{sk,ks}\} \ipa{kj} \change\ \ipa{ts\super h tS\super h} (?)\\
\{\ipa{sr,rs}\} \change\ \ipa{r:}\\
\ipa{r} \change\ \ipa{r:} / _N\\
\ipa{l} \change\ \ipa{\textsuperimposetilde{l}} / \{C,\ipa{l}V\}_\\
\ipa{l} \change\ \ipa{\textsuperimposetilde{l}} / V_V\\
N\ipa{s s}N \change\ \ipa{s} N\\
N \change\ \ipa{w} / S_S\\
VN \change\ V[+nasal] \change\ V \change\ (?) / _\#, in polysyllables\\
N \change\ \ipa{n} / _\#, in monosyllables\\
\s{N} \change\ \ipa{n} / _\#\\
\ipa{\s{m} \s{n} \s{r} \s{l}} \change\ \ipa{am an ar a\textsuperimposetilde{l}} \\
V \change\ (?) / _(C)\#

\subsubsection{Proto-Indo-European to Istanbul Armenian}{\it Mecislau \& Pogostick Man}, the latter citing Wikipedia contributors (2013), ``Armenian Language''. {\it Wikipedia, the Free Encyclopedia}. \textless\url{http://en.wikipedia.org/w/index.php?title=Armenian_language&oldid=582063933}\textgreater

\tab {\it NB: The changes in plosives are the most contentious; the Wikipedia article gives differences between the seven dialects in initial position for only the alveolar series by way of comparison, so take plosive changes with a huge grain of salt.}

\{\ipa{e,i}\}\ipa{:} \{\ipa{u,o}\}\ipa{:} \change\ \ipa{i u}\\
\{\ipa{e,o}\}\ipa{j Ew} \change\ \ipa{Ej ow}\\
\{\ipa{e,o}\} \change\ \ipa{a} (rare)\\
\ipa{a:} \change\ \ipa{a}\\
\ipa{e} \change\ \ipa{E}\\
\ipa{E o} \change\ \ipa{i u} / _N\\
\ipa{ej ia} \change\ \ipa{e Ea}\\
\{\ipa{i,u}\} \change\ \ipa{@} / in some unstressed syllables\\
\ipa{e oj Ea} \change\ \ipa{i u E} / when unstressed\\
\ipa{t} \change\ \ipa{t\super h} / \{\ipa{aw,ow}\}_ ! \#_\\
\ipa{k\super w} \change\ \ipa{tS\super h} / _\{\ipa{e,i}\}\\
\ipa{t k(\super w)} \change\ \ipa{d g} / \{N,L\}_\\
\ipa{p \'{k} k(\super w)} \change\ \{\ipa{w,v}\} \ipa{s k\super h}\\
\ipa{\'{g}} \change\ \ipa{ts} (?)\\
\ipa{b d g(\super w)} \change\ \ipa{p t k}\\
\ipa{g\super w\super H} \change\ \ipa{dZ} / \#_\{\ipa{e,i}\}\\
\ipa{b\super H \'{g}\super H g(\super w)\super H} \change\ \ipa{b j k} / \#_\\
\ipa{b\super H d\super H \'{g}\super H g\super H g\super w\super H} \change\ \{\ipa{w,v}\} \ipa{d z g Z}\\
\ipa{j} \change\ \ipa{w} / _\ipa{o}\\
\ipa{j} \change\ ?\\
\{\ipa{sk,ks}\} \ipa{kj} \change\ \ipa{ts\super h tS\super h} (?)\\
\{\ipa{sr,rs}\} \change\ \ipa{r:}\\
\ipa{r} \change\ \ipa{r:} / _N\\
\ipa{l} \change\ \ipa{\textsuperimposetilde{l}} / \{C,\ipa{l}V\}_\\
\ipa{l} \change\ \ipa{\textsuperimposetilde{l}} / V_V\\
N\ipa{s s}N \change\ \ipa{s} N\\
N \change\ \ipa{w} / S_S\\
VN \change\ V[+nasal] \change\ V \change\ (?) / _\#, in polysyllables\\
N \change\ \ipa{n} / _\#, in monosyllables\\
\s{N} \change\ \ipa{n} / _\#\\
\ipa{\s{m} \s{n} \s{r} \s{l}} \change\ \ipa{am an ar a\textsuperimposetilde{l}} \\
V \change\ (?) / _(C)\#

\subsubsection{Proto-Indo-European to Kharpert Armenian}{\it Mecislau \& Pogostick Man}, the latter citing Wikipedia contributors (2013), ``Armenian Language''. {\it Wikipedia, the Free Encyclopedia}. \textless\url{http://en.wikipedia.org/w/index.php?title=Armenian_language&oldid=582063933}\textgreater

\tab {\it NB: The changes in plosives are the most contentious; the Wikipedia article gives differences between the seven dialects in initial position for only the alveolar series by way of comparison, so take plosive changes with a huge grain of salt.}

\{\ipa{e,i}\}\ipa{:} \{\ipa{u,o}\}\ipa{:} \change\ \ipa{i u}\\
\{\ipa{e,o}\}\ipa{j Ew} \change\ \ipa{Ej ow}\\
\{\ipa{e,o}\} \change\ \ipa{a} (rare)\\
\ipa{a:} \change\ \ipa{a}\\
\ipa{e} \change\ \ipa{E}\\
\ipa{E o} \change\ \ipa{i u} / _N\\
\ipa{ej ia} \change\ \ipa{e Ea}\\
\{\ipa{i,u}\} \change\ \ipa{@} / in some unstressed syllables\\
\ipa{e oj Ea} \change\ \ipa{i u E} / when unstressed\\
\ipa{p t} \change\ \ipa{h t\super h} / \#_ (?)\\
\ipa{t} \change\ \ipa{t\super h} / \{\ipa{aw,ow}\}_\\
\ipa{k\super w} \change\ \ipa{tS\super h} / _\{\ipa{e,i}\}\\
\ipa{t k(\super w)} \change\ \ipa{d g} / \{N,L\}_\\
\ipa{p \'{k} k(\super w)} \change\ \{\ipa{w,v}\} \ipa{s k\super h}\\
\ipa{\'{g}} \change\ \ipa{ts} (?)\\
\ipa{b d g(\super w)} \change\ \ipa{p t k}\\
\ipa{b\super H d\super H \'{g}\super H g(\super w)\super H} \change\ \ipa{p t j k} / \#_\\
\ipa{b\super H d\super H \'{g}\super H g\super H g\super w\super H} \change\ \{\ipa{w,v}\} \ipa{d z g Z}\\
\ipa{j} \change\ \ipa{w} / _\ipa{o}\\
\ipa{j} \change\ ?\\
\{\ipa{sk,ks}\} \ipa{kj} \change\ \ipa{ts\super h tS\super h} (?)\\
\{\ipa{sr,rs}\} \change\ \ipa{r:}\\
\ipa{r} \change\ \ipa{r:} / _N\\
\ipa{l} \change\ \ipa{\textsuperimposetilde{l}} / \{C,\ipa{l}V\}_\\
\ipa{l} \change\ \ipa{\textsuperimposetilde{l}} / V_V\\
N\ipa{s s}N \change\ \ipa{s} N\\
N \change\ \ipa{w} / S_S\\
VN \change\ V[+nasal] \change\ V \change\ (?) / _\#, in polysyllables\\
N \change\ \ipa{n} / _\#, in monosyllables\\
\s{N} \change\ \ipa{n} / _\#\\
\ipa{\s{m} \s{n} \s{r} \s{l}} \change\ \ipa{am an ar a\textsuperimposetilde{l}} \\
V \change\ (?) / _(C)\#

\subsubsection{Proto-Indo-European to Sebastia Armenian}{\it Mecislau \& Pogostick Man}, the latter citing Wikipedia contributors (2013), ``Armenian Language''. {\it Wikipedia, the Free Encyclopedia}. \textless\url{http://en.wikipedia.org/w/index.php?title=Armenian_language&oldid=582063933}\textgreater

\tab {\it NB: The changes in plosives are the most contentious; the Wikipedia article gives differences between the seven dialects in initial position for only the alveolar series by way of comparison, so take plosive changes with a huge grain of salt.}

\{\ipa{e,i}\}\ipa{:} \{\ipa{u,o}\}\ipa{:} \change\ \ipa{i u}\\
\{\ipa{e,o}\}\ipa{j Ew} \change\ \ipa{Ej ow}\\
\{\ipa{e,o}\} \change\ \ipa{a} (rare)\\
\ipa{a:} \change\ \ipa{a}\\
\ipa{e} \change\ \ipa{E}\\
\ipa{E o} \change\ \ipa{i u} / _N\\
\ipa{ej ia} \change\ \ipa{e Ea}\\
\{\ipa{i,u}\} \change\ \ipa{@} / in some unstressed syllables\\
\ipa{e oj Ea} \change\ \ipa{i u E} / when unstressed\\
\ipa{p t} \change\ \ipa{h t\super h} / \#_ (?)\\
\ipa{t} \change\ \ipa{t\super h} / \{\ipa{aw,ow}\}_\\
\ipa{k\super w} \change\ \ipa{tS\super h} / _\{\ipa{e,i}\}\\
\ipa{t k(\super w)} \change\ \ipa{d g} / \{N,L\}_\\
\ipa{p \'{k} k(\super w)} \change\ \{\ipa{w,v}\} \ipa{s k\super h}\\
\ipa{\'{g}} \change\ \ipa{ts} (?)\\
\ipa{b d g(\super w)} \change\ \ipa{p t k}\\
\ipa{b\super H d\super H \'{g}\super H g\super H g\super w\super H} \change\ \{\ipa{w,v}\} \ipa{d z g Z} / ! _\#\\
\ipa{j} \change\ \ipa{w} / _\ipa{o}\\
\ipa{j} \change\ ?\\
\{\ipa{sk,ks}\} \ipa{kj} \change\ \ipa{ts\super h tS\super h} (?)\\
\{\ipa{sr,rs}\} \change\ \ipa{r:}\\
\ipa{r} \change\ \ipa{r:} / _N\\
\ipa{l} \change\ \ipa{\textsuperimposetilde{l}} / \{C,\ipa{l}V\}_\\
\ipa{l} \change\ \ipa{\textsuperimposetilde{l}} / V_V\\
N\ipa{s s}N \change\ \ipa{s} N\\
N \change\ \ipa{w} / S_S\\
VN \change\ V[+nasal] \change\ V \change\ (?) / _\#, in polysyllables\\
N \change\ \ipa{n} / _\#, in monosyllables\\
\s{N} \change\ \ipa{n} / _\#\\
\ipa{\s{m} \s{n} \s{r} \s{l}} \change\ \ipa{am an ar a\textsuperimposetilde{l}} \\
V \change\ (?) / _(C)\#

\subsubsection{Proto-Indo-European to Southeast Armenian}{\it Mecislau \& Pogostick Man}, the latter citing Wikipedia contributors (2013), ``Armenian Language''. {\it Wikipedia, the Free Encyclopedia}. \textless\url{http://en.wikipedia.org/w/index.php?title=Armenian_language&oldid=582063933}\textgreater

\tab {\it NB: The changes in plosives are the most contentious; the Wikipedia article gives differences between the seven dialects in initial position for only the alveolar series by way of comparison, so take plosive changes with a huge grain of salt.}

\{\ipa{e,i}\}\ipa{:} \{\ipa{u,o}\}\ipa{:} \change\ \ipa{i u}\\
\{\ipa{e,o}\}\ipa{j Ew} \change\ \ipa{Ej ow}\\
\{\ipa{e,o}\} \change\ \ipa{a} (rare)\\
\ipa{a:} \change\ \ipa{a}\\
\ipa{e} \change\ \ipa{E}\\
\ipa{E o} \change\ \ipa{i u} / _N\\
\ipa{ej ia} \change\ \ipa{e Ea}\\
\{\ipa{i,u}\} \change\ \ipa{@} / in some unstressed syllables\\
\ipa{e oj Ea} \change\ \ipa{i u E} / when unstressed\\
\ipa{p t} \change\ \ipa{h t\super h} / \#_ (?)\\
\ipa{t} \change\ \ipa{t\super h} / \{\ipa{aw,ow}\}_\\
\ipa{k\super w} \change\ \ipa{tS\super h} / _\{\ipa{e,i}\}\\
\ipa{t k(\super w)} \change\ \ipa{d g} / \{N,L\}_\\
\ipa{p \'{k} k(\super w)} \change\ \{\ipa{w,v}\} \ipa{s k\super h}\\
\ipa{\'{g}} \change\ \ipa{ts} (?)\\
\ipa{b d g(\super w)} \change\ \ipa{p t k}\\
\ipa{g\super w\super H} \change\ \ipa{dZ} / \#_\{\ipa{e,i}\}\\
\ipa{b\super H \'{g}\super H g\super w\super H} \change\ \ipa{b j g} / \#_\\
\ipa{b\super H d\super H \'{g}\super H g\super H g(\super w)\super H} \change\ \ipa{p t j k} / \#_ \\
\ipa{b\super H d\super H \'{g}\super H g\super H g\super w\super H} \change\ \{\ipa{w,v}\} \ipa{d z g Z}\\
\ipa{j} \change\ \ipa{w} / _\ipa{o}\\
\ipa{j} \change\ ?\\
\{\ipa{sk,ks}\} \ipa{kj} \change\ \ipa{ts\super h tS\super h} (?)\\
\{\ipa{sr,rs}\} \change\ \ipa{r:}\\
\ipa{r} \change\ \ipa{r:} / _N\\
\ipa{l} \change\ \ipa{\textsuperimposetilde{l}} / \{C,\ipa{l}V\}_\\
\ipa{l} \change\ \ipa{\textsuperimposetilde{l}} / V_V\\
N\ipa{s s}N \change\ \ipa{s} N\\
N \change\ \ipa{w} / S_S\\
VN \change\ V[+nasal] \change\ V \change\ (?) / _\#, in polysyllables\\
N \change\ \ipa{n} / _\#, in monosyllables\\
\s{N} \change\ \ipa{n} / _\#\\
\ipa{\s{m} \s{n} \s{r} \s{l}} \change\ \ipa{am an ar a\textsuperimposetilde{l}} \\
V \change\ (?) / _(C)\#

\subsubsection{Proto-Indo-European to Southwest Armenian}{\it Mecislau \& Pogostick Man}, the latter citing Wikipedia contributors (2013), ``Armenian Language''. {\it Wikipedia, the Free Encyclopedia}. \textless\url{http://en.wikipedia.org/w/index.php?title=Armenian_language&oldid=582063933}\textgreater

\tab {\it NB: The changes in plosives are the most contentious; the Wikipedia article gives differences between the seven dialects in initial position for only the alveolar series by way of comparison, so take plosive changes with a huge grain of salt.}

\{\ipa{e,i}\}\ipa{:} \{\ipa{u,o}\}\ipa{:} \change\ \ipa{i u}\\
\{\ipa{e,o}\}\ipa{j Ew} \change\ \ipa{Ej ow}\\
\{\ipa{e,o}\} \change\ \ipa{a} (rare)\\
\ipa{a:} \change\ \ipa{a}\\
\ipa{e} \change\ \ipa{E}\\
\ipa{E o} \change\ \ipa{i u} / _N\\
\ipa{ej ia} \change\ \ipa{e Ea}\\
\{\ipa{i,u}\} \change\ \ipa{@} / in some unstressed syllables\\
\ipa{e oj Ea} \change\ \ipa{i u E} / when unstressed\\
\ipa{p t} \change\ \ipa{h t\super h} / \#_ (?)\\
\ipa{t} \change\ \ipa{t\super h} / \{\ipa{aw,ow}\}_\\
\ipa{k\super w} \change\ \ipa{tS\super h} / _\{\ipa{e,i}\}\\
\ipa{t k(\super w)} \change\ \ipa{d g} / \{N,L\}_\\
\ipa{p \'{k} k(\super w)} \change\ \{\ipa{w,v}\} \ipa{s k\super h}\\
\ipa{\'{g}} \change\ \ipa{ts} (?)\\
\ipa{b d g(\super w)} \change\ \ipa{p t k}\\
\ipa{g\super w\super H} \change\ \ipa{dZ} / \#_\{\ipa{e,i}\}\\
\ipa{b\super H \'{g}\super H g\super w\super H} \change\ \ipa{b j g} / \#_\\
\ipa{b\super H d\super H \'{g}\super H g\super H g\super w\super H} \change\ \{\ipa{w,v}\} \ipa{d z g Z}\\
\ipa{j} \change\ \ipa{w} / _\ipa{o}\\
\ipa{j} \change\ ?\\
\{\ipa{sk,ks}\} \ipa{kj} \change\ \ipa{ts\super h tS\super h} (?)\\
\{\ipa{sr,rs}\} \change\ \ipa{r:}\\
\ipa{r} \change\ \ipa{r:} / _N\\
\ipa{l} \change\ \ipa{\textsuperimposetilde{l}} / \{C,\ipa{l}V\}_\\
\ipa{l} \change\ \ipa{\textsuperimposetilde{l}} / V_V\\
N\ipa{s s}N \change\ \ipa{s} N\\
N \change\ \ipa{w} / S_S\\
VN \change\ V[+nasal] \change\ V \change\ (?) / _\#, in polysyllables\\
N \change\ \ipa{n} / _\#, in monosyllables\\
\s{N} \change\ \ipa{n} / _\#\\
\ipa{\s{m} \s{n} \s{r} \s{l}} \change\ \ipa{am an ar a\textsuperimposetilde{l}} \\
V \change\ (?) / _(C)\#

\subsection{Avestan}

\subsubsection{Proto-Indo-European to Avestan}{\it Pogostick Man, Alex Fink, and Tropylium}, the former two citing Wikipedia contributors (2013), ``Proto-Indo-Iranian language". {\it Wikipedia, the Free Encyclopedia}. \textless\url{https://en.wikipedia.org/w/index.php?title=Proto-Indo-Iranian_language&oldid=543625693}\textgreater; and Alex Fink citing \url{https://en.wikipedia.org/wiki/Avestan_phonology}

\tab {\it NB: Tropylium wishes to note that his sound changes are subject to change.}

\ipa{b\super H d\super H \'{g}\super H} \change\ \ipa{b d z}\\
\ipa{\'{k} \'{g}} \change\ \ipa{s z}\\
\ipa{k(\super w) g(\super w)(\super H)} \change\ \ipa{tS dZ} / _E\\
\ipa{k k\super w g(\super w)(\super H)} \change\ \ipa{x k g} / else\\
\ipa{rt} \change\ \ipa{\v{s}} (Alex Fink says that the realization of /\ipa{\v{s}}/ ``is unclear")\\
\ipa{s} \change\ \{\ipa{s,h}\}\\
\ipa{\textsubarch{u}} \change\ \ipa{v}\\
\ipa{l} \change\ \ipa{r}\\
\{\ipa{\s{n},\s{m}}\} \change\ \ipa{a}\\
\{\ipa{\s{l},\s{r}}\} \change\ \ipa{@r(@(r))}\\
\ipa{e e:} \change\ \ipa{a a:}\\
\ipa{o o:} \change\ \{\ipa{a},\ipa{a:}\} \ipa{a:}\\
\ipa{h(j)} \change\ \ipa{Nh} / \ipa{a}_\ipa{a}\\
\ipa{hw} \change\ \ipa{N\super wh} / \ipa{a}_\ipa{a}\\
\ipa{h} \change\ \ipa{N} / \ipa{a}_\ipa{ra}\\
h$_x$ \change\ \O

\subsection{Proto-Indo-European to Proto-Celtic}{\it dhokarena56}, from Matasovi\'{c} (2009), {\it Etymological Dictionary of Proto-Celtic} (ed. Lubotsky).

``PIE Dialectal"\\
--- h$_1$\ipa{e} h$_2$\ipa{e} h$_3$\ipa{e} \change\ \ipa{e a o}\\
--- \ipa{e}h$_1$ \ipa{e}h$_2$ \ipa{e}h$_3$ \change\ \ipa{e: a: o:}\\
--- H \change\ \ipa{a} / C_C ! \#_\\
--- SS \change\ \ipa{s:}\\
--- \O\ \change\ \ipa{a} / CR_HC\\
--- H \change\ \O\ / V_C when pretonic\\
--- H \change\ \ipa{a} / \#R_C\\
--- \'{K} \change\ K

Early Proto-Celtic\\
--- \ipa{g\super w} \change\ \ipa{b}\\
--- \ipa{h} \change\ \O\ / C_\\
--- \O\ \change\ \ipa{i} / C\{\ipa{l,r}\}S\\
--- \ipa{e} \change\ \ipa{a} / _R\ipa{a} (short \ipa{a} only), though ``[t]he e was often restored by analogy''\\
--- \O\ \change\ \ipa{a} / C_RC\\
--- H \change\ \O\ / ``if not in a syllabic position"\\
--- \ipa{p}\textellipsis \ipa{k\super w} \change\ \ipa{k\super w}\textellipsis \ipa{k\super w}\\
--- \ipa{e:} \change\ \ipa{i:}\\
--- \ipa{o:} \change\ \ipa{u:} / in U\#\\
--- V\ipa{:} \change\ V[-long] / _RC\\
--- C$_1$C$_2$ \change\ \ipa{x}C$_2$ / if C$_2$ was a plosive or \ipa{s}\\
--- \ipa{p} \change\ \ipa{b} / _\{\ipa{r,l}\}

Late Proto-Celtic\\
--- \ipa{p} \change\ \ipa{w} / B_N\\
--- \ipa{p} \change\ \ipa{f} \\
--- \ipa{o: ej} \change\ \ipa{a: e:}\\
--- \ipa{e} \change\ \ipa{o} / _\ipa{w}\\
--- \ipa{u} \change\ \ipa{o} / _\ipa{w}O

\subsubsection{Proto-Indo-European to Old Irish}{\it dhokarena56}

``Laryngeal rules (the ones common to all branches except Anatolian)''\\
K\ipa{\super w} \change\ K\\
``The PIE rules for the voicing of s \change\ z, as in [nizdos] for *nisdos, are assumed to apply"\\
C\ipa{\super H} \change\ C\\
\ipa{e:} \change\ \ipa{i:} / ! _\{\ipa{i,u}\}\\
Obstruent clusters assimilate in voicing to that of the final obstruent\\
\ipa{t:} \change\ \ipa{s:}\\
\ipa{p} \change\ \ipa{f} / \{V,\#\}_\\
\ipa{f} \change\ \ipa{x} / _O\\
\ipa{f} \change\ \O\ / else\\
\ipa{\s{r} \s{l}} \change\ \{\ipa{ri,ra}\} \{\ipa{li,la}\} / _\{S,R\} (which vowel crops up is unpredictable)\\
\ipa{\s{r} \s{l}} \change\ \{\ipa{ra,ar}\} \{\ipa{la,al}\} / _\{\ipa{s},CC,V,\#\} (the results are unpredictable)\\
\ipa{\s{m} \s{n}} \change\ \ipa{am an} / _\{\ipa{s},(\{\ipa{m,j,w})V\}\\
\ipa{\s{m} \s{n}} \change\ \ipa{em en} / else\\
Stress change:\\
--- Pronouns, articles, and conjunctions become unstressed.\\
--- First syllables stress in all verbal imperatives.\\
--- First syllables stress in all other parts of speech except preverbs and the exceptions noted above.\\
--- Second syllables receive stress otherwise.\\
--- ``This, unlike the preceding rules, remained a morphologically conditioned rule in Old Irish."\\
\ipa{g\super w} \change\ \ipa{b} / \#_V ! _\ipa{u(:)}\\
\ipa{g\super w} \change\ \ipa{b} / \#_N\\
\ipa{g\super w} \change\ \ipa{b} / C_V\\
\ipa{g\super w} \change\ \ipa{g}\\
\ipa{p t k k\super w b d g m n l r s} \change\ \ipa{f T x x\super w v D G} M N L R \ipa{h} / V(\#)_\{R,V\} (``We don't know the exact values of lenited /m n l r/. We can guess that lenited m became a nasalized labial continuant of some sort, but beyond that, we don't know.")\\
\ipa{k} \change\ \ipa{x} / V_\ipa{t}\\
\ipa{m} \change\ \ipa{n} / V_\#; ``[i]t is thought that the vowel needs to be unstressed, but this is not certain"\\
V\ipa{:} \change\ V / _N\#; ``[i]t is thought that the long vowel probably needed to be unstressed- again, this is uncertain"\\
\ipa{p t k k\super w b d g} \O\ \change\ \ipa{b d g g\super w mb nd Ng n} / \ipa{n}\#_"V\\
\ipa{o:} \change\ \ipa{u:} / _(C\textellipsis)\#\\
\ipa{o:i} \change\ \ipa{u:} / _\#\\
\ipa{o:} \change\ \ipa{a:} / else\\
V\ipa{:} \change\ V[-long] / _H (includes diphthongs)\\
``The following three rules only apply if the vowel is unstressed":\\
--- \ipa{e} \change\ \ipa{i} / _(C\textellipsis)\#\\
--- \ipa{o} \change\ \ipa{a} / _\{(C\textellipsis),\ipa{u}\}\#\\
--- \{\ipa{ai,oi}\} \change\ \ipa{i:} / _\#\\
``The following two rules apply if the vowel in question is stressed or follows the stressed syllable"; consonant clusters cannot be /\ipa{nt nd}/:\\
--- \ipa{i u} \change\ \ipa{e o} / _C(\textellipsis C)\{\ipa{a(:),e(:),o(:)}\}\\
--- \ipa{e o} \change\ \ipa{i u} / _C(\textellipsis C)\{H,\ipa{j}\}\\
C \change\ C\ipa{\super j} / _\{F,\ipa{j}\}\\
C \change\ C\ipa{\super w} / _\{B,\ipa{w}\}\\
K\ipa{\super w} \change\ K\\
For the following: ``The book says nothing about length in the input vowels, but I think they could be either short or long from the examples given."\\
--- \ipa{n} \change\ \O\ / \{\ipa{i,o,u}\}_\{\ipa{p,t,k,s}\}\\
--- \{\ipa{a,e}\}\ipa{n} \change\ \ipa{e:} / _\{\ipa{p t k s}\}\\
\ipa{w} \change\ \ipa{f} / \#_\\
\ipa{w} \change\ \O\ / \{\#,C\}C_\\
\ipa{w} \change\ \O\ / \{\ipa{T,x}\}_\\
\ipa{w} \change\ \O\ / V_\{V,\#\}\\
\ipa{w} \change\ \ipa{v} / else\\
``The following changes. . .are, quoth the book, `somewhat approximative'":\\
--- \{\ipa{p,t}\} \change\ \O\ / \#\ipa{s}_\ipa{r}\\
--- \{\ipa{p,t}\} \change\ \O\ / \#\ipa{s}_ ``(although it says that occasionally st \textgreater\ t / \#_)''\\
--- \ipa{s} \change\ \O\ / [anything]\{\ipa{l,r}\}_O\\
--- \ipa{hn hm} \change\ \ipa{n: m:} / [anything]_ (``[t]his change is a bit speculative")\\
--- ``[A] sequence of two plosives becomes a geminate of the second one"\\
--- \ipa{st zd} \change\ \ipa{s: d:} / [anything]_\\
--- \{\ipa{l,h}\}\ipa{l} \{\ipa{l,h}\}\ipa{r l}\{\ipa{p,s,n}\} \ipa{r}\{\ipa{p,s}\} \ipa{ln} \change\ \ipa{l: r: l: r:} (\ipa{l:}?) / [anything]_\\
--- C\ipa{:} \change\ C[-long]\\
V \change\ \O\ / C_\# when unstressed ! C = \ipa{j}\\
C(\textellipsis C) \change\ \O\ / _\# ! /\ipa{l r}/ and clusters containing them; ``[t]his remained a phonologically conditioned rule in OIr"\\
\{\ipa{au,eu,ou}\} \change\ \ipa{o:}\\
\ipa{ei} \change\ \ipa{e:}\\
\ipa{o:} \change\ \ipa{ua} / _[anything], when stressed\\
\ipa{e:} \change\ \ipa{ia} / _\{\#,C\ipa{\super j}\} when stressed; ``ai and oi remain, but are written as \textless ae ai oe oi\textgreater\, seemingly randomly"\\
\ipa{j} \change\ \O\\
The second and third rule below ``may well have been for the most part optional"; every one of the three ``only applies to unstressed vowels" and ``remained as a phonologically conditioned rule":\\
--- V\ipa{:} \change\ V[-long]\\
--- \ipa{a} \change\ \ipa{e} / _\#\\
--- \{\ipa{e,o}\} \change\ \ipa{a} / _[anything]\\
V \change\ V\ipa{:} / _\#, when stressed; ``[t]his remained as a phonologically conditioned rule in OIr"\\
VOR \change\ V\ipa{:}R; ``this is a tad unclear, because in some instances it didn't seem to apply"\\
V \change\ \O\ / \#UU(_)U(U(_)U) / unstressed; this ``remained as a phonologically conditioned rule in OIr"; ``[t]hat's a little unclear, so let me try and enumerate: in words of more than three syllables, every other vowel (only the even ones) dropped, if it's unstressed. In some words, syncope didn't apply because it would create an unwieldy consonant cluster: so PIE *komaktyom \change\ OIr cumachte, not *cumchte"

\subsubsection{Proto-Celtic to Middle Welsh}{\it Dewrad \& Pogostick Man}, the latter citing Willis (David), ``Old and Middle Welsh"

\ipa{k\super w} \change\ \ipa{p}\\
V\ipa{:} \change\ V / _\#\\
\ipa{ei} \change\ \ipa{e:}\\
\ipa{st} \change\ \ipa{s:} (with some exceptions)\\
\ipa{ai} \change\ \ipa{E}\\
\ipa{s} \change\ \O\ / V_V\\
V \change\ \ipa{@} / _(C)\#, also in proclitics\\
\ipa{s} \change\ \O\ / \ipa{x}_\\
\{\ipa{au,eu,ou}\} \change\ \O\\
\ipa{u:} \{\ipa{oi,O:}\} \change\ \ipa{y: u:}\\
\ipa{j} \change\ \ipa{D} / V_\\
\ipa{i u} \change\ \ipa{e o} / _C\ipa{a}\\
\ipa{y:} \change\ \ipa{1}\\
\ipa{p t k} \{\ipa{b,m}\} \ipa{d g} \change\ \ipa{b d g v D G} / _V\\
\ipa{a:} \change\ \ipa{O:}\\
\ipa{a o} \change\ \ipa{ei} \{\ipa{1,ei}\} / _(C\textellipsis)\ipa{j}(C\textellipsis)\#\\
\ipa{a} \change\ \{\ipa{1,ei}\} / _(C\textellipsis)\ipa{j}(C\textellipsis)\#\\
V \change\ \ipa{1} / _(C\textellipsis)\ipa{j}(C\textellipsis)\#\\
\{\ipa{a,o}\} \change\ \ipa{e} / _(C\textellipsis)\ipa{i(:)}\\
\{\ipa{a,e,o}\} \change\ \ipa{ei} / _(C\textellipsis)\ipa{j}\\
V \change\ \O\ / _\#\\
\ipa{mb nd Ng} \change\ \ipa{m: n: N:}\\
\ipa{e} \change\ \ipa{i} / _N\\
\$ \change\ \ipa{h} / V_ (what \$ is is unclear)\\
V \change\ \O\ / _[+intertonic]\\
\ipa{p: t: k:} \change\ \ipa{f T x}\\
\ipa{p t k} \change\ \ipa{f T x} / \{\ipa{r,l}\}_\\
\ipa{G} \change\ \ipa{i} / _C\\
\ipa{xt} \change\ \ipa{iT}\\
\ipa{G} \change\ \ipa{i} / C_V\\
\ipa{E:} \change\ \ipa{ui}\\
\ipa{O:} \change\ \ipa{au} / when stressed\\
\ipa{l} \change\ \ipa{\textbeltl} / _\ipa{t}\\
\ipa{w} \change\ \ipa{gw} / \#_\\
\ipa{mp nt Nk} \change\ \ipa{\r*{m} \r*{n} \r{N}}\\
\ipa{O} \change\ \ipa{@} / \#_\ipa{s}C\\
\ipa{l r} \change\ \ipa{\textbeltl\ \r*{r}} / \#_\\
\ipa{G} \change\ \ipa{@} / _\#

\subsection{Proto-Indo-European to Dacian}{\it Pogostick Man}, from Wikipedia contributors (2013), ``Dacian language". {\it Wikipedia, the Free Encyclopedia}. \textless\url{https://en.wikipedia.org/w/index.php?title=Dacian_language&oldid=582406161}\textgreater

\ipa{o} \change\ \ipa{a}\\
\ipa{e} \change\ \ipa{je} / in open syllables, when stressed\\
\ipa{e} \change\ \ipa{ja} / in closed syllables, when stressed\\
\ipa{e:} \change\ \ipa{a:}\\
\ipa{oi wo wj ow} \change\ \ipa{ai wa vi aw}\\
\ipa{ei} \change\ \{\ipa{ei,i}\} (``PIE \textbf{*ei} evolution is not well reconstructed yet")\\
\ipa{b\super H d\super H \'{g}\super H g\super H g\super w\super H} \change\ \ipa{b d \'{g} g g\super w}\\
\ipa{\'{k} \'{g}} \change\ \ipa{ts dz}\\
\{\ipa{k\super w,kw}\} \{\ipa{g\super w,gw}\} \change\ \ipa{tS dZ} (\change\ \ipa{s}\raisebox{-0.6ex}{\textasciitilde}\ipa{z} \ipa{z} ?) / _E\\
\{\ipa{k\super w,kw}\} \{\ipa{g\super w,gw}\} \change\ \ipa{k g} / else

\subsection{Proto-Indo-European to Common Germanic}{\it Siride}

\ipa{b\super H d\super H g\super H} \change\ \ipa{B D G}\\
\ipa{b d g} \change\ \ipa{p t k}\\
\ipa{p t k} \change\ \ipa{f T x}\\
\ipa{f T s x} \change\ \ipa{B D z G} ``(Except initially or following IE stress)''\\
\{\ipa{i,j}\} \{\ipa{u,w}\} \change\ \ipa{j w} / V[+short]C_\\
\{\ipa{i,j}\} \{\ipa{u,w}\} \change\ \ipa{ij uw}\\
\ipa{a:} \change\ \ipa{o:}\\
\ipa{e} \change\ \ipa{i}\\
\ipa{e:} \change\ \ipa{\ae:}\\
\ipa{o} \change\ \ipa{a}\\
\ipa{ei oi} \change\ \ipa{i: ai}\\
\ipa{e:i} \{\ipa{o:i,a:i}\} \change\ \ipa{e: o:} (?)\\
\ipa{eu ou} \change\ \ipa{iu au}

\subsubsection{Common Germanic to Gothic}{\it Pogostick Man}, from Wright, Joseph (1910). {\it Grammar of the Gothic Language}, 2nd Ed.; and Wikipedia contributors (2014). ``Gothic language". {\it Wikipedia, the Free Encyclopedia}. \textless\url{http://en.wikipedia.org/w/index.php?title=Gothic_language&oldid=635946920}\textgreater

\tab {\it NB: Wright seems to regard Germanic labiovelars as sequences of velar + \ipa{w} if I'm reading this right; additionally, it looks like some of what Wright considers diphthongs may have been long monophthongs.}

Stressed vowels:\\
--- \ipa{o e} \change\ \ipa{u i}\\
--- \ipa{u} \change\ \ipa{O} / _\{\ipa{r,h}\} (unless this \ipa{r} ``arose from older {\bf s} by assimilation'')\\
--- \ipa{i} \change\ \ipa{E} / _\{\ipa{r,h,\textturnw}\}\\
--- \ipa{\ae:} \change\ \ipa{e:}\\
--- \ipa{ew} \change\ \ipa{iw}

Unstressed vowels:\\
--- V[- long] \change\ \O\ / _\# ! V = \ipa{u}\\
--- V[- long] \change\ \O\ / U_C\# ! V = \ipa{u}\\
--- Inherited ``long final vowels\textellipsis became shortened in polysyllabic words, when the vowels in question originally had the `broken' accent, but remained unshortened when they originally had the `slurred' accent"\\
--- \ipa{aj} \change\ \ipa{a} / U_\#\\
--- ``Originally long diphthongs became shortened in final syllables"

\ipa{iw} \change\ \ipa{ju} / [- stress]\\
\ipa{w} \change\ \ipa{\textsubarch{u}} / V[- long]_\{\#,C\}\\
\ipa{w} \change\ \O\ / \ipa{o:}_\ipa{j}\\
\ipa{o:w \ae:j} \change\ \ipa{O: E:} / _V\\
\ipa{j} \change\ \ipa{i} / C_\# ``after the loss of a final vowel or syllable"\\
\ipa{ij} \change\ \ipa{i:} / _s ``after the loss of a vowel in final syllables"\\
\ipa{ij} \change\ \ipa{i} / _\# ``after the loss of a final vowel or syllable"\\
V\ipa{w} \change\ \ipa{u} / _\ipa{s} (to wit, the vowel is deleted and the *\ipa{w} syllabifies)\\
``In a few instances medial {\bf -w-} (or {\bf -ww-} the origin of which is uncertain) after short vowels became {\bf -ggw-} in Gothic\textellipsis"; similarly, medial *-j(j)- became -ddj- in uncertain conditions\\
\ipa{iji} \change\ \ipa{i:} / U[- stress](C\textellipsis)_\\
\ipa{iji} \change\ \ipa{i:} / U[+ long + closed]_ in the stem\\
\ipa{i} \change\ \O\ / _\ipa{ji}\\
\ipa{m} \change\ \ipa{B} / C[- voiced]_\ipa{n}, when medial\\
\ipa{m} \change\ \ipa{F} / C[+ voiced]_\ipa{n}, when medial\\
\ipa{n:} \change\ \ipa{n} / _C ! _\ipa{j}\\
\ipa{B} \change\ \ipa{b} / \{\ipa{r,l}\}_\\
\ipa{D} \change\ \ipa{d} / C[+ voiced]_\\
\ipa{B D G} \change\ \ipa{F T x} / V_(\ipa{s})\#\\
``The final {\bf -h} [= /\ipa{h}/?] in unaccented particles was often assimilated to the initial consonant of the following word"\\
\ipa{G} \change\ \ipa{g} / \#_\\
\ipa{G} \change\ \ipa{g} / C_V\\
``In the forms of the strong verbs, medial {\bf z} was supplanted by {\bf s} through the levelling out of the {\bf s}-forms\textellipsis{\bf z} was also supplanted by {\bf s} in several weak verbs, which in some cases was due to the influence of the corresponding strong verbs"\\
\ipa{z} \change\ \ipa{s} / _\#, though ``[t]his {\bf s} was dropped when it came to stand after an original {\bf s} through the loss of a vowel", though it ``remained when protected by a particle"\\
\ipa{s} \change\ \O\ / V[- long]\ipa{r}_\#\\
\ipa{s} \change\ \ipa{r} / in ``[t]he prep[osition]. {\bf us}\textellipsis before {\bf r} in compounds"\\
\ipa{s} \change\ \O\ / in ``[t]he prep[osition]. {\bf us}\textellipsis in compounds before {\bf st}", though this seems to have been less common

\subsubsection{Common Germanic to West Germanic}{\it Siride}

\ipa{B D G} \change\ \ipa{b d g} / \{\#,"V\}_\\
\ipa{z} \change\ \{\ipa{r},\O\}\\
C \change\ C\ipa{:} / _\ipa{j} ! C = \ipa{r}\\
\ipa{i u} \change\ \ipa{e o} / _\%\{\ipa{a,o}\}\\
\ipa{o:} \change\ \ipa{u:} / _\#

\paragraph{West Germanic to Anglo-Frisian}{\it Siride? \& Pogostick Man}, the latter citing Wikipedia contributors (2014), ``Anglo-Frisian languages". {\it Wikipedia, the Free Encyclopedia}. \textless\url{https://en.wikipedia.org/w/index.php?title=Anglo-Frisian_languages&oldid=602286013}\textgreater; Wikipedia contributors (2014), ``Old Frisian". \textit{Wikipedia, the Free Encyclopedia}. \textless\url{https://en.wikipedia.org/w/index.php?title=Old_Frisian&oldid=559739599}\textgreater; and Wikipedia contributors (2014), ``Old English phonology". {\it Wikipedia, the Free Encyclopedia}. \textless\url{https://en.wikipedia.org/w/index.php?title=Old_English_phonology&oldid=602537992}\textgreater

\ipa{a} \change\ \ipa{\~{A}} / _N (short only)\\
VN \change\ \~{V}\ipa{:} / _F\\
\ipa{a} \change\ \ipa{\ae:} / short only, includes diphthongs ! B or *\ipa{\~{a}} in next syllable\\
\ipa{k g} \change\ \ipa{tS j}(?)\\
\ipa{\ae:} \change\ \ipa{a:} ``under to [{\it sic}] the influence of neighboring consonants", but the article doesn't elaborate\\
\ipa{\ae:} \change\ \ipa{e:}\\
\ipa{\ae u} \change\ \ipa{au} (\ipa{\ae} \change\ \ipa{a} / _B in general?)\\
\ipa{a:} \change\ \ipa{\ae:} / ! _N or if nasalized\\
\ipa{i o} \change\ \ipa{e a} / unstressed\\
\ipa{ai au eu} \change\ \{\ipa{e:,a:}\} \ipa{a: ia}\\
\ipa{ia iu} \change\ \ipa{ja: ju:}\\
\ipa{a} \change\ \ipa{\ae} / ! _N or if nasalized, or if *B or *\ipa{\~{a}} in next syllable\\
\ipa{h} \change\ \O\ / V_V\\
\{\ipa{i,u}\} \change\ \O\ / -\# ! VC_\\
\ipa{T} resists change to \ipa{d} until the 14th Century

\subparagraph{Anglo-Frisian to Old English}{\it Pogostick Man}, from Wikipedia contributors (2011), ``Phonological history of English". {\it Wikipedia, the Free Encyclopedia}. \textless\url{http://en.wikipedia.org/w/index.php?title=Phonological_history_of_English&oldid=453796112}\textgreater

\ipa{\~{A}:} \change\ \ipa{\~{o}:}\\
V[+nas] \change\ V[-nas]\\
\{\ipa{i,u}\} \change\ \O / _\# ! V[-long]C_\#\\
\ipa{k G g} \change\ \ipa{tS J dZ} / ``in certain complex circumstances''

\subparagraph{Old English to Kentish Middle English}{\it Pogostick Man}, from Moore, Samuel (1919), {\it Historical Outlines of English Phonology and Midd le English Grammar for Courses in Chaucer, Middle English, and the History of the English Language}; and Wikipedia contributors (2011), ``Middle English phonology". {\it Wikipedia, the Free Encyclopedia}. \textless\url{http://en.wikipedia.org/w/index.php?title=Middle_English_phonology&oldid=456896605}\textgreater

V\ipa{:} \change\ V[-long] / _C\{\ipa{:},C\} ! _\ipa{st}\{\#,V\} or when preceding a cluster which had triggered a vowel to become long in Old English; the book gives ``Christ" vs. ``Christmas" as an example\\
\ipa{eA e:A eo e:o} \change\ \ipa{A E: e e:}\\
\ipa{\ae j} \change\ \ipa{aj} \change\ \ipa{ej}\\
\{\ipa{\ae:j,e(:)j}\} \change\ \ipa{ej}\\
\ipa{AG} \change\ \ipa{Aw}\\
\{\ipa{eAh,eA\c{c},eAx,eAJ,eAG}\} \change\ \ipa{Aw}\\
\ipa{e:Aw i:w} \change\ \ipa{ew ju}\\
\{\ipa{A:w,A:G,o:w}\} \change\ \ipa{O:w}\\
\ipa{oG} \change\ \ipa{O:w} / _V\\
\{\ipa{o(:)ht,A:ht}\} \change\ \ipa{ow}\\
\ipa{A: y(:)} \change\ \ipa{O: e(:)}\\
\ipa{A e o} \change\ \ipa{A: E: o:} / in U[+open] ! in \#U with the following U containing /\ipa{i:}/ or ending in one of /\ipa{m n r l}/\\
\ipa{e:A e:o i:e} become sounds of uncertain identity; Moore says they were probably diphthongs\\
V\ipa{:} \change\ V[-long] / in \#U before a U with /\ipa{i:}/\\
\ipa{m} \change\ \ipa{n} \change\ \O\ / _\# when unstressed\\
\ipa{hn} \{\ipa{wl,hl}\} \ipa{hr} \change\ \ipa{w l r}\\
\ipa{G} \change\ \ipa{g} / \#_\\
\ipa{G} \change\ \ipa{w} / C_V

\subparagraph{Old English to Midlands Middle English}{\it Pogostick Man}, from Moore, Samuel (1919), {\it Historical Outlines of English Phonology and Midd le English Grammar for Courses in Chaucer, Middle English, and the History of the English Language}; and Wikipedia contributors (2011), ``Middle English phonology". {\it Wikipedia, the Free Encyclopedia}. \textless\url{http://en.wikipedia.org/w/index.php?title=Middle_English_phonology&oldid=456896605}\textgreater

V\ipa{:} \change\ V[-long] / _C\{\ipa{:},C\} ! _\ipa{st}\{\#,V\} or when preceding a cluster which had triggered a vowel to become long in Old English; the book gives ``Christ" vs. ``Christmas" as an example\\
\ipa{eA e:A eo e:o} \change\ \ipa{A E: e e:}\\
\ipa{\ae j} \change\ \ipa{aj} \change\ \ipa{ej}\\
\{\ipa{\ae:j,e(:)j}\} \change\ \ipa{ej}\\
\ipa{AG} \change\ \ipa{Aw}\\
\{\ipa{eAh,eA\c{c},eAx,eAJ,eAG}\} \change\ \ipa{Aw}\\
\ipa{e:Aw i:w} \change\ \ipa{ew ju}\\
\{\ipa{A:w,A:G,o:w}\} \change\ \ipa{O:w}\\
\ipa{oG} \change\ \ipa{O:w} / _V\\
\{\ipa{A:ht,o(:)ht}\} \change\ \ipa{ow} \\
\ipa{A: y(:)} \change\ \ipa{O: i(:)}\\
\ipa{A e o} \change\ \ipa{A: E: o:} / in U[+open] ! in \#U with the following U containing /\ipa{i:}/ or ending in one of /\ipa{m n r l}/\\
V\ipa{:} \change\ V[-long] / in \#U before a U with /\ipa{i:}/\\
\ipa{m} \change\ \ipa{n} \change\ \O\ / _\# when unstressed\\
\ipa{hn} \{\ipa{wl,hl}\} \ipa{hr} \change\ \ipa{w l r}\\
\ipa{G} \change\ \ipa{g} / \#_\\
\ipa{G} \change\ \ipa{w} / C_V\\
\{\ipa{e,A,o}\} \change\ \ipa{@} \change\ \O\ / _\#

\subparagraph{Midlands Middle English to Early Modern English}{\it Pogostick Man}, from FireSpeakerWiki contributors (2013), ``English sound changes". {\it FireSpeakerWiki}. \textless\url{http://wiki.firespeaker.org/English_sound_changes}\textgreater

\ipa{U} \change\ \ipa{7} ! P_ and _\ipa{l}\\
\ipa{mb Ng} \change\ \ipa{m N} / _\#\\
\ipa{tj sj dj zj} \change\ \ipa{tS S dZ Z} / ! _\ipa{u:} (perhaps only before stressed \ipa{u:}?)\\
\ipa{a A} \{\ipa{E,I,7}\} \change\ \ipa{A: O: 3:} / _\ipa{\*r}\{C,\#\}\\
\ipa{AU} \change\ \ipa{A:} / _P\\
\ipa{AU} \change\ \ipa{A:} / _N (sometimes)\\
\ipa{AU} \change\ \ipa{O:} / else\\
``[A] large number of cases that were \ipa{A:} have become \ipa{O:} subsequently for non-phonetic reasons, like laundry"\\
\ipa{a} \change\ \ipa{A:} / ``in a few words, like `father'"\\
\ipa{a} \change\ \ipa{\ae} / else\\
\ipa{@I @U} \change\ \ipa{Ai \ae U} / ``in some parts of South-Eastern England"\\
\ipa{@I @U} \change\ \ipa{aI aU} / ``in most of Britain"\\
\ipa{e: o:} \change\ \ipa{eI oU} / ! _\ipa{\*r}

\subparagraph{Early Modern English to American English}{\it Pogostick Man}, from FireSpeakerWiki contributors (2013), ``English sound changes". {\it FireSpeakerWiki}. \textless\url{http://wiki.firespeaker.org/English_sound_changes}; and my Phonetic Description class\textgreater

\ipa{\ae} \change\ \ipa{\ae:} (e.g., NYC) or \ipa{A:} (e.g., Boston) / _\{F[-voiced],N[-voiced]\} (``words which change vary between dialects")\\
\ipa{6} \change\ \ipa{6:} \change\ \ipa{O:} / _F[-voiced]\\
\ipa{\ae: A: O:} \change\ \ipa{\ae @}\raisebox{-0.6ex}{\textasciitilde}\ipa{e@ A O}\\
\ipa{\textturnw} \change\ \ipa{w} (regional)\\
\ipa{l} \change\ \ipa{\textsuperimposetilde{l}} / ``in some conditions''\\
\ipa{I} \change\ \ipa{i} / _\# when unstressed\\
\{\ipa{t,d}\} \change\ \ipa{R} / V_V[-stress]\\
\ipa{i u e} \change\ \ipa{I U E} / _\ipa{\*r}\\
\ipa{o O} \change\ \ipa{O 6} / _\ipa{\*r} (most dialects have at least one if not both)\\
\ipa{\ae} \change\ \ipa{E} / _\ipa{\*r}\\
\ipa{j} \change\ \O\ / \{\ipa{T,s,z,l,n,t,d}\}_ when in onset position\\
\ipa{6} \change\ \ipa{O} / _K ``(partial)"\\
\ipa{\ae} \change\ \ipa{E@} / _\{\ipa{n,m}\} ``and others depending on dialect"\\
\ipa{e@} \change\ \ipa{e:}\\
\ipa{i@} \change\ \ipa{I} (ongoing)\\
\ipa{w} \change\ \O\ / C_\ipa{\*r} for some C (toward(s), quart(er), sword)\\
\ipa{t} \change\ \O\ / \ipa{f}_\ipa{\s{n}}\\
Stuff regarding syllabification (e.g., of /\ipa{\*r}/) and hiatus\\
Loss of pretonic /\ipa{@}/ in \#U (ongoing)

\subparagraph{Early Modern English to Australian English}{\it Pogostick Man}, from FireSpeakerWiki contributors (2013), ``English sound changes". {\it FireSpeakerWiki}. \textless\url{http://wiki.firespeaker.org/English_sound_changes}\textgreater; and my Phonetic Description class

\ipa{3:\*r A:\*r O:\*r e:\*r o:\*r i:\*r u:\*r} \change\ \ipa{3: A: O: E@ O@ I@ U@} / syllable-finally\\
\ipa{A} \change\ \ipa{6}\\
\ipa{\ae} \change\ \ipa{\ae:} \change\ \ipa{A:} / _\{F[-voiced],N[-voiced]\}\\
\ipa{6} \change\ \ipa{6:} \change\ \ipa{O:} / _F[-voiced]\\
\ipa{\textturnw} \change\ \ipa{w}\\
\ipa{l} \change\ \ipa{\textsuperimposetilde{l}} (the conditions of this are not elaborated upon)\\
\ipa{oU i:} \change\ \ipa{OU I@} / _\ipa{\textsuperimposetilde{l}}\\
\ipa{oU i:} \change\ \ipa{@U Ii} / else\\
\ipa{u:} \change\ \ipa{U@} \change\ \ipa{u:} / _\ipa{\textsuperimposetilde{l}} ! in Queensland and New South Wales\\
\ipa{u:} \change\ \ipa{U0} \change\ \ipa{0:} / else\\
\ipa{\textsuperimposetilde{l}} \change\ \ipa{@\textsuperimposetilde{l}} / ! if one of the above vowel changes after the formation of /\ipa{\textsuperimposetilde{l}}/ apply\\
\ipa{I} \change\ \ipa{i:} / _\# when unstressed\\
\ipa{I} \change\ \ipa{i:} / ``unstressed foot-finally if the next syllable is stressed and begins with /k \ipa{g tS dZ S Z}/"\\
\ipa{I} \change\ \ipa{@} / unstressed\\
\ipa{@} \change\ \ipa{I} / _\{\ipa{k,g,tS,dZ,S,Z,v}\}\\
\ipa{t d} \change\ \ipa{R} / V_V[-stress]\\
\ipa{O@} \change\ \ipa{O:}\\
\ipa{E@ U@} \change\ \ipa{E: O:} (ongoing)\\
\ipa{U@} \change\ \ipa{o:} / ! \{\ipa{j,dZ}\}_\\
\ipa{U@} \change\ \ipa{0:w@} / ``almost always otherwise, but see [above vowel changes after /\ipa{\textsuperimposetilde{l}}/ is formed]"\\
\ipa{o@} \change\ \ipa{o:}\\
\ipa{j} \change\ \O\ / \%\{\ipa{T,s,z,l}\}_"V\\
\ipa{sj zj lj} \change\ \ipa{S Z j}\raisebox{-0.6ex}{\textasciitilde}\ipa{\ipa{\textsuperimposetilde{l}}j} / else ``(\ipa{j}\raisebox{-0.6ex}{\textasciitilde}\ipa{\textsuperimposetilde{l}j} fluctuation is formality)''\\
\ipa{\textsuperimposetilde{l}j} \change\ \ipa{\textsuperimposetilde{l}i} / ``after any segment after which coda-/\ipa{\textsuperimposetilde{l}}/ is forbidden, e.g. failure [\ipa{f\ae i\textsuperimposetilde{l}i5}]"\\
\ipa{tj dj} \change\ \ipa{tS dZ}\\
\ipa{O:} \change\ \ipa{O} / _\ipa{\*r}V[-stress]\\
\ipa{O:} \change\ \ipa{O} / _F[-voiced]\\
\ipa{O} \change\ \ipa{O:} / ``in `gone' and some derivatives"\\
\ipa{\ae} \change\ \ipa{\ae:} / _\{\ipa{n,m,g,\textsuperimposetilde{l}}\% ! _\ipa{n,m,g,\textsuperimposetilde{l}}\}\%\{\ipa{j,w}\} or a form of a strong verb\\
\ipa{\ae} \change\ \ipa{\ae:} / _\ipa{d} (rare) ! form of a strong verb

\subparagraph{Early Modern English to British English}{\it Pogostick Man}, from FireSpeakerWiki contributors (2013), ``English sound changes". {\it FireSpeakerWiki}. \textless\url{http://wiki.firespeaker.org/English_sound_changes}\textgreater

\ipa{3:\*r A:\*r O:\*r e:\*r o:\*r i:\*r u:\*r} \change\ \ipa{3: A: O: E@ O@ I@ U@} / syllable-finally\\
\ipa{A} \change\ \ipa{6}\\
\ipa{\ae} \change\ \ipa{\ae:} \change\ \ipa{A:} / _\{F[-voiced],N[-voiced]\}\\
\ipa{6} \change\ \ipa{6:} \change\ \ipa{O:} / _F[-voiced]\\
\ipa{\textturnw} \change\ \ipa{w}
\ipa{l} \change\ \ipa{\textsuperimposetilde{l}} / ``in coda"\\
\ipa{oU} \change\ \ipa{@U}\\
``LOT-CLOTH split reversed properly"

\subparagraph{Old English to Northern Middle English}{\it Pogostick Man}, from Moore, Samuel (1919), {\it Historical Outlines of English Phonology and Midd le English Grammar for Courses in Chaucer, Middle English, and the History of the English Language}; and Wikipedia contributors (2011), ``Middle English phonology". {\it Wikipedia, the Free Encyclopedia}. \textless\url{http://en.wikipedia.org/w/index.php?title=Middle_English_phonology&oldid=456896605}\textgreater

V\ipa{:} \change\ V[-long] / _C\{\ipa{:},C\} ! _\ipa{st}\{\#,V\} or when preceding a cluster which had triggered a vowel to become long in Old English; the book gives ``Christ" vs. ``Christmas" as an example\\
\ipa{eA e:A eo e:o} \change\ \ipa{A E: e e:}\\
\ipa{\ae j} \change\ \ipa{aj} \change\ \ipa{ej}\\
\{\ipa{\ae:j,e(:)j}\} \change\ \ipa{ej}\\
\ipa{AG} \change\ \ipa{Aw}\\
\{\ipa{eAh,eA\c{c},eAx,eAJ,eAG}\} \change\ \ipa{Aw}\\
\ipa{e:Aw i:w} \change\ \ipa{ew ju}\\
\{\ipa{A:w,A:G,o:w}\} \change\ \ipa{O:w}\\
\ipa{oG} \change\ \ipa{O:w} / _V\\
\{\ipa{A:ht,o(:)ht}\} \change\ \ipa{ow} \\
\ipa{A:} most likely became one of \{\ipa{e:,E:}\}\\
\ipa{A e o} \change\ \ipa{A: E: o:} / in U[+open] ! in \#U with the following U containing /\ipa{i:}/ or ending in one of /\ipa{m n r l}/\\
\ipa{y(:)} \change\ \ipa{i(:)}\\
V\ipa{:} \change\ V[-long] / in \#U before a U with /\ipa{i:}/\\
\ipa{n} \change\ \O\ / _\# when unstressed (not clear as to whether \ipa{m} \change\ \ipa{n} beforehand in this position or not)\\
\ipa{j tS} \change\ \ipa{g k}\\
\ipa{S} \change\ \ipa{s} / in unstressed syllables\\
\ipa{\textturnw} became a sound spelled $\langle$qu$\rangle$\\
\ipa{hn} \{\ipa{wl,hl}\} \ipa{hr} \change\ \ipa{w l r}\\
\ipa{G} \change\ \ipa{g} / \#_\\
\ipa{G} \change\ \ipa{w} / C_V\\
\{\ipa{e,A,o}\} \change\ \ipa{@} \change\ \O\ / _\#, when unstressed (it appears this sound may also have gone to /\ipa{I}/)

\subparagraph{Early Northern Middle English to Scots}{\it Pogostick Man}, from Wikipedia contributors (2014), ``Phonological history of Scots". {\it Wikipedia, the Free Encyclopedia}. \textless\url{https://en.wikipedia.org/w/index.php?title=Phonological_history_of_Scots&oldid=582962563}\textgreater; and Wikipedia contributors (2014), ``Scottish Vowel Length Rule". {\it Wikipedia, the Free Encyclopedia}. \textless\url{https://en.wikipedia.org/w/index.php?title=Scottish_vowel_length_rule&oldid=589349104}\textgreater

\ipa{b} \change\ \O\ / \ipa{m}_\ipa{l}\\
\ipa{t} \change\ \O\ / \{\ipa{p,k}\}_\# (``except in some inflected forms" for *\ipa{kt})\\
\ipa{d} \change\ \O\ / \ipa{n}_\\
\ipa{d} \change\ \O\ / \ipa{l}_\#\\
\ipa{s} \change\ \ipa{S} / _E (E_ also?)\\
\ipa{f} \change\ \O\ ``in certain contexts"\\
/\ipa{k g}/ remain unpalatalized when E_\\
\{\ipa{F,x}\} \change\ \O\ / _\# (seems to be sporadic)\\
\ipa{\textturnw} \change\ \ipa{xw} (some speakers seem to have resisted this)\\
\ipa{og} \change\ \ipa{2u}\\
\ipa{ul} became some sort of diphthong or vowel (possibly one of \ipa{u:, uw, 2w}), but the article isn't very clear\\
\ipa{ol al} \change\ \ipa{ou A:} \change\ \ipa{2u} \{\ipa{A,O}\}\\
Vowel shift:\\
--- \ipa{ai} \change\ \ipa{Ei} \change\ \ipa{@i} / when stem-final\\
--- \ipa{u:} \change\ \ipa{2u} / when-stem final, in northern varieties\\
--- \ipa{\o:} \change\ \ipa{wi} / \{\ipa{k,g}\}_ (in Mid Northern dialects)\\
--- \ipa{\o:} \change\ \ipa{i} (in northern dialects)\\
--- \ipa{\o:} \change\ (\ipa{j})\{\ipa{u,2}\} / _\{\ipa{k,x}\} (outcome varies depending upon dialect)\\
--- \ipa{a} \change\ \ipa{i} / _\ipa{n} (in northern varieties)\\
--- \ipa{a} \change\ \ipa{e} / _\ipa{n} (otherwise)\\
--- \ipa{a} \change\ \{\ipa{E,e}\} / _\ipa{r}C\\
--- \ipa{ai oi ui ei au ou iu E(o)u} \change\ \ipa{e: oe @i i:} \{\ipa{A:,O:}\} \ipa{2u ju j(2)u}\\
--- \ipa{E:} \change\ \ipa{Ei} (\change\ \ipa{@i}?) / in some northern varieties\\
--- \ipa{i: e: E: a: o: u:} \{\ipa{\o:,y:}\} \change\ \ipa{@i i} \{\ipa{i,e}\} \ipa{e o u \o}\\
--- \ipa{\ae} \change\ \ipa{E} / _C[+alveolar]\\
--- \ipa{a O u} \change\ \{\ipa{a,A}\} \ipa{O 2}\\
Application of the Scottish vowel-length rule:\\
--- V \change\ V\ipa{:} / _\{\ipa{r},F[+voiced],\$,\#\}\\
--- \ipa{@i} \change\ \ipa{aI} / _\{\ipa{r},F[+voiced],\$,\#\} (pursuant to the above)

\subparagraph{Old English to Scots}{\it Marcas Brian MacStiof\'{a}in \'{O} Mhaiti\'{u} \'{O} Domhnaill}, from personal research

\tab {\it NB: This is an alternate listing of sound changes from Old English to Scots presented by a native speaker, which leads into a listing of sound changes to the Falkirk dialect.}

\ipa{\ae:} \change\ \ipa{E:}\\
\ipa{A} \change\ \ipa{a} / ! _\{\ipa{l,r}\} (sporadic)\\
\ipa{o:} \change\ \ipa{ju} / \{\ipa{n,x}\}_\\
\ipa{o:} \change\ \ipa{iu} / _K\\
\ipa{o:} \change\ \ipa{\o}\\
\ipa{\o:} \change\ \ipa{e:} ``(not a thorough change)''\\
\ipa{\o} \change\ \ipa{I}\\
\ipa{\ae:(A) e:(o)} \change\ \ipa{E i:}\\
\ipa{E} \change\ \ipa{3i} \change\ \ipa{i} / _\{\ipa{m,\textltailn}\}\\
\ipa{ai} \change\ \ipa{a:}\\
\ipa{e} \change\ \ipa{E} / _\ipa{nt}\\
\{\ipa{y,i}\} \change\ \ipa{I}\\
\ipa{I} \change\ \ipa{3} / _\{K,\ipa{r}\}\\
\ipa{e(o)} \change\ \ipa{E}\\
\ipa{u} \change\ \ipa{U} \change\ \ipa{8} \change\ \ipa{2}\\
\ipa{u:} \change\ \ipa{u}\\
\ipa{o} \change\ \ipa{2} / P_\ipa{r}\\
\ipa{\ae} \change\ \ipa{A} / _\{\ipa{x,l}\}\\
\ipa{\ae(A)} \change\ \ipa{e}\\
\ipa{O\{g,j\}} \change\ \ipa{2u}\\
\ipa{A:} \change\ \ipa{e} / ! _\{\ipa{N,n}\}\\
\ipa{2} \change\ \ipa{Ii} (sporadic)\\
\ipa{a} \change\ \ipa{I} / ``unstressed and/or final''\\
N \change\ \O\ / _C ! _\%C\\
\ipa{xw} \change\ \ipa{\textturnw}\\
S[+ voice] \change\ S[- voice] / _\#\\
\ipa{d}\raisebox{-0.6ex}{\textasciitilde}\ipa{D} \change\ \ipa{d} / V_\ipa{u}\\
\ipa{d}\raisebox{-0.6ex}{\textasciitilde}\ipa{D} \change\ \ipa{D} / V_V\\
\ipa{t} \change\ \O\ / \ipa{p}_\\
\ipa{dZ} \change\ \ipa{tS} / _\#\\
\{\ipa{f,v}\} \change\ \O\ / \{\ipa{l,r},V\}_ (sometimes blocked)\\
\ipa{D} \change\ \O\ / \{\ipa{l,r},V\}_C\\
\ipa{T} \change\ \ipa{h} / _\ipa{I}\\
\ipa{T} \change\ \ipa{f} / \{V,\ipa{r}\}_\# (sometimes blocked)\\
\ipa{T} \change\ \O\\
V \change\ V\ipa{:} / _\{\ipa{r},F[+ voice],V,\#\}\\
\ipa{m} \change\ \ipa{n} / _\ipa{f}\\
\ipa{e} \change\ \ipa{E} / ``unstressed''\\
\O\ \change\ \ipa{\u{@}} / _\{\ipa{n,r}\}\\
\ipa{k} \change\ \O\ / \ipa{n}_\ipa{t} (sporadic)\\
\ipa{m} \change\ \O\ / _\ipa{n}\\
\ipa{l} \change\ \O\ / \ipa{u}_\\
\ipa{l} \change\ \ipa{u} / \{\ipa{O,A}\}_\\
\{\ipa{Ou,Au}\} \change\ \ipa{A}\\
\ipa{t} \change\ \ipa{d} / \ipa{r}_\\
\{\ipa{w,k}\} \change\ \O\ / _\{\ipa{n,r}\}\\
\ipa{m} \change\ \ipa{n} / _\ipa{f}\\
\ipa{s} \change\ \ipa{\:s} / _\{\ipa{t,r}\}\\
\ipa{t} \change\ \ipa{\:t} / _\ipa{r}\\
\ipa{t} \change\ \ipa{P} / V_V\\
\ipa{t} \change\ \ipa{P} / _\#\\
\ipa{u} \change\ \ipa{Y}

\subparagraph{Scots to Falkirk Scots}{\it Marcas Brian MacStiof\'{a}in \'{O} Mhaiti\'{u} \'{O} Domhnaill}, from personal research

\ipa{p}V\ipa{n t}V\ipa{n k}V\ipa{n} \change\ \ipa{P\s{m} P\s{n} P\s{N}} / _\#\\
V \change\ \~{V}\ipa{:} / _\ipa{n}C ``(works across word boundaries)''\\
\ipa{n} \change\ \O\ / V_C (in words of more than one syllable)\\
\ipa{k} \change\ \ipa{\c{c}} / V_\# ! _C ``(sometimes)''; ``(except when phonemic)''\\
\ipa{g} \change\ \ipa{J} / V_\# ! _C ``(most times)''\\
\ipa{p b} \change\ \ipa{F B} / \{\#,V\}_ ! _C\\
V \change\ \~{V} / N_\\
\ipa{n} \change\ \O\ / V_V\# ``in some disyllabic words''\\
\ipa{l} \change\ \ipa{\;L}\\
\ipa{\;L} \change\ \ipa{U} / \{\ipa{a,E}\}_, typically ! _V

\subparagraph{Old English to Southern Middle English}{\it Pogostick Man}, from Moore, Samuel (1919), {\it Historical Outlines of English Phonology and Midd le English Grammar for Courses in Chaucer, Middle English, and the History of the English Language}; and Wikipedia contributors (2011), ``Middle English phonology". {\it Wikipedia, the Free Encyclopedia}. \textless\url{http://en.wikipedia.org/w/index.php?title=Middle_English_phonology&oldid=456896605}\textgreater

V\ipa{:} \change\ V[-long] / _C\{\ipa{:},C\} ! _\ipa{st}\{\#,V\} or when preceding a cluster which had triggered a vowel to become long in Old English; the book gives ``Christ" vs. ``Christmas" as an example\\
\ipa{eA e:A eo e:o} \change\ \ipa{A E: e e:}\\
\ipa{\ae j} \change\ \ipa{aj} \change\ \ipa{ej}\\
\{\ipa{\ae:j,e(:)j}\} \change\ \ipa{ej}\\
\ipa{AG} \change\ \ipa{Aw}\\
\{\ipa{eAh,eA\c{c},eAx,eAJ,eAG}\} \change\ \ipa{Aw}\\
\ipa{e:Aw i:w} \change\ \ipa{ew ju}\\
\{\ipa{A:w,A:G,o:w}\} \change\ \ipa{O:w}\\
\ipa{oG} \change\ \ipa{O:w} / _V\\
\{\ipa{A:ht,o(:)ht}\} \change\ \ipa{ow} \\
\ipa{A:} \change\ \ipa{O:}\\
\ipa{A e o} \change\ \ipa{A: E: o:} / in U[+open] ! in \#U with the following U containing /\ipa{i:}/ or ending in one of /\ipa{m n r l}/\\
\ipa{y(:)} \change\ \ipa{i(:)}\\
V\ipa{:} \change\ V[-long] / in \#U before a U with /\ipa{i:}/\\
\ipa{m} \change\ \ipa{n} \change\ \O\ / _\# when unstressed\\
\ipa{hn} \{\ipa{wl,hl}\} \ipa{hr} \change\ \ipa{w l r}\\
\ipa{f T s G} \change\ \ipa{v D z g} / \#_\\
\ipa{G} \change\ \ipa{w} / C_V\\
\{\ipa{e,A,o}\} \change\ \ipa{@} / _\#\\
\ipa{e} \change\ \O\ / if another /\ipa{e}/ one syllable previous

\subparagraph{Middle English to Yola}{\it Pogostick Man and Marcas Brian MacStiof\'{a}in \'{O} Mhaiti\'{u} \'{O} Domhnaill}, the former from Wikipedia contributors (2016), ``Forth and Bargy dialect''. {\it Wikipedia, the Free Encyclopedia}. \textless\url{https://en.wikipedia.org/w/index.php?title=Forth_and_Bargy_dialect&oldid=703468711}\textgreater; and the latter from personal research

C \change\ \O\ / C_\%\\
\ipa{t d} \change\ \ipa{T D} (conditioning unclear)\\
\O\ \change\ \ipa{E} / \ipa{u:}_\ipa{d}\\
F \change\ F[+ voice] / \#_ ! F = \ipa{\textturnw}\\
\ipa{\textturnw} \change\ \ipa{f} (at least one instance of \change\ \ipa{w}, before a high front vowel)\\
U \change\ U[+ stress] / \#U_ (often)

\subparagraph{Anglo-Frisian to Old Frisian}{\it Pogostick Man}, from Wikipedia contributors (2011), ``Old Frisian". {\it Wikipedia, the Free Encyclopedia}. \textless\url{http://en.wikipedia.org/w/index.php?title=Old_Frisian&oldid=461768402}\textgreater

\ipa{k g} \change\ \ipa{tS j} / _E\\
\ipa{g} \change\ \ipa{j} / E_\\
\ipa{aj aw ew} \change\ \{\ipa{e:,a:}\} \ipa{a: ja}\\
\ipa{h} \change\ \O\ / V_V

\subparagraph{North Frisian Lenition}{\it TzirTzi}, from Goblirsch, Kurt Gustav (2002), ``The North Frisian lenition and Danish linguistic hegemony''. In Carr, Gerald F., and Irmengard Raugh (2002), {\it New Insights in Germanic Linguistics} III:46 -- 65

\ipa{p t k} \change\ \ipa{b d g} \change\ \ipa{v r G} / V\ipa{:}_\{V,\#\}\\
Vowel length neutralized (to long vowels?)

\paragraph{West Germanic to Old Low Franconian}{\it Pogostick Man}, from Wikipedia contributors (2014), ``Old Dutch". {\it Wikipedia, the Free Encyclopedia}. \textless\url{https://en.wikipedia.org/w/index.php?title=Old_Dutch&oldid=588537679}\textgreater; and Wikipedia contributors (2014), ``Germanic umlaut". {\it Wikipedia, the Free Encyclopedia}. \textless\url{https://en.wikipedia.org/w/index.php?title=Germanic_umlaut&oldid=602634218}\textgreater

\ipa{e: o:} \change\ \ipa{ie uo}\\
\ipa{ai au} \change\ \ipa{e: o:}\\
\ipa{h} \change\ \O\ / \#_C\\
\ipa{jan} \change\ \ipa{en} / CC_\#\\
\ipa{j} \change\ \O\ / CC_\\
\ipa{h} \change\ \O\ / V_V\\
\ipa{xs} \change\ \ipa{s:}\\
Final obstruents devoice\\
\ipa{a} \change\ \ipa{6}? (\change\ \ipa{o}) / _\ipa{\textsuperimposetilde{l}}\\
Some vowel reduction seems to have occurred in unstressed syllables\\
\ipa{ai u} \change\ \ipa{ei Y} / _(C\textellipsis)\{\ipa{i(:),j}\} (short only; in the case of [\ipa{Y}] at least this was not yet phonemic)\\
\ipa{a} \change\ \ipa{E} / _(C\textellipsis)\{\ipa{i(:),j}\} (conjectured based on date from the ``Germanic umlaut" article)\\
\ipa{u:} \change\ \ipa{0w} / _V (probably, in most areas)\\
\ipa{u:} \change\ \ipa{Uw} / _V (probably, in areas that did not undergo the above change, such as Limburg)\\
\ipa{u:} \change\ \ipa{0:} (probably, in areas with \ipa{u:} \change\ \ipa{0w} / _V)\\
\ipa{ei ou} \change\ \ipa{e: o:} (except in southeastern dialects; *\ipa{ei} as a result of the umlaut of *\ipa{ai} was not affected)

\subparagraph{Old Low Franconian to Middle Dutch}{\it Pogostick Man}, from Wikipedia contributors (2014), ``Middle Dutch". {\it Wikipedia, the Free Encyclopedia}. \textless\url{https://en.wikipedia.org/w/index.php?title=Middle_Dutch&oldid=602536434}\textgreater; and Wikipedia contributors (2014), ``Hieronymous Bosch". {\it Wikipedia, the Free Encyclopedia}. \textless\url{https://en.wikipedia.org/w/index.php?title=Hieronymus_Bosch&oldid=601403790}\textgreater

\ipa{u:} \change\ \ipa{y:}\\
\ipa{iu} \change\ \ipa{ju} / \#_ (in some northern dialects)\\
\ipa{iu} \change\ \{\ipa{y:,io}\} (outcome varies depending upon dialect; the former seems more typical)\\
\ipa{iw} \change\ \ipa{yw} (dialectal)\\
\{\ipa{ie,ia,io}\} \ipa{uo} \change\ \ipa{i@ u@}\\
Umlaut phonemicizes, but only for umlauts of non-dipthongal short vowels (except in extreme eastern dialects); [\ipa{Y}] becomes a phoneme\\
\ipa{f T s} \change\ \ipa{v D z} / syllable-initially (\ipa{h} \change\ \ipa{H}?)\\
V \change\ \ipa{@} / if short and unstressed\\
\ipa{f} \change\ \{\ipa{x,\c{c}}\} / _\ipa{t} (the former seems to have occurred in northern dialects, the latter in southern ones)\\
\ipa{T D} \change\ \ipa{t d}\\
\{\ipa{u:,uw}\} \ipa{u} \change\ \ipa{Ow o} (except in the southeast)\\
\{\ipa{ol,al}\} \{\ipa{ar,er}\} \ipa{or} \change\ \ipa{Ou a:r o:r} / _C[+dental]\\
V[-long +stress] \change\ V\ipa{:} / in open syllables (\ipa{Y} \change\ \{\ipa{\oe:,\o:}\} here but this is not phonemically important; there seem to have been qualitative differences between original long vowels and long vowels resulting from this change---lengthened \ipa{i:} seems to have become \ipa{e:}, but lengthened \ipa{a:} merged with original \ipa{a:}); does not affect original long vowels or vowels in diphthongs

\subparagraph{Middle Dutch to Modern Dutch}{\it Pogostick Man}, from Wikipedia contributors (2014), ``Dutch Phonology". {\it Wikipedia, the Free Encyclopedia}. \textless\url{https://en.wikipedia.org/w/index.php?title=Dutch_phonology&oldid=602553868}\textgreater; and Wikipedia contributors (2014), ``Hard and soft G in Dutch". {\it Wikipedia, the Free Encyclopedia}. \textless\url{https://en.wikipedia.org/w/index.php?title=Hard_and_soft_G_in_Dutch&oldid=594028971}\textgreater

\tab {\it NB: This is likely highly incomplete, but the source materials did not have much to say.}

\ipa{l} \change\ \ipa{u} / \ipa{o}_\{\ipa{t,d}\}\#\\
The change of /\ipa{f}/ to a velar fricative is often reverted by analogy\\
\ipa{i: y:} \change\ \ipa{Ei \oe y}\\
\ipa{u:} \change\ \ipa{2u} (? conjectured based on the above diphthongization and on developments in Polder Dutch vowels)\\
Hard-vs.-soft-G phenomena:\\
--- \ipa{x G} \change\ \{\ipa{x,X}\} \{\ipa{G,x,X}\} / in northern dialects\\
--- \ipa{x G} \change\ \ipa{\c{c} J} / in southern dialects (the articles use velar phonemes here but describes them as ``front velar"; based on the description and on representations in other articles, the palatal phonemes are used here)

\subparagraph{Modern Dutch to Polder Dutch Vowel Shift}{\it Pogostick Man}, from Wikipedia contributors (2014), ``Dutch Phonology". {\it Wikipedia, the Free Encyclopedia}. \textless\url{https://en.wikipedia.org/w/index.php?title=Dutch_phonology&oldid=602553868}\textgreater

\ipa{Ei \oe y 2u} \change\ \ipa{ai ay au}\\
\ipa{e: \o: o:} \change\ \ipa{Ei \oe y  Ou}

\subparagraph{Belgian and Netherlandish Dutch Monophthongization}{\it Pogostick Man}, from Wikipedia contributors (2014), ``Dutch Phonology". {\it Wikipedia, the Free Encyclopedia}. \textless\url{https://en.wikipedia.org/w/index.php?title=Dutch_phonology&oldid=602553868}\textgreater

\ipa{Ei \oe y Ou} \change\ \ipa{E: \oe: O:}

\paragraph{Middle High German to Standard German}{\it Pawe\l\ Ciupak}, from Behr, Hans-Joahim, Ingrid Bennewitz, {\it et al.} (2004). {\it Die Bamberg (BA)-Braunschweiger (BS) Grammatik des Alt- und Mittelhochdeutschen im Internet}. \textless\url{https://www.tu-braunschweig.de/Medien-DB/germanistik/babs260304.pdf}\textgreater; Kundert, Ursula (2009). {\it Einf\"{u}hrung in das Mittelhochdeutsche}. \textless\url{http://www.germsem.uni-kiel.de/mediaevistik/materialien/Kundert_Mhdreader_090330.pdf}\textgreater; and Anonymous (2009). {\it Mittelhochdeutsche Kurzgrammatik}. \textless\url{https://www.uni-frankfurt.de/47053276/Kurzgrammatik-HA_09_2009.pdf}\textgreater

\ipa{s} \change\ \ipa{S} / \#_\{\ipa{l,m,n,w,p,t}\}\\
\ipa{s} \change\ \ipa{S} / \ipa{r}_\\
\ipa{t} \change\ \{\ipa{ts,k}\} / _\ipa{w}\\
\ipa{x} \change\ \ipa{k} / _\ipa{s}\\
\ipa{\{h,j\}} \change\ \O\ / V_V\\
\ipa{w j} \change\ \ipa{b g} / \ipa{\{l,r\}}_ (occasionally otherwise)\\
\ipa{w} \change\ \O\ / \ipa{\{ou,\o y,y:\}}_\\
\ipa{w} \change\ \ipa{v}\\
\ipa{a:} \change\ \ipa{o:} / _\{N,C[+ dental],P,\ipa{h}\} (sporadic?)\\
\ipa{a:} \change\ \ipa{o:} / \{N,C[+ dental],P,\ipa{h}\}_ (sporadic?)\\
\ipa{e(:) i} \change\ \ipa{\o(:) y} / _C[+ affricate]\\
\ipa{e(:) i} \change\ \ipa{\o(:) y} / _\{P,\ipa{l,S}\} (sporadic?)\\
\ipa{e(:) i} \change\ \ipa{\o(:) y} / \{P,\ipa{l,S}\}_ (sporadic?)\\
\ipa{y(:) y@ \o(:) \o y} \change\ \ipa{i(:) i@ e(:) ei} (intermittent)\\
\ipa{u: y: i:} \change\ \ipa{ou \o y ei}, except in certain unstressed endings and monosyllables, _C\{C,V,\#\} (``especially before /xt/''), and Low German borrowings\\
\ipa{u@ y@ i@} \change\ \ipa{u: y: i:}\\
\ipa{ou \o y ei} \change\ \ipa{au oy ai}\\
\ipa{u y} \change\ \ipa{o \o} / _N (with some occasional exceptions)\\
\ipa{a:w} \change\ \ipa{au}\\
V\ipa{:} \change\ V[- long] / _CC (some exceptions; the change was more common around _\ipa{xt} and _\ipa{r}C)\\
V\ipa{:} \change\ V[- long] / _\%C\ipa{@\{r,l,n\}}\# (some exceptions)\\
V \change\ V\ipa{:} / _\%, when stressed (except for /\ipa{@}/?)\\
V \change\ V\ipa{:} / _\ipa{r\{t,d,s,ts\}} (except /\ipa{@}/)\\
V \change\ V\ipa{:} / in some monosyllables ending in alveolar resonants or vowels\\
V \change\ V\ipa{:} / by analogy in some cases\\
\ipa{@} \change\ \O\ / unstressed, but not in every case\\
\O\ \change\ \ipa{@} / M_\ipa{r}\% (I don't know what Mr. Ciupak means by $\langle$M$\rangle$)

\paragraph{High German Cosonant Shift and Umlaut}{\it Pogostick Man}, from \textless\url{http://en.wikipedia.org/wiki/High_German_consonant_shift}\textgreater; and Wikipedia contributors (2014), ``Germanic umlaut". {\it Wikipedia, the Free Encyclopedia}. \textless\url{https://en.wikipedia.org/w/index.php?title=Germanic_umlaut&oldid=602634218}\textgreater

\ipa{p t k} \change\ \ipa{f: z: x:} / V_V\\
\ipa{p t k} \change\ \ipa{f z x} / _\#\\
\ipa{p t k} \change\ \ipa{pf ts kx} / \#_\\
\ipa{p t k} \change\ \ipa{pf ts kx} / \{L,N\}_\\
\ipa{p: t: k:} \change\ \ipa{pf ts kx}\\
\ipa{b d g} \change\ \ipa{p t k}\\
\ipa{G} \change\ \ipa{g}\\
\ipa{B} \change\ \ipa{b} / V_V\\
\ipa{B} \change\ \ipa{b} / _\ipa{l}\\
\ipa{s} \change\ \ipa{S} / \#_\{\ipa{p,t}\}\\
\ipa{sk} \change\ \ipa{S} / \#_\\
\{\ipa{T,D}\} \change\ \ipa{d}\\
\ipa{a u o} \change\ \ipa{e y \o} / _(C\textellipsis)\{\ipa{i(:),j}\}

\paragraph{West Germanic to Old Low German}{\it Pogostick Man}, from Wikipedia contributors (2014), ``Old Saxon phonology". {\it Wikipedia, the Free Encyclopedia}. \textless\url{https://en.wikipedia.org/w/index.php?title=Old_Saxon_phonology&oldid=598609310}\textgreater; Wikipedia contributors (2014), ``Old Saxon". {\it Wikipedia, the Free Encyclopedia}. \textless\url{https://en.wikipedia.org/w/index.php?title=Old_Saxon&oldid=598557577}\textgreater; and Wikipedia contributors (2014), ``Germanic umlaut". {\it Wikipedia, the Free Encyclopedia}. \textless\url{https://en.wikipedia.org/w/index.php?title=Germanic_umlaut&oldid=602634218}\textgreater

\ipa{ai au} \change\ \ipa{e: o:}\\
\ipa{B} \change\ \ipa{v}\\
\ipa{v: G: h:} \change\ \ipa{b: g: x:} (perhaps not strictly a sound change, but worth noting)\\
\ipa{f T s} \change\ \ipa{v D z} / syllable-initially\\
\ipa{v} \change\ \ipa{f} / _C ! _\ipa{d}\\
\ipa{b d} \change\ \ipa{p t} / _C[-voice]\\
\ipa{k} \change\ \ipa{ts} / _E (\ipa{g} \change\ \ipa{dz} here?)\\
\ipa{n} \change\ \ipa{N} / _\{\ipa{k,g}\}\\
\ipa{g} \change\ \ipa{k} / \ipa{N}_\#\\
\ipa{g} \change\ \ipa{J} / _E (singleton only)\\
\ipa{g} \change\ \ipa{G} / _V (singleton only)\\
\ipa{g} \change\ \ipa{x} / _\#\\
F[-voice] \change\ F[+voice] / X[+voiced]_X[+voiced]?\\
Umlaut applies; going by the orthography, only \ipa{a} \change\ \ipa{e} / _(C\textellipsis)\{\ipa{i(:),j}\} is often marked (and even then haphazardly), but based upon reflexes in the daughter languages it seems that the umlaut had to apply to the other back vowels too

\subsubsection{Common Germanic to Proto-Norse}{\it Pogostick Man}, from Theiling, Henrik, \url{http://www.kunstsprachen.de/s17/rules.sch}

\ipa{wi} \change\ \ipa{u} / C\ipa{i}_C\\
E\ipa{Bu} E\ipa{Bo} \change\ \ipa{ju: jo:}\\
\ipa{aB\{u,o\}} \change\ \ipa{au}\\
\ipa{B} \change\ \O\ / V_B\\
\ipa{\{\ae,e\}:(w(a))} \change\ \ipa{a:}\\
\ipa{o:} \change\ \ipa{a} / _\ipa{n}\%\\
\ipa{z x} \change\ \ipa{\;R h}\\
\ipa{i} \change\ \ipa{j} / _V ! in \#U\\
\ipa{j} \change\ \ipa{i} / C_\\
\ipa{n} \change\ \O\ / V\{\ipa{:},V\}_\ipa{h}V\\
V\ipa{n} \change\ V\ipa{:} / _\ipa{h}V\\
\ipa{(w)u(:) i(:)} \change\ \ipa{(w)o(:) e(:)} / _(C)(C)\ipa{a} ! CC = NC or one C = \{\ipa{\;R,j\}}\\
\ipa{iu} \change\ \ipa{y:}\\
\ipa{\{\ae,e\}:u: \{\ae,e\}:i:} \change\ \ipa{eu ai}\\
\ipa{w\{o,u\}:wu: j\{e,i\}:ji:} \change\ \ipa{u: i:}\\
\ipa{w} \change\ \O\ / _\ipa{w}\\
\ipa{j} \change\ \O\ / _\ipa{i}\\
\ipa{o(u) \{O,A,au,ai,\ae\} \{ja,j}E\ipa{,\ae(i),e(i),y\}} \change\ \ipa{u a i}\\
V\ipa{:} \change\ V[- long] / ! \#U, U\#

\paragraph{Proto-Norse to Old Norse}{\it Pogostick Man}, from Theiling, Henrik, \url{http://www.kunstsprachen.de/s17/rules.sch}

\ipa{T} \change\ \ipa{f} / _\ipa{l}\\
\ipa{i} \change\ \ipa{I} / _NS[- voice] ! _NS(C)\{\ipa{o,i,j}\}\\
\ipa{i} \change\ \ipa{e:} / _\ipa{\;R}\#\\
\ipa{b \{w,v\} d D g} \change\ \ipa{p f t T k} / _\#\\
\ipa{j} \change\ \O\ / \#_\\
E\ipa{:}B\ipa{:} E\ipa{:a:} \change\ \ipa{jo: ja:} / \{\ipa{v,w}\}_\\
B\ipa{:}B \ipa{a:\{o,a,\ae,e\} \{\ae,e\}:\{\ae,e\}: \{\ae,e\}:i i:}E\ipa{:} \change\ \ipa{o: a: e: e: i:}\\
\ipa{e} \change\ \ipa{i} / \#(C)(C)(C)_(C)(C)(C)\{\ipa{i,j}\}\\
\ipa{e} \change\ \ipa{ja} / ! \{\ipa{\{h,k,N\}n,w,v,l,r}\}_, _\{\ipa{u,o,i}\}\\
\ipa{a(i) \{e,w\{\ae,i\}\} \{we,ei\} (w)I} \change\ \ipa{ey \o\ y Y} / \ipa{w}_ ! \ipa{hw}_\\
\ipa{a} \change\ \O\ / C(C)_\{\ipa{\;R,s,t,T}\}\#\\
VN \change\ \~{V} / _\# ! in \#U\\
\ipa{u \{o,6\} a au ju:} \change\ \ipa{y \o\ \ae\ y y:} / _(C)(C)(C)\ipa{j}\\
\ipa{a} \change\ \O\ / _\{\ipa{\;R,s,t,T}\}\#\\
\ipa{o:} \change\ \ipa{u} / _\#\\
\ipa{\;R} \change\ \ipa{r}_\ipa{n}\\
\ipa{\;R} \change\ \ipa{r} / C_\ipa{n}\\
\ipa{\;Rn} \change\ \ipa{n:}\\
\ipa{w} \change\ \O\ / C_\ipa{o}\\
\ipa{a} \change\ \O\ / CC_U\#\\
\ipa{a} \change\ \ipa{u} / \%\ipa{u} / ! in \#U\\
\ipa{wa: na:} \change\ \ipa{o: no:} / \%\ipa{u} in \#U\\
\ipa{a(:) ae} \change\ \ipa{o(:) 6\o} / \#(C)(C)(C)_(C)(C)(C)\ipa{u}\\
\ipa{a} \change\ \O\ / _U\#\\
\ipa{\{(j)u,we\}: \{o,6\}: a: au} \change\ \ipa{y: \o: \ae: \ae y}\\
\ipa{u \{o,6\} a} \change\ \ipa{y \o\ \ae} / _(C)(C)(C)\ipa{i}\\
\{B,E\} \change\ \O\ / CC_\{\ipa{\;R,s,t,T}\}\# ! B = \ipa{6}\\
\ipa{u \{o,6\} a au ju:} \change\ \ipa{y \o\ \ae\ \ae y y:} / _(C)(C)(C)\ipa{i}\\
E \change\ \O\ / _\{\ipa{\;R,s,t,T}\}\\
\ipa{u} \change\ \ipa{o} / _\ipa{m}\#\\
\ipa{u} \change\ \O\ / _(\{\ipa{\;R,s,t,T}\})\#\\
\{B,E\} \change\ \O\ / CC_U\# ! B = \ipa{6}\\
V \change\ V\ipa{:} / _\ipa{l}\{P,\ipa{w,k},\#\}\\
\ipa{\{a,\ae,e\}:hi:} \change\ \ipa{\ae:}\\
\ipa{6:h\{u,a\} a:h\{u,a\}} \change\ \ipa{6: a:}\\
\{B,E\} \change\ \O\ / _U\# ! B = \ipa{6}\\
\ipa{u} \change\ \ipa{o} / _\ipa{m}\#\\
V\ipa{:} \change\ V[- long] / ! in \#U\\
\ipa{\{u,we,wi\} \{o,6\} a au ju:} \change\ \ipa{y \o\ \ae\ \ae y y:} / _(C)(C)(C)\ipa{i}\\
\ipa{wa we wi} \change\ \ipa{6: \o: y:} / \#P_\\
\ipa{w} \change\ \O\ / \#P_V\ipa{:}\\
\ipa{e} \change\ \ipa{j6} / _(C)(C)(C)\ipa{u} ! \{\ipa{\{h,k,N\}n,w,v,l,r}\}_\\
\ipa{e} \change\ \ipa{ja} / _(C)(C)(C)\ipa{u} ! \{\ipa{\{h,k,N\}n,w,v,l,r}\}_\\
\ipa{au \{ai,ey,ei\} \ae\{y,i\} \o y} V\ipa{:} \change\ \ipa{o e \ae\ \o} V / \#(C)(C)(C)_CC\\
V[- long] \change\ \O\ / \#U_UU\\
\ipa{u \{o,6\} a au ju:} \change\ \ipa{y \o\ \ae\ \ae y y:} / _(C)(C)(C)\ipa{j}\\
\ipa{u \{o,6\} a au ju:} \change\ \ipa{y \o\ \ae\ \ae y y:} / \#(C)(C)_\ipa{\;R}\\
\ipa{b} \change\ \O\ / \ipa{m}_\ipa{s}\\
\ipa{d} \change\ \O\ / \{\ipa{l,m}\}_\{\ipa{b,g,k,l,m,n,s}\}\\
\ipa{D} \change\ \O\ / \ipa{n}_ ! \ipa{g}_\\
\ipa{D} \change\ \O\ / \ipa{r}_\ipa{\{m,l,g,n\}}\\
\{\ipa{f,B,p}\} \change\ \O\ / \ipa{r}_\ipa{n}\\
\{\ipa{f,B,p}\} \change\ \O\ / \ipa{l}_\{\ipa{d,g,n,D,t}\}\\
\{\ipa{g,G}\} \change\ \O\ / \ipa{l}_\{\ipa{D,t}\}\\
\{\ipa{g,G}\} \change\ \O\ / \ipa{r}_\{\ipa{d,n,t}\}\\
\ipa{k} \change\ \O\ / \ipa{l}_\ipa{s}\\
\ipa{k} \change\ \O\ / \ipa{r}_\{\ipa{m,s,t}\}\\
\ipa{k} \change\ \O\ / \ipa{s}_\ipa{l,t}\\
\ipa{l} \change\ \O\ / \ipa{N}_\ipa{s}\\
\ipa{l} \change\ \O\ / \ipa{r}_\{\ipa{m,s}\}\\
\ipa{l} \change\ \O\ / \ipa{s}_\ipa{t}\\
\ipa{n} \change\ \O\ / \ipa{f}_\{\ipa{d,s,t}\}\\
\ipa{n} \change\ \O\ / \ipa{l}_\ipa{b}\\
\ipa{n} \change\ \O\ / \ipa{m}_\{\ipa{s,b}\}\\
\ipa{n} \change\ \O\ / \ipa{N}_\{\ipa{s,w}\}\\
\ipa{n} \change\ \O\ / \ipa{r}_\{\ipa{s,t,w}\}\\
\ipa{n} \change\ \O\ / \ipa{t}_\ipa{s}\\
\ipa{r} \change\ \O\ / \{\ipa{D,f}\}_\{\ipa{g,G}\}\\
\ipa{r} \change\ \O\ / \ipa{k}_\{\ipa{n,s}\}\\
\ipa{r} \change\ \O\ / \ipa{m}_\ipa{m}\\
\ipa{r} \change\ \O\ / \ipa{t}_\{\ipa{k,s}\}\\
\ipa{t} \change\ \O\ / \{\ipa{g,G}\}_\ipa{s}\\
\ipa{t} \change\ \O\ / \ipa{p}_\{\ipa{g,G,n}\}\\
\ipa{t} \change\ \O\ / \ipa{r}_\ipa{k}\\
\ipa{t} \change\ \O\ / \ipa{s}_\{\ipa{k,l,n,s}\}\\
\{\ipa{s,z}\} \change\ \O\ / \{\ipa{r,\;R}\}_N\\
\{\ipa{v,w}\} \change\ \O\ / \#_V[+ round]\\
\ipa{r:\{r,\;R\} s\;R} \change\ \ipa{r: s:}\\
\ipa{l(:)\;R n(:)\;R} \change\ \ipa{l: n:} / "V\ipa{:}_ (or all V_ ?)\\
\{\ipa{l(:),n\}\{r,\;R\}} \change\ \O\ / V\{\ipa{:},V\}_\\
\ipa{n: l: r: s:} \change\ \ipa{n l r s} / C_\\
\{\ipa{t,T,d,D}\} \change\ \O\ / \ipa{n}_\ipa{l}\\
\{\ipa{t,T,d,D}\} \change\ \O\ / \ipa{l}_\ipa{n}\\
\ipa{l: n:} \change\ \ipa{l n} / C_\#\\
\ipa{ai wi (w)}V \change\ \ipa{e: we: (w)}V\ipa{:} / _\ipa{h}\#\\
\ipa{6} \change\ \ipa{o:} / \ipa{n}_\ipa{h}\\
\ipa{\{\ae,e\}i ai au w\{I,i\} wy w}V \ipa{iu} V\ipa{:} \change\ \ipa{e: a: o: we: wo: w}V\ipa{: e: o:} V\\
\ipa{ey} \change\ \{\ipa{jo,\ae\}:} / _\ipa{\;R}\\
\ipa{i} \change\ \ipa{e:} / _\#\\
\ipa{Dl(:) Dn(:)} \change\ \ipa{l: n:}\\
\ipa{Y I} \change\ \ipa{\o\ e}\\
V\{\ipa{T,D}\} \change\ V\ipa{:} / \#(C)(C)(C)_\{\ipa{l,r}\}\\
\ipa{ai} \change\ \ipa{a:} / _\ipa{r}\\
\ipa{ai} \change\ \ipa{a:} / _\ipa{h}\{C,V\}\\
\ipa{\ae} \change\ \ipa{e}\\
\ipa{lT nT} \change\ \ipa{l: n:}\\
\~{V} \change\ V\ipa{:} / in \#U (maybe only \ipa{\~{\i}}?)\\
\~{V} \change\ V[- nas]\\
\ipa{B D G} \change\ \ipa{f T k} / _\{\ipa{p,t,k,s}\}\\
\ipa{B D G} \change\ \ipa{b d g} / \#_\\
\ipa{B D G} \change\ \ipa{b d g} / \{\ipa{m,n,N,l}\}_\\
\ipa{B G} \change\ \ipa{b g} / \ipa{r}_\\
\ipa{G} \change\ \ipa{g} / _\{\ipa{r,\;R,T,D}\}\\
\ipa{B G} \change\ \ipa{f h} / _\%\\
\ipa{G} \change\ \ipa{g} / _\{E,\ipa{j}\}\\
\ipa{B D G} \change\ \ipa{v T h}\\
(V\ipa{:})\ipa{Tt} \change\ (V)\ipa{t:}\\
E\ipa{:\{u,o\}:} E\ipa{:a:} \change\ \ipa{jo: ja:} / \{\ipa{v,w}\}_\\
\ipa{T} \change\ \ipa{t} / \{\ipa{p,t,k}\}_\\
\ipa{T: D:} \change\ \ipa{t: d:}\\
\ipa{T f} \change\ \ipa{D v} / \{V,C[+ voiced]\}_\{V,C[+ voiced]\}\\
\ipa{h} \change\ \O\ / C_\ipa{t}\\
\ipa{ht} \change\ \ipa{t:}\\
\ipa{hw} \change\ \O\ / ! \#_\\
F[- voice] \change\ \O\ / \{\ipa{s,f,x,h,t}\}_\{\ipa{p,t,k}\}\\
\ipa{n:} \change\ \ipa{D} / _\{\ipa{r,\;R}\}\\
\ipa{wo: w\o: j\ae:} V\ipa{:} \change\ \ipa{wo w\o je} V[- long] / _\%\\
NS[- voice] \change\ S[- voice]\ipa{:}\\
\{\ipa{t(:),g:\}k} \change\ \ipa{k:}\\
\ipa{ts} \change\ \ipa{s:} / V_V\\
\ipa{u}N \ipa{y}N \ipa{i}N VN \change\ \ipa{o \o: e:} V\ipa{:} / _\{\ipa{s,f}\}\\
S\ipa{:} \change\ S[- long] / U[- stress]_\\
\ipa{lT nT} \change\ \ipa{l: n:}\\
\ipa{p} \change\ \ipa{f} / _\{\ipa{t,k}\}\\
\ipa{t} \change\ \ipa{T} / _\{\ipa{p,k}\}\\
\ipa{k} \change\ \ipa{x} / _\{\ipa{p,t}\}\\
\ipa{m} \change\ \ipa{f} / _\{\ipa{n,N}\}\\
\ipa{n} \change\ \ipa{T} / _\{\ipa{m,N}\}\\
\ipa{N} \change\ \ipa{x} / _\{\ipa{m,n}\}\\
\ipa{s} \change\ \ipa{ts} / \{\ipa{l,n}\}\ipa{:}_\\
\O\ \change\ \ipa{t} / \ipa{s}_\ipa{r}\\
\ipa{G} \change\ \ipa{g:} / _\ipa{j}\\
\ipa{\;R} \change\ \ipa{r}\\
\ipa{w} \change\ \ipa{v} / \#_\\
S\ipa{:} \change\ S[- long] / _\{\ipa{r}\}\ipa{:}\\
\ipa{g} \change\ \O\ / \#_\ipa{n}\\
\ipa{r} \change\ \O\ / \{\ipa{p,t,k}\}_V\ipa{r}\\
\ipa{r: l:} \change\ \ipa{r l} / \{\ipa{p,t,k,f,s}\}_\\
\ipa{o \{\ae,e\}} \change\ \ipa{u i} / ! in \#U\\
\ipa{o} \change\ \ipa{u} / V_\\
\ipa{a} \change\ \ipa{e} / _\ipa{i}\\
\ipa{IU} \change\ \ipa{y:}\\
\ipa{\o\ \o:} \change\ \ipa{e \ae:}\\
\ipa{e(:)}\{B,\ipa{i(:)}\} \change\ \ipa{e(:)u} / \{\ipa{v,w}\}_\\
\ipa{e:}\{B\ipa{:,i(:)}\} \change\ \ipa{jo:} / _C[+ alveolar]\#\\
\ipa{e:}\{B\ipa{:,i(:)}\} \change\ \ipa{ju:}\\
\O\ \change\ \ipa{j} / \{\ipa{y,e}\}\ipa{(:)}_\ipa{a}\\
N\ipa{(:) k k(:)} N\ipa{(:)g g(:) G} \change\ \ipa{\textltailn c(:) c(:) \textltailn\textbardotlessj(:) \textbardotlessj(:) J} / _\{\ipa{i,j}\}\\
\ipa{j} \change\ \O\ / \'{K}_\\
\ipa{a} \change\ \ipa{e} / _\{\ipa{i,j,}\'{K}\}\\
N\ipa{(:) k k(:)} N\ipa{(:)g g(:) G} \change\ \ipa{\textltailn c(:) c(:) \textltailn\textbardotlessj(:) \textbardotlessj(:) \textbardotlessj} / _\{\ipa{i,j}\}\\
\ipa{f T} \change\ \ipa{v D} / \#_\\
\ipa{f} \change\ \ipa{v} / _\{\ipa{p,t,k}\}\\
\ipa{j} \change\ \O\ / \{\ipa{c,\textbardotlessj}\}_\\
\ipa{j} \change\ \O\ / _\ipa{e:}\\
\ipa{au} \change\ \ipa{6} / \ipa{j}_\\
\ipa{e:} \change\ \ipa{e} / \'{K}_\\
\ipa{e: j} \change\ \ipa{e} \O\ / C\ipa{w}_\\
\ipa{k} \change\ \ipa{h} / \#_\{\ipa{v,n}\}

\subparagraph{Old Norse to Early Icelandic}{\it Pogostick Man}, from Theiling, Henrik, \url{http://www.kunstsprachen.de/s17/rules.sch}

\ipa{k} \change\ \ipa{\c{c}} / _\#\\
\ipa{n} \change\ \ipa{N} / _\{\ipa{k,g}\}\\
\ipa{t} \change\ \ipa{D} / V_\# ``in some verbal endings''\\
\O\ \change\ \ipa{u} / C_\ipa{r}\#\\
\ipa{wa:} \change\ \ipa{wo}\\
\ipa{6 6:} \change\ \ipa{\o\ a:}\\
\ipa{\o:} \change\ \ipa{\ae}\\
\ipa{u o a \o\ y e i} \change\ \ipa{u: o: a: \o i y: ei i:} / _\{\ipa{\textltailn c,\textltailn\textbardotlessj,Nk,Ng}\}\\
\ipa{g \textbardotlessj} \change\ \ipa{G j} / V_V\\
\ipa{g} \change\ \ipa{G} / V_\#\\
\ipa{hj} \change\ \ipa{\c{c}}\\
\ipa{u o a \o: e: y i} \change\ \ipa{YI oi ai \o i ei y: i:} / _\ipa{j}\\
\ipa{a:} \change\ \ipa{ai} / _\ipa{j}\\
\ipa{e} \change\ \ipa{ei} / _\{\ipa{G,j}\}\\
O \change\ \O\ / \{F[- same POA],\ipa{r,l}\}_O\\
S \change\ \O\ / N[+ same POA]_S\\
\{\ipa{l(:),rl}\} \change\ \ipa{\textbeltl} / \ipa{v}_\#\\
N \change\ N[- voice] / O_\#\\
\{\ipa{l(:),rl}\} C \change\ \ipa{\textbeltl} C[- voice] / _\{S[- voice],\ipa{s}\}\\
\ipa{r} \change\ \ipa{\r*{r}} / S[- voice] / _\#\\
\{\ipa{l(:),rl}\} C \change\ \ipa{\textbeltl} C[- voice] / S_ ! S\ipa{:}_\\
\ipa{\{l(:),rl\} r:} \change\ \ipa{d\textbeltl\ \r*{r}} / _\#\\
S[- voice]\ipa{:} S[+ voice]\ipa{:} \change\ \ipa{\super h}S S[- voice]\ipa{:}\\
S[- voice] \change\ \ipa{\super h}S / _\{\ipa{l,\textbeltl,m,n}\}\\
S[- voice] \change\ S\ipa{\super h} / \#\ipa{v}_V\\
S[+ voice] \change\ S[- voice]\\
\ipa{p t k} \change\ \ipa{f T x} / _\{S,F[- voice]\}\\
\ipa{b d \{g,G\}} \change\ \ipa{p t k} / _S\\
F \change\ S[- voice] / _\{\ipa{l},N\}\\
\ipa{u o: a: YI \{y,i\} \{y,i\}: \ae: e: ey} \change\ \ipa{Y ou au ai y I i je ei}\\
\ipa{w} \change\ \ipa{v}\\
V \change\ V\ipa{:} / _(C)\#, in monosyllables\\
V \change\ V\ipa{:} / ! _CCV, in polysyllables\\
\{\ipa{\r*{n}n,n\r*{n}} \change\ \ipa{t\r*{n}} / V_\\
\ipa{n:} \change\ \ipa{t\r*{n}} / V_\#\\
\ipa{n:} \change\ \ipa{tn} / V_\\
\ipa{n\r*{n},\r*{n}n} \change\ \ipa{\r*{n}}\\
\ipa{n:} \change\ \ipa{n}\\
\{\ipa{\r*{r}n,\r*{r}\r*{n},r\r*{n}} \change\ \ipa{t\r*{n}}\\
\ipa{rn} \change\ \ipa{t\r*{n}} / _\#\\
\ipa{r} \change\ \ipa{t} / _\ipa{n}\\
\{\ipa{r,\r*{r},l\}\textbeltl} \change\ \ipa{t\textbeltl}\\
\{\ipa{\textbeltl l,\textbeltl:} \change\ \ipa{t\textbeltl}\\
\ipa{\r*{r}l} \change\ \ipa{t\textbeltl}\\
\{\ipa{l:,rl}\} \change\ \ipa{t\textbeltl} / _\#\\
\{\ipa{l:,rl}\} \change\ \ipa{tl}\\
\ipa{h} \change\ \ipa{k} / \#_\{\ipa{v,w}\}\\
\ipa{hl} \change\ \ipa{\textbeltl} / \#_\\
\ipa{hr hn} \change\ \ipa{\r*{r} \r*{n}} / \#_\\
\ipa{v} \change\ \O\ / \{\ipa{u,o,a}\}\ipa{:}_\\
\ipa{\r*{n}} \change\ \ipa{\r*{m}} / \ipa{p}_\#\\
C\ipa{:} \change\ C[- long]

\subparagraph{Old Norse to Orkney Norn}{\it Pogostick Man and Marcas Brian MacStiof\'{a}in \'{O} Mhaiti\'{u} \'{O} Domhnaill}, from \url{http://nornlanguage.x10.mx/index.php?ork_phon}, citing Marwick, Hugh, ``Orkney Norn''

\tab {\it NB: For the most part, these changes are not in chronological order and are often tendencies more than strict sound-change laws.}

\ipa{ny} \change\ \ipa{in}\\
\ipa{f} \change\ \ipa{m} / _\ipa{n}\\
\ipa{n} \change\ \O\ / \ipa{m}_\\
\ipa{p(:) t(:) k(:)} \change\ \ipa{b(:) d(:) g(:)} / \{V,R\}_\{V,R\}\\
\{\ipa{t,d}\}\ipa{j} \change\ \ipa{tS}\\
\ipa{d} \change\ \O\ / \ipa{n}_\# (sometimes)\\
\ipa{b d g} \change\ \ipa{p t k} / \#_\\
\ipa{k} \change\ \ipa{s} / _\ipa{n}\\
\ipa{g} \change\ \ipa{k} / _\# (sporadic)\\
\ipa{gn gl} \change\ \ipa{nj lj}\\
\ipa{k g} \change\ \ipa{c \textbardotlessj} / _\{E,\ipa{j}\}\\
\ipa{f} \change\ \O\ / _\{\ipa{l,b,v,n},\#\}\\
\ipa{f} \change\ \O\ / V\ipa{:}_V\\
\ipa{fd} \change\ \ipa{d:}\\
\ipa{h} \change\ \O\ / _\{\ipa{l,n,r}\}\\
\ipa{hw hj} \change\ \ipa{\{w,\textturnw\}} \ipa{S}\\
\ipa{h} \change\ \ipa{x} / _\ipa{i}\\
\ipa{T} \change\ \ipa{h} / \#_B\\
\ipa{T} \change\ \ipa{D} / V_V\\
\ipa{T} \change\ \ipa{t}\\
\ipa{D} \change\ \ipa{T} / _\#\\
\ipa{D} \change\ \O\ /\{\ipa{a},E\}_\\
\ipa{D} \change\ \ipa{d} / ! V_V\\
\ipa{sk} \change\ \{\ipa{sk,S,ks}\}\\
\ipa{s} \change\ \O\ / \{\ipa{t,k,r}\}_\ipa{l}\\
\ipa{s} \change\ \ipa{S} / _V\ipa{r} ?\\
\ipa{G} \change\ \O\\
\ipa{l} \change\ \O\ / V_V\\
\ipa{l} \change\ \O\ / _\{\ipa{m,s,k}\}\\
\ipa{l} \change\ \O\ / _\# ?\\
\ipa{lm} \change\ \ipa{ml} (sporadic)\\
\ipa{l:} \change\ \ipa{L}\\
\ipa{v} \change\ \ipa{w}\\
\ipa{u a e e:} \change\ \ipa{2 A I E} / _C\ipa{:}\\
\ipa{u} \change\ \ipa{\o} / _\ipa{l(:)}\\
\ipa{u} \change\ \ipa{2} / _C\{\ipa{:},CC\}\\
\ipa{u:} \change\ \ipa{\o:} / _CC\\
\ipa{o a} \change\ \ipa{\o\ E} / _(C)(C)\ipa{i}\\
\ipa{o} \change\ \ipa{\o} / _C\ipa{r}\\
\ipa{o:} \change\ \ipa{\o:} / _(C)(C)\#\\
\ipa{o:} \change\ \ipa{u(:)}\\
\ipa{jo:} \change\ \{\ipa{u,o,\o}\} (looks like being in the ultima or the penult may have had something to do with it, but it isn't clear to me)\\
\ipa{6} \change\ \ipa{O} / _C(\ipa{:})C\\
\ipa{6} \change\ \ipa{2} / _\ipa{r(:)}\\
\ipa{6} \change\ \ipa{\{E,e\}} / _C\ipa{:}V (V can be a syllabic consonant)\\
\ipa{6} \change\ \ipa{I} / _\ipa{r}C\\
\ipa{au} \change\ \ipa{(O)u}\\
\ipa{a a:} \change\ \ipa{E \o:} / _\ipa{r}\\
\ipa{a} \change\ \ipa{O} / _\{\ipa{l,nd}\}C\\
\ipa{ja} \change\ \ipa{i}\\
\ipa{\ae} \change\ \ipa{e:} / _\ipa{\{D,r\}}\\
\ipa{\ae} \change\ \ipa{E:}\\
\ipa{\oe} \change\ \ipa{E:} / _CC\\
\ipa{\oe} \change\ \ipa{\o} / _\ipa{l}\\
\ipa{\oe} \change\ \ipa{i:} / _N\\
\ipa{\oe} \change\ \ipa{e:}\\
\ipa{y:} \change\ \ipa{\o} / _\ipa{j}\\
\ipa{y:} \change\ \ipa{i}\\
\ipa{y} \change\ \{\ipa{I,i}\}\\
\ipa{e i:} \change\ \ipa{E i(:)}\\
\ipa{ey} \change\ \ipa{e:} / _F\\
\ipa{ey} \change\ \ipa{E} / _\ipa{r}\\
\ipa{ey} \change\ \ipa{ai}\\
\ipa{ei} \change\ \ipa{e:} / _C(\ipa{:},V)\# (V can be a syllabic consonant)\\
\ipa{e} \change\ \ipa{a} / _\ipa{i}\\
\ipa{i} \change\ \ipa{I} / _CC\\
V[- long] \change\ \O\ / _\#

\subparagraph{Old Norse to Shetland Norn}{\it Pogostick Man and Marcas Brian MacStiof\'{a}in \'{O} Mhaiti\'{u} \'{O} Domhnaill}, from \url{http://nornlanguage.x10.mx/index.php?shet_phon}, citing Jakobsen, Jakob, {\it An Etymological Dictionary of the Norn Language}

\tab {\it NB: For the most part, these changes are not in chronological order and are often tendencies more than strict sound-change laws. Further, I'm assuming that $\langle$\"a$\rangle$ is /\ipa{\ae}/ and that $\langle$\c{o}$\rangle$ and $\langle$\.{o}$\rangle$ are /\ipa{6}/, and since I'm not sure what the conditions are for (apparent) reflexes with long vowels, I'm ignoring the vowel length in the Shetland Norn reflexes.}

\ipa{p t k} \change\ \ipa{b d g} / V_V (the second V at least can be a syllabic consonant)\\
\ipa{b d} \change\ \ipa{p t} / N_\\
\ipa{b} \change\ \ipa{v} / \#_ (sporadic?)\\
\ipa{lm} \change\ \ipa{ml}\\
\O\ \change\ \ipa{b} / \ipa{m}_\ipa{l}\\
\ipa{nd ld} (\change\ \ipa{n: l:} ?) \change\ \ipa{\textltailn\ L}\\
\ipa{dj} \change\ \ipa{dZ}\\
\ipa{g(:)} \change\ \ipa{dZ} / _\ipa{i}V\\
\ipa{k g} \change\ \ipa{c \textbardotlessj} / _E\\
\ipa{tr} \change\ \ipa{rd}\\
\ipa{p: t: k:} \change\ \ipa{b d} \{\ipa{g,G}\} / V_V\\
\ipa{t:} \change\ \{\ipa{t\super j,d\super j}\}\\
\ipa{tj} \change\ \ipa{tS} / _\#\\
\ipa{tj} \change\ \ipa{S}\\
\ipa{gl} \change\ \ipa{lg}\\
\ipa{f} \change\ \ipa{v} / \#_ (sporadic?)\\
\ipa{f} \change\ \ipa{m} / _\ipa{n}\\
\ipa{f} \change\ \ipa{p} / _\ipa{t}\\
\ipa{v} \change\ \ipa{w} / \#_\\
\ipa{n} \change\ \O\ / _\ipa{m}\#\\
\ipa{vl} \change\ \ipa{lv}\\
\ipa{T} \change\ \{\ipa{t,d}\}\\
\ipa{D} \change\ \O\ / _\#\\
\ipa{D} \change\ \ipa{d}\\
\ipa{s} \change\ \O\ / \ipa{k}_\ipa{l}\\
\O\ \change\ \ipa{h} / \#_V\\
\ipa{h} \change\ \O\ / _\{V,\ipa{w,j}\} (sporadic)\\
\ipa{h} \change\ \O\ / _\ipa{l}\\
\ipa{h} \change\ \{\O,\ipa{h,k}\} / _\ipa{r}\\
\ipa{h} \change\ \{\O,\ipa{h,k,s}\} / _\ipa{n}\\
\ipa{hv} \change\ \{\ipa{h,k,s}\}\ipa{w}\\
\ipa{hj} \change\ \ipa{S}\\
\ipa{n l} \change\ \ipa{\textltailn\ L} / _C\\
\ipa{n:} \change\ \ipa{\textltailn(d)}\\
\ipa{rn} \{\ipa{l:,rl}\} \change\ \ipa{\textltailn\ L}\\
\ipa{ms} \change\ \ipa{N(k)s}\\
\ipa{r} alternates with \ipa{l}\\
\ipa{\{u,o\}(:) a a: \{6,\oe,y\} e i(:)} \change\ \ipa{\{o,O\}(i) \ae(i) \{O,6\}(i) \{o,O\}(i) \{\ae,e\} } / _\{\'{K},C\ipa{\super j}\}\\
\ipa{u} \change\ \{\ipa{o,6}\} / _CC\\
\ipa{u:} \change\ \{\ipa{u,o,6,\o}\} (conditioning unclear; it seems the presence of a velar consonant may have helped to retain the quality of /\ipa{u}/)\\
\ipa{o:} \change\ \ipa{u}\\
\ipa{6} \change\ \ipa{E} / _C\ipa{:}\\
\ipa{6} \change\ \ipa{\o} / _O[+ dental/alveolar]\\
\ipa{\{6,ey\} j\{u,o,a\}: y:} \change\ \ipa{o \o\ u} / \'K_\\
\ipa{j\{u,o,a\}:} \change\ \ipa{\o}\\
\ipa{a} \change\ \O\ / C[+ dental/alveolar]_\ipa{u}\\
\ipa{a} \change\ \{\ipa{o,O}\} / _\{K,\ipa{r}\} (! K = \ipa{w} ?)\\
\ipa{a:} \change\ \ipa{wo} (dialectal)\\
\ipa{a:} \change\ \ipa{O(u)} / _\{\ipa{l,r}\}\\
\ipa{au} \change\ \{\ipa{o,O,6}\} / \ipa{j}_\\
\ipa{au} \change\ \ipa{j\{o,O\}}\\
\ipa{y} \change\ \ipa{@} / _\ipa{r(:)}\\
\ipa{y:} \change\ \ipa{\o} / _O[+ dental/alveolar]\\
\ipa{\ae} \change\ \ipa{e} / ! _\{\'K,C\ipa{\super j}\}\\
\ipa{e} \change\ \{\ipa{o,6}\} / _\ipa{w}\\
\ipa{e:} \change\ \{\ipa{6,@}\} / \ipa{w}_\\
\ipa{e e:} \change\ \ipa{\{\ae,E,e\} \{(j)E,je\}}\\
\ipa{ey} \change\ \ipa{\o}\\
Final short vowels drop

\subsubsection{Common Germanic to Vandalic}{\it Jaceb Kilpatrick \& Pogostick Man}, from Wikipedia contributors (2015), ``Vandalic language''. {\it Wikipedia, the Free Encyclopedia}. \textless\url{https://en.wikipedia.org/w/index.php?title=Vandalic_language&oldid=686359598}\textgreater

\tab {\it NB: This is likely incomplete.}

\ipa{h} \change\ \O\ / \#_\\
\ipa{e:} \change\ \ipa{i} / unstressed\\
\ipa{e} \change\ \ipa{i} / ! \{\ipa{w,r,h}\}_\\
\ipa{o:} \change\ \ipa{u}\\
\ipa{w:} \change\ \ipa{g}\\
\ipa{w} \change\ \{\ipa{gw,v}\} / \#_\\
\ipa{tj} \change\ \ipa{tsj}\\
\ipa{T D} \change\ \ipa{t d} (not a complete change; apparently due to Latin)\\
\ipa{z} \change\ \O\ (seems to have been complete by the Sixth Century)

\subsection{Greek}\tab It is entirely possible that I utterly failed to interpret the source documents correctly. If so, please do not hesitate to correct me.

\subsubsection{Proto-Indo-European to Aeolian Greek}{\it Pogostick Man}, from Tucker, R. Whitney (1969), ``Chronology of Greek Sound Changes". {\it The American Journal of Philology} 90(1):36 -- 47; and Wikipedia contributors (2013), ``Ancient Greek dialects". {\it Wikipedia, the Free Encyclopedia}. \textless\url{https://en.wikipedia.org/w/index.php?title=Ancient_Greek_dialects&oldid=575325271}\textgreater

\{H$_x$,\ipa{\s{m},\s{n}}\} \change\ \ipa{a}\\
\ipa{b\super H d\super H g\super H} \change\ \ipa{p\super h t\super h k\super h}\\
\ipa{s} \change\ \ipa{h} / \#_\\
\ipa{s} \change\ \ipa{h} / V_V\\
\ipa{t} \change\ \ipa{ts} / _\ipa{i}\\
\ipa{j} \change\ \ipa{h}\\
\ipa{k\super w k\super h\super w g\super w} \change\ \ipa{t t\super h d} / _E\\
\ipa{k\super w k\super h\super w g\super w} \change\ \ipa{p p\super h b} / _\{\ipa{a,o},C\}\\
\ipa{k\super w k\super h\super w g\super w} \change\ \ipa{k k\super h g} / _\ipa{u}\\
\ipa{k\super w k\super h\super w g\super w} \change\ \ipa{k k\super h g} / \ipa{u}_\\
\ipa{ts} \change\ \ipa{s}\\
\ipa{h} \change\ \O\\
V\ipa{ns} \change\ V\ipa{:s}\\
\ipa{n} \change\ \O\ / _\ipa{s}\\

\subsubsection{Proto-Indo-European to Attic Greek}{\it Pogostick Man}, from Tucker, R. Whitney (1969), ``Chronology of Greek Sound Changes". {\it The American Journal of Philology} 90(1):36 -- 47; and Wikipedia contributors (2013), ``Ancient Greek dialects". {\it Wikipedia, the Free Encyclopedia}. \textless\url{https://en.wikipedia.org/w/index.php?title=Ancient_Greek_dialects&oldid=575325271}\textgreater

\{H$_x$,\ipa{\s{m},\s{n}}\} \change\ \ipa{a}\\
\ipa{b\super H d\super H g\super H} \change\ \ipa{p\super h t\super h k\super h}\\
\ipa{s} \change\ \ipa{h} / \#_\\
\ipa{s} \change\ \ipa{h} / V_V\\
\ipa{t} \change\ \ipa{ts} / _\ipa{i}\\
\ipa{j} \change\ \ipa{h}\\
\ipa{k\super w k\super h\super w g\super w} \change\ \ipa{t t\super h d} / _E\\
\ipa{k\super w k\super h\super w g\super w} \change\ \ipa{p p\super h b} / _\{\ipa{a,o},C\}\\
\ipa{k\super w k\super h\super w g\super w} \change\ \ipa{k k\super h g} / _\ipa{u}\\
\ipa{k\super w k\super h\super w g\super w} \change\ \ipa{k k\super h g} / \ipa{u}_\\
\ipa{ts} \change\ \ipa{s}\\
\ipa{h} \change\ \O\\
\ipa{A:} \change\ \ipa{\ae:}\\
V\ipa{ns} \change\ V\ipa{:s}\\
\ipa{n} \change\ \O\ / _\ipa{s}\\
\ipa{tS} \change\ \ipa{t} / \#_\\
\ipa{tS} \change\ \ipa{t:} / medial\\
\ipa{w} \change\ \O\\
Vowel contraction (on which the author does not elaborate much)\\
Some ``metathesis of quality as well as of quantity" with regards to vowels\\
\ipa{u(:)(j)} \change\ \ipa{y(:)(j)}\\
\ipa{\ae:} \change\ \ipa{E:} (includes diphthongs)\\
\ipa{ej ow} \change\ \ipa{e: o:}\\
\ipa{e:} \change\ \ipa{i:} / _C\\
\ipa{j} \change\ \O\ / V\ipa{:}_\\
\ipa{e:} \change\ \ipa{i:} / _V\\
\ipa{E:} \change\ \ipa{e:}\\
\ipa{Aj} \change\ \ipa{E:}\\
\ipa{h} \change\ \O\\
\ipa{oj} \change\ \ipa{\o j} \change\ \ipa{yj} (\change\ \ipa{y:} sometimes)\\
\ipa{e o} \change\ \ipa{E O}\\
\ipa{p\super h t\super h k\super h} \change\ \ipa{f T x}\\
Pitch-accent lost\\
\ipa{b d g} \change\ \ipa{v D G} / V_V\\
\ipa{dz} \change\ \ipa{z}\\
V\ipa{:} \change\ V[-long]\\
C\ipa{:} \change\ C[-long]\\
\ipa{Au Eu eu} \change\ \ipa{Av Ev ev}\\
\ipa{O} \change\ \O\ / in the suffixes {\it -ios} and {\it -ion}\\
\ipa{n} \change\ \O\ / _\#\\
\ipa{e} \change\ \ipa{i}\\
\ipa{p t k} \change\ \ipa{b d g} / N_\\
\ipa{y} \change\ \ipa{i}\\
\ipa{g x} \change\ \ipa{j \c{c}} / _\{\ipa{E,i}\}\\
\ipa{p k} \change\ \ipa{f x} / _\ipa{t}\\
"\{\ipa{i,e}\}V \change\ \ipa{j}"V

\subsubsection{Proto-Indo-European to Boeotian Greek}{\it Pogostick Man}, from Tucker, R. Whitney (1969), ``Chronology of Greek Sound Changes". {\it The American Journal of Philology} 90(1):36 -- 47; and Wikipedia contributors (2013), ``Ancient Greek dialects". {\it Wikipedia, the Free Encyclopedia}. \textless\url{https://en.wikipedia.org/w/index.php?title=Ancient_Greek_dialects&oldid=575325271}\textgreater

\{H$_x$,\ipa{\s{m},\s{n}}\} \change\ \ipa{a}\\
\ipa{b\super H d\super H g\super H} \change\ \ipa{p\super h t\super h k\super h}\\
\ipa{s} \change\ \ipa{h} / \#_\\
\ipa{s} \change\ \ipa{h} / V_V\\
\ipa{t} \change\ \ipa{ts} / _\ipa{i}\\
\ipa{j} \change\ \ipa{h}\\
\ipa{k\super w k\super h\super w g\super w} \change\ \ipa{t t\super h d} / _E\\
\ipa{k\super w k\super h\super w g\super w} \change\ \ipa{p p\super h b} / _\{\ipa{a,o},C\}\\
\ipa{k\super w k\super h\super w g\super w} \change\ \ipa{k k\super h g} / _\ipa{u}\\
\ipa{k\super w k\super h\super w g\super w} \change\ \ipa{k k\super h g} / \ipa{u}_\\
\ipa{ts} \change\ \ipa{s}\\
\ipa{h} \change\ \O\\
V\ipa{ns} \change\ V\ipa{:s}\\
\ipa{n} \change\ \O\ / _\ipa{s}\\
\{\ipa{e:,ej}\} \ipa{E: A(:)j} \{\ipa{oj,O:j}\} \change\ \ipa{i: e: E:} \{\ipa{y,\o}\}\\
\ipa{o:} \change\ \ipa{u:}\\

\subsubsection{Proto-Indo-European to Coan Greek}{\it Pogostick Man}, from Tucker, R. Whitney (1969), ``Chronology of Greek Sound Changes". {\it The American Journal of Philology} 90(1):36 -- 47; and Wikipedia contributors (2013), ``Ancient Greek dialects". {\it Wikipedia, the Free Encyclopedia}. \textless\url{https://en.wikipedia.org/w/index.php?title=Ancient_Greek_dialects&oldid=575325271}\textgreater

\tab {\it NB: This assumes that the adjective ``Coan" refers to the ``Ceos" Tucker mentions in the source.}

\{H$_x$,\ipa{\s{m},\s{n}}\} \change\ \ipa{a}\\
\ipa{b\super H d\super H g\super H} \change\ \ipa{p\super h t\super h k\super h}\\
\ipa{s} \change\ \ipa{h} / \#_\\
\ipa{s} \change\ \ipa{h} / V_V\\
\ipa{t} \change\ \ipa{ts} / _\ipa{i}\\
\ipa{j} \change\ \ipa{h}\\
\ipa{k\super w k\super h\super w g\super w} \change\ \ipa{t t\super h d} / _E\\
\ipa{k\super w k\super h\super w g\super w} \change\ \ipa{p p\super h b} / _\{\ipa{a,o},C\}\\
\ipa{k\super w k\super h\super w g\super w} \change\ \ipa{k k\super h g} / _\ipa{u}\\
\ipa{k\super w k\super h\super w g\super w} \change\ \ipa{k k\super h g} / \ipa{u}_\\
\ipa{ts} \change\ \ipa{s}\\
\ipa{h} \change\ \O\\
V\ipa{ns} \change\ V\ipa{:s}\\
\ipa{n} \change\ \O\ / _\ipa{s}\\
\ipa{\ae:} \change\ \ipa{E:}\\

\subsubsection{Proto-Indo-European to Cretan Greek}{\it Pogostick Man}, from Tucker, R. Whitney (1969), ``Chronology of Greek Sound Changes". {\it The American Journal of Philology} 90(1):36 -- 47; and Wikipedia contributors (2013), ``Ancient Greek dialects". {\it Wikipedia, the Free Encyclopedia}. \textless\url{https://en.wikipedia.org/w/index.php?title=Ancient_Greek_dialects&oldid=575325271}\textgreater

\{H$_x$,\ipa{\s{m},\s{n}}\} \change\ \ipa{a}\\
\ipa{b\super H d\super H g\super H} \change\ \ipa{p\super h t\super h k\super h}\\
\ipa{s} \change\ \ipa{h} / \#_\\
\ipa{s} \change\ \ipa{h} / V_V\\
\ipa{t} \change\ \ipa{ts} / _\ipa{i}\\
\ipa{j} \change\ \ipa{h}\\
\ipa{k\super w k\super h\super w g\super w} \change\ \ipa{t t\super h d} / _E\\
\ipa{k\super w k\super h\super w g\super w} \change\ \ipa{p p\super h b} / _\{\ipa{a,o},C\}\\
\ipa{k\super w k\super h\super w g\super w} \change\ \ipa{k k\super h g} / _\ipa{u}\\
\ipa{k\super w k\super h\super w g\super w} \change\ \ipa{k k\super h g} / \ipa{u}_\\
\ipa{ts} \change\ \ipa{s}\\
\ipa{h} \change\ \O\\
V\ipa{ns} \change\ V\ipa{:s}\\
\ipa{n} \change\ \O\ / _\ipa{s}\\

\subsubsection{Proto-Indo-European to Doric Greek}{\it Pogostick Man}, from Tucker, R. Whitney (1969), ``Chronology of Greek Sound Changes". {\it The American Journal of Philology} 90(1):36 -- 47; and Wikipedia contributors (2013), ``Ancient Greek dialects". {\it Wikipedia, the Free Encyclopedia}. \textless\url{https://en.wikipedia.org/w/index.php?title=Ancient_Greek_dialects&oldid=575325271}\textgreater

\{H$_x$,\ipa{\s{m},\s{n}}\} \change\ \ipa{a}\\
\ipa{b\super H d\super H g\super H} \change\ \ipa{p\super h t\super h k\super h}\\
\ipa{s} \change\ \ipa{h} / \#_\\
\ipa{s} \change\ \ipa{h} / V_V\\
\ipa{t} \change\ \ipa{ts} / _\ipa{i}\\
\ipa{j} \change\ \ipa{h}\\
\ipa{k\super w k\super h\super w g\super w} \change\ \ipa{t t\super h d} / _E\\
\ipa{k\super w k\super h\super w g\super w} \change\ \ipa{p p\super h b} / _\{\ipa{a,o},C\}\\
\ipa{k\super w k\super h\super w g\super w} \change\ \ipa{k k\super h g} / _\ipa{u}\\
\ipa{k\super w k\super h\super w g\super w} \change\ \ipa{k k\super h g} / \ipa{u}_\\
\ipa{ts} \change\ \ipa{s}\\
\ipa{h} \change\ \O\\
V\ipa{ns} \change\ V\ipa{(:)s} (Tucker says that ``[i]n a few Doric dialects the lengthening did not occur")\\
\ipa{n} \change\ \O\ / _\ipa{s}\\
\ipa{tS} \change\ \ipa{t} / \#_\\
\ipa{tS} \change\ \ipa{t:} / medial\\
\ipa{h} \change\ \O\ (in those ``dialects of the western fringe of Asia Minor and the near-by islands")\\
Vowel contraction (on which the author does not elaborate much)\\

\subsubsection{Proto-Indo-European to Elian Greek}{\it Pogostick Man}, from Tucker, R. Whitney (1969), ``Chronology of Greek Sound Changes". {\it The American Journal of Philology} 90(1):36 -- 47; and Wikipedia contributors (2013), ``Ancient Greek dialects". {\it Wikipedia, the Free Encyclopedia}. \textless\url{https://en.wikipedia.org/w/index.php?title=Ancient_Greek_dialects&oldid=575325271}\textgreater

\{H$_x$,\ipa{\s{m},\s{n}}\} \change\ \ipa{a}\\
\ipa{b\super H d\super H g\super H} \change\ \ipa{p\super h t\super h k\super h}\\
\ipa{s} \change\ \ipa{h} / \#_\\
\ipa{s} \change\ \ipa{h} / V_V\\
\ipa{t} \change\ \ipa{ts} / _\ipa{i}\\
\ipa{j} \change\ \ipa{h}\\
\ipa{k\super w k\super h\super w g\super w} \change\ \ipa{t t\super h d} / _E\\
\ipa{k\super w k\super h\super w g\super w} \change\ \ipa{p p\super h b} / _\{\ipa{a,o},C\}\\
\ipa{k\super w k\super h\super w g\super w} \change\ \ipa{k k\super h g} / _\ipa{u}\\
\ipa{k\super w k\super h\super w g\super w} \change\ \ipa{k k\super h g} / \ipa{u}_\\
\ipa{ts} \change\ \ipa{s}\\
\ipa{h} \change\ \O\\
V\ipa{ns} \change\ V\ipa{:s}\\
\ipa{n} \change\ \O\ / _\ipa{s}\\
\ipa{h} \change\ \O\\

\subsubsection{Proto-Indo-European to Ionic Greek}{\it Pogostick Man}, from Tucker, R. Whitney (1969), ``Chronology of Greek Sound Changes". {\it The American Journal of Philology} 90(1):36 -- 47; and Wikipedia contributors (2013), ``Ancient Greek dialects". {\it Wikipedia, the Free Encyclopedia}. \textless\url{https://en.wikipedia.org/w/index.php?title=Ancient_Greek_dialects&oldid=575325271}\textgreater

\{H$_x$,\ipa{\s{m},\s{n}}\} \change\ \ipa{a}\\
\ipa{b\super H d\super H g\super H} \change\ \ipa{p\super h t\super h k\super h}\\
\ipa{s} \change\ \ipa{h} / \#_\\
\ipa{s} \change\ \ipa{h} / V_V\\
\ipa{t} \change\ \ipa{ts} / _\ipa{i}\\
\ipa{j} \change\ \ipa{h}\\
\ipa{k\super w k\super h\super w g\super w} \change\ \ipa{t t\super h d} / _E\\
\ipa{k\super w k\super h\super w g\super w} \change\ \ipa{p p\super h b} / _\{\ipa{a,o},C\}\\
\ipa{k\super w k\super h\super w g\super w} \change\ \ipa{k k\super h g} / _\ipa{u}\\
\ipa{k\super w k\super h\super w g\super w} \change\ \ipa{k k\super h g} / \ipa{u}_\\
\ipa{ts} \change\ \ipa{s}\\
\ipa{h} \change\ \O\\
\ipa{A:} \change\ \ipa{\ae:}\\
V\ipa{ns} \change\ V\ipa{:s}\\
\ipa{n} \change\ \O\ / _\ipa{s}\\
\ipa{tS} \change\ \ipa{s} / \#_\\
\ipa{tS} \change\ \ipa{s:} / medial\\
VC\ipa{w} \change\ V\ipa{:}C\\
\ipa{w} \change\ \O\\
\ipa{h} \change\ \O\ (in Eastern Ionic)\\
Vowel contraction (on which the author does not elaborate much)\\
\ipa{\ae:} \change\ \ipa{E:}\\
\ipa{ej ow} \change\ \ipa{e: o:} happened ``in the various Ionic dialects at various dates"\\
Some ``metathesis of quality as well as of quantity" with regards to vowels; did not occur to the same degree as it did in Attic\\
\ipa{u(:)(j)} \change\ \ipa{y(:)(j)}\\
\ipa{o:} \change\ \ipa{u:} (?)\\
\ipa{j} \change\ \O\ / V\ipa{:}_\\
\ipa{e:} \change\ \ipa{i:} / _V\\
\ipa{E:} \change\ \ipa{e:}\\
\ipa{Aj} \change\ \ipa{E:}\\
\ipa{h} \change\ \O\\
\ipa{oj} \change\ \ipa{\o j} \change\ \ipa{yj} (\change\ \ipa{y:} sometimes)\\
\ipa{e o} \change\ \ipa{E O}\\

\subsubsection{Proto-Indo-European to Laconian Greek}{\it Pogostick Man}, from Tucker, R. Whitney (1969), ``Chronology of Greek Sound Changes". {\it The American Journal of Philology} 90(1):36 -- 47; and Wikipedia contributors (2013), ``Ancient Greek dialects". {\it Wikipedia, the Free Encyclopedia}. \textless\url{https://en.wikipedia.org/w/index.php?title=Ancient_Greek_dialects&oldid=575325271}\textgreater

\{H$_x$,\ipa{\s{m},\s{n}}\} \change\ \ipa{a}\\
\ipa{b\super H d\super H g\super H} \change\ \ipa{p\super h t\super h k\super h}\\
\ipa{s} \change\ \ipa{h} / \#_\\
\ipa{s} \change\ \ipa{h} / V_V\\
\ipa{t} \change\ \ipa{ts} / _\ipa{i}\\
\ipa{j} \change\ \ipa{h}\\
\ipa{k\super w k\super h\super w g\super w} \change\ \ipa{t t\super h d} / _E\\
\ipa{k\super w k\super h\super w g\super w} \change\ \ipa{p p\super h b} / _\{\ipa{a,o},C\}\\
\ipa{k\super w k\super h\super w g\super w} \change\ \ipa{k k\super h g} / _\ipa{u}\\
\ipa{k\super w k\super h\super w g\super w} \change\ \ipa{k k\super h g} / \ipa{u}_\\
\ipa{ts} \change\ \ipa{s}\\
\ipa{h} \change\ \O\\
V\ipa{ns} \change\ V\ipa{:s}\\
\ipa{n} \change\ \O\ / _\ipa{s}\\
\ipa{p\super h t\super h k\super h} \change\ \ipa{f T x}\\

\subsubsection{Proto-Indo-European to Mycenaean Greek}{\it Pogostick Man}, from Tucker, R. Whitney (1969), ``Chronology of Greek Sound Changes". {\it The American Journal of Philology} 90(1):36 -- 47; and Wikipedia contributors (2013), ``Ancient Greek dialects". {\it Wikipedia, the Free Encyclopedia}. \textless\url{https://en.wikipedia.org/w/index.php?title=Ancient_Greek_dialects&oldid=575325271}\textgreater

\{H$_x$,\ipa{\s{m},\s{n}}\} \change\ \ipa{a}\\
\ipa{b\super H d\super H g\super H} \change\ \ipa{p\super h t\super h k\super h}\\
\ipa{s} \change\ \ipa{h} / \#_\\
\ipa{s} \change\ \ipa{h} / V_V\\
\ipa{t} \change\ \ipa{ts} / _\ipa{i}\\
\ipa{j} \change\ \ipa{h}\\
\ipa{k\super w k\super h\super w g\super w} \change\ \ipa{t t\super h d} / _E\\
\ipa{k\super w k\super h\super w g\super w} \change\ \ipa{p p\super h b} / _\{\ipa{a,o},C\}\\
\ipa{k\super w k\super h\super w g\super w} \change\ \ipa{k k\super h g} / _\ipa{u}\\
\ipa{k\super w k\super h\super w g\super w} \change\ \ipa{k k\super h g} / \ipa{u}_\\
\ipa{ts} \change\ \ipa{s}\\
\ipa{h} \change\ \O\\

\subsection{Proto-Indo-European to Hittite}{\it Goatface}

\ipa{\'{k} \'{g} \'{g}\super H} \change\ \ipa{k g g\super H}\\
\ipa{b\super H d\super H g\super H} \change\ \ipa{p t k}\\
\ipa{k\super w g\super w g\super w\super H} \change\ \ipa{ku gu ku}\\
\ipa{t} \change\ \ipa{ts} / _\{\ipa{i,e}\}\\
\ipa{m} \change\ \O\ / _\#\\
\ipa{e(:)} \change\ \ipa{a(:)} / _h$_2$\\
\ipa{e(:)} \change\ \ipa{a(:)} / h$_2$_\\
\ipa{e(:)} \change\ \ipa{o(:)} / _h$_3$\\
\ipa{e(:)} \change\ \ipa{o(:)} / h$_3$_\\
h$_3$ \change\ \O\ / _o {\it ``(according to Kortlandt)''}\\
h$_2$ \change\ \ipa{x} (or some sort of dorsal or laryngeal fricative?)\\
\ipa{o(:)} \change\ \ipa{a(:)}\\
\{\ipa{u:,eu,au}\} \change\ \ipa{u}\\
\ipa{\s{m} \s{n} \s{r} \s{l}} \change\ \ipa{am an ar al}\\
\ipa{w} \change\ \ipa{m} / \ipa{u}_\\
{\it ``Changes I'm less sure of"}\\
--- \ipa{r} \change\ \O\ / \#_\\
--- \ipa{r} \change\ \O\ / _\# {\it ``sometimes??"}\\
--- \ipa{e(:)} \change\ \ipa{a(:)} / _R {\it ``sometimes??"}\\
--- \ipa{e(:)} \change\ \ipa{a(:)} / {\it ``when unstressed?"}

\subsection{Proto-Indo-European to Proto-Indo-Iranian}{\it Tropylium}, from Kobayashi, Masato (2004), {\it Historical Phonology of Old Indo-Aryan Consonants}

\ipa{e} \change\ \ipa{a} / _\{h$_2$,h$_3$\}\\
\ipa{e} \change\ \ipa{a} / \{h$_2$,h$_3$\}_\\
\ipa{p} \change\ \ipa{b} / _h$_3$\\
H \change\ \ipa{@} / ``syllabic''\\
\ipa{@} \change\ \O\ / \#_\\
\{h$_1$,h$_3$\} \change\ *H\\
*H \change\ \O\ / S_\\
K\ipa{\super w} \change\ K\\
\ipa{s} \change\ \ipa{S} / \{\ipa{u,i,l,r,}K,\'{K}\}\\
\'{K} \change\ T\v{S}\\
B\ipa{\super H}$_1$P$_2$ \change\ B$_1$B\ipa{\super H}$_2$ / ``includes s \ipa{S} \textgreater\ z\ipa{\super H Z\super H}''\\
B\ipa{\super H}B\ipa{\super H} \change\ BB\ipa{\super H}\\
\ipa{ptS} \change\ \ipa{pS}\\
\ipa{ttS ddZ\super H} \change\ \ipa{t.S d.Z\super H}\\
\ipa{t} \change\ \O\ / _\ipa{St}\\
\ipa{d} \change\ \O\ / _\ipa{Zd\super H}\\
\ipa{tSS} \change\ \ipa{S:}\\
\ipa{k g g\super H} \change\ \ipa{c \textbardotlessj\ \textbardotlessj\super H} / _\ipa{e,i,j}\\
\ipa{o} \change\ \ipa{a:} / _CV, ``does not affect o$_2$ \textless\ eh$_3$''\\
\{\ipa{e,o,o}$_2$,\r{N}\} \change\ \ipa{a}

\subsubsection{Proto-Indo-Iranian to Proto-Indo-Aryan}{\it Tropylium}, from Kobayashi, Masato (2004), {\it Historical Phonology of Old Indo-Aryan Consonants}

\ipa{@} \change\ \ipa{i}\\
Sh$_2$ \change\ S[+ aspirated]\\
h$_2$ \change\ *H\\
VH \change\ V\ipa{:} / _\{C,\#\}\\
\r{R}H \change\ \{\ipa{u,i}\}R / _V (sporadic)\\
\r{R}H \change\ \{\ipa{u,i}\}\ipa{:}R / _C\\
\ipa{S Z} \change\ \ipa{\:s \:z}\\
\ipa{n} \change\ \ipa{\:n} / R(V)_\\
\ipa{t d(\super H) n} \change\ \ipa{\:t \:d(\super H) \:n} / \d{C}_ ! _\ipa{r}\\
\ipa{s t d(\super H) n} \change\ \ipa{\:s \:t \:d(\super H) \:n} / _\d{C}\\
\ipa{ls lt ld(\super H) ln} \change\ \ipa{\:s \:t \:d(\super H) \:n} / ``disputed''\\
\ipa{s} \change\ \ipa{\:s} / _V\{\ipa{\:s,\:t,\:d(\super H)}\}\\
\O\ \change\ \ipa{a} / \#_\ipa{z}\\
\ipa{u\:z a\:z i\:z} \change\ \ipa{u: @: i:}\\
\ipa{@:} \change\ \ipa{o:} / \ipa{w}_\\
\ipa{@:} \change\ \ipa{e:}\\
\ipa{tS dZ(\super H)} \change\ \ipa{tC d\textctz(\super H)} \change\ \ipa{C \textbardotlessj(\super H)}\\
\ipa{tst dzd\super H} \change\ \ipa{t: d\super H:}\\
\ipa{k} \change\ \O\ / \#_\ipa{t}\\
\ipa{p} \change\ \O\ / \#_\ipa{st}\\
\ipa{\textbardotlessj\super H} \change\ \ipa{H}\\
\ipa{d\super H} \change\ \ipa{H} / (unclear environment)\\
\ipa{b\super H} \change\ \ipa{H} (very rare)\\
V\ipa{m} \change\ \~{V} / _(\#)\{\ipa{s,C}\}\\
\ipa{n} \change\ \ipa{\textltailn} / \ipa{\textbardotlessj}_

\paragraph{Proto-Indo-Aryan to Central Middle Indo-Aryan}{\it Pogostick Man}, from Shukla, Shaligram (1974), ``Phonological change and dialect variation in Middle-Indo-Aryan". In Anderson, J., and C. Jones (Eds.), {\it Historical Linguistics} II:391-401.

C(C) \change\ \O\ / C_\#\\
VN VC[-nas] \change\ V[+nas] V\ipa{:} / _\#\\
\ipa{a}\{\ipa{i,j}\}(\ipa{a}) \ipa{a}\{\ipa{u,w}\}(\ipa{a}) \change\ \ipa{e o}\\
\ipa{j w} \change\ \ipa{dZ b} / V_V\\
C \change\ C[+voiced] / V_V\\
\{\ipa{b\super H,d\super H,g\super H}\} \{\ipa{j,v}\} \change\ \ipa{h} \O\ / V_V\\
V\ipa{m} \change\ V\ipa{\!{v}} \change\ V[+nas]\ipa{v} / _V\\
\ipa{e o} \change\ \ipa{i u} / _\#\\
V\ipa{:} \change\ V[-long] / _\#\\
\ipa{ah} \change\ \ipa{o} \\
\ipa{\:r} \change\ \ipa{i}\\
\ipa{\:s} \change\ \ipa{x} / \ipa{k}_\\
\{\ipa{\:s,\c{c}}\} \change\ \ipa{s}\\
\ipa{v} \change\ \O\ / \{\ipa{t,d}\}_\\
C$_1$C$_2$ \change\ C$_2$C$_2$ / V_V\\
C\ipa{n} \change\ CC / V_V ! C = \ipa{dZ}\\
\ipa{dZ\textltailn} \change\ \ipa{\:n:} / V_V

\paragraph{Proto-Indo-Aryan to Eastern Middle Indo-Aryan}{\it Pogostick Man}, from Shukla, Shaligram (1974), ``Phonological change and dialect variation in Middle-Indo-Aryan". In Anderson, J., and C. Jones (Eds.), {\it Historical Linguistics} II:391-401.

C(C) \change\ \O\ / C_\#\\
VN VC[-nas] \change\ V[+nas] V\ipa{:} / _\#\\
\ipa{a}\{\ipa{i,j}\}(\ipa{a}) \ipa{a}\{\ipa{u,w}\}(\ipa{a}) \change\ \ipa{e o}\\
\ipa{j w} \change\ \ipa{dZ b} / V_V\\
C \change\ C[+voiced] / V_V\\
\{\ipa{b\super H,d\super H,g\super H}\} \{\ipa{j,v}\} \change\ \ipa{h} \O\ / V_V\\
\ipa{b} \{\ipa{d,dZ,g}\} \change\ \ipa{v j} / V_V\\
V\ipa{m} \change\ V\ipa{\!{v}} \change\ V[+nas]\ipa{v} / _V\\
\ipa{e o} \change\ \ipa{i u} / _\#\\
V\ipa{:} \change\ V[-long] / _\#\\
\ipa{ah} \change\ \ipa{e} \\
\ipa{\:r} \change\ \ipa{i}\\
\ipa{k\:s} \change\ \ipa{hk}\\
\ipa{\:s s} \change\ \ipa{s \c{c}}\\
\ipa{r} \change\ \ipa{l}\\
\ipa{v} \change\ \O\ / \{\ipa{t,d}\}_\\
C$_1$C$_2$ \change\ C$_2$C$_2$ / V_V\\
C\ipa{n} \change\ CC / V_V ! C = \ipa{dZ}\\
\ipa{dZ\textltailn} \change\ \ipa{\textltailn:} / V_V

\paragraph{Proto-Indo-Aryan to Northwestern Middle Indo-Aryan}{\it Pogostick Man}, from Shukla, Shaligram (1974), ``Phonological change and dialect variation in Middle-Indo-Aryan". In Anderson, J., and C. Jones (Eds.), {\it Historical Linguistics} II:391-401.

C(C) \change\ \O\ / C_\#\\
VN VC[-nas] \change\ V[+nas] V\ipa{:} / _\#\\
\ipa{a}\{\ipa{i,j}\}(\ipa{a}) \ipa{a}\{\ipa{u,w}\}(\ipa{a}) \change\ \ipa{e o}\\
\ipa{j w} \change\ \ipa{dZ b} / V_V\\
C \change\ C[+voiced] / V_V\\
\{\ipa{b\super H,d\super H,g\super H}\} \{\ipa{j,v}\} \change\ \ipa{h} \O\ / V_V\\
V\ipa{m} \change\ V\ipa{\!{v}} \change\ V[+nas]\ipa{v} / _V\\
\ipa{e o} \change\ \ipa{i u} / _\#\\
V\ipa{:} \change\ V[-long] / _\#\\
\ipa{ah} \change\ \ipa{o} \\
\ipa{\:r} \change\ \ipa{i}\\
\ipa{k\:s} \change\ \ipa{tS:}\\
\{\ipa{\:s,\c{c}}\} \change\ \ipa{s}\\
\ipa{s}C \change\ C\ipa{h}\\
\ipa{v} \change\ \O\ / \{\ipa{t,d}\}_\\
C$_1$C$_2$ \change\ C$_2$C$_2$ / V_V\\
C\ipa{n} \change\ CC / V_V ! C = \ipa{dZ}\\
\ipa{dZ\textltailn} \change\ \ipa{\:n:} / V_V

\paragraph{Proto-Indo-Aryan to Western Middle Indo-Aryan}{\it Pogostick Man}, from Shukla, Shaligram (1974), ``Phonological change and dialect variation in Middle-Indo-Aryan". In Anderson, J., and C. Jones (Eds.), {\it Historical Linguistics} II:391-401.

C(C) \change\ \O\ / C_\#\\
VN VC[-nas] \change\ V[+nas] V\ipa{:} / _\#\\
\ipa{a}\{\ipa{i,j}\}(\ipa{a}) \ipa{a}\{\ipa{u,w}\}(\ipa{a}) \change\ \ipa{e o}\\
\ipa{j w} \change\ \ipa{dZ b} / V_V\\
C \change\ C[+voiced] / V_V\\
\{\ipa{b\super H,d\super H,g\super H}\} \{\ipa{j,v}\} \change\ \ipa{h} \O\ / V_V\\
V\ipa{m} \change\ V\ipa{\!{v}} \change\ V[+nas]\ipa{v} / _V\\
\ipa{e o} \change\ \ipa{i u} / _\#\\
V\ipa{:} \change\ V[-long] / _\#\\
\ipa{ah} \change\ \ipa{o} \\
\ipa{\:r} \change\ \ipa{i}\\
\ipa{k\:s} \change\ \ipa{tS:}\\
\{\ipa{\:s,\c{c}}\} \change\ \ipa{s}\\
\ipa{tv dv} \change\ \ipa{p b}\\
C$_1$C$_2$ \change\ C$_2$C$_2$ / V_V\\
C\ipa{n} \change\ CC / V_V ! C = \ipa{dZ}\\
\ipa{dZ\textltailn} \change\ \ipa{\:n:} / V_V

\paragraph{Proto-Indo-Aryan to Vedic Sanskrit}\textit{Tropylium}, from Kobayashi, Masato (2004), \textit{Historical Phonology of Old Indo-Aryan Consonants}

\{\ipa{s,\:s}\} \change\ \ipa{h} / _\#C[- voice]\\
\{\ipa{s,\:s}\} \change\ \ipa{r} / _\#C[+ voice]\\
\ipa{l} \change\ \ipa{r}\\
\ipa{\:d(\super H)} \change\ \ipa{\:l(\super H)}\\
\ipa{sC C\:s} \change\ \ipa{C: \:s:}\\
\ipa{s: \:s: C:} \change\ \ipa{t.s t.\ipa{\:s} t.C}\\
\ipa{t.\:s t.C} \change\ \ipa{k\:s c\super h:}\\
\ipa{c\super h:} \change\ \ipa{c\super h} / C_\\
\ipa{bz\super H} \change\ \ipa{ps}\\
\{\ipa{p\:s,c\:s,\textbardotlessj\:z\super H,g\:z\super H}\} \change\ \ipa{k\:s}

\subparagraph{Vedic Sanskrit to Classical Sanskrit}\textit{Tropylium}, from Kobayashi, Masato (2004), \textit{Historical Phonology of Old Indo-Aryan Consonants}

\ipa{\:l(\super H)} \change\ \ipa{\:d(\super H)}\\
H \change\ \O\\
\ipa{au a:u ai a:i} \change\ \ipa{au o: ai e:} / ! _V\\
\ipa{w} \change\ \ipa{v}

\subsection{Proto-Indo-European to Proto-Slavic}{\it Hwhatting}

\tab {\it NB: ``Not in chronological order''}

\ipa{b\super H d\super H} \{\ipa{g\super H,\'{g}\super H}\} \ipa{g\super w\super H} \change\ \ipa{b d g g\super w}\\
K\ipa{\super w \'{k} \'{g}} \change\ K \ipa{s z}\\
\ipa{s} \change\ \ipa{x} / \{\ipa{i,u,r,k}\}_\\
\ipa{k g x} \change\ \ipa{tS Z S} / _\{\ipa{e(:)(i),i(:)}\}\\
\{\ipa{a,o,@}\} \change\ \ipa{e} / \ipa{j}_\\
\{\ipa{a,o,@}\} \change\ \ipa{o}\\
\ipa{i u} \change\ \ipa{\textsoftsign\ \texthardsign}\\j
\ipa{i: u:} \change\ \ipa{i 1}\\
\ipa{u} \change\ \ipa{\textsoftsign} / \ipa{j}_\\
\ipa{e(:)i} \change\ \ipa{i}\\
\{\ipa{ai,oi}\} \change\ \ipa{i} / \ipa{j}_\\
\{\ipa{ai,oi}\} \change\ \ipa{\ae:}\\
\{\ipa{a:i,o:i}\} \change\ \{\ipa{\ae:,a}\} (the former seems to be more common)\\
\{\ipa{a(:)u,o(:)u}\} \ipa{e(:)u} \change\ \ipa{u ju}\\
\ipa{e} \change\ \ipa{\textsoftsign} / _\ipa{j}V\\
\ipa{e} \change\ \ipa{o} / _\ipa{w}V\\
\ipa{w} \change\ \ipa{v}\\
\ipa{\s{l} \s{r}} \change\ \{\ipa{\textsoftsign l,\texthardsign l}\} \{\ipa{\textsoftsign r,\texthardsign r}\}\\
\{\ipa{\s{m},\s{n}}\} \change\ \{\ipa{\~{e},\~{o}}\} / _C\$\\
\ipa{\s{m} \s{n}} \change\ \{\ipa{\textsoftsign m,\texthardsign m}\} \{\ipa{\textsoftsign n,\texthardsign n}\}\\
\ipa{\ae:} \change\ \ipa{a} / ``After palatal fricatives and affricates''\\
\{\ipa{e(:),i(:)}\} \{\ipa{a(:),o(:),u(:)}\} \change\ \ipa{\~{e} \~{o}} / _N\$\\
\ipa{oi o} \change\ \ipa{i \texthardsign} / ``Sometimes in final syllables''\\
\{O,N\} \change\ \O\ / _\$\\
\ipa{k g x} \change\ \ipa{ts dz s} / _\{\ipa{\ae:,i}\}\\
\ipa{k g x} \change\ \ipa{ts dz s} / ``After some syllables with front vowels''\\
\ipa{sj zj} \change\ \ipa{S Z}\\
\ipa{kj gj xj} \change\ \ipa{tS Z S}

\subsubsection{Proto-Slavic to Polish}{\it Xi\textpolhook{a}dz Faust}, in \url{http://pittmirg.ovh.org/inne/psc.pdf http://pittmirg.ovh.org/inne/psc.pdf}, mainly citing Klemensewicz {\it et al.} (1955), ``Gramatyka historyczna j\textpolhook{e}zyka polskiego'', and Dubisz and D\l ugosz-Kurczabowa (2003?), ``Gramatyka historyczna j\textpolhook{e}zyka polskiego''

\tab {\it NB: The original document heavily uses Slavistic notation as opposed to IPA; I've done the best I could in figuring this stuff out but be warned of possible errors.}

\ipa{sk x} \change\ \ipa{CtC C} / _E\\
\ipa{x} \change\ \ipa{C} / E_\\
\ipa{El} \change\ \ipa{lO} / T_T ``in certain cases (mostly after a PSl. palato-alveolar")\\
\ipa{Ol Or El Er} \change\ \ipa{lO rO lE rE} / T_T\\
\ipa{Or Ol} \change\ \ipa{ra la} / \#_T ``in syllables with long vowels"\\
\ipa{Or Ol} \change\ \ipa{rO lO} / \#_T\\
C \change\ C\super j / _E ! /\ipa{j C \textctz}\\
\ipa{j} \change\ \ipa{l\super j} / \{\ipa{p,b,m,v}\}_ (sporadic)\\
\ipa{E \~{E} E:} \change\ \ipa{O \~{O} a} / _C[-palatalized +dental] (also sporadically before plain non-dentals)\\
\ipa{E:} \change\ \ipa{E}\\
Havlik's law:\\
--- \{\ipa{\textsoftsign,\texthardsign}\} \change\ \ipa{e} / iambic counting from U\# or a syllable not containing a yer\\
--- \{\ipa{\textsoftsign,\texthardsign}\} \change\ \O] / in even syllables counting iambic from U\# or a syllable not containing a yer\\
--- ``[H]owever: in the vicinity of *\ipa{j} the development of yers did not comply with the aforementioned law"\\
\ipa{\texthardsign} \change\ \ipa{a:} / _\ipa{r}\\
\ipa{\texthardsign l} \change\ \ipa{O(:)\textsuperimposetilde{l}} / P_\\
\ipa{\texthardsign l} \change\ \ipa{E\textsuperimposetilde{l}} / K_\\
\ipa{\texthardsign l} \change\ \ipa{\textsuperimposetilde{l}u} / else\\
\ipa{\textsoftsign l} \change\ \ipa{\textsuperimposetilde{l}u} / C[+dental]_\\
\ipa{\textsoftsign l} \change\ \ipa{E\textsuperimposetilde{l}} / P_C[+dental -palatalized]\\
\ipa{\textsoftsign l} \change\ \ipa{il} / P_\\
\ipa{\textsoftsign l} \change\ \ipa{O(:)\textsuperimposetilde{l}}\\
\ipa{\textsoftsign} \change\ \ipa{a:} / \ipa{r}_C[+dental -palatalized]\\
\ipa{\textsoftsign r} \change\ \ipa{i(:)\:z} \change\ \{\ipa{E(:)r,E(:)\:z}\} \change\ \{\ipa{Er,E\:z}\}\\
\{\ipa{\~{E},\~{O}\}} \change\ \ipa{\~{a}}\\
\ipa{a E i O u 1 \~{a}} \change\ \ipa{a: E: i: O: u: 1: \~{a}:} / _\{C/U\}[+voiced][lost yer] (i.e., a voiced consonant or a cluster with one)\\
\ipa{ajE} \change\ \ipa{E:} in adjectives, \ipa{a:} in verbs\\
\{\ipa{aja,Oja}\} \{\ipa{OjE,1jE}\} \ipa{1jE} \change\ \ipa{a: E: i:}\\
\{\ipa{E(:)jE,\textsoftsign jE,OjE,ujE,1jE}\} \change\ \ipa{E:}\\
\{\ipa{Oj\~{O},\~{O}j\~{O},\textsoftsign j\~{O}}\} \change\ \ipa{\~{O}:}\\
\ipa{\textsoftsign j\textsoftsign\ \texthardsign j\textsoftsign} \change\ \ipa{i 1}\\
\ipa{j\textsoftsign} \change\ \ipa{i} / utterance-initially (cf. English utterance-initial glottal stops before vowels)\\
\ipa{\textsoftsign\ \texthardsign} \change\ \ipa{i 1} / _\ipa{j}\\
\ipa{ji} \change\ \ipa{i} / \#_\\
\{\ipa{aja,Eja,\textsoftsign ja,Oja}\} \change\ ipa{a:}\\
\ipa{iji 1j1} \change\ \ipa{i 1}\\
\ipa{Ovi} \change\ \O\\
O[+voice] \change\ O[-voice] / _\# (unless followed by some type of voiced consonant, be it any type of consonant or just an obstruent---this differs by location)\\
\ipa{O} \change\ \ipa{O:} / _\{\ipa{r,l}\} (sporadic, perhaps analogical)\\
\O\ \change\ \ipa{h} / _\ipa{\texthardsign}\\
Mobile stress \change\ initial stress \change\ penultimate stress (in most areas)\\
V \change\ \O\ / unstressed (sporadic)\\
\ipa{i} \change\ \O\ / _\# ``in the infinite and imperative desinences. . .some verbs have never been affected due to a potential `difficult' cluster that would result, instead they got an analogical final -j extension"\\
\ipa{tsi \:zi} \change\ \ipa{tC \:z} / V_\\
\ipa{i u} \change\ \ipa{u i} / \{\ipa{l\super j,j}\}_ (sporadic)\\
\ipa{t\super j d\super j s\super j z\super j n\super j r\super j l l\super j} \change\ \ipa{tC dZ C \textctz\ \textctn\ r \textsuperimposetilde{l} l} (this last probably not before /\ipa{i}/)\\
\ipa{i: u: 1:} \change\ \ipa{i u 1}\\
\ipa{\~{a}} \change\ \ipa{\~{E}} / short only\\
\ipa{\~{a}:} \change\ \ipa{\~{a}} \change\ \ipa{\~{O}}\\
\ipa{a: O: E:} \change\ \ipa{6 o e}\\
V\ipa{:} \change\ V in certain frequently-used words\\
Sporadic (de)nasalization of vowels; ``there were certain environments which favoured nasality changes: in the vicinity of nasal consonants. . .and before sibilants"\\
\{\ipa{i,1}\} \change\ \ipa{E} / _C[+rhotic]\\
\ipa{k g} \change\ \ipa{k\super j g\super j} / _\ipa{E} where the vowel is from a yer or a borrowing\\
\ipa{k1 g1} \change\ \ipa{k\super ji g\super ji}\\
\ipa{Si Zi tSi dZi Ci \textctz i} \change\ \ipa{\:s1 \:z1 \:t\:s1 \:d\:z1 ts1 Z1}\\
\ipa{S Z tS dZ C \textctz} \change\ \ipa{\:s \:z \:t\:s \:d\:z ts Z}\\
\ipa{\|'r} \change\ \ipa{\:s} / C[-voiced]_\\
\ipa{\|'r} \change\ \ipa{\:s} / _C[+voiced]\\
\ipa{\'r} \change\ \ipa{\:z} / else\\
\ipa{E} \change\ \ipa{O} / _\ipa{\textsuperimposetilde{l}} (if the vowel was from a yer)\\
\{\ipa{E,a}\} \change\ \ipa{O} (sporadic)\\
V \change\ \ipa{E} (sporadic, analogical)\\
\ipa{\textsuperimposetilde{l}} \change\ \ipa{w}\\
\ipa{6 e} \change\ \ipa{a E}\\
\ipa{o} \change\ \ipa{O} / _N\\
\ipa{o} \change\ \ipa{u} / else\\
\ipa{u} \change\ \ipa{O} (rare, sporadic)\\
\ipa{p\super j m\super j f\super j} \change\ \ipa{p m f} / _\#\\
\{\ipa{i,1}\} \change\ \O\ / _\ipa{j}V when unstressed\\
\ipa{\~{E}} \change\ \ipa{E} / _\{\#,\ipa{l,\textsuperimposetilde{l}}\}\\
\ipa{\~{O}} \change\ \ipa{O} / _\{\ipa{l,\textsuperimposetilde{l}}\}\\
\ipa{\~{O}} \change\ \ipa{O} / _\# (in some regions or dialects)\\
\ipa{\~{E} \~{O}} \change\ \ipa{\~{E}}N \ipa{\~{O}}N / _\{S,A\}\\
\ipa{\~{E} \~{O}} \change\ \ipa{E\~{\textturnmrleg} O\~{\textturnmrleg}} / _F[-palatal]\\
\ipa{\~{O}} \change\ \ipa{O\~{\textturnmrleg}} / _\# (in standard registers/pronunciations)\\
\ipa{\~{E} \~{O}} \change\ \ipa{E\~{\j} O\~{\j}} / _F[+palatal]\\
\ipa{\textctn} \change\ \ipa{\~{\j}} / _F\\
\ipa{n} \change\ \ipa{N} / _S[+velar] (regional)

\tab``The following sections are structured according to respective sound change types without much chronology, as the sound changes tend to sporadic, irregular or inconsistent or to be trends spreading over considerable time spans."\\\\
C\super j \change\ C / _C[+dental] with developments of yers in ablaut environments\\
C\super j \change\ C in select words due to prestige influence of Czech in the Middle Ages\\
C\super j \change\ C in select words otherwise, possibly by analogy\\
\ipa{t} \change\ \ipa{r} / \ipa{t}V_\\
\ipa{n} \change\ \ipa{m} / \{\ipa{b,p}\}\{\ipa{l,r,\:z}\}V_\\
\ipa{j} \change\ \O\ / _\ipa{E} ``in participial and deverbal forms originally with alveolopalatal consonants in the onsets of two consecutive syllables"\\
Oscillations involving:\\
--- Dentals and postalveolars\\
--- Postalveolars and alveolopalatals\\
--- Voicing\\
OR \change\ RO / V_C\\
RO \change\ OR / C_V\\
\ipa{v}C \change\ C\ipa{v} / _V\\
C\ipa{v} \change\ \ipa{v}C / _C\\
\ipa{tCts \textctz\|'r} \change\ \ipa{jts j\|'r}\\
\ipa{CtC} \change\ \ipa{js} / _\{\ipa{\t*{ts},s}\}\\
\ipa{z\:z} \change\ \ipa{\:z\:d\:z}\\
\ipa{\textctz\ z} \change\ \ipa{d\textctz\ dz} / _\ipa{v}\\
\O\ \change\ \ipa{d} / \ipa{r}_\ipa{z}\\
Regressive voicing/devoicing of obstruents in consonantal clusters\\
\ipa{v \|'r} \change\ \ipa{f \r*{r}} / C[-voiced]_\\
\ipa{v \|'r} \change\ \ipa{f \r*{r}} / _\{C[-voiced],\#\}\\
\ipa{\r{\|'r} \|'r} \change\ \ipa{\:s \:z}\\
L\ipa{v} \change\ L[-voiced]\ipa{f} / O[-voiced]_ ``for many speakers''\\
\ipa{v} \change\ \O\ / \ipa{x}_\ipa{o}\\
\{\ipa{xv,pv}\} \change\ \ipa{f}\\
\ipa{plv} \change\ \ipa{pf}\\
\ipa{p} \change\ \O\ / \#_\ipa{p}\\
\ipa{\:t\:ss} \change\ \ipa{ts}\\
\{\ipa{z,s,\:s}\} \change\ \O\ / _\ipa{s}C\\
\ipa{x} \change\ \O\ / _\ipa{r} ``in the word `robak'"\\
\ipa{\|'r} \change\ \ipa{r} / \{\ipa{C,\textctz}\}_\\
\ipa{\:zr} \change\ \ipa{\textctz r}\\
\ipa{C\|'r \textctz\|'r} \change\ \ipa{\:sr \:zr} / ```szron' and `\.{z}re\'{c}'", respectively\\
\ipa{t} \change\ \O\ / \ipa{s}_\{\ipa{\textsuperimposetilde{l},w}\}\\
\ipa{\:t\:s} \change\ \ipa{t} / _\ipa{\|'r}\\
\ipa{C} \change\ \O\ / \ipa{t\|'r}_\ipa{tC}\\
\ipa{g} \change\ \O\ / _\ipa{d}\\
\ipa{w} \change\ \O\ / \{\ipa{r,b}\}_\\
\ipa{d} \change\ \O\ / \ipa{\textsuperimposetilde{l}}_\ipa{\textltailn}\\
\ipa{d} \change\ \O\ / \ipa{r}_\ipa{ts}\\
\ipa{d} \change\ \O\ / _\ipa{n} ``in arch. `jeno'"\\
\O\ \change\ \ipa{t} / \ipa{s}_\ipa{r} ``in `str\textpolhook{e}czy\'{c}'"\\
\ipa{st \:s\:t\:s} \change\ \ipa{z \:z} / _\ipa{b}\\
\{\ipa{b,p}\} \change\ \O\ / _\ipa{n} ``in verbs in -n\textpolhook{a}\'{c}"\\
\ipa{v} \change\ \O\ / _\ipa{stv}\\
\ipa{t} \change\ \O\ / \ipa{\:t\:s}_\ipa{v} in ``czworo"\\
\ipa{s} \change\ \O\ / _\ipa{\textsuperimposetilde{l}za} in ``s\l za"\\
\ipa{trk} \change\ \ipa{kr} / in the name of the river ``Skrwa"\\
\ipa{zd\super j st\super j} \change\ \{\ipa{\textctz,C}\} \{\ipa{s,C}\} / _\ipa{n}\\
\ipa{d\super j} \change\ \O\ / \ipa{r}_\ipa{n}\\
\ipa{st\super j} \change\ \ipa{C} / _\ipa{l}\\
\ipa{sl\super j} \change\ \O\ / _\ipa{s}\\
\ipa{pv} \change\ \ipa{f}\\
\ipa{\:t\:s} \change\ \ipa{\:s} / _\ipa{p}\\
\ipa{st\super jkl} \change\ \ipa{CtCkl} \change\ \{\ipa{C,s}\}\ipa{k\textsuperimposetilde{l}} \change\ \ipa{\:sk\textsuperimposetilde{l}}\\
\ipa{dz ts} \change\ \ipa{\t*{dz} \t*{ts}}\\
C[+sibilant]P \change\ C[+alveolopalatal] / _C[+coronal]\\
C[+sibilant]P \change\ C[+dental]\\
\ipa{w} \change\ \O\ / C_\#\\
\ipa{w} \change\ \O\ / C_C (sporadic)\\
``Oscillations between dental and alveo[lo]palatals" / _C\\
\ipa{n} \change\ \ipa{s} \change\ \ipa{C} / \ipa{k}_\ipa{\~{E}} where the vowel was from Proto-Slavic\\
\ipa{n} \change\ \ipa{s} / \ipa{k}_V\ipa{n}\\
\ipa{d} \change\ \ipa{g} / _\ipa{n}\\
\ipa{\:t\:s} \change\ \ipa{t} / _\ipa{r}\\
\ipa{C} \change\ \O\ / \ipa{t}_\ipa{CtC}\\
``Insertion of epenthetic vowels" in some situations, typically one of /\ipa{E u}/, the latter written as either $\langle$u$\rangle$ or $\langle$\'{o}$\rangle$\\
Epenthetic \ipa{d g} appears in some circumstances\\
P\super j \change\ P / _C\\
\ipa{r\super j} \change\ \ipa{r} / _\{\ipa{s,\t*{ts},l,w,n,\textltailn}\}\\
\ipa{t\super j d\super j} \change\ \ipa{t d} / _\{\ipa{l,n,\textltailn,r,\:z}\}\\
\ipa{C \textctz} \change\ \ipa{s z} / _C (sometimes)

\subsubsection{Proto-Slavic to Old Russian}{\it Hwhatting}

\tab {\it NB: ``Not in chronological order''}

\{\ipa{t,d}\} \change\ \O\ / V_\ipa{l}V\\
\ipa{or ol er el} \change\ \ipa{oro olo ere ele} / _\$\\
\ipa{mj pj bj} \change\ \ipa{ml\super j pl\super j bl\super j}\\
\ipa{tj dj} \change\ \ipa{tS, Z}\\
\ipa{kt gd} \change\ _E\\
\ipa{\~{e} \~{o}} \change\ \ipa{ja u}\\
\ipa{je} \change\ \ipa{o} / \#_\\
\ipa{je} \change\ \ipa{o} / V_ (sporadic)\\
\ipa{j} \change\ \O\ / \#_\ipa{u}\\
\ipa{j} \change\ \O\ / V_\ipa{u} (sporadic)\\
\ipa{j\textsoftsign} \change\ \ipa{i}\\
\O\ \change\ \ipa{j} / \#_\ipa{a}

\subsection{Proto-Indo-European to Proto-Italic}{\it Pogostick Man}, from \url{http://gillesquentel.org/docs/PIE_to_Italic_C.pdf} and \url{http://gillesquentel.org/docs/PIE_to_Italic_V.pdf}

\tab {\it NB: This is likely incomplete.}

p \change\ \{\ipa{p,k\super w}\}\\
\'{k} \'{g} \change\ \ipa{k g}\\
\'{g}\super h g\super w\super h \change\ \ipa{g\super h x\super w}\\
b\super h d\super h g\super h \change\ \ipa{p\super h t\super h d\super h} \change\ \ipa{F T x}\\
s \change\ \ipa{z} / medial (I'm assuming between vowels or when *s voiced in PIE)\\
\ipa{eu} \change\ \ipa{ou}

\subsubsection{Proto-Italic to Proto-Latino-Falsican}{\it Pogostick Man}, from \url{http://gillesquentel.org/docs/PIE_to_Italic_C.pdf} and \url{http://gillesquentel.org/docs/PIE_to_Italic_V.pdf}

\tab {\it NB: This is likely incomplete.}

\ipa{x} \change\ \ipa{h}\\
\ipa{g\super w} \change\ \ipa{w}\\
\ipa{g\super h} \change\ \ipa{f} / \#_\\
\ipa{g\super h} \change\ \{\ipa{d,h,g}\}\\
\{\ipa{F,T}\} \change\ \ipa{f} / \#_\\
\ipa{F T} \change\ \ipa{b} \{\ipa{d,b}\} / V_V\\
\ipa{z} \change\ \ipa{r}\\
\ipa{x\super w} \change\ \ipa{f}\\
\ipa{x\super w} \change\ \{\ipa{w,g\super w}\}\\
\textsubring{l} \textsubring{r} \change\ \ipa{ol} \{\ipa{or,er}\} / _\#\\
\textsubring{m} \textsubring{n} \change\ \ipa{em en}\\
\ipa{e} \change\ \{\ipa{e,i}\}

\paragraph{Proto-Indo-European to Latin}{\it Mecislau}, from Ramat, Anna Giacrole and Paolo Ramat, {\it The Indo-European Languages}, and other sources

\ipa{e o} \change\ \ipa{i u} / _\ipa{N}\\
\ipa{e} \change\ \ipa{o} / _\ipa{\textsuperimposetilde{l}}\\
\ipa{o} \change\ \ipa{u} / _\{\ipa{mb,mk,\textsuperimposetilde{l}}\}\\
\ipa{o} \change\ \ipa{e} / \ipa{w}_\{\ipa{r,s,t}\}\\
\ipa{o:} \change\ \ipa{u:} / _\ipa{r}\\
\ipa{aj} \change\ \ipa{ai} \change\ \ipa{e:} (in rustic dialects)\\
\ipa{aj} \change\ \ipa{ai} \change\ \ipa{ae}\\
\ipa{oj} \change\ \ipa{oi} \change\ \ipa{oe} \change\ \ipa{u:}\\
\ipa{aw} \change\ \ipa{o:} (in rustic dialects)\\
\ipa{aw} \change\ \ipa{au}\\
\{\ipa{ew,ow}\} \change\ \ipa{ou} \change\ \ipa{u:}\\
V \change\ \ipa{i} / \%(C)(C)_\% when unstressed\\
V \change\ \ipa{o} / \%(C)(C)V_\% when unstressed \\
V \change\ \ipa{e} / \%(C)(C)_\%\ipa{r} when unstressed (with some exceptions)\\
V \change\ \{\ipa{i,u}\} / \%(C)(C)_\%P when unstressed\\
\ipa{a o} \change\ \ipa{e u} / \%(C)(C)_C(C)\% when unstressed\\
\ipa{a} \change\ \ipa{e} \change\ \ipa{i} / \%(C)(C)_\ipa{N} when unstressed\\
\ipa{a} \change\ \ipa{e} \change\ \ipa{u} / \%(C)(C)_\ipa{\textsuperimposetilde{l}} when unstressed\\
\ipa{e} \change\ \ipa{u} / \%(C)(C)_\ipa{\textsuperimposetilde{l}} when unstressed\\
\ipa{ai} \change\ \ipa{ei} \change\ \ipa{i:} / \%(C)(C)_ when unstressed\\
\ipa{ei oi ou} \change\ \ipa{i: e: u:} / \%(C)(C)_ when unstressed\\
\{\ipa{i,o}\} \change\ \ipa{e} / _\#\\
\{\ipa{i,e}\} \change\ \O\ / _\# (sometimes)\\
\ipa{a} \change\ \ipa{e} / _C(C)\#\\
\ipa{e} \change\ \ipa{i} / _\{\ipa{s,t}\}\#\\
\ipa{o} \change\ \ipa{u} / _C(C)\# ! \{\ipa{u,w}\}_\\
\{\ipa{ai,ei,oi}\} \change\ \ipa{ei} \change\ \ipa{i:} / _(C)(C)\#\\
V\ipa{:} \change\ V[-long] / _\{\ipa{m,(n)t,l,r}\}\#\\
V\ipa{:} \change\ V[-long] / _\#\\
\ipa{j} \change\ \ipa{i} / C_\\
\ipa{w} \change\ \ipa{u} / \ipa{t}_\\
\ipa{e} \change\ \ipa{o} / _\ipa{w}\\
\ipa{e} \change\ \ipa{o} / \ipa{w}_\\
\ipa{w} \change\ \O\ / \ipa{s}_\ipa{o}\\
\ipa{\s{m} \s{n}} \change\ \ipa{em en}\\
\ipa{\s{n}:} \change\ \ipa{n}\\
\ipa{\s{l} \s{l}: \s{r} \s{r}:} \change\ \ipa{ol l or r}\\
\ipa{b\super H} \change\ \ipa{h} / \#_ (in rustic dialects)\\
\{\ipa{b\super H,d\super H,g\super w\super H}\} \change\ \ipa{f} / \#_\\
\ipa{h} \change\ \O\ / \ipa{b}_\\
\ipa{t} \change\ \ipa{k} / _\ipa{l}\\
\ipa{t} \change\ \O\ / C_\#\\
\ipa{t} \change\ \ipa{d} / V_\\
\ipa{dw} \change\ \ipa{b}\\
\ipa{d} \change\ \O\ / V\ipa{:}_\#\\
\ipa{d} \change\ \O\ / C_\\
\ipa{d} \change\ \ipa{l} ``in many dialects''\\
\ipa{d\super H} \change\ \ipa{b} / \ipa{r}V_\\
\ipa{d\super H} \change\ \ipa{b} / _V\ipa{r}\\
\ipa{d\super H} \change\ \ipa{b} / _\ipa{l}\\
\ipa{d\super H} \change\ \ipa{b} / \ipa{u:}_\\
\ipa{d\super H} \change\ \ipa{d}\\
\ipa{\'{k} \'{g}} \change\ \ipa{k g}\\
\ipa{g\super H} \change\ \ipa{g} / \ipa{N}_\\
\ipa{g} \change\ \O\ / _\ipa{h}\\
\ipa{k\super w} \change\ \O\ / C_C\\
\ipa{k\super w} \change\ \ipa{k} / _\{\ipa{o,i},C\}\\
\ipa{g\super w(\super H)} \change\ \ipa{gu} / \ipa{N}_\\
\ipa{w} \change\ \O\ / \ipa{g}V_\{\ipa{l,r}\}\\
\ipa{g\super w} \change\ \ipa{v}\\
\ipa{g\super w\super H} \change\ \ipa{f} / _\ipa{r}\\
\ipa{g\super w\super H} \change\ \ipa{v} / V_V\\
\ipa{s} \change\ \ipa{z} \change\ \ipa{r} / V_V\\
\ipa{s} \change\ \ipa{T} \change\ \ipa{f} / \#_\ipa{r}\\
\ipa{s} \change\ \ipa{T} \change\ \ipa{b} / _\ipa{r}\\
\ipa{s} \change\ \ipa{z} / _C[+voiced]\\
V \change\ V\ipa{:} / _\ipa{z}C[+voiced]\\
\ipa{z} \change\ \O\ / _C[+voiced]\\
p\textellipsis k\super w \change\ k\super w\textellipsis k\super w\\
V$_1$\textellipsis V$_2$ \change\ V$_2$\textellipsis V$_2$ (rare)\\
V \change\ V\ipa{:} / _S[+voiced]\{S[-voiced],F[-voiced]\}; ``(i, e, and o sometimes bypass this)"\\
S[+voiced] \change\ S[-voiced] / _\{S[-voiced],F[-voiced]\}\\
S[-voiced] \change\ S[+voiced] / _N\\
\ipa{s} \change\ \ipa{z} / _\{N,\ipa{l,r}\}\\
V \change\ V\ipa{:} / _\ipa{z}\{\ipa{l,r}\}\\
\ipa{z} \change\ \O\ / _\{\ipa{l,r}\}\\
S \change\ \ipa{f} / _\ipa{f}\\
\{\ipa{t,d}\} \change\ \ipa{s} / _\ipa{s}\\
\{\ipa{p,b}\} \{\ipa{t,d}\} \change\ \ipa{m n} / _\{\ipa{m,n}\}\\
\{\ipa{k,g}\} \change\ \ipa{N} / _\ipa{n}\\
\ipa{m:} \change\ \ipa{n} / \{W,V\ipa{:}\}_\\
\{\ipa{d,n,r}\} \change\ \ipa{l} / _\ipa{l}\\
\ipa{n} \change\ \ipa{r} / _\ipa{r}\\
\ipa{s} \change\ \ipa{z} \change\ \ipa{l} / \ipa{l}_\\
\ipa{s} \change\ \ipa{z} \change\ \ipa{r} / \ipa{r}_\\
\ipa{n} \change\ \ipa{l} / \ipa{l}_\\
V \change\ V\ipa{:} / _\{\ipa{t,d}\}\ipa{t}\\
\{\ipa{t,d}\}\ipa{t} \change\ \ipa{tst} \change\ \ipa{s:}\\
\O\ \change\ \ipa{t} / \ipa{s:}_\ipa{r}\\
\O\ \change\ \ipa{p} / \ipa{m}_\{\ipa{s,t,l}\}\\
\ipa{s:} \change\ \ipa{s} / _\#\\
\ipa{s:} \change\ \ipa{s} / \{W,V\ipa{:}\}_\\
\ipa{l} \change\ \ipa{r} \change\ _V\ipa{l}\\
\ipa{l} \change\ \ipa{r} / \ipa{l}V_ ``(in suffixes with l if root already has l)"\\
\ipa{r}\textellipsis \ipa{r} \change\ \ipa{r}\textellipsis\O\\
\{\ipa{n,d}\}\textellipsis\ipa{r} \change\ \ipa{r}\textellipsis\ipa{r}\\
V\ipa{:} \change\ V / _C(C)\# ``(irregular: often before -m, -t, -nt, but never before ?s)"\\
V \change\ \{V\ipa{:},V[+nas]\} / _\ipa{n}\{\ipa{f,s}\}\\
\ipa{n} \change\ \O\ / V[+nas]_\\
C$_1$C$_2$C$_3$C$_4$ \change\ (C$_3$)C$_4$\\
C$_1$C$_2$C$_3$ \change\ C$_1$C$_3$

\subparagraph{Classical Latin vs. Vulgar Latin}

\tab ``The following relate to the changes of vowels as found in the evolution to the written medieval languages of Iberia, Gallia and Italia (Anglo-Norman, Old Spanish, etc.). The Latin of Africa, Sardinia and the easternmost parts of the Empire exhibited different mergers."

\{\ipa{e,i}\} \change\ \ipa{j} / C_V when unstressed\\
\{\ipa{e,ai}\} \change\ \ipa{E}\\
\{\ipa{i,e:,oi}\} \change\ \ipa{e}\\
\ipa{i:} \change\ \ipa{i}\\
\ipa{o} \change\ \ipa{O}\\
\{\ipa{u,o:}\} \change\ \ipa{o}\\
\ipa{u:} \change\ \ipa{u}\\
\ipa{a:} \change\ \ipa{a}\\
\ipa{m} \change\ \ipa{n} / _\# ``(in certain common monosyllabic words, as well as some common compounds of them)"\\
\ipa{m} \change\ \O\ / _\#\\
\ipa{h} \change\ \O\\
\ipa{w} \change\ \ipa{B}\\
\ipa{E O} \change\ \ipa{e o} / when unstressed\\
\ipa{j} \change\ \ipa{J} / \#_V\\
\ipa{j} \change\ \ipa{J:} / V_V

\tab ``In contrast, Romanian exhibits u, u\ipa{:} \change\ u (and ultimately also \ipa{O}, o\ipa{:} \change\ o); and Sardinian and African Latin underwent a straight merger of the vowels by length without considering quality (e, e\ipa{:} \change\ e; i, i\ipa{:} \change\ i; u, u\ipa{:} \change\ u; etc.)"

\subparagraph{Latin to Catalan}{\it Mecislau}

\tab {\it NB: Due to problems when the board migrated to a different system, a lot of the special characters were replaced with $\langle$?$\rangle$. In many cases these have been replaced with $\langle$\O$\rangle$ because it was likely that this was what was meant, but conditional $\langle$?$\rangle$ has either been left alone or attempted to have been filled in from context. In some cases, conditional $\langle$?$\rangle$ may have been used to mark stress or syllable boundaries. Take such changes with a grain of salt and use at your own risk.}

\ipa{h} \change\ \O\\
\ipa{n} \change\ \O\ / _\ipa{s}\\
\{\ipa{m,n,t}\} \change\ \O\ / _\#\\
V \change\ \O\ / "V\%C_L(C)V(C)\# (irregular)\\
V \change\ \O\ / "V\%L_C(C)V(C)\# (irregular)\\
V \change\ \O\ / "V\%\ipa{s}_\ipa{t}(C)V(C)\# (irregular)\\
\ipa{u} \change\ \ipa{w} \change\ \O\ / (``when in unstressed penult or between first and tonic syllables; irregular")\\
\ipa{i:} \change\ \ipa{i} / stressed\\
\ipa{i:} \change\ \ipa{i} / _\%"V\\
\{\ipa{i,e:}\} \ipa{e} \change\ \{\ipa{e,E}\} \{\ipa{E,e}\} / stressed\\
\{\ipa{i,e:,e}\} \change\ \ipa{e} / _\%"V\\
\ipa{i} \change\ \ipa{j} / "V_\#\\
\ipa{u:} \change\ \ipa{u} / stressed\\
\ipa{u:} \change\ \ipa{u} / _\%"V\\
\ipa{au} \change\ \ipa{a} / _\%"\ipa{u}\\
\{\ipa{u,o:}\} \ipa{o} \change\ \ipa{o O} / stressed\\
\{\ipa{u,o(:)}\} \change\ \ipa{u} / _\%"V in East Catalan\\
\{\ipa{u,o(:)}\} \change\ \ipa{o} / _\%"V else\\
\ipa{u} \change\ \ipa{w} / "\ipa{E}_\#\\
\ipa{a:} \change\ \ipa{a}\\
\ipa{oe} \change\ \{\ipa{e,E}\}\\
\ipa{ae au} \change\ \ipa{e O} / stressed\\
\ipa{ae au} \change\ \ipa{e o} / _\%"V\\
\ipa{o} \change\ \ipa{u} / _\ipa{a}\\
\ipa{o} \change\ \ipa{u} / _\%"V (irregular)\\
VV \change\ V\ipa{:} (``For outcomes of word-final vowels, see down below")\\
\ipa{ndj} \change\ \ipa{\textltailn}\\
\ipa{dj} \change\ \ipa{dZ} \change\ \ipa{Z}\\
\O\ \change\ \ipa{e} / \#_\ipa{s}C\\
\ipa{l} \change\ \O\ / \{\ipa{o,u}\}_CV\\
\ipa{l} \change\ \ipa{w} / V_CV (``although l was usually restored later'')\\
\ipa{mn} \change\ \ipa{n:} \change\ \ipa{\textltailn}\\
\ipa{p b t d k g} \change\ \ipa{B} \{\ipa{B,w}\} \O\ \{\ipa{j,w}\} \ipa{G} \{\ipa{j},\O,\ipa{g}\} / V_\ipa{r}V\\
\ipa{N} \change\ \ipa{\textltailn} / _\{\ipa{i,e}\}\\
\ipa{p b} \change\ \ipa{b w} / V_\ipa{l}V (the latter is irregular)\\
\{\ipa{kl,gl}\} \change\ \ipa{L} / V_V (the latter is irregular)\\
\ipa{sk} \change\ \ipa{S} / V_\{\ipa{i,e}\}\\
\ipa{p k} \change\ \O\ \ipa{j} / V_\ipa{t}V\\
\ipa{k} \change\ \O\ / V\ipa{n}_\ipa{t}V\\
\ipa{ks} \change\ \ipa{S} / V_V\\
\ipa{k} \change\ \ipa{j} / _\ipa{s}\#\\
\ipa{gn tj} \change\ \ipa{\textltailn} \O\ / V_V\\
\ipa{stj} \change\ \ipa{S}\\
\ipa{tj} \change\ \ipa{s} / C_\\
\ipa{sj ssj jn} \change\ \ipa{js jS \textltailn} / V_V\\
\ipa{mnj} \change\ \{\ipa{mni,\textltailn}\} / V_V\\
\ipa{lj rj kj gj} \change\ \ipa{L jr ts Z} / V_V\\
\{\ipa{bj,vj}\} \change\ \ipa{wZ} / _\%"V\\
\ipa{b} \change\ \ipa{v} / \"V\%_\ipa{j}\\
\ipa{ja} \change\ \ipa{je} / \#_ (irregular)\\
V \change\ \O\ / _\%"V (rare)\\
\O\ \change\ \{\ipa{e,o}\} / CL_\#\\
\O\ \change\ \{\ipa{e,o}\} / \ipa{r:}_\#\\
\ipa{a} \change\ \ipa{e} / ``in the penult"\\
V \change\ \O\ / "V\%_(C)(C)V(C)\# (``irregular; e is kept before n")\\
\ipa{b} \change\ \ipa{v} / V_V\\
\ipa{p t} \change\ \ipa{b d} / V_V\\
\ipa{f} \change\ \ipa{v} / V_V (irregular)\\
\ipa{s} \change\ \ipa{z} / "V\%V_V\\
\ipa{s} \change\ \O\ / V_V\%"V\\
\ipa{k g} \change\ \O\ \{\O,\ipa{Z}\} / V_\{\ipa{i,e}\} (\ipa{g} \change\ \ipa{Z} is learned)\\
\ipa{g} \change\ \O\ / V_V\%"V\\
\ipa{k j} \change\ \ipa{g Z} / V_V\\
``These next two changes are awkward - Basically, when the final vowel drops off down below, the newly-final d should become w; BUT d should also have become z and disappeared before the final vowels drop off, leaving a dilem[m]a... I'm not certain how this should be [interpreted]"\\
--- \ipa{d} \change\ \ipa{z} \change\ \O\ / V_V\\
--- \ipa{d} \change\ \ipa{w} / _V\#\\
\ipa{i:} \change\ \O\ / _\#\\
\{\ipa{i,e(:),ae}\} \change\ \O\ / _(C)\#\\
\{\ipa{u(:),o(:)}\} \change\ \O\ / _\#\\
V \change\ \O\ / ``between first and tonic syllables; except when C_CC, _n"; ``if there are multiple vowels between the initial and tonic syllables, the vowel directly before the tonic is usually dropped" ! V = \ipa{a}\\
\ipa{w} \change\ \O\ / \ipa{u}_\#\\
\ipa{j gj ts z n} \change\ \ipa{tS i w s} \O\ / _\#\\
\ipa{t} \change\ \O\ / V_\ipa{s}V\\
\{\ipa{b,v}\} \change\ \ipa{w} / V_\#\\
\ipa{d} \change\ \ipa{t} / _\#\\
\ipa{l} \change\ \ipa{L} / \#_\\
\ipa{k} \change\ \ipa{ts} \change\ \ipa{s} / \#_\{\ipa{i,e}\}\\
\ipa{g} \change\ \ipa{Z} / \#_\{\ipa{i,e}\}\\
\ipa{j} \change\ \ipa{dZ} \change\ \ipa{Z} / \#_\\
\ipa{k\super w g\super w} \change\ \ipa{k g} / \#_\{\ipa{i,e}\}\\
\ipa{k\super w} \change\ \ipa{k} / C_V\\
\ipa{k\super w} \change\ \ipa{g} / V_\{\ipa{i,e}\}\\
\ipa{k\super w} \change\ \ipa{k} / \#_\ipa{a}\%"V\\
\ipa{k\super w} \change\ \ipa{gw} / V_\ipa{a}\\
\ipa{k\super w} \change\ \ipa{kw} / \#_"\ipa{a}\\
\ipa{g\super w} \change\ \ipa{gw} / \#_\ipa{a}\\
\ipa{g\super w} \change\ \ipa{g} / C_\{\ipa{i,e}\}\\
\ipa{g\super w} \change\ \ipa{gw} / C_\ipa{a}\\
\ipa{b} \change\ \ipa{m} \change\ \O\ / V\ipa{m}_V\\
\ipa{n} \change\ \ipa{r} \change\ \ipa{br} / \ipa{m}_\\
\ipa{k} \change\ \ipa{w} / V_\ipa{r}V\\
\{\ipa{b,v}\} \change\ \ipa{w} / V_\ipa{t}V\\
\ipa{g} \change\ \O\ / V_\ipa{d}V\\
\ipa{l: n:} \change\ \ipa{L \textltailn}\\
\ipa{L} \change\ \ipa{l} / "\ipa{i}_\\
C \change\ \O\ / C$_1$_C$_2$ ! C$_2$ = L\\
\{\ipa{a,o}\} \change\ \O\ / \#_ (rare)\\
\ipa{O} \change\ \ipa{o} / _N\$C\\
\ipa{e} \change\ \ipa{E} / _\ipa{v}\\
\ipa{o} \change\ \ipa{u} / _\{\ipa{\textltailn,nk,N}\} when stressed\\
\ipa{e} \change\ \ipa{i} / _\{\ipa{nk,N}\} when stressed (irregular)\\
\ipa{aj} \change\ \ipa{ej} \change\ \ipa{ee} \change\ \ipa{e} (irregular)\\
\ipa{aj} \change\ \ipa{ej} / _\ipa{S} when stressed (irregular)\\
\ipa{Ej Oj} \change\ \ipa{jEj uei} / \ipa{i} \{\ipa{u,ui}\} / stressed\\
\ipa{E} \change\ \ipa{e} / ! _\{\ipa{r:,l,r}C[-labial],\ipa{nr}\} or _ ? \ipa{w}\#\\
\ipa{e} \change\ \ipa{E} (in Eastern Catalan)

\subparagraph{Latin to French}{\it pharazon}

\tab {\it NB: The vowels here marked $\langle$\'{o}$\rangle$ and $\langle$\`{o}$\rangle$ seem to have had some sort of open-close distinction similar to /\ipa{o O}/.}

Vulgar Latin:\\
--- \ipa{h} \change\ \O\\
--- V$_0$V$_0$ \change\ V$_0$\ipa{:}\\
--- \ipa{n} \change\ \O\ / _\{\ipa{f,v,s}\}\\
--- \ipa{r} \change\ \ipa{s} / _\ipa{s}\\
--- \{\ipa{m,n}\} \change\ \O\ / _\# in polysyllables\\
--- \ipa{m} \change\ \ipa{n} / _\#\\
--- \ipa{u} \change\ \O\ / CC_V\\
--- \ipa{w} \change\ \ipa{gu} / ``from Germanic loanwords"\\
--- V \change\ "V / "VS\ipa{r}_\\
--- V \change\ "V / _C*"\{\ipa{i,e}\}V\\
--- \{\ipa{i,e}\} \change\ \ipa{j} / _V

Stressed vowels:\\
--- \ipa{a:} \change\ \ipa{a}\\
--- \ipa{(a)e} \change\ \ipa{\`{e}}\\
--- \{\ipa{e:,i,oe}\} \change\ \ipa{\'{e}}\\
--- \ipa{i: o:} \change\ \ipa{i \'{o}}\\
--- \ipa{o} \change\ \ipa{\`{o}}\\
--- \ipa{u} \change\ \ipa{\'{o}} / !_\ipa{i:}\\
--- \ipa{u:} \change\ \ipa{u}

Initial vowels (first vowel of a word):\\
--- \ipa{a:} \change\ \ipa{a}\\
--- \{\ipa{e(:),i,ae,oe}\} \change\ \ipa{e}\\
--- \ipa{i:} \change\ \ipa{i}\\
--- \{\ipa{o(:),u}\} \change\ \ipa{o}

Final vowels:\\
--- \ipa{a:} \change\ \ipa{a}\\
--- \{\ipa{e(:),i,ae,oe}\} \change\ \ipa{e}\\
--- \ipa{i: o:} \change\ \ipa{i o}\\
--- \ipa{u(:)} \change\ \ipa{o} / except _V (?)

\ipa{k g} \change\ \ipa{tj dj} / _E\\
\ipa{\`{e}} \change\ \ipa{iE} / in U[+open]\\
\ipa{\`{e}} \change\ \ipa{iE} / _C\#\\
\ipa{\`{e}} \change\ \ipa{E} / in U[+closed]\\
\ipa{\`{o}} \change\ \ipa{uo} \change\ \ipa{uE} / in U[+open] ! _N\\
\ipa{\`{o}} \change\ \ipa{O} / in U[+closed]\\
\ipa{dj} \change\ \ipa{dZ} / \ipa{r}_\\
\ipa{d} \change\ \O\ / _\ipa{j}\\
\ipa{j} \change\ \O\ / V_"E\\
\ipa{j tj} \change\ \ipa{dZ ts} / \#_\\
\ipa{j} \change\ \ipa{dZ} / V_V (rare)\\
\O\ \change\ \ipa{s} / \ipa{t}_\ipa{j}\\
\ipa{t} \change\ \ipa{s} / \ipa{s}_\ipa{j}\\
\{\ipa{gn,nj}\} \change\ \ipa{\textltailn}\\
\ipa{nk} \change\ \ipa{\textltailn} / _\ipa{t}\\
V \change\ \O\ / in the unstressed penult\\
V \change\ \O\ / intertonic ! V = \ipa{a}\\
\ipa{a} \change\ \ipa{@} / intertonic\\
\O\ \change\ \ipa{b} / \ipa{m}_\{\ipa{r,l}\}\\
\O\ \change\ \ipa{d} / \{\ipa{n,l,\textltailn,z}_\ipa{r}\\
\O\ \change\ \ipa{t} / \ipa{s}_\ipa{r}\\
\ipa{k g} \change\ \ipa{t d} / \{\ipa{n,r}\}_\ipa{r}\\
\ipa{n} \change\ \ipa{r} / \{\ipa{g,p}\}_\\
``[T]wo obstruents in contact with different voicing assimilate to the voicing of the second"\\
C \change\ \O\ / C$_1$_C$_2$ ! C$_2$ = \{\ipa{r,l}\}\\
\ipa{t} \change\ \ipa{s} / _\{\ipa{n,m}\}\\
\{\ipa{kl,gl,lj}\} \change\ \ipa{L}\\
\{\ipa{p,b}\} \{\ipa{t,d}\} \change\ \ipa{v D} / V_\{V,\ipa{r}\}\\
\ipa{v} \change\ \O\ / V_B\\
\ipa{p} \change\ \ipa{b} / _\ipa{l}\\
\ipa{D} \change\ \O\ / _\ipa{r}\\
\ipa{(t)s} \change\ \ipa{(d)z} / V_V\\
\ipa{k} \change\ \ipa{js} / V_\ipa{s}V\\
\ipa{k} \change\ \ipa{j} / _\ipa{s}\#\\
\{\ipa{k,g}\} \change\ \O\ / V_B\\
\{\ipa{k,g}\} \change\ \O\ / B_\ipa{a}\\
\{\ipa{k,g}\} \change\ \ipa{j} / _\{\ipa{a},C\}\\
\ipa{k\super w} \change\ \{\ipa{v,u}\} / V_E\\
\ipa{k\super w} \change\ \ipa{j}\{\ipa{v,u}\} / V_\ipa{a}\\
``[N]ote that the [following] clusters are the only case where a consonant does not receive intervocalic treatment before /j/":\\
--- \ipa{(k)kj} \change\ \ipa{ts}\\
--- \ipa{g} \change\ \O\ / _\ipa{j}\\
--- \ipa{pj} \change\ \ipa{tS}\\
--- \{\ipa{b,v}\}\ipa{j} \change\ \ipa{dZ}\\
--- \ipa{m}\{\ipa{\textltailn,j}\} \change\ \ipa{ndZ}\\
V"\ipa{e} \change\ "V\ipa{i}\\
V$_0$V$_0$ \change\ V$_0$\\
\O\ \change\ \ipa{e} / \#_\ipa{s}C\\
\ipa{k g} \change\ \ipa{tS dZ} / _\ipa{a}\\
\ipa{t} \change\ \O\ / \{\ipa{S,s}\}\\
\ipa{d} \change\ \O\ / _\{\ipa{z,Z}\}\\
\ipa{E O} \change\ \ipa{iE uE} / _\{C\ipa{j,j}C\}\\
\O\ \change\ \ipa{j} / \{\ipa{S,Z,sj,zj}\}"\{\ipa{a,\'{e}}\}_ in U[+open]\\
\ipa{s:j zj rj} \change\ \ipa{js: jz jr}\\
\ipa{j} \change\ \O\ / \ipa{s}_ (\ipa{s:}_?)\\
\ipa{E} \change\ \ipa{Ea} / _\ipa{l}\{C,\#\}\\
\ipa{l} \change\ \ipa{u} / _\{C,\#\}\\
\ipa{l} \change\ \O\ / \{\ipa{i,u}\}_\\
\{\ipa{l:e,l:o}\} \change\ \ipa{u} / \{\ipa{e,o}\}_\# ``[this is actually an analogical development, but it applies as regularly as a sound law]"\\
\ipa{(E)au} \change\ \ipa{O}\\
\ipa{\'{e}} \change\ \ipa{Ei} / in U[+open]\\
\ipa{\'{e}} \change\ \ipa{E} / in U[+closed]\\
\ipa{\'{o}} \change\ \ipa{ou} \change\ \ipa{Eu} / in U[+open]\\
\ipa{\'{o}} \change\ \ipa{O} / _N\\
\ipa{\'{o}} \change\ \ipa{ou} / in U[+closed]\\
\ipa{e} \change\ \ipa{@} / \#(C\textellipsis)_(\%\textellipsis)" in U[+open]\\
\ipa{e} \change\ \ipa{E} / \#(C\textellipsis)_(\%\textellipsis)" in U[+closed] or _V (?)\\
\ipa{o} \change\ \{\ipa{ou,O}\} ``(the outcome fluctuates, but \ipa{O} is often the result of analogy rather than strict sound change; always ou before another vowel)"\\
\ipa{a} \change\ \ipa{@} / \#\{\ipa{tS,dZ}\}_(\%\textellipsis)" in U[+open]\\
\ipa{a} \change\ \ipa{E} / in U[+open] ``(but a following \ipa{L} creates a [closed] syllable)"\\
\ipa{k\super w g\super w} \change\ \ipa{k g}\\
C$_0$C$_0$ \change\ C$_0$\\
\ipa{t} \change\ \O\ / V_\#\\
\ipa{E} \change\ \ipa{i} / _C(C\textellipsis)\ipa{i}\#\\
V \change\ \O\ / _\# ``(except in monosyllables or after another vowel)" ! V = \ipa{a}\\
\ipa{a} \change\ \ipa{@}\\
V \change\ \ipa{@} / _\{CC,\ipa{tS,dZ}\} ! _\{\ipa{nt,ng,mp,rt,rd}\}\\
\ipa{s} \change\ \O\ / _C\\
\{\ipa{p,b}\} \change\ \O\ / _\{\ipa{t,d}\}\\
\ipa{v} \change\ \O\ / _C\\
\ipa{v} \change\ \O\ / C_\\
\ipa{D} \change\ \O\\
\ipa{uE} \change\ \ipa{Eu}\\
\ipa{ai} \change\ \ipa{e} / _\#\\
\ipa{ai iEI} \change\ \ipa{E i}\\
\ipa{ou Eu u uEi} \change\ \ipa{u \oe\ y yi}\\
\{\ipa{ei,Oi}\} \change\ \ipa{oi} / C[-nas]\\
\ipa{O} \change\ \ipa{u} / _"V\\
V[-high] \change\ \ipa{@} \change\ \O\ / _V ``(except that a is kept before o)"\\
\ipa{\textltailn} \change\ \ipa{in} / _\{C,\#\}\\
V\{\ipa{n,m}\} / V[+nas] / _\{C,\#\}\\
\ipa{\~{E}} \change\ \ipa{\~{a}}\\
\{\ipa{a\~{\i},e\~{\i}}\} \change\ \ipa{\~{E}}\\
\ipa{\~{y}} \change\ \ipa{\~{\oe}}\\
O[+voiced] \change\ O[-voiced] / _\#\\
\{\ipa{t,s}\} \change\ \O\ / _\#\\
\ipa{k} \change\ \O\ / V[+nas]_\#\\
\{\ipa{n,m}\} \change\ \O\ / C_\#\\
\ipa{j} \change\ \O\ / \{\ipa{S,Z}\}_V[-nas]\\
\ipa{L r} \change\ \ipa{j K}\\
\ipa{oi} \change\ \ipa{wE} \change\ \ipa{wa}\\
\ipa{o\~{\i}} \change\ \ipa{w\~{E}}\\
``([pharazon has] omitted the loss of \ipa{@} in various contexts, since it often resurfaces)"

\subparagraph{Vulgar Latin to Italian}{\it Dewrad}, from Boyd-Bownam, P. {\it From Latin to Romance in Sound Charts}

\tab {\it NB: Dewrad says, ``It should be noted that due to my source they are not in any sort of chronological order, nor do they indicate some of the more sporadic changes."}

\ipa{r} \change\ \O\ / \ipa{a}_\ipa{ju}\#\\
\ipa{t}V\ipa{k} \change\ \ipa{dZ} / unstressed\\
\ipa{au} \change\ \ipa{u} / \#_ (sporadically, e.g. audire \change\ udire)\\
\ipa{au} \change\ \ipa{o}\\
\ipa{k g} \change\ \ipa{tS dZ} / _E\\
\ipa{k}V\ipa{l} \change\ \ipa{k:j} / unstressed\\
\ipa{kt} \change\ \ipa{t:}\\
\ipa{E} \change\ \ipa{jE} / unstressed ! _\{\ipa{dZ,L,\textltailn}\}\\
\ipa{g} \change\ \O\ / \ipa{a}_V\\
\ipa{j} \change\ \ipa{dZ} / \#_\\
\ipa{j} \change\ \ipa{dZ} / V_V\\
\{\ipa{dj,gj}\} \ipa{lj} \{\ipa{nj,gn}\} \change\ \ipa{dZ L \textltailn}\\
\ipa{O} \change\ \ipa{uo} / stressed ! \ipa{j}_ or _\{\ipa{dZ},L\}\\
\ipa{b} \change\ \ipa{v} / V_\\
\ipa{l} \change\ \ipa{j} / \#C_\\
C \change\ C\ipa{:} / V_\ipa{j}V\\
\ipa{sj} \change\ \ipa{dZ}\\
\{\ipa{t,d,k,m,n,s}\} \change\ \O\ / _\#\\
\ipa{r} \change\ \O\ / _\# (in polysyllables only)\\
\ipa{ta:te} \change\ "\ipa{ta} / _\#\\
\ipa{t k} \change\ \ipa{d g} / V_\ipa{r}\\
\{\ipa{skj,stj,s:j} \change\ \ipa{S}\\
\ipa{tj ks w} \change\ \ipa{ts s: gw}

\subparagraph{Latin to Portuguese}{\it Mecislau}

\tab {\it NB: Due to problems when the board migrated to a different system, a lot of the special characters were replaced with $\langle$?$\rangle$. In many cases these have been replaced with $\langle$\O$\rangle$ because it was likely that this was what was meant, but conditional $\langle$?$\rangle$ has either been left alone or attempted to have been filled in from context. In some cases, conditional $\langle$?$\rangle$ may have been used to mark stress or syllable boundaries. Take such changes with a grain of salt and use at your own risk. Further, Mecislau gives some dual-output changes, which distinguish between vulgar and ``semi-learned'' outcomes.}

\ipa{h} \change\ \O\\
\ipa{rs} \change\ \ipa{s:}\\
\ipa{n} \change\ \O\ / _\ipa{s}\\
V$_0$V$_0$ \change\ V$_0$\ipa{:}\\
V \change\ \O\ / "V\%L(C)(C)V(C)\# (irregular)\\
V \change\ \O\ / _L(C)(C)V(C)\# (irregular)\\
V \change\ \O\ "V\%\ipa{s}_\ipa{t}(C)V(C)\# (irregular)\\
\ipa{u} \change\ \ipa{w} / _V (between first and stressed syllables)\\
\ipa{w} \change\ \ipa{u} / _"V\\
\ipa{w} \change\ \O\ / _V\\
\ipa{au} \change\ \ipa{a} / _\%"\ipa{u}\\
\ipa{au} \change\ \ipa{o}\\
\ipa{e} \change\ \ipa{i:} / "_\%\ipa{I:}\#\\
\ipa{i:} \{\ipa{i,e:}\} \ipa{e} \change\ \ipa{i e E} / stressed\\
\ipa{i:} \{\ipa{i,e(:)}\} \change\ \ipa{i e} / _\%"V\\
\ipa{i:} \change\ \O\ / \{\ipa{k,s}\}_\#\\
\{\ipa{i(:),e(:),ae}\} \change\ \ipa{e} / _\#\\
\ipa{u:} \change\ \ipa{u}\\
\ipa{ui:} \change\ \ipa{ui} / _\#\\
\{\ipa{u,o:}\} \ipa{o} \change\ \ipa{o O} / stressed\\
\{\ipa{u,o(:)}\} \change\ \ipa{o} / _\%"V\\
\{\ipa{u(:),o(:)}\} \change\ \ipa{o} \change\ \ipa{u} / _\#\\
\ipa{a: oe} \change\ \ipa{a e}\\
\ipa{ae} \change\ \ipa{E} / stressed\\
\ipa{E O} \change\ \ipa{e o} / _("\ipa{u})\#\\
\{\ipa{olt,okt}\} \change\ \ipa{ujt} \change\ \ipa{ut}\\
\ipa{a\textsuperimposetilde{l}} \change\ \ipa{o}\\
\ipa{\textsuperimposetilde{l}} \change\ \ipa{w} / V_C\ipa{a}\\
\ipa{o} \change\ \ipa{u} / _("V)\\
\ipa{e} \change\ \O\ / \ipa{el}_\#\\
V \change\ \ipa{a} / _\{\ipa{n,r}\}(C)V(C)\# (irregular)\\
V \change\ V[+nas] / _N\$C when stressed\\
\ipa{\~{O}} \change\ \ipa{\~{o}}\\
N \change\ \O\ / V[+nas]_\$C ! C = S\\
V \change\ V[+nas] / _N\$V\\
V \change\ V[+nas] / \#N_ (rare)\\
N \change\ \O\ / V[+nas]_\$V\\
\{\ipa{\~{a},\~{a}e,\~{o}e}\} \change\ \ipa{\~{a}o} / _\#\\
V[+nas] \change\ V[-nas] / unstressed\\
V[+nas] \change\ V[-nas] / in U\#\\
V$_0$[+nas]V$_0$[-nas] \change\ V$_0$[+nas]\\
\ipa{\~{\i}} \change\ \ipa{i\textltailn}\\
\ipa{e} V \change\ \ipa{o} \O\ / _? (irregular)\\
\ipa{e} \change\ \ipa{o} / _\ipa{m}"V (irregular)\\
V \change\ \O\ / _"V (irregular)\\
\ipa{e} \change\ \O\ / \{\ipa{l,n,r,s,k}\}_\#\\
\ipa{e} \change\ \O\ / "\{\ipa{i,e}\}_\#\\
\ipa{e} \change\ \ipa{i} / _(C)(C)V(C)\#\\
\{\ipa{e,i}\} \change\ \O\ / \{\ipa{l,m,r}\}_ when between \#U and U[+stress]\\
\{\ipa{e,i}\} \change\ \O\ / \ipa{k}_\ipa{t} when between \#U and U[+stress]\\
\ipa{o} \change\ \O\ / _\{\ipa{r,l}\} when between \#U and U[+tonic]\\
``[I]f there are multiple vowels between the initial and tonic syllables, the vowel directly before the tonic is dropped"\\
\ipa{k} \change\ \ipa{ts} \change\ \ipa{s} / \#_\{\ipa{i,e}\}\\
\ipa{k} \change\ \ipa{g} / \#_\{\ipa{a,r}\} (rare)\\
\ipa{g} \change\ \ipa{g\super j} \change\ \ipa{d\super j} \change\ \ipa{dZ} \change\ \ipa{Z} / \#_\{\ipa{i,e}\}\\
\ipa{j} \change\ \ipa{dZ} \change\ \ipa{Z} / \#_\\
\ipa{pl} \change\ \{\ipa{S,pr}\} / \#_\\
\ipa{l} \change\ \ipa{r} / \ipa{b}_\\
\ipa{fl} \change\ \{\ipa{S,fr}\} / \#_\\
\{\ipa{fl,skl}\} \change\ \ipa{S}\\
\ipa{ngi} \change\ \ipa{\textltailn}\\
\ipa{s} \change\ \ipa{S} / V_C[-voiced]V\\
\ipa{s} \change\ \ipa{Z} / V_C[+voiced]V\\
\ipa{kl} \change\ \ipa{kL} \change\ \ipa{tS} \change\ \ipa{S} / \#_\\
\{\ipa{kl,gl}\} \change\ \ipa{L}\\
\ipa{g} \change\ \O\ / \#_\ipa{l}\\
\ipa{k\super w} \change\ \ipa{kw} / \#_"\ipa{a}\\
\ipa{k\super w g\super w} \change\ \ipa{k g} / \#_\{\ipa{i,e,o}\}\\
\ipa{k\super w} \change\ \ipa{g} / V_\{\ipa{i,e}\}\\
\ipa{k\super w} \change\ \ipa{k} / VC_\{\ipa{a,i,e}\}\\
\ipa{k\super w} \change\ \ipa{gw} / V_\ipa{a}
\ipa{g\super w} \change\ \ipa{gw} / \#_\\
\ipa{g\super w} \change\ \ipa{gw} / C_\ipa{a}\\
\ipa{b} \change\ \ipa{v} / V_\{V,\ipa{r}\}\\
\ipa{d} \change\ \O\ / V_V\\
\ipa{g} \change\ \{\O,\ipa{Z}\} / V_\{\ipa{i,e}\} (\ipa{g} \change\ \ipa{Z} is learned)\\
\ipa{g} \change\ \ipa{j} / V_\ipa{r}\\
\ipa{pl bl p t} \change\ \ipa{br} \{\ipa{br,l}\} \ipa{b d} / V_V (\ipa{bl} \change\ \ipa{l} is learned)\\
\ipa{p t k} \change\ \ipa{b d g} / V_\ipa{r}\\
\ipa{p} \change\ \O\ / V_\{\ipa{t,s}\}V\\
\ipa{k} \change\ \ipa{j} / V_\ipa{t}V\\
\ipa{k} \change\ \O\ / V\ipa{n}_\ipa{t}V\\
\ipa{ks gn tj} \change\ \ipa{S \textltailn} \{\ipa{z,s}\} / V_V (\ipa{tj} \change\ \ipa{s} is learned)\\
\ipa{tj} \change\ \ipa{s} / C_V\\
\ipa{dj} \change\ \ipa{dZ} \change\ \ipa{Z} / V_V\\
\ipa{dj} \change\ \ipa{dz} \change\ \ipa{ts} \change\ \ipa{s} / \ipa{r}_V\\
\ipa{ndj} \change\ \ipa{nts} \change\ \ipa{ns} \change\ \ipa{\textltailn} / V_V\\
\ipa{sj} \change\ \ipa{jZ} / V_V\\
\ipa{j} \change\ \O\ / \ipa{i}_\ipa{Z}V\\
\ipa{s:j (m)nj lj rj} \change\ \ipa{jS \textltailn L jr} / V_V\\
\ipa{kj} \change\ \ipa{ts} \change\ \ipa{s} / V_V\\
\ipa{gj} \change\ \{\ipa{Z,j}\} / V_V (\ipa{gj} \change\ \ipa{j} is learned)\\
\ipa{pj} \{\ipa{bj,vj}\} \change\ \ipa{jb jv} / V_V\\
\ipa{mj} \change\ \ipa{jm} (irregular)\\
C$_0$C$_0$ \change\ C$_0$ / ! C = \ipa{r}\\
C \change\ \O\ / C$_1$_C$_2$ ! C$_2$ = L\\
\ipa{k} \change\ \ipa{j} / _\ipa{s}\#\\
\ipa{f} \change\ \ipa{v} / V_V (irregular)\\
\ipa{s l} \change\ \ipa{z} \O\ / V_V\\
\ipa{a} \change\ \O\ / "\ipa{O}_\#\\
\ipa{sk k} \change\ \ipa{jS z} / V_\{\ipa{i,e}\}\\
\ipa{k} \change\ \ipa{g} / V$_1$_V$_2$ ! V$_2$ = \ipa{O}\\
\ipa{j} \change\ \{\O,\ipa{Z}\} / V_V (\ipa{j} \change\ \ipa{Z} is learned)\\
\ipa{b} \change\ \ipa{v} / VL_V\\
\ipa{m} \change\ \O\ / _\ipa{n}\\
\{\ipa{e,i}\} \change\ \O\ / \{L,N\}_(C)(C)V(C)\#\\
\ipa{o} \change\ \O\ / _(C)(C)V(C)\#\\
\ipa{e} \change\ \ipa{j} / \{\ipa{a,o,u}\}_\\
\ipa{a} \change\ \ipa{e} / _\ipa{j} when stressed\\
\ipa{o} \change\ \ipa{u} / _\ipa{\textltailn} when stressed\\
V$_0$V$_0$ \change\ V$_0$ (irregular)\\
\ipa{d} \change\ \O\ / V_V (in Portugal)

\subparagraph{Vulgar Latin to Old Proven\c{c}al}{\it Pogostick Man}, from Grandgent, Charles Hall (1905), {\it An outline of the phonology and morphology of old Proven\c{c}al}, Revised Edition

\tab {\it NB: Use at your own peril. Trying to put a chronology to this is sort of like what I imagine undergoing a root canal would be like, as is figuring out the conditioning on a lot of these things because of the convention Grandgent uses. Nevertheless, I have tried�and probably largely failed. In any case the sections regarding the development of the vowels are placed first, because the source does that and other Romance changes posted here do similarly, and the grouping of the consonants is in large part informed by the surce. Also, I use \% here to denote a syllable boundary because I didn't want to have to open yet another window so I could throw a sigma into my document.}

Stress shift: Secondary stress shifts to two syllables away from the penult. If the secondary tonic precedes the tonic, that vowel is considered stressed for the purpose of subsequent sound changes, and at some point the intervening vowel drops. Vowel changes assume the changes in Vulgar Latin as listed elsewhere in this thread.\\
VN \change\ V\ipa{:} / _S (except for the prefixes con-, in-); I'm assuming this change happened in Vulgar Latin and then vowel length went to quality\\\\
STRESSED VOWELS\\
--- \ipa{I} \change\ \ipa{e}\\
--- \ipa{U} \change\ \ipa{o} (this change seems to have happened later, hence is listed separately)\\
--- \ipa{O} \change\ \ipa{y} / _\{\ipa{u},P,\ipa{k,g,i}\} (in northwestern dialects)\\
--- \ipa{O} \change\ \ipa{4e} / _\{\ipa{u},P,\ipa{k,g,i}\} (in western dialects, Limousin, and Auvergne)\\
--- \ipa{O} \change\ \ipa{4o} / _\{\ipa{u},P,\ipa{k,g,i}\} (in Languedoc)\\
--- \ipa{O} \change\ \{\ipa{4e,4o,O}\} / _\{\ipa{u},P,\ipa{k,g,i}\} (in southern dialects)\\
--- \ipa{a} \change\ \ipa{a} / _N (Rouergue, Limousin, Auvergne, Dauphin\'{e})\\
--- \ipa{a} \change\ \ipa{a} / _\# in monosyllables and oxytones (Rouergue, Limousin, Auvergne, Dauphin\'{e})\\
--- \ipa{a} \change\ \aa\ (I have no idea what is going on here. Grandgent seems to distinguish an open and close /\ipa{a}/, and I have listed his open a as /\aa/, which seems to have been distinct from /\ipa{O}/, but beyond this section it doesn't really seem to matter very much)\\
--- \aa\ \ipa{E O} \change\ \ipa{a e o} / _N (in Limousin and neighboring regions, the last two particularly in Limousin, Languedoc, and Gascon, though when _\ipa{\textltailn} this change may have been blocked)\\
--- \ipa{E O} \change\ \ipa{jE wO} (intermittent, ``least common in the southwest")\\
--- \ipa{e} \change\ \ipa{i} / _V (and possibly V_?)\\
--- \ipa{e} \change\ \ipa{i} / _(C\textellipsis)\ipa{i}\{C(C\textellipsis)V,\#\}\\
--- \ipa{E} \change\ \ipa{jE} / _\{\ipa{u,i,L,r\super j,S,Z,j,tS,dZ}\} (except in some northern and western dialects, or if this \ipa{u} \textleftarrow\ \ipa{l} or if this \ipa{i} \textleftarrow\ \ipa{D})\\
--- \ipa{ju} \change\ \ipa{jeu}\\
--- \ipa{o} \change\ \ipa{y} / _\{\ipa{tS,dZ,it,id}\} (did not occur in Dauphin\'{e})\\
--- \ipa{o} \change\ \ipa{y} / _\ipa{\textltailn} (in northern and western dialects)\\
--- \ipa{o} \change\ \ipa{y} / _\ipa{i}\# (in Bordeaux, Auvergne, and some of Languedoc)\\
--- \ipa{O} \change\ \{\ipa{O,we}\} (in southwestern dialects)\\
--- \ipa{u} \change\ \ipa{y}\\
--- \ipa{o} \change\ \ipa{u} (during the literary period)\\\\
UNSTRESSED VOWELS\\
--- E \change\ \O\ / _\ipa{e}\\
--- B \change\ \O\ / _\ipa{o}\\
--- E \change\ \ipa{j} (presumably in the vicinity of another vowel)\\
--- V \change\ \O\ (though /\ipa{a}/ seemed to resist this)\\\\
INITIAL-SYLLABLE VOWELS\\
--- \{\ipa{a,\oe,e,i}\} \change\ \ipa{e} (here, $\langle$\ipa{\oe}$\rangle$ denotes the reflex of the Latin vowel written this way, not a front rounded vowel)\\
--- \ipa{u} \change\ \ipa{o}\\
--- \ipa{au} \change\ \ipa{a} / _(C\textellipsis)\ipa{u}\\
--- V \change\ \O\ / _\ipa{r} (seemed to be an intermittent change)\\
--- Lots and lots of analogical formations\\\\
INTERTONIC VOWELS\\
--- V \change\ \O\ (again, /\ipa{a}/ seemed to resist this change, which was itself generally blocked by analogy)\\\\
PENULT VOWELS\\
--- V \change\ \O\ / ``penult of paroxytones'', though /\ipa{a}/ often remained ``as an indistinct e'', probably /\ipa{@}/\\
--- V often remained when \{\ipa{tS,dZ,j}\}_, especially if CC_, or when \{\ipa{(k)s,s:,sj}\}_\\
--- V is retained when P_C[+dental]\\
--- CVK \change\ CV\ipa{j} (intermittent if C was a resonant)\\
--- V \change\ \O\ / \ipa{lv}_\ipa{r} (dialect-dependent)\\
--- \ipa{e} \change\ \ipa{i} / _(C\textellipsis)\ipa{i}(C\textellipsis)\#\\\\
FINAL VOWELS\\
--- (Again, /\ipa{a}/ seems to be a persistent exception to these)\\
--- V \change\ \O\ / R_(C\textellipsis)\\
--- Grandgent remarks that /\ipa{i}/ was probably the last vowel to drop out\\
--- \ipa{a} \change\ \ipa{o} in most dialects except Gascon and Languedoc; final -\ipa{as} remained in ``Limousin and some others"\\
--- /\ipa{i}/ and /\ipa{u}/ remain when "V_ (then \ipa{u} is subject to the changes listed above---\ipa{u} \change\ \ipa{y}, \&c.)\\
--- \ipa{e u} \change\ \ipa{e o} / _\ipa{nt}\#\\
--- -\ipa{os} remains ``in the extreme east''\\
--- Final -\ipa{i} remains ``as late as the 12th century" in some regions (Aude, Tarn, Aveyron, Corr\`{e}ze, and some pockets of Haute-Garrone)\\
--- Epenthesis of /\ipa{e}/ in 2{\sc sg} ``of some verbs"\\
--- V \change\ ``indistinct e'' (probably /\ipa{@}/) if dropping it would create ugly consonant clusters:\\
------ C_L, P_C[+dental], C_\ipa{tS}, C_\ipa{k}, C_\ipa{m}, C_\ipa{n} where ``originally separated by the vowel of the penult" (proparoxytones)\\
------ \'{K}_\ipa{r} (paroxytones)\\
------ Where the cluster would be unwieldy otherwise, apocope happened\\
------ Final /\ipa{(m)bj mnj pj mj}/ ``required a supporting vowel" (dialect-dependent)\\
--- V \change\ \ipa{e} ``in many late words"\\\\
K \change\ \O\ / _\ipa{l} (not always, but this was a general change)\\
\ipa{v} \change\ \ipa{b} / \ipa{r}_ (sporadic)\\
h \change\ \O\\
\ipa{d} \change\ \O\ / V_V (seems to have happened in the north and northeast at some point)\\
\ipa{b d g} \change\ \ipa{B D G} / V_V\\
\ipa{Bj} \change\ \ipa{j} (in forms of {\it habeo} and {\it debeo})\\
\ipa{k g} \change\ \ipa{tS dZ} / _E\\
\ipa{tS} \change\ \ipa{ts} (sometimes)\\
\ipa{g} \change\ \O\ / V_(VC\textellipsis)"V\\
\ipa{j} \change\ \O\ / V_"E\\
N \change\ \O\ / _\#, in polysyllables\\
\ipa{k\super w g\super w} \change\ \ipa{k g} / _B\\
\ipa{rs} \change\ \ipa{s:}\\
\ipa{s:} \change\ \ipa{s} / V\ipa{:}_\\
\ipa{p t k s} \change\ \ipa{b d g z} / V_V (this \ipa{b} \change\ \ipa{v}?)\\
\O\ \change\ \ipa{i} / \#_\ipa{s}C\\
\ipa{w} \change\ \ipa{v} \change\ \O\ / _\ipa{u} (``restored by analogy in many words")\\
\ipa{w} \change\ \ipa{v} \change\ \O\ / _"\ipa{o}\\
\ipa{w} \change\ \ipa{v} \change\ \ipa{B}\\
\ipa{B} \change\ \ipa{w} / V_C\\
\ipa{w} \change\ \O\ / C_B (when from earlier B?)\\
\ipa{k} \change\ \O\ / _\ipa{s}\{C,\#\} (the latter in polysyllables only)\\
\{\ipa{d,g}\} \change\ \O\ / _\ipa{j}\\
Loan phonemes:\\
--- Loaned /\ipa{b}/ did not lenite\\
--- Loaned /\ipa{k}/ did not palatalize\\
--- Loaned \ipa{w} \change\ \ipa{gw}\\
Original \ipa{z} (/\ipa{ts}/?) \change\ \ipa{dj} \change\ \ipa{j}\\
Greek /\ipa{k}/ shows up variously as /\ipa{k g}/\\
\ipa{p\super h} \change\ \ipa{f}\\
Some reanalysis of initials as medials if a prefixed form was reanalyzed as a single morpheme\\
\ipa{(s)k g} \change\ \ipa{(s)tS dZ} / \#_\ipa{a} (in the north and northeast)\\
\ipa{tS} \change\ \ipa{ts} \change\ \ipa{s} / \#_\\
\ipa{j} \change\ \ipa{dZ} / \#_ (but not in B\'{e}arn)\\
\ipa{f} \change\ \ipa{h} (in B\'{e}arn and Gascon)\\
\ipa{B} \change\ \ipa{b} (in B\'{e}arn, Gascon, and Languedoc)\\
\ipa{B} \change\ \ipa{v} (though sometimes \change\ \ipa{gw} instead if analogy interfered)\\
\ipa{k\super w g\super w} \change\ \ipa{kw gw} (in western dialects)\\
\ipa{k\super w g\super w} \change\ \ipa{k g} (else)\\
\O\ \change\ \ipa{k} / \#\ipa{s}_\ipa{l}\\
\ipa{i} \change\ \ipa{e} / \#_\ipa{s}C\\
\ipa{b} \{\ipa{d,dz,dZ}\} \ipa{z Z g} \change\ \ipa{p} \{\ipa{t,ts,tS}\} \ipa{s S k} / _(\ipa{s})\#\\
\ipa{tSs} \change\ \{\ipa{ts,tS}\} (varies)\\
\ipa{j} \change\ \ipa{i} / _\ipa({s})\#\\
\ipa{D} \change\ \O\ / _\#\\
\ipa{D} \change\ \ipa{t} / _\ipa{s}\#\\
\ipa{B} \change\ \ipa{u} / V_(\ipa{s})\#\\
\ipa{B} \change\ \{\O,\ipa{f}\} / C_(\ipa{s})\# (the latter is rare)\\
\ipa{n} \change\ \O\ / V_\# (did not occur in extreme western areas, some northern areas, the southeast, and the east)\\
\ipa{n} \change\ \O\ / _\ipa{s}\# (except for eastern and southeastern dialects)\\
\ipa{n} \change\ \O\ / \ipa{r}_(\ipa{s})\#?\\
\ipa{k:} \change\ \ipa{tS} / _\ipa{a} (in the east and northeast)\\
\ipa{l:} \change\ \ipa{l} (in the south)\\
\ipa{r:}\raisebox{-0.6ex}{\textasciitilde}\ipa{r} stuff---not sure what was going on here, but it seems like this distinction lingered on into the literary period, but the two may have been in the process of merging\\
\ipa{mn} \change\ \ipa{mpn} \change\ \ipa{n:} (dialectal)\\
C\ipa{:} \change\ C\\
\ipa{g} \change\ \ipa{k} / B_ (your guess is probably better than mine)\\
\ipa{g} \change\ \{\ipa{k,j}\} / V_ (if /\ipa{j}/ resulted, it dropped after /\ipa{i}/; forms with \ipa{k} are ``most persistent in the west'' and more common overall)\\
\ipa{l} \change\ \ipa{w} / V_\ipa{s}\\
\ipa{ms ns} \change\ \ipa{mps nts} (sporadic?)\\
\{\ipa{p,b}\} \change\ \O\ / R_\ipa{s}\# (unless blocked by analogy)\\
\ipa{ts} \change\ \ipa{s} / _\# (Provence, Limousin, some Languedoc and Gascon)\\
\ipa{B} \change\ \O\ / when pretonic and immediately adjacent to a back vowel\\
\ipa{B} \change\ \ipa{b} / V_V (in western and some central dialects)\\
\ipa{B} \change\ \ipa{v} (otherwise)\\
\ipa{tS} \change\ \ipa{i} / _C\\
\ipa{tS} \change\ \ipa{i} / C_\\
\ipa{tS} \change\ \ipa{jdz} \change\ \ipa{jz} / V_V (in the south and northwest)\\
\ipa{tS} \change\ \ipa{dz} \change\ \ipa{z} / V_V (otherwise)\\
\ipa{D} \change\ \ipa{i} / C_\\
\ipa{D} \change\ \ipa{i} / _C ! _\ipa{s}\#\\
\ipa{D} \change\ \ipa{z} (except in some northern and eastern dialects where \change\ \O)\\
\ipa{g} \change\ \ipa{j} / _\ipa{a} (in the north and east; further \change\ \ipa{dZ} in the north)\\
\ipa{g} \change\ \ipa{g} / _\{\ipa{o,u/y}\}\\
\ipa{g} \change\ \ipa{j} \change\ \ipa{dZ} / ``[b]etween the last two vowels of a proparoxytone", though it dropped early in some dialects\\
\ipa{dZ} \change\ \ipa{j}\\
\ipa{l} \change\ \ipa{u} / _\ipa{s}\# (in many dialects)\\
\ipa{s} \change\ \ipa{r} / _\ipa{n} (in a few dialects)\\
\ipa{j} \change\ \ipa{dZ} / _"E (in the west)\\
\ipa{j} \change\ \O\ / _"E (else)\\
\ipa{j} \change\ \ipa{i} / _C\\
\ipa{j} \change\ \ipa{dZ} / V_V (did not occur in the northeast and some northern dialects)\\
\ipa{B} \change\ \ipa{u} / _\ipa{l}\\
\{\ipa{t,d}\}\ipa{l} \{\ipa{k,g}\}\ipa{l} \change\ \ipa{l: L}\\
\O\ \change\ \ipa{b} / \ipa{m}_\ipa{l}\\
\ipa{p j} \change\ \ipa{b i} / _\ipa{l}\\
\ipa{p k tS b g B j} \change\ \ipa{b g i} \{\ipa{b,u,u}\change\ipa{y}\change\ipa{i}\} \O\ \{\ipa{u,u}\change\ipa{y}\change\ipa{i}\} \ipa{i} / _\ipa{r}\\
\{\ipa{t,d}\} \change\ \ipa{D} \change\ \ipa{i} / _\ipa{r}\\
\ipa{D} \change\ \O\ / \ipa{au}_\\
\O\ \change\ \ipa{d} / \ipa{z}_\ipa{r}\\
\O\ \change\ \ipa{b} / \ipa{m}_\ipa{r}\\
\O\ \change\ \ipa{t} / \ipa{s}_\ipa{r}\\
\ipa{Bw tw} \change\ \ipa{w: dw} \change\ \ipa{g\super w g\super w} \change\ \ipa{g g}\\
\{\ipa{d,k}\}\ipa{w} \change\ \ipa{g\super w} \change\ \ipa{g}\\
(\ipa{k\super w} \change\ \ipa{g\super w} \change\ \ipa{g} ?)\\
\{\ipa{l,r}\}\ipa{w} \{\ipa{n,Nk,Ng}\}\ipa{w} \change\ \ipa{lg\super w Ng\super w} \change\ \ipa{lg Ng}\\
\ipa{pw} \change\ \ipa{upw} \change\ \ipa{up} \change\ \ipa{ub}\\
\ipa{w} \change\ \O\ / \ipa{s}_ ?\\
\ipa{Bj} \change\ \{\ipa{udZ,uj}\} (in northern dialects)\\
\ipa{B} \change\ \ipa{u} / _\ipa{j} (in western dialects)\\
\ipa{B} \change\ \{\ipa{b,v}\} / _\ipa{j} (in southern and eastern dialects)\\
\ipa{ktj klj} \change\ \ipa{is L}\\
\{\ipa{tS(:),k\super w}\}\ipa{j} \change\ \ipa{ts}\\
\ipa{d}V\ipa{g (n)d}V\ipa{g} \change\ \ipa{dZ(?) ndZ}\\
\ipa{l}\{\ipa{tj,tSj}\} \ipa{ldZ} \change\ \ipa{lts ldz} \change\ \ipa{uts udz} \change\ \ipa{us uz}\\
\ipa{l:}\{\ipa{j},V\ipa{dZ}\} \ipa{lnj} \change\ \ipa{L \textltailn}\\
\ipa{lvj} \change\ \ipa{lbj} \change\ \ipa{ubj}\\
\ipa{j} \change\ \O\ / \ipa{n}_\# (in many dialects)\\
\ipa{nj} \change\ \ipa{\textltailn}\\
\ipa{mbj} \change\ \{\ipa{mbj,mdZ,ndZ}\}\\
\ipa{mnj} \change\ \ipa{\textltailn} (Limousin, as well as extreme eastern and southwest dialects)\\
\ipa{mnj} \change\ \ipa{ndZ} (else)\\
\ipa{mj} \change\ \{\ipa{mj,\textltailn}\}\\
\{\ipa{ntSj,ndj}\} \change\ \ipa{nts} \change\ \ipa{ns}\\
\{\ipa{ndj,ndZj}\} \ipa{ndZ} \change\ \ipa{\textltailn} \{\ipa{\textltailn,ndZ}\}\\
\ipa{p} \change\ \ipa{b} / _\ipa{rj}\\
\ipa{ptj} \change\ \ipa{ts} \change\ \ipa{s}\\
\ipa{pj} \change\ \ipa{ptS} \change\ \ipa{tS} (except in western and some southern dialects)\\
\ipa{rtSj} \change\ \ipa{rts} (\change\ \ipa{rs} ?)\\
\ipa{rdj r}V\ipa{g} \change\ \ipa{rdz rdZ}\\
\ipa{rdZ} \change\ \{\ipa{rdZ,rdz}(\change\ \ipa{rz})\}\\
\ipa{rtj} \change\ \ipa{rts} \change\ \ipa{rs}\\
\ipa{r:}V\ipa{g r:j} \change\ \ipa{rdZ ir}\\
\ipa{rt}V\ipa{g} \change\ \{\ipa{rdZ,rts}(\change\ \ipa{rs})\}\\
\ipa{rtj} \change\ \ipa{rts} \change\ \ipa{rs}\\
\ipa{v} \change\ \{\ipa{v,b}\} / \ipa{r}_\ipa{j}\\
\ipa{rj} \change\ \ipa{r\super j} \change\ \ipa{ir} / V_V\\
\ipa{rj} \change\ \ipa{r\super j} \change\ \ipa{r} / _\#\\
\{\ipa{s:j,stSj,stj}\} \change\ \ipa{S} \change\ \ipa{is} (except in the west or extreme east, where the outcome was some flavor of \ipa{(i)(t)S})\\
\ipa{sj} \change\ \ipa{Z} \change\ \ipa{(i)(d)Z} (in some northeastern, northern, and western dialects)\\
\ipa{sj} \change\ \ipa{Z} \change\ \{\ipa{r,z}\} (rare)\\
\ipa{sj} \change\ \ipa{Z} \change\ \ipa{iz} (the usual outcome)\\
\ipa{t}V\ipa{g trj} \change\ \ipa{dZ ir}\\
\ipa{t:j} \change\ \ipa{ts} \change\ \ipa{s}\\
\ipa{tj} \change\ \ipa{tS} \change\ \ipa{dj} \change\ \ipa{djZ} (?) \change\ \ipa{dz} (in the north and west) or \ipa{idz} (in the south and east)---medial \ipa{(i)dz} became \ipa{(i)z}; \ipa{i}-less forms ``prevail in the literary language'' and seem to have become common if the \ipa{dz} follows the stress\\
\ipa{l} \change\ \ipa{u} / _\{\ipa{t,s}\} (Languedoc)\\
\ipa{l} \change\ \ipa{u} / _\{\ipa{d,s}\} (Rouergue)\\
\ipa{l} \change\ \ipa{u} / _\{\ipa{t,d,s}\} (else)\\
\ipa{ndt ndtS ntS nkt nf} \change\ \ipa{n}\{\ipa{d,t}\} \{\ipa{nts,ndz}(\change\ \ipa{nz}),\ipa{ndZ}\} \ipa{n(t)s} \{\ipa{\textltailn,(i)nt,ntS}\} \{\ipa{nf,f:}(\change\ \ipa{f})\}\\
\ipa{nt nd} \change\ \{\O,\ipa{n}\} \O\ / _\# (in some Languedoc and Gascon areas)\\
\ipa{nd} \change\ \ipa{n} / _\# (in western areas and for some speakers of Limousin)\\
\ipa{d} \change\ \ipa{t} / \ipa{n}_\#\\
\ipa{rtS rdtS} \change\ \ipa{r(t)s} \{\ipa{rdz}(\change\ \ipa{rz}),\ipa{rts,rdZ}\}\\
\ipa{rdg} \change\ \ipa{r}\{\ipa{g,dZ}\} / _\ipa{a}\\
\ipa{t} \change\ \O\ / \ipa{rd}_\\
\ipa{t} \change\ \O\ / \ipa{r}_\ipa{m} (sometimes)\\
\ipa{stS} \change\ \ipa{s} (in some northern and northeastern dialects)\\
\ipa{stS} \change\ \ipa{(i)(t)S} (for western and extreme eastern regions)\\
\ipa{stS} \change\ \ipa{is} (otherwise)\\
\ipa{k} \change\ \O\ / \ipa{s}_\ipa{b} (sporadic?)\\
\{\ipa{p,t}\} \change\ \O\ / \ipa{s}_\ipa{m} (sporadic?)\\
\ipa{p} \change\ \O\ / \ipa{s}_\ipa{t}\\
\ipa{stg} \change\ \ipa{s}\{\ipa{g,dZ}\}\\
\ipa{bk} \change\ \ipa{pts} / _\ipa{a}\\
\ipa{b} \change\ \{\O,\ipa{u}\} / _\ipa{rg}\\
\ipa{b} \change\ \O\ / _\ipa{s}\{\ipa{t,k}\}\\
\ipa{b} \change\ \{\O,\ipa{b}\} / _\ipa{t}\\
\ipa{b} \change\ \O\ / _\ipa{ts}\\
\ipa{B} \change\ \ipa{u} / _\ipa{k}\\
\ipa{BtS} \change\ \ipa{u}\{\ipa{ts,dz}\} \change\ \ipa{u}\{\ipa{s,z}\}\\
\ipa{Bt Bd} \change\ \ipa{pt bd} (in the west)\\
\ipa{Bt Bd} \change\ \ipa{ut ud} (else)\\
\ipa{ktS} \change\ \ipa{(i)tS} (in western and extreme eastern dialects)\\
\ipa{ktS} \change\ \ipa{its} \change\ \ipa{is} (else)\\
\ipa{tS k} \change\ \{\ipa{i,s}\} \{\O,\ipa{k}\} / _\ipa{m}\\
\ipa{kt gd} \change\ \ipa{it id} (in the north, northeast, and southwest)\\
\{\ipa{tS,k}\}\ipa{t gd} \change\ \ipa{tS dZ} (else)\\
\ipa{dtS} \change\ \ipa{ts} (in Auvergne and some western areas)\\
\ipa{dtS} \change\ \ipa{dZ} (for some southeastern and southwestern speakers)\\
\ipa{dtS} \change\ \ipa{dz} \change\ \ipa{z} (else)\\
\ipa{g} \change\ \O\ / _\ipa{m}\\
\ipa{gn}\{\ipa{d,t}\} \change\ \ipa{Nn}\{\ipa{d,t}\} \change\ \ipa{\textltailn}\{\ipa{d,t}\} \change\ \{\ipa{(i)nd,\textltailn d,ndZ}\} or \ipa{t(S)}\\
\ipa{gn} \change\ \ipa{Nn} \change\ \ipa{\textltailn}\\
\ipa{ksk} \change\ \ipa{stS} / _\ipa{a} (in the north and northeast)\\
\ipa{k} \change\ \O\ / _\ipa{sk} (else)\\
\ipa{ks} \change\ \ipa{S} \change\ \ipa{s} / _\ipa{m}\\
\ipa{ks:} \change\ \ipa{is}\\
\ipa{p} \change\ \O\ / _\ipa{f}\\
\ipa{pk} \change\ \ipa{ptS} / _\ipa{a}\\
The outcome of \ipa{ps} varied; some dialects preserved it, while others changed it to \ipa{(i)S} (typical of the west), \ipa{us} (the east), or \ipa{is}\\
\ipa{pt} \change\ \{\ipa{ut,it}\} ``in a few words"\\
\ipa{p} \change\ \O\ / _\ipa{t} ``except in parts of Languedoc and Gascony"\\
\ipa{td} \change\ \ipa{t:} \change\ \ipa{t}\\
\ipa{jd} \change\ \{\ipa{dZ,id}\}\\
Some dialects dropped all final \ipa{k}, while others only dropped it when B_\#, changing it to \ipa{i} when \{\ipa{a},E\}_\#\\
\{\ipa{d,l}\} \change\ \O\ / _\#\\
\ipa{t} \change\ \O\ / _\# ! ``in the preterit of verbs", though it tended to drop ``in strong preterits"\\
The outcome of final \ipa{nt} was usually \ipa{n}, but in the extreme north and some areas of the south, the full cluster was kept as part of the ending -\ipa{ant}; further, ``in some dialects the n fell after o, u"

\subparagraph{Vulgar Latin to Rhaeto-Romance}{\it Pogostick Man}, from Wikipedia contributors (2014), ``Rhaeto-Romance languages". {\it Wikipedia, the Free Encyclopedia}. \textless\url{https://en.wikipedia.org/w/index.php?title=Rhaeto-Romance_languages&oldid=607581179}\textgreater

\tab {\it NB: These are probably very incomplete and too general, but they seem to be the key distinguishing features of Rhaeto-Romance.}

\ipa{E e} \change\ \ipa{ej je}\\
\ipa{a} \change\ \ipa{e} / stressed, usually when \'{K}_\\
\ipa{u:} \change\ \ipa{y} (\change\ \ipa{i} in most descendants, with the exception of Engadine)\\
\ipa{a} V \change\ \ipa{e} (= /\ipa{@}/?) \O\ / in final syllables (though Friulian preserves the ending -\ipa{is})\\
\ipa{k g} \change\ \ipa{tS dZ} / _\ipa{a} (note the similarity with some varieties of Old Proven\c{c}al)\\
/\ipa{pl fl kl}/ preserved\\
Germanic loaned /\ipa{w}/ preserved---{\it i.e.}, it did not become /\ipa{gw}/\\
C[+ voiced] \change\ \O\ / V_V (only for obstruents?)\\
C[- voiced] \change\ C[+ voiced] / V_V\\
Final /\ipa{s}/ preserved

\subparagraph{Latin to Romanian}{\it pharazon}, from Jensen, {\it A Comparative Study of Romance}, as well as other sources

\ipa{h} \change\ \O\\
V$_0$V$_0$ \change\ V$_0$\ipa{:}\\
\ipa{n} \change\ \O\ / _\{\ipa{f,v,s}\}\\
\ipa{r} \change\ \ipa{s} / _\ipa{s}\\
\{\ipa{m,n,s}\} \change\ \O\ / _\# in polysyllables\\
\ipa{m s} \change\ \ipa{n i} / _\#\\
\ipa{u} \change\ \O\ / CC_V\\
V \change\ "V / "VS\ipa{r}_\\
V \change\ "V / _(C\textellipsis)"\{\ipa{i,e}\}V

``Stressed vowels (note the difference from the French development)":\\
--- \ipa{a:} \change\ \ipa{a}\\
--- \ipa{(a)e} \change\ \ipa{E}\\
--- \ipa{e:,i,oe} \change\ \ipa{e}\\
--- \ipa{i: o(:) u(:)} \change\ \ipa{i o u}

Word-initial vowels:\\
--- \ipa{a:} \change\ \ipa{a}\\
--- \{\ipa{e(:),i,ae,oe}\} \change\ \ipa{e}\\
--- \ipa{i:} \change\ \ipa{i}\\
--- \{\ipa{o(:),u}\} \change\ \ipa{o}

Word-final vowels:\\
--- \ipa{a:} \change\ \ipa{a}\\
--- \{\ipa{e(:),i,ae,oe}\} \change\ \ipa{e} / _\#\\
--- \ipa{i:} \change\ \ipa{i}\\
--- \{\ipa{o(:),u:}\} \change\ \ipa{o}\\
--- \ipa{u} \change\ \ipa{o} / ! V_

\ipa{s} \change\ \ipa{i} / \#(C\textellipsis)V_\#\\
\ipa{k} \change\ \O\ / _\ipa{s}\#\\
\{\ipa{s,t}\} \change\ \O\ / _\#\\
\ipa{k\super w g\super w} \change\ \ipa{p b} / V_\ipa{a}\\
\ipa{g} \change\ \ipa{m} / _\ipa{n}\\
\ipa{w} \change\ \O\ / \{\ipa{k,g}\}_\\
\ipa{k g} \change\ \ipa{tS dZ} / _E\\
\ipa{E} \change\ \ipa{ie}\\
\ipa{t d s} \change\ \ipa{ts dz S} / _\ipa{i}\\
\ipa{o e} \change\ \ipa{u i} / _N\\
\ipa{a} \change\ \ipa{1} _N ! _\{\ipa{n:,mn}\}\\
\ipa{i} \change\ \O\ / OL_\ipa{e}\\
\ipa{e} \change\ \ipa{a} / \ipa{i}_(C\textellipsis)\{\ipa{a,e}\}\#\\
\ipa{i} \change\ \O\ / \{\ipa{ts,dz},S\}_V\\
\ipa{li} \change\ \ipa{lj} / _V\\
\ipa{l} \change\ \ipa{lj} / _\ipa{i}\\
\ipa{l:} \change\ \O\ / _\ipa{i}\\
"\ipa{el:a} \change\ \ipa{e}"\ipa{a} / _\#\\
\{\ipa{b,v}\} \change\ \O\ / V_\{V,\ipa{t}\}\\
\ipa{l} \change\ \ipa{r} / V_V\\
\ipa{m} \change\ \ipa{u} / \ipa{a}_\ipa{n}V\\
\ipa{b} \change\ \ipa{u} / V_\{\ipa{l,r}\}\\
\ipa{p} \change\ \O\ / _\ipa{s}\\
\ipa{c} \change\ \ipa{p} / _\{\ipa{s,t}\}\\
\{\ipa{sc,st}\} \change\ \ipa{St} / _F\\
\ipa{s} \change\ \ipa{S} / _\ipa{kl}\\
\ipa{c} \change\ \O\ / \ipa{n}_\ipa{t}\\
\ipa{kj gj} \change\ \ipa{tS} \{\ipa{j},\O\} (\ipa{gj} \change\ \O\ is rare)\\
P\ipa{j} \change\ \ipa{4}\\
\{\ipa{sj,stj,s:j}\} \change\ \ipa{S}\\
\ipa{j} \change\ \ipa{s} / \ipa{t}_\\
\ipa{j} \change\ \ipa{z} / \ipa{rd}_\\
\ipa{dj} \change\ \ipa{Z} / _"B\\
\ipa{dj} \change\ \ipa{z} / V_V\\
\{\ipa{n,l}\} \change\ \O\ / _\ipa{j}\\
\ipa{ja} \change\ \ipa{e} / \ipa{r}_\#\\
\ipa{j} \change\ \O\ / \ipa{r}_\\
\ipa{d} \change\ \O\ / _\ipa{z}\\
\ipa{l} \change\ \ipa{j} / \{\ipa{k,g}\}_\\
\ipa{a} \change\ \ipa{e} / C[+palatal]_\#\\
\ipa{a o} \change\ \ipa{@ u} / "U\textellipsis_\#\\
\ipa{e} \change\ \ipa{@} / "U\textellipsis P_(C\textellipsis)V\# ! V = \ipa{i}\\
\ipa{u} \change\ \O\ / \ipa{o}_\ipa{e}\\
\ipa{e} \change\ \ipa{@} / \ipa{ou}_\#\\
\ipa{u} \change\ \O\ / ! \{OL,"V\}_\#\\
\ipa{a} \change\ \ipa{a} / \#(C\textellipsis)V\textellipsis C[+palatal]_\textellipsis"U\\
\ipa{a o} \change\ \ipa{@ u} / \#(C\textellipsis)V\textellipsis_\textellipsis"U\\
\ipa{e} \change\ \ipa{i} / \#(C\textellipsis)V\textellipsis_\ipa{n}\textellipsis"U\\
\ipa{e} \change\ \ipa{@} / \#(C\textellipsis)V\textellipsis\{\ipa{t,d,n}\}_\textellipsis"U\\
\ipa{e} \change\ \ipa{@} / P"_(C\textellipsis)B\\
\ipa{e} \change\ \O\ / P_\ipa{a}\\
\ipa{e} \change\ \ipa{@} / \#\{\ipa{r},P\}_\textellipsis"U\\
\ipa{o} \change\ \ipa{u} / \#C\textellipsis_\textellipsis"U\\
\ipa{a} \change\ \ipa{@} \#C(C\textellipsis)_\textellipsis"U\\
\{\ipa{t,d}\} \change\ \O\ / \ipa{n}_\#\\
C\ipa{i} \change\ C\ipa{\super j} / _\# ! R\textellipsis R_\#

\subparagraph{Latin to Sardinian}{\it qwed117}, ``mainly'' from \url{http://www.sardegnacultura.it/documenti/7_25_20060427093224.pdf}

V\ipa{:} \change\ V[- long]\\
\ipa{e} \change\ \ipa{i} / C_V\\
\ipa{i} \change\ \ipa{j} / V_V\\
\ipa{i} \change\ \ipa{j} / _V\\
\ipa{\{b,v,w\}} \change\ \O\ / V_V\\
\ipa{au ai} \change\ \ipa{o e}\\
\ipa{h} \change\ \O\ / \{\#,C\}_\\
\ipa{m} \change\ \O\ / _\#\\
\O\ \change\ \ipa{i} / _\ipa{s} (``[m]ainly Logudorese'')\\
\ipa{nd \{l:,ld\}} \change\ \ipa{\:n\:d \:d:} / V_V\\
\ipa{t} \change\ \ipa{k} / \ipa{s}_\ipa{l} (sporadic)\\
\ipa{l} \change\ \O\ / \ipa{rk}_ (sporadic)\\
\ipa{r} \change\ \ipa{l} / _C (sporadic)\\
\ipa{l} \change\ \ipa{r} / C_ (sporadic)\\
\ipa{\{i,j\}} \change\ \ipa{dZ} / V\ipa{r}_V (``dialectal'')\\
\ipa{v} \change\ \ipa{b} / \#_\\
S[- voice] \change\ S[+ voice] / V_"V\\
\ipa{kw gw} \change\ \ipa{p: b:} / \#_ (``[o]nly Logudorese'')\\
\ipa{w} \change\ \O\ / \#\ipa{k}_V\\
\ipa{k} \change\ \ipa{ts} / _\ipa{i}\\
\ipa{\{p,k\}s} \change\ \ipa{s:}\\
\ipa{o} \change\ \ipa{u} / _\ipa{k} (possibly restricted in occurrence)\\
\ipa{k} \change\ \O\ / _\ipa{t} ?\\
\ipa{l\{i,j\}} \change\ \ipa{l:} \change\ \{\ipa{ts,dz,ldz,dZ,l:}\} (``varies'')\\
\ipa{e} \change\ \O\ / \ipa{u}_\#\\
\ipa{t} \change\ \ipa{d} / V_\ipa{r}V\\
\ipa{sk} \change\ \ipa{s:}\\
\ipa{\{i,j\}} \change\ \ipa{g} / \#_\ipa{e}\\
\ipa{d} \change\ \O\ / V_\{\ipa{i,j}\}\\
\ipa{n\{i,j\} t\{i,j\}} \change\ \ipa{ndZ ts} / _V\\
\ipa{d} \change\ \ipa{r} / V_V (``[s]ome Campidanese'')\\
\ipa{b d g} \change\ \ipa{B D G} / ``except in Nuorese''\\
\ipa{gn} \change\ \ipa{n:}\\
\ipa{r} \change\ \ipa{ur:} / \#_\ipa{e} / Logudorese\\
\ipa{r} \change\ \ipa{ar:} / \#_(j)B / Logudorese

\subparagraph{Vulgar Latin to Spanish}{\it ? and Seraf\'{\i}n}, the former citing Penny, Ralph (2002), {\it A History of the Spanish Language}, 2nd Ed. Cambridge University Press; and Lipski, John (1994), {\it Latin American Spanish}. Longman Pub Group.

\ipa{b} \change\ \ipa{B} / V_V\\
\{\ipa{tj,kj}\} \{\ipa{t:j,k:j,ptj,ktj,skj}\} \change\ \ipa{ts t:s}\\
\ipa{k g} \change\ \ipa{tS dZ} \change\ \ipa{ts dz} / _\{\ipa{j,i,e,E}\}\\
\{\ipa{t,k}\} \change\ \O\ / _\#\\
V \change\ \O\ / C_\{\ipa{r,l}\} when unstressed and not at a word boundary\\
V \change\ \O\ / \{\ipa{r,l}\}_C when unstressed and not at a word boundary\\
V \change\ \O\ / C_\ipa{s} when unstressed and not at a word boundary (sporadic)\\
V \change\ \O\ / \ipa{s}_C when unstressed and not at a word boundary (sporadic)\\
\{\ipa{k,g}\} \change\ \ipa{x} \change\ \ipa{j} / _\{\ipa{t,s,n,l}\}\\
\ipa{pt} \{\ipa{Rs,ps}\} \change\ \ipa{t: s:}\\
\ipa{ns} \change\ \ipa{s} (with a few exceptions)\\
\ipa{mb mn} \change\ \ipa{m: n:}\\
\{\ipa{jl,lj}\} \{\ipa{jn,nj}\} \{\ipa{jg,gj}\} \change\ \ipa{L \textltailn\ J:}\\
\ipa{bj} \change\ \ipa{J:} (sporadic)

Raising of \ipa{e} \{\ipa{E,a}\} \ipa{O o} \change\ \ipa{i e o u}; near \ipa{j}, in particular environments:\\
--- \ipa{e} \change\ \ipa{i} / _C\ipa{j} ! C = \ipa{p}\\
--- \ipa{E O u} \change\ \ipa{e o u} / _(C)\ipa{j}\\
--- \ipa{a} \change\ \ipa{e} / _\ipa{j}

\ipa{oj} \change\ \ipa{we} (sporadic)\\
\ipa{E O} \change\ \ipa{je we}\\
"\ipa{je.o} "\ipa{je.a} \change\ \{"\ipa{i.o},"\ipa{jo}\} "\ipa{i.a}\\
\ipa{jt js} \change\ \ipa{tS S}\\
\ipa{f} \change\ \ipa{h} / ! _\{\ipa{ue},L\}\\
\ipa{Rj pj} \change\ \ipa{jR jp} / V_\\
\ipa{L} \change\ \ipa{Z}\\
\ipa{J} \change\ \{\O,\ipa{Z}\} (the latter is rare)\\
\ipa{J:} \change\ \O\ / E_\\
\ipa{d} \change\ \{\O,\ipa{D}\} / V_V\\
\ipa{g} \change\ \{\O,\ipa{G}\} / V_V\\
\ipa{p t k s ts} \change\ \ipa{b d g z dz} / V_V\\
\ipa{p: t: k: s: t:s J:} \change\ \ipa{p t k s ts J}\\
\ipa{n: l: RR} \change\ \ipa{\textltailn\ L r}\\
\ipa{kl pl} \change\ \ipa{L} \{\ipa{L,tS}\}\\
\ipa{fl} \change\ \ipa{L} (sporadic)\\
V \change\ \O\ / unstressed ! V = \ipa{a}\\
\ipa{sj} \change\ \ipa{js} / V_\\
\ipa{i u} \change\ \ipa{e o} / _(C)\#\\
V\ipa{R} \change\ \ipa{R}V / C_\#\\
\ipa{e} \change\ \O\ / V\{\ipa{d,s,n,l,R}\}_\#\\
\ipa{d g} \change\ \ipa{D G} / V_V\\
/\ipa{J}/ ``gains a fortified [\ipa{dZ}] allophone" by analogy with the voiced-stop/voiced-fricative allophony in Spanish\\
``Complex resolution of many consonant clusters created with the previous loss of unstressed vowels":

\tab ``With deletion or assimilation or both":\\
--- \ipa{t} \change\ \O\ / _\ipa{m}\\
--- \ipa{d} \change\ \O\ / _\ipa{n} (sometimes)\\
--- \ipa{mn} \change\ \ipa{\textltailn} (sometimes)\\
--- \ipa{tst dzd} \change\ \ipa{ts dz}\\
--- V\ipa{dz} \change\ \O\ / \{\ipa{nts,ndz,rdz}_\\
--- \ipa{ndz}V\ipa{g} \change\ \ipa{ng}\\
--- \ipa{mp}V\ipa{t sk}V\ipa{p sp}V\ipa{t st}V\ipa{k} \change\ \ipa{nt sp st sk}

\tab ``With dissimilation":\\
--- \ipa{n} \change\ \{\ipa{l,R}\} / _\ipa{m}\\
--- \ipa{n} \change\ \{\ipa{R,l}\} / \ipa{ng}_\\
--- \ipa{n} \change\ \ipa{R} / \ipa{nd}_

\tab ``With metathesis":\\
--- \ipa{dn dl} \change\ \ipa{nd ld}\\
--- \ipa{ml nR} \change\ \ipa{lm Rn} (sometimes)\\
--- \ipa{BG} \change\ \ipa{wG} \change\ \ipa{Gw}

\tab ``With epenthesis":\\
--- \O\ \change\ \ipa{b} / \ipa{m}_\ipa{R}\\
--- \ipa{mn ml} \change\ \ipa{mbR mbl}\\
--- \O\ \change\ \ipa{d} / \ipa{n}_\ipa{R}

\ipa{t} \change\ \O\ / _\#\\
/\ipa{b}/ [\ipa{b}], /\ipa{B}/ [\ipa{b}\raisebox{-0.6ex}{\textasciitilde}\ipa{B}] \change\ /\ipa{b}/ [\ipa{b}\raisebox{-0.6ex}{\textasciitilde}\ipa{B}]\\
\ipa{b} \change\ \ipa{u} / _C\\
\ipa{l} \change\ \ipa{u} / _C (sometimes)\\
\ipa{ts dz} \change\ \ipa{\|[s \|[z}\\ %]]
\ipa{\|[z z Z} \change\ \ipa{\|[s s S} \\ %]]
\ipa{S} \change\ \{\ipa{x,X}\}

\tab ``None of the following sound changes is universal to all dialects. If the same sound appears twice or more with an apparent contradiction, this accounts for different dialects. In all cases there are dialects that conserve the original sound at the beginning of the 21st century, with the exception of the old phonemic [\ipa{\|[s} - s] distinction (though kept in another way today, as [\ipa{T} - s] in many parts of Spain)."%]

\ipa{h} \change\ \O\ ``(just a reminder: from the f \change\ h change above)"\\
\ipa{x} \change\ \ipa{h}\\
/\ipa{dR}/ [\ipa{DR}] \change\ \ipa{R} / V_V\\
\ipa{L} \change\ /\ipa{J}/ (``merged with")\\
\ipa{L} \change\ \ipa{j}\\
/\ipa{J}/ [\ipa{J}\raisebox{-0.6ex}{\textasciitilde}\ipa{dZ}] \change\ [\ipa{Z}\raisebox{-0.6ex}{\textasciitilde}\ipa{dZ}]\\
\ipa{Z} \change\ \ipa{S}\\
\{\ipa{\|[s,s}\} \change\ /\ipa{s}/ [\ipa{\|[s}] ``(merged as)"\\ %]]\\
\{\ipa{\|[s,s}\} \change\ \ipa{h} / _\$\\ %]\\
\{\ipa{\|[s,s}\} \change\ \ipa{h} ``(in all environments)" \\ %]\\
\ipa{\|[s} \change\ \ipa{T} \\ %]\\
\{\ipa{\|[s,s}\} \change\ \ipa{T} [\ipa{T}] ``(merged as)'' \\ %]\\
\ipa{h} \change\ \O\ / _\ipa{d}\\
\ipa{h} \change\ \O\ / _\ipa{t\super h}\\
\ipa{h} \change\ \O\ / _\$\\
\ipa{n} \change\ \ipa{N} / _\#\\
\ipa{b g} \change\ \ipa{B G} / \{\ipa{l,R}\}_\\
\ipa{d} \change\ \ipa{D} / \ipa{R}_\\
\ipa{tR} \change\ \{\ipa{t\r*{R},tS,t\:s}\}\\
\ipa{r} \change\ \{\ipa{\:z,X}\}\\
\ipa{l} \change\ \ipa{R} / _\$\\
\ipa{R} \change\ \ipa{l} / _\$\\
\ipa{s} \change\ \ipa{R} / _\ipa{T}\\
\ipa{kT} \change\ \ipa{T:}

\subsubsection{Proto-Italic to Proto-Sibellian}{\it Pogostick Man}, from \url{http://gillesquentel.org/docs/PIE_to_Italic_C.pdf} and \url{http://gillesquentel.org/docs/PIE_to_Italic_V.pdf}

\tab {\it NB: This is likely incomplete.}

\ipa{t} \change\ \ipa{f}\\
\ipa{\'{k} \'{g} k\super w g\super w} \change\ \ipa{k g p b}\\
\{\ipa{\'{g}\super h,x}\} \{\ipa{F,T,g\super w\super h}\} \change\ \ipa{h f}\\
\textsubring{r} \change\ \ipa{er} / _\#\\
\ipa{eu} \change\ \ipa{ou}

\paragraph{Proto-Sibellian to Oscan}{\it Pogostick Man}, from \url{http://gillesquentel.org/docs/PIE_to_Italic_C.pdf} and \url{http://gillesquentel.org/docs/PIE_to_Italic_V.pdf}

\tab {\it NB: This is likely incomplete.}

\ipa{z} \change\ \ipa{r}

\paragraph{Proto-Sibellian to Umbrian}{\it Pogostick Man}, from \url{http://gillesquentel.org/docs/PIE_to_Italic_C.pdf} and \url{http://gillesquentel.org/docs/PIE_to_Italic_V.pdf}

\tab {\it NB: This is likely incomplete.}

\ipa{d} \change\ \ipa{rs} / V_V

\subsection{Proto-Indo-European to Proto-Tocharian}{\it Nortaneous}, from \url{http://www.utexas.edu/cola/centers/lrc/eieol/tokol-TC-X.html} and \url{https://azargoshnasp.net/history/Tocharian/positionoftocharian.pdf}

\'K \change\ K\\
C\ipa{h} \change\ C / _(V)C\ipa{h}\\
\ipa{d} \change\ \O\ / _N\\
\ipa{dz} \change\ \O\ / B_\\
\ipa{dz} \change\ \ipa{ts}\\
K\ipa{\super w} \change\ K / _\{C,\ipa{o,a}\} ! C = syllabic\\
K\ipa{\super w} \change\ \ipa{C} / _\ipa{e(:)}\\
K\ipa{\super wy} \change\ \ipa{C}\\
\ipa{p \{ts,k(\super w)\} m n l r y} \change\ \ipa{pj s\super j \:s mj \textltailn\ lj rj wj} / _\{E(:),\ipa{y}\}\\
\ipa{\{t,d\super H}\} \change\ \ipa{t\super j} / _E(\ipa{:})\\
\ipa{\{t,d\super H}\} \change\ \ipa{ts} / _\ipa{y}\\
D(\ipa{\super H}) \change\ T\\
\ipa{\s{n}} \change\ \ipa{@} \change\ \O\ / C_\#\\
R H \change\ \ipa{u}R \ipa{1} / C_\{C,\#\} when syllabic\\
R H\ipa{\s{n}} \change\ \ipa{E}R \ipa{1n} / \#_C\\
H \change\ \ipa{E} / _R, when R = syllabic\\
H \change\ \O\ / V_V\\
\ipa{h}$_2$\ipa{e} \change\ \ipa{@} / _\#\\
\ipa{h}$_2$\ipa{e a} \change\ \ipa{a 1}\\
\{\ipa{eh}$_2$,\ipa{a}H\} \ipa{a:} \change\ \ipa{a: O}\\
\ipa{h}$_3$\ipa{e o} \change\ \ipa{o E}\\
\{\ipa{eh}$_3$,\ipa{o}H\} \change\ \ipa{o:}\\
\ipa{o:(s,y) o:n} \change\ \ipa{u} \{\ipa{u,\~{o}}\}\\
Something about *\ipa{\~{o}} and umlaut\\
\ipa{\~{o} o:} \change\ \ipa{o a}\\
\ipa{u} \change\ \ipa{w@} / \#_\\
\ipa{u} \change\ \{\ipa{@,u}\}\\
\ipa{uh}$_1$ \ipa{u}\{\ipa{h}$_2$,\ipa{h}$_3$\} \change\ \ipa{u:} \change\ \ipa{w@ w1}\\
\ipa{i} \change\ \ipa{@} / \{P,K(\ipa{\super w}),\ipa{s}\}_\\
C\ipa{i} \change\ C\ipa{\super j@}\\
\ipa{s} \change\ \ipa{s\super j} / _\ipa{t\super j}\\
\ipa{ih}$_1$ \ipa{i\{h}$_2$,\ipa{h}$_3$\} \change\ \ipa{j@ j1}\\
(\ipa{h}$_1$)\ipa{e} (\ipa{h}$_1$)\ipa{e:} \change\ \ipa{j@} \ipa{jE:} / \#_\\
C\ipa{e} C\ipa{e:} \change\ C\ipa{\super j@} C\ipa{\super jE}\\
\ipa{e}H \change\ \ipa{e:}\\
\ipa{ow aw ew} \change\ \ipa{Eu au @w}\\
\ipa{oy ay ey} \change\ \ipa{Ei ai @j}\\
\ipa{E} \change\ \ipa{o} / _\$B\\
\ipa{E} \change\ \ipa{@} / _\$\ipa{1}\\
"\ipa{1} "\ipa{@ 1}[- stress] \ipa{@}[- stress] \change\ \ipa{1 a a @}

\subsubsection{Proto-Tocharian to Tocharian A}{\it Nortaneous}, from \url{http://www.utexas.edu/cola/centers/lrc/eieol/tokol-TC-X.html} and \url{https://azargoshnasp.net/history/Tocharian/positionoftocharian.pdf}

\ipa{s} \change\ \ipa{\:s} / _\ipa{t}\\
\ipa{k} \change\ \ipa{p} / _\{\ipa{s,\:s}\}\\
C\ipa{\super jj} \change\ C\ipa{\super j:}\\
\ipa{n} \change\ \ipa{j} / V_\ipa{s}V ! E_\\
V \change\ j / V_\ipa{n\super jt\super j} ! V = E\\
\ipa{j} \change\ \O\ / \ipa{w}_\\
\ipa{\{a,E\}i @j} \change\ \ipa{e i}\\
\ipa{\{a,E\}u @w} \change\ \ipa{o u}\\
\ipa{\{O,E\}} \change\ \ipa{a}\\
V \change\ [+ round] / K\ipa{\super w}_\\
\ipa{\{k\super w,kw\}} \change\ \ipa{k}\\
\ipa{@} \change\ \O\ / _\%\\
V \change\ \O\ / _\#\\
``[E]penthesis [of /\ipa{@}/] to break up `difficult' consonant clusters (mostly in the coda?)''

\subsubsection{Proto-Tocharian to Tocharian B}{\it Nortaneous}, from \url{http://www.utexas.edu/cola/centers/lrc/eieol/tokol-TC-X.html} and \url{https://azargoshnasp.net/history/Tocharian/positionoftocharian.pdf}

\ipa{\{a,E\}u @w} \change\ \ipa{au u}\\
\ipa{\{a,E\}i @j} \change\ \ipa{ai i}\\
\ipa{O E} \change\ \ipa{o e}\\
\ipa{w} \change\ \O\ / _\ipa{j}\\
K\ipa{\super w} ``usually but not always retained''\\
C[+ coronal]\ipa{w} \change\ C\ipa{:}\\
\ipa{mn} \change\ \ipa{nm} / V_V\\
\ipa{s} \change\ \O\ / \ipa{n}_\#\\
\O\ \change\ \ipa{t} / \{N,L\}_S\\
\ipa{@} \change\ \O\ / _\%, when unstressed\\
``[E]penthesis [of /\ipa{@}/] to break up `difficult' consonant clusters (mostly in the coda?)''

\clearpage

\section{Je-Tup\'{\i}-Carib}

\subsection{Cariban}

\subsubsection{Pre-Bakairi to Eastern Bakairi}{\it Pogostick Man}, from Meira, S\'{e}rgio (2005), ``Reconstructing Pre-Bakairi Segmental Phonology''. {\it Anthropological Linguistics} 47(3):261 -- 291

VNV \change\ \~{V}\~{V} / ! _(C)\#\\
\ipa{P} \change\ \O\ / _C[+ fricative - voiced]\\
\ipa{R} \change\ \O\ / V_V, when neither vowel is stressed

\subsubsection{Pre-Bakairi to Western Bakairi}{\it Pogostick Man}, from Meira, S\'{e}rgio (2005), ``Reconstructing Pre-Bakairi Segmental Phonology''. {\it Anthropological Linguistics} 47(3):261 -- 291

V[- stress]NV \change\ \~{V}\~{V}\\
\ipa{z} \change\ \ipa{h} / V_\ipa{a}\\
\ipa{z} \change\ \O\ / V_V\\
C[+ fricative - voiced] \change\ \O\ / \ipa{P}_\\
\ipa{1} \change\ \ipa{@} / P_\\
\ipa{1} \change\ \ipa{i}\\
\ipa{Z} \change\ \O\\
\ipa{R} \change\ \O\ / V_V, where at least one of the vowels is nasalized

\subsection{Ofai\'{e}-J\^{e}}

\subsubsection{Proto-Ofai\'{e}-J\^{e} to Proto-J\^{e}}{\it Pogostick Man}, from Gudschinsky, Sarah C. (1971), ``Ofai�-Xavante, a J� Language''

V\ipa{m} \change\ \~{V} / _\#\\
VS \change\ \ipa{r} / C_V\\
\ipa{c} \change\ \{\ipa{c,z}\}\\
\ipa{N\super w \{k\super w,h\super w\}} \change\ \ipa{m p}\\
\ipa{@} \change\ \O\ / C_CV (not sure if this happened all the time or not)

\subsubsection{Proto-Ofai\'{e}-J\^{e} to Ofai\'{e}-Xavante}{\it Pogostick Man}, from Gudschinsky, Sarah C. (1971), ``Ofai�-Xavante, a J� Language''

\ipa{m} \change\ \ipa{w} / _\#\\
\ipa{m} \change\ \{\ipa{w,p}\}\\
\ipa{\textltailn}V \change\ \ipa{j}\~{V}\\
\ipa{k(\super w)} \change\ \ipa{P} / _\#\\
\ipa{k\super w} \change\ \ipa{k}\\
\ipa{N} \change\ \ipa{n} / V_V\\
\ipa{N} \change\ \ipa{\~{j}} / \#_ (not sure if this nasalizes the following vowel or not)\\
\ipa{N\super w h\super w} \change\ \ipa{\~{j} h}

\subsection{Tupar\'{i}}

\tab As pertains to this section, the vowels given in the form $\langle$\{V$_1$/V$_2$\}$\rangle$\hspace{0pt} herein may have apparently been some sort of alternation in vowel grade or quality. Also, the names of these languages were researched on the Wikipedia; they are in many cases different from the names cited within the source papers proper.

\tab Moore and Galucio (1994) give the following inventory for Proto-Tupar\'{i}:

\begin{center}\begin{tabular}{c | c c c c c c}
& Bilabial & Alveolar & Palatal & Velar & Labiovelar & Glottal\\\hline
Stop & \ipa{p} & \ipa{t} & & \ipa{k} & \ipa{k\super w} & \ipa{P}\\
Nasal & \ipa{m} & \ipa{n} & & \ipa{N} & \ipa{N\super w}\\
Fricative & \ipa{B} & & & & & \ipa{h}\\
Affricate & & \ipa{ts (n)dz}\\
Liquid & & \ipa{r}\raisebox{-0.6ex}{\textasciitilde}D & \ipa{j}\raisebox{-0.6ex}{\textasciitilde}\ipa{\~\j}\raisebox{-0.6ex}{\textasciitilde}\ipa{\textltailn}\end{tabular}\end{center}

\begin{center}\begin{tabular}{c | c c c}
& Front & Central & Back\\ \hline
High & \ipa{i \~{\i}} & \ipa{1 \~{1}} & \ipa{u \~{u}}\\
Mid & \ipa{e \~{e}} \\
Low & & \ipa{a \~{a}}\end{tabular}\end{center}

\tab *\ipa{u} *\ipa{\~{u}} may have actually been *\ipa{o} *\ipa{\~{o}}, respectively. Additionally, the following ablaut pairs have been reconstructed:\\\\
*\ipa{a}\raisebox{-0.6ex}{\textasciitilde}*\ipa{e}\\
*\ipa{e}\raisebox{-0.6ex}{\textasciitilde}*\ipa{a}\\
*\ipa{\~{\i}}\raisebox{-0.6ex}{\textasciitilde}*\ipa{\~{e}}

\tab (From Moore, Denny and Ana Vilacy Galucio (1994), ``Reconstruction of Proto-Tupari Consonants and Vowels". {\it Report 8: Survey of California and Other Indian Languages: Proceeds of the Meeting of the Society for the Study of the Indigenous Languages of the Americas, July 2 -- 4, 1993, and the Hokan-Penutian Workshop, July 3, 1993}, 119 -- 137)

\subsubsection{Proto-Tupar\'{i} to Makur\'{a}p}{\it Pogostick Man}, from Moore, Denny and Ana Vilacy Galucio (1994), ``Reconstruction of Proto-Tupari Consonants and Vowels". {\it Report 8: Survey of California and Other Indian Languages: Proceeds of the Meeting of the Society for the Study of the Indigenous Languages of the Americas, July 2 -- 4, 1993, and the Hokan-Penutian Workshop, July 3, 1993}, 119 -- 137.

\ipa{t} \change\ \ipa{r} / _"V\\
\ipa{t} \change\ \ipa{l} / _V\\
\ipa{t} \change\ \O\ / else\\
\ipa{k} \change\ \O\ / _\#\\
\ipa{k\super w} \change\ \O\\
\ipa{b} \change\ \ipa{B} / V_V\\
\ipa{g\super w} \change\ \ipa{B} / _V[-nas]\\
\ipa{ts (n)dz} \change\ \ipa{t nd}\\
\ipa{B} \change\ \O\ / _\ipa{i}\\
\ipa{h} \change\ \O\ / V_C\\
\ipa{P} \change\ \O\\
\ipa{r} \change\ \ipa{l} / V[+nas]_V[+nas]\\
\ipa{D nN\super w} \change\ \ipa{c B} / \#_V[-nas]\\
\ipa{D} \change\ \{\O,\ipa{c}\} / else\\
\ipa{j N\super w m n} \change\ \ipa{\textltailn\ m} \{\ipa{m,p}\} \{\O,\ipa{t}\} / _V[+nas]\\
\ipa{\super nd N} \change\ \ipa{t} \{\ipa{g,k}\} / _V[-nas]\\
\ipa{u} \change\ \ipa{o} / _\{\ipa{p,b}\}\ipa{i}\\
\ipa{1} \change\ \O\ / ! \#_\{\ipa{p,B}\}\ipa{e}\\
\{\ipa{a/e}\} \{\ipa{e/a}\} \{\ipa{\~{\i},\~{e}}\} \change\ \ipa{e a \~{e}}

\subsubsection{Proto-Tupar\'{i} to Mekens}{\it Pogostick Man}, from Moore, Denny and Ana Vilacy Galucio (1994), ``Reconstruction of Proto-Tupari Consonants and Vowels". {\it Report 8: Survey of California and Other Indian Languages: Proceeds of the Meeting of the Society for the Study of the Indigenous Languages of the Americas, July 2 -- 4, 1993, and the Hokan-Penutian Workshop, July 3, 1993}, 119 -- 137.

\ipa{t} \change\ \ipa{r} / _"V\\
\ipa{g} \change\ \ipa{k}\\
\ipa{k} \change\ \ipa{g} / in U[+stress]\\
\ipa{g\super w} \change\ \ipa{k} / _\ipa{o}\\
\ipa{g\super w Ng} \change\ \ipa{k\super w k} / _V[-nas]\\
\ipa{g\super w} \change\ \ipa{k\super w} / \#_V[+nas]\\
\ipa{ts (n)dz} \change\ \{\ipa{s,ts}\} \ipa{s}\\
\ipa{B} \change\ \O\ / \ipa{i}_\\
\ipa{h} \change\ \O\ / V_C\\
\ipa{P} \change\ \O\\
\{\ipa{\super mb,\super nd,D}\} \change\ \ipa{t} / _V[+nas]\\
\ipa{D} \change\ \ipa{s} / _\ipa{i}\\
\ipa{D} \change\ \ipa{h} / else\\
\ipa{N N\super w} \change\ \ipa{k m} / _V[+nas]\\
\ipa{N\super w} \change\ \ipa{k\super w} / \#_V[-nas]\\
\ipa{1} \change\ \ipa{i} / \#_\{\ipa{p,Be}\}\\
\ipa{1} \change\ \O\ / else\\
\{\{\ipa{a/e}\},\{\ipa{e/a}\}\} \{\ipa{\~{\i},\~{e}}\} \change\ \ipa{a \~{e}}

\subsubsection{Proto-Tupar\'{i} to Tupar\'{i}}{\it Pogostick Man}, from Moore, Denny and Ana Vilacy Galucio (1994), ``Reconstruction of Proto-Tupari Consonants and Vowels". {\it Report 8: Survey of California and Other Indian Languages: Proceeds of the Meeting of the Society for the Study of the Indigenous Languages of the Americas, July 2 -- 4, 1993, and the Hokan-Penutian Workshop, July 3, 1993}, 119 -- 137.

\ipa{t} \change\ \ipa{r} / _V\\
\ipa{k\super w g} \change\ \O\ \ipa{k} \\
\ipa{g\super w} \change\ \O\ / _\ipa{o}\\
\ipa{g\super w Ng} \change\ \ipa{B k} / _V[-nas]\\
\{\ipa{(n)dz,ts}\} \change\ \ipa{s} / _\ipa{i}\\
\{\ipa{(n)dz,ts}\} \change\ \ipa{t} / else\\
\ipa{B D} \change\ \O\ \{\ipa{s,h}\} / _\ipa{i}\\
\ipa{D} \change\ \ipa{h}\\
\ipa{\super mb N\super w} \change\ \ipa{p B} / \#_V[-nas]\\
\ipa{n} \change\ \O\ / ! \#_V[-nas]\\
\ipa{N} \change\ \ipa{k} / \#_V[+nas]\\
\ipa{N\super w} \change\ \ipa{m} / V[+nas]_V[+nas]\\
\ipa{u} \change\ \ipa{o} / _\{\ipa{p,b}\}\ipa{i}\\
\{\ipa{a/e}\} \change\ \ipa{e}

\subsubsection{Proto-Tupar\'{i} to Wayor\'{o}}{\it Pogostick Man}, from Moore, Denny and Ana Vilacy Galucio (1994), ``Reconstruction of Proto-Tupari Consonants and Vowels". {\it Report 8: Survey of California and Other Indian Languages: Proceeds of the Meeting of the Society for the Study of the Indigenous Languages of the Americas, July 2 -- 4, 1993, and the Hokan-Penutian Workshop, July 3, 1993}, 119 -- 137.

\ipa{p} \change\ \ipa{B} / V_\\
\ipa{t} \change\ \ipa{r} / _"V\\
\ipa{p t} \change\ \O\ \ipa{l} / _V\\
\ipa{k} \change\ \ipa{g} / in U[+stress]\\
\ipa{b} \change\ \O\ / V_V\\
\ipa{g\super w} \change\ \ipa{g} / _\ipa{o}\\
\ipa{b\super w} \change\ \ipa{B} / \#_V[+nas]\\
\ipa{ts (n)dz} \change\ \ipa{t nd}\\
\ipa{h} \change\ \O\ / V_C\\
\ipa{P} \change\ \O\\
\ipa{r N\super w} \change\ \ipa{n B} / V[+nas]_V[+nas]\\
\ipa{D} \change\ \ipa{(n)d}\\
\ipa{Ng} \change\ \ipa{k} / ! _V[-nas]\\
\ipa{u} \change\ \ipa{1} / _\{\ipa{p,b}\}\ipa{i}\\
\{\ipa{a/e}\} \{\ipa{e/a}\} \{\ipa{\~{\i},\~{e}}\} \change\ \ipa{a e \~{\i}}

\subsection{Tup\'{i}-Guaran\'{i}}

\subsubsection{Proto-Tup\'{i}-Guaran\'{i} to Akw\'{a}ra}{\it Pogostick Man}, from Lemle, Miriam (1971), �Internal Classi�cation of the Tupi-Guarani Linguistic Family''. In {\it Tupi Studies I}, from {\it Summer Institute of Linguistics Publications in Linguistics and Related Fields Publication Number 29}.

\ipa{p} \change\ \ipa{k} / _\ipa{w}\\
\ipa{t} \change\ \ipa{tS} / _\{\ipa{i,\~{\i}}\}\\
\ipa{k b r} \change\ \{\ipa{N},\O\} \{\ipa{w,m}\} \{\ipa{n,r,t}\} / _\#\\
\ipa{b} \change\ \ipa{w}\\
\ipa{ts} \change\ \{\ipa{h},\O\}\\
\ipa{a} \change\ \{\ipa{1,o}\} / _N\#\\
\ipa{o} \change\ \ipa{a} / ! \ipa{o}(C\textellipsis)_(C\textellipsis)\#\\
\ipa{u} \change\ \O\ / \ipa{k}_\ipa{w}\\
\ipa{a} \change\ \ipa{o} / ! C\ipa{w}_\\
\ipa{\~{a}} \change\ \ipa{a} / C\ipa{w}_\\
\ipa{\~{a}} \change\ \ipa{\~{o}}\\
\{\ipa{\~{e},\~{\i}}\} \ipa{\~{1}} \{\ipa{u,\~{u},\~{o}}\} \change\ \ipa{i 1 o}

\subsubsection{Proto-Tup\'{i}-Guaran\'{i} to Cocama}{\it Pogostick Man}, from Lemle, Miriam (1971), �Internal Classi�cation of the Tupi-Guarani Linguistic Family''. In {\it Tupi Studies I}, from {\it Summer Institute of Linguistics Publications in Linguistics and Related Fields Publication Number 29}.

\ipa{p} \change\ \ipa{k} / _\ipa{w}\\
\ipa{t} \change\ \ipa{tS} / _\{\ipa{i,\~{\i}}\}\\
\ipa{P ts N} \change\ \O\ \{\ipa{ts,tS}\} \ipa{n}\\
\O\ \change\ \ipa{i} / \ipa{j}_\#, in monosyllables\\
\ipa{j} \change\ \ipa{i} / _\#, in polysyllables\\
\ipa{b} \change\ \O\ / _\#\\
\ipa{b} \change\ \ipa{w} / else\\
\ipa{w} \change\ \O\ / \ipa{k}_\ipa{w}\\
\ipa{w} \change\ \ipa{u} / \ipa{k}_\\
\ipa{a} \change\ \O\ / _\ipa{j}\#\\
\ipa{e}N \change\ \ipa{y} / _\#\\
\ipa{e} \change\ \ipa{1} / \{\ipa{k,j}\}_\\
\ipa{o} \change\ \ipa{u(a)} / ! \ipa{o}(C\textellipsis)_(C\textellipsis)\#\\
\ipa{\~{a}} \{\ipa{\~{e},\~{\i}}\} \ipa{\~{1}} \change\ \ipa{a i} \O\\
\ipa{u} \change\ \ipa{\~{u}} (? possibly backwards?)\\
\ipa{iP uP} \change\ \ipa{j w} / C_V\\
V$_0$\ipa{P}V$_0$ \change\ V$_0$\ipa{:}

\subsubsection{Proto-Tup\'{i}-Guaran\'{i} to Guajajara}{\it Pogostick Man}, from Lemle, Miriam (1971), �Internal Classi�cation of the Tupi-Guarani Linguistic Family''. In {\it Tupi Studies I}, from {\it Summer Institute of Linguistics Publications in Linguistics and Related Fields Publication Number 29}.

\ipa{t} \change\ \ipa{ts} / _\{\ipa{i,\~{\i}}\}\\
\ipa{ts} \change\ \{\ipa{h},\O\}\\
\ipa{b} \change\ \O\ / \ipa{u}_\#\\
\ipa{b} \change\ \ipa{w} / else\\
\ipa{u} \change\ \O\ / \ipa{k}_\ipa{w}\\
\ipa{a} \change\ \ipa{@} / _N\#\\
\ipa{a} \change\ \ipa{@} / if N in U\#\\
\ipa{o} \change\ \ipa{u} / ! \ipa{o}(C\textellipsis)_(C\textellipsis)\#\\
\ipa{\~{a} \~{e} \~{\i} \~{1}} \{\ipa{\~{o},\~{u}}\} \change\ \ipa{@ e i 1 o}

\subsubsection{Proto-Tup\'{i}-Guaran\'{i} to Guaran\'{i}}{\it Pogostick Man}, from Lemle, Miriam (1971), �Internal Classi�cation of the Tupi-Guarani Linguistic Family''. In {\it Tupi Studies I}, from {\it Summer Institute of Linguistics Publications in Linguistics and Related Fields Publication Number 29}.

\ipa{m p} \change\ \ipa{N k} / _\ipa{w}\\
\ipa{t} \change\ \ipa{tS} / _\{\ipa{i,\~{\i}}\}\\
\ipa{k} \change\ \O\\
\ipa{ts} \change\ \{\ipa{tS},\O\}\\
\{\ipa{b,r}\} \change\ \O\ / _\#\\
\ipa{u} \change\ \O\ / \ipa{k}_\ipa{w}\\
V\{\ipa{m,n}\} \change\ V[+nas] / _\#\\
V\ipa{N} \change\ V[+nas]

\subsubsection{Proto-Tup\'{i}-Guaran\'{i} to Guarayo}{\it Pogostick Man}, from Lemle, Miriam (1971), �Internal Classi�cation of the Tupi-Guarani Linguistic Family''. In {\it Tupi Studies I}, from {\it Summer Institute of Linguistics Publications in Linguistics and Related Fields Publication Number 29}.

\ipa{m p} \change\ \ipa{N k} / _\ipa{w}\\
\ipa{t} \change\ \ipa{tS} / _\{\ipa{i,\~{\i}}\}\\
\{\ipa{b,k}\} \ipa{r} \change\ \O\ \{\ipa{r},\O\} / _\#\\
\ipa{P ts} \change\ \{\ipa{P},\O\} \{\ipa{ts,tS}\}\\
\{V\ipa{m},V\ipa{N}\} V\ipa{n} \change\ V[+nas] \{V\ipa{r},V[+nas]\} / _\#\\
\ipa{a}N \ipa{e}N \ipa{i}N \ipa{1}N \ipa{u}N \change\ \ipa{\~{a} \~{e} \~{\i} \~{1} \~{u}} / _\#\\ %DOUBLE-CHECK
\ipa{\~{o}} \change\ \ipa{o}


\subsubsection{Proto-Tup\'{i}-Guaran\'{i} to Kamayur\'{a}}{\it Pogostick Man}, from Lemle, Miriam (1971), �Internal Classi�cation of the Tupi-Guarani Linguistic Family''. In {\it Tupi Studies I}, from {\it Summer Institute of Linguistics Publications in Linguistics and Related Fields Publication Number 29}.

\ipa{p} \change\ \ipa{h} / _\{o,u,w\}\\
\ipa{t} \change\ \ipa{tS} / _\{\ipa{i},\~{\i}\}\\
\ipa{ts} \change\ \{\ipa{h,j,}\O\}\\
\ipa{b r} \change\ \ipa{p t} / _\#\\
\ipa{b} \change\ \ipa{w} / else\\
\ipa{\~{a}} \change\ \ipa{a} / C\ipa{w}_\\
\ipa{\~{e} \~{\i} \~{o}} lost nasalization sometimes, kept it in others\\
\ipa{u} \change\ \ipa{\~{o}} (? possibly backwards?)

\subsubsection{Proto-Tup\'{i}-Guaran\'{i} to Parintint\'{i}n}{\it Pogostick Man}, from Lemle, Miriam (1971), �Internal Classi�cation of the Tupi-Guarani Linguistic Family''. In {\it Tupi Studies I}, from {\it Summer Institute of Linguistics Publications in Linguistics and Related Fields Publication Number 29}.

\ipa{k} \change\ \{\ipa{N},\O\} / _\#\\
\ipa{ts} \change\ \{\ipa{h},\O\}\\
V\ipa{n} \change\ V[+nas] (sometimes)\\
\O\ \change\ \ipa{N} / \{\#,V\}_\ipa{w}\\
\ipa{b r} \change\ \{\ipa{b},\O\} \{\ipa{r,t}\} / _\#\\
\ipa{u} \change\ \O\ / \ipa{k}_\ipa{w}\\
\ipa{e}N \change\ \ipa{\~{\i}} / _\#\\
\ipa{\~{e} \~{\i} \~{u}} \change\ \{\ipa{\~{e},e}\} \{\ipa{\~{\i},i}\} \{\ipa{\~{u},\~{o}}\}

\subsubsection{Proto-Tup\'{i}-Guaran\'{i} to Sirion\'{o}}{\it Pogostick Man}, from Lemle, Miriam (1971), �Internal Classi�cation of the Tupi-Guarani Linguistic Family''. In {\it Tupi Studies I}, from {\it Summer Institute of Linguistics Publications in Linguistics and Related Fields Publication Number 29}.

\ipa{p} \change\ \{\ipa{h},\O\} / _\{\ipa{u,o}\}\\
\ipa{p} \change\ \{\ipa{k},\O\} / _\ipa{w}\\
\ipa{p} \change\ \ipa{h} / else\\
\ipa{t} \change\ \{\ipa{ts,tS}\} / _\{\ipa{i,\~{\i}}\}\\
\ipa{k} \change\ \O\ / _\#\\
\ipa{P ts} \change\ \O\ \{\ipa{s,S}\}\\
\ipa{u} \change\ \O\ / \ipa{k}_\ipa{w}\\
V\{\ipa{m,n}\} \change\ V[+nas] / _\#\\
V\ipa{N} \change\ V[+nas]\\
\ipa{j} \change\ \{\ipa{j,i}\} / _\#\\
\ipa{j} \change\ \{\ipa{\textltailn,tS}\} / else\\
\ipa{w} \change\ \{\ipa{g,k}\} / \{\#,V\}_\\
\{\ipa{b,r}\} \change\ \O\ / _\#\\
\ipa{a} \change\ \{\O,\ipa{o,e}\} / _\ipa{j}\#\\
\ipa{o u} \change\ \{\ipa{u,o}\} \{\ipa{u,o,i}\}\\
\ipa{a}N \ipa{e}N \ipa{i}N \ipa{1}N \ipa{u}N \change\ \ipa{\~{a} \~{e} \~{\i}} \{\ipa{\~{1},\~{i}}\} \ipa{\~{o}} / _\#\\
\{\ipa{\~{1},\~{o}}\} \ipa{\~{u}} \change\ \{\ipa{\~{o},\~{e}}\} \ipa{\~{o}}

\subsubsection{Proto-Tup\'{i}-Guaran\'{i} to Classical Tupi}{\it Pogostick Man}, from Lemle, Miriam (1971), �Internal Classi�cation of the Tupi-Guarani Linguistic Family''. In {\it Tupi Studies I}, from {\it Summer Institute of Linguistics Publications in Linguistics and Related Fields Publication Number 29}.

\ipa{t} \change\ \ipa{tS} / _\{\ipa{i,\~{\i}}\}\\
\ipa{ts} \change\ \{\ipa{s,S}\}\\
\ipa{i} \change\ \ipa{\~{\i}} / \ipa{P}_\#\ (sporadic)

\paragraph{Tupian}

\subparagraph{Proto-Monde to Gavi\~{a}o}{\it Pogostick Man}, from Anonby, Stan, and David J. Holbrook (2013), ``A report and comparative-historical look at the Cinta Larga, Suru\'{i}, Gavi\~{a}o and Zor\'{o} languages". {\it Working Papers of the Linguistics Circle of the University of Victoria} 23:14 -- 31

\ipa{p} \change\ \ipa{v} / _\#\\
\ipa{h} \change\ \O\ / V_ (sporadic, likely an areal feature)

\subparagraph{Proto-Monde to Proto-Cinta Larga-Suru\'{i}-Zor\'{o}}{\it Pogostick Man}, from Anonby, Stan, and David J. Holbrook (2013), ``A report and comparative-historical look at the Cinta Larga, Suru\'{i}, Gavi\~{a}o and Zor\'{o} languages". {\it Working Papers of the Linguistics Circle of the University of Victoria} 23:14 -- 31

V \change\ \~{V} / _\ipa{h}\\
\ipa{h} \change\ \O\ / V_\\
\ipa{v} \change\ \ipa{w} / \#_\\
\ipa{tS} \change\ \ipa{S}

\subparagraph{Proto-Cinta Larga-Suru\'{i}-Zor\'{o} to Cinta Larga}{\it Pogostick Man}, from Anonby, Stan, and David J. Holbrook (2013), ``A report and comparative-historical look at the Cinta Larga, Suru\'{i}, Gavi\~{a}o and Zor\'{o} languages". {\it Working Papers of the Linguistics Circle of the University of Victoria} 23:14 -- 31

\ipa{o} \change\ \ipa{u}\\
V\ipa{h} \change\ V\ipa{:}\\
V \change\ V\ipa{:} / _\#\\

\subparagraph{Proto-Cinta Larga-Suru\'{i}-Zor\'{o} to Suru\'{i}}{\it Pogostick Man}, from Anonby, Stan, and David J. Holbrook (2013), ``A report and comparative-historical look at the Cinta Larga, Suru\'{i}, Gavi\~{a}o and Zor\'{o} languages". {\it Working Papers of the Linguistics Circle of the University of Victoria} 23:14 -- 31

\ipa{h} \change\ \O\ / V_\\
\ipa{\super Ng} \change\ \ipa{g} / \#_ (possibly all prenasalized consonants?)\\
\ipa{b} \change\ \ipa{m} / \#_

\subparagraph{Proto-Cinta Larga-Suru\'{i}-Zor\'{o} to Zor\'{o}}{\it Pogostick Man}, from Anonby, Stan, and David J. Holbrook (2013), ``A report and comparative-historical look at the Cinta Larga, Suru\'{i}, Gavi\~{a}o and Zor\'{o} languages". {\it Working Papers of the Linguistics Circle of the University of Victoria} 23:14 -- 31

\ipa{h} \change\ \O\ / V_\\
\ipa\O\ \change\ \ipa{P} / V_\#\\
\ipa{\super Ng} \change\ \ipa{g} / \#_ (possibly all prenasalized consonants?)\\
\ipa{S} \change\ \ipa{tS} (sporadic, areal feature from Gavi\~{a}o influence)

\subsubsection{Proto-Tup\'{i}-Guaran\'{i} to Urubu}{\it Pogostick Man}, from Lemle, Miriam (1971), �Internal Classi�cation of the Tupi-Guarani Linguistic Family''. In {\it Tupi Studies I}, from {\it Summer Institute of Linguistics Publications in Linguistics and Related Fields Publication Number 29}.

\ipa{p k} \change\ \ipa{k} \O\ / _\ipa{w}\\
\ipa{t} \change\ \ipa{S} / _\{\ipa{i,\~{\i}}\}\\
\ipa{k} \change\ \{\ipa{k},\O\} / _\#\\
\ipa{k} \change\ \{\ipa{k,S}\} / else\\
\ipa{ts} \change\ \{\ipa{s,h}\}\\
V\ipa{n} \change\ V[+nas] / _\# (sometimes)\\
V\ipa{N} \change\ V[+nas]\\
\ipa{j b} \change\ \{\ipa{j,i}\} \O\ / _\#\\
\ipa{b} \change\ \ipa{w} / else\\
\ipa{u} \change\ \O\ / \ipa{k}_\ipa{w}\\
\ipa{u} \change\ \ipa{o} / ! \ipa{o}(C\textellipsis)_(C\textellipsis)\#\\
\ipa{a}N \ipa{i}N \ipa{u}N \change\ \{\ipa{a}N,\ipa{\~{a}}\} \ipa{\~{\i}} \{\ipa{u}N,\ipa{\~{u}}\} / _\#\\
\ipa{\~{1} \~{e} \~{o}} \change\ \O\ \{\ipa{\~{e},e}\} \{\ipa{o,\~{o},u,\~{u}}\}

\clearpage

\section{Kartvelian}\tab Wikipedia presents the following phonemic inventory for Proto-Kartvelian.

\begin{center}\begin{tabular}{c | c c c c c c c c}
& Bilabial & Alveolar & Postalveolar & Retroflex & Palatal & Velar & Uvular & Glottal\\ \hline
Nasal & \ipa{m} & \ipa{n}\\
Plosive & \ipa{p p' b} & \ipa{t t' d} & & & & \ipa{k k' g} & \ipa{q q'}\\
Fricative & & \ipa{s z} & \ipa{S} & \ipa{\:s \:z} & & \ipa{x G} & & \ipa{h}\\
Lateral Fricative & & \ipa{\textbeltl}\\
Affricate & & \ipa{ts ts' dz} & \ipa{tS tS' dZ} & \ipa{\:t\:s \:t\:s' \:d\:z}\\
Lateral Affricate & & \ipa{t\textbeltl'}\\
Liquid & & \ipa{l r} & & & (\ipa{j}) & \ipa{w}\\
\end{tabular}

\begin{tabular}{c | c c c}
& Front & Central & Back\\ \hline
High & (\ipa{i}) & & (\ipa{u}) \\
Mid & \ipa{E E:} & & \ipa{O O:} \\
Low & & & \ipa{A A:}\end{tabular}\end{center}

\tab The presence of *\ipa{j} is denoted in the article on the protolanguage proper as ``dubious''; the page on the language family does not include it in its list of regular correspondences, nor does it list the long vowels or *\ipa{h}.

\tab (From Wikipedia contributors (2013), ``Kartvelian languages''. {\it Wikipedia, the Free Encyclopedia}. \textless\url{http://en.wikipedia.org/w/index.php?title=Kartvelian_languages&oldid=580201868}\textgreater; and Wikipedia contributors (2013), ``Proto-Kartvelian language''. {\it Wikipedia, the Free Encyclopedia}. \textless\url{http://en.wikipedia.org/w/index.php?title=Proto-Kartvelian_language&oldid=574800306}\textgreater)

\subsection{Proto-Kartvelian to Georgian}{\it Pogostick Man}, from Wikipedia contributors (2013), ``Kartvelian languages''. {\it Wikipedia, the Free Encyclopedia}. \textless\url{http://en.wikipedia.org/w/index.php?title=Kartvelian_languages&oldid=580201868}\textgreater

\ipa{q} \change\ \ipa{x}\\
\{\ipa{\textbeltl,\:s}\} \change\ \ipa{s}\\
\ipa{\:t\:s} \{\ipa{\:t\:s',t\textbeltl'}\} \change\ \ipa{ts ts'}\\
\ipa{\:d\:z} \change\ \ipa{dz}\\
\ipa{\:z} \change\ \ipa{z}\\
\ipa{w} \change\ \ipa{v}

\subsection{Proto-Kartvellian to Svan}{\it Pogostick Man}, from Wikipedia contributors (2013), ``Kartvelian languages''. {\it Wikipedia, the Free Encyclopedia}. \textless\url{http://en.wikipedia.org/w/index.php?title=Kartvelian_languages&oldid=580201868}\textgreater

\ipa{k k'} \change\ \{\ipa{k,tS}\} \{\ipa{k',tS'}\}\\
\ipa{g} \change\ \{\ipa{g,dZ}\}\\
\ipa{\textbeltl\ S \:s} \change\ \ipa{l} \{\ipa{sg,Sg}\} \ipa{S}\\
\ipa{tS \:t\:s t\textbeltl' tS' \:t\:s'} \change\ \{\ipa{tSk,Sg}\} \ipa{tS h} \{\ipa{Sk',tS'k'}\} \ipa{tS'}\\
\ipa{dz dZ \:d\:z} \change\ \{\ipa{dz,z}\} \{\ipa{dZg,sg}\} \{\ipa{dZ,Z}\}\\
\ipa{\:z} \change\ \ipa{Z}

\subsection{Proto-Kartvelian to Zan}{\it Pogostick Man}, from Wikipedia contributors (2013), ``Kartvelian languages''. {\it Wikipedia, the Free Encyclopedia}. \textless\url{http://en.wikipedia.org/w/index.php?title=Kartvelian_languages&oldid=580201868}\textgreater

\ipa{E A} \change\ \ipa{A O}\\
\ipa{q q'} \change\ \ipa{x} \{\ipa{k',q',P}\}\\
\ipa{\textbeltl\ S \:s} \change\ \O\ \{\ipa{sk,Sk}\} \ipa{S}\\
\ipa{tS \:t\:s} \{\ipa{t\textbeltl',\:t\:s'}\} \ipa{tS''} \change\ \ipa{tSk tS tS'} \{\ipa{ts'k',tS'k'}\}\\
\ipa{dZ \:d\:z} \change\ \{\ipa{dZg,dzg}\} \ipa{dZ}\\
\ipa{\:z} \change\ \ipa{Z}\\
\ipa{w} \change\ \ipa{v}

\clearpage

\section{Khoisan}\tab For the following section, all clicks change regardless of secondary articulation or associated articulations with the exception of when such is specifically noted.

\subsection{Khoe}

\subsubsection{Proto-Khoe to \ipa{\v*{||}}Ana}{\it Pogostick Man}, from Rainer (1984), ``Studying the linguistic and ethno-history of the Khoe-speaking (central Khoisan) peoples of Botswana, research in progress". In {\it Botswana Notes and Records} 16:19 -- 35.

\ipa{\~{|}(n)} \change\ \ipa{\~{\textdoublebarpipe}n}\\
\ipa{!} \change\ \ipa{!}\raisebox{-0.6ex}{\textasciitilde}\ipa{k}\\
\ipa{!x} \change\ \ipa{x}\\
\ipa{\v*{!}} \change\ \ipa{g}\\
\ipa{\~{!}(n)} \change\ \ipa{N}\\
\ipa{ts} \change\ \{\ipa{ts}\raisebox{-0.6ex}{\textasciitilde}\ipa{ts\super h,ts,s}\}\\
\ipa{h} \change\ \ipa{j} / _E

\subsubsection{Proto-Khoe to \ipa{||}Ani}{\it Pogostick Man}, from Rainer (1984), ``Studying the linguistic and ethno-history of the Khoe-speaking (central Khoisan) peoples of Botswana, research in progress". In {\it Botswana Notes and Records} 16:19 -- 35.

\ipa{\~{|}(n)} \change\ \ipa{\~{\textdoublebarpipe}n}\\
\ipa{! \v*{!} \~{!}} \change\ \ipa{!}\raisebox{-0.6ex}{\textasciitilde}\ipa{k \v*{!}}\raisebox{-0.6ex}{\textasciitilde}\ipa{g \~{!}(n)}\raisebox{-0.6ex}{\textasciitilde}\ipa{Ng}\\
\ipa{!x} \change\ \ipa{!x}\raisebox{-0.6ex}{\textasciitilde}\ipa{x}\\
\ipa{ts} \change\ \{\ipa{ts,s}\}\\
\ipa{h} \change\ \ipa{j}\raisebox{-0.6ex}{\textasciitilde}\ipa{P} / _E\\
\ipa{h} \change\ \ipa{h}\raisebox{-0.6ex}{\textasciitilde}\ipa{P}

\subsubsection{Proto-Khoe to Buga}{\it Pogostick Man}, from Rainer (1984), ``Studying the linguistic and ethno-history of the Khoe-speaking (central Khoisan) peoples of Botswana, research in progress". In {\it Botswana Notes and Records} 16:19 -- 35.

\ipa{\~{|}(n)} \change\ \ipa{\~{\textdoublebarpipe}}\\
\ipa{! !\super P \v*{!} \~{!}} \change\ \ipa{k} \O\ \ipa{g Ng}\\
\ipa{!x} \change\ \ipa{x}\\
\ipa{\~{!}n} \change\ \ipa{Ngj}\raisebox{-0.6ex}{\textasciitilde}\ipa{\~{!}}\\
\ipa{ts} \change\ \{\ipa{ts}\raisebox{-0.6ex}{\textasciitilde}\ipa{ts\super h,ts,s}\}\\
\ipa{h} \change\ \ipa{j}

\subsubsection{Proto-Khoe to Kxoe}{\it Pogostick Man}, from Rainer (1984), ``Studying the linguistic and ethno-history of the Khoe-speaking (central Khoisan) peoples of Botswana, research in progress". In {\it Botswana Notes and Records} 16:19 -- 35.

\ipa{\~{!}(n)} \change\ \ipa{\~{\textdoublebarpipe}}\\
\ipa{! !\super P ! \~{!}} \change\ \ipa{k} \O\ \ipa{g Ng}\\
\ipa{!x \~{!}n} \change\ \ipa{x Ngj}\raisebox{-0.6ex}{\textasciitilde}\ipa{\~{!}}\\
\ipa{ts dz} \change\ \{\ipa{\c{c},t\c{c}}\} \ipa{dZ}\\
\ipa{k\super h} \change\ \ipa{kx}\\
\ipa{h} \change\ \ipa{j}

\subsubsection{Proto-Khoe to Nama}{\it Pogostick Man}, from Rainer (1984), ``Studying the linguistic and ethno-history of the Khoe-speaking (central Khoisan) peoples of Botswana, research in progress". In {\it Botswana Notes and Records} 16:19 -- 35.

\ipa{k} \change\ \ipa{g}\\
\{\ipa{|\super P,|x'}\} \change\ \ipa{|}\\
\ipa{\~{|}(n)} \change\ \{\ipa{\~{\textdoublebarpipe},\textdoublebarpipe}\}\\
\ipa{! !\super P \~{!}n} \change\ \ipa{!g ~ \~{!}} \\
\ipa{\textdoublebarpipe} \{\ipa{\~{\textdoublebarpipe}n,\textdoublebarpipe\super P,\textdoublebarpipe x'}\} \change\ \ipa{\textdoublebarpipe g \textdoublebarpipe}\\
\{\ipa{||\super P,||x'}\} \change\ \ipa{||}\\
\ipa{ts dz kx'} \change\ \{\ipa{ts,s}\} \ipa{d} \O

\subsubsection{Proto-Khoe to Naro}{\it Pogostick Man}, from Rainer (1984), ``Studying the linguistic and ethno-history of the Khoe-speaking (central Khoisan) peoples of Botswana, research in progress". In {\it Botswana Notes and Records} 16:19 -- 35.

\ipa{\~{|}(n)} \change\ \ipa{\~{\textdoublebarpipe}}\\
\ipa{ts dz} \change\ \{\ipa{ts}\raisebox{-0.6ex}{\textasciitilde}\ipa{ts\super h,ts,s}\} \ipa{dz}\raisebox{-0.6ex}{\textasciitilde}\ipa{ts}\\
\ipa{k\super h} \change\ \{\ipa{kx,k}\}

\subsubsection{Proto-Khoe to !Ora}{\it Pogostick Man}, from Rainer (1984), ``Studying the linguistic and ethno-history of the Khoe-speaking (central Khoisan) peoples of Botswana, research in progress". In {\it Botswana Notes and Records} 16:19 -- 35.

\ipa{\~{|}(n)} \change\ \ipa{\~{\textdoublebarpipe}}\\
\ipa{! \~{!}n} \change\ \ipa{! \~{!}}\\
\ipa{\~{\textdoublebarpipe}n} \change\ \ipa{\textdoublebarpipe}\\
\ipa{||\super P} \change\ \{\ipa{||\super P,||}\}\\
\ipa{ts} \change\ \{\ipa{ts,s}\}

\subsubsection{Proto-Khoe to Teti}{\it Pogostick Man}, from Rainer (1984), ``Studying the linguistic and ethno-history of the Khoe-speaking (central Khoisan) peoples of Botswana, research in progress". In {\it Botswana Notes and Records} 16:19 -- 35.

\ipa{\~{|}(n)} \change\ \ipa{j}\\
\ipa{! !\super P \v*{!} \~{!}(n) !x} \change\ \ipa{k} \O\ \ipa{g N x}\\
\ipa{\textdoublebarpipe\ \~{\textdoublebarpipe}n \textdoublebarpipe\super P} \change\ \ipa{c \textltailn\ Pj}\\
\ipa{||\super P ||x'} \change\ \{\ipa{||\super P},\O\} \ipa{||\super P}\\
\ipa{ts dz kx'} \change\ \{\ipa{ts}\raisebox{-0.6ex}{\textasciitilde}\ipa{ts\super h,ts,s}\} \ipa{z k'}\\
\ipa{h} \change\ \ipa{j}\raisebox{-0.6ex}{\textasciitilde}\ipa{Pj} / _E\\
\ipa{h} \change\ \ipa{h}\raisebox{-0.6ex}{\textasciitilde}\ipa{Pj}

\subsubsection{Proto-Khoe to Ts\ipa{P}ixa}{\it Pogostick Man}, from Rainer (1984), ``Studying the linguistic and ethno-history of the Khoe-speaking (central Khoisan) peoples of Botswana, research in progress". In {\it Botswana Notes and Records} 16:19 -- 35.

\ipa{\~{|}(n) |x'} \change\ \ipa{j |\super P}\\
\ipa{! !\super P \v*{!} \~{!}(n) !x} \change\ \ipa{k} \O\ \ipa{g Ng x}\\
\ipa{\~{\textdoublebarpipe}n \textdoublebarpipe x'} \change\ \ipa{\textltailn\ \textdoublebarpipe\super P}\\
\ipa{||x'} \change\ \ipa{||\super P}\\
\ipa{ts dz kx'} \change\ \{\ipa{ts}\raisebox{-0.6ex}{\textasciitilde}\ipa{ts\super h,ts,s}\} \ipa{z k'}\\
\ipa{h} \change\ \ipa{j}\raisebox{-0.6ex}{\textasciitilde}\ipa{Pj} / _E\\
\ipa{h} \change\ \ipa{h}\raisebox{-0.6ex}{\textasciitilde}\ipa{j}

\subsubsection{Proto-Khoe to \ipa{\v*{|}}Ui}{\it Pogostick Man}, from Rainer (1984), ``Studying the linguistic and ethno-history of the Khoe-speaking (central Khoisan) peoples of Botswana, research in progress". In {\it Botswana Notes and Records} 16:19 -- 35.

\ipa{\~{|}(n)} \change\ \ipa{\~{\textdoublebarpipe}n}\\
\ipa{\~{!} !x} \change\ \ipa{\~{!}(n) !x}\raisebox{-0.6ex}{\textasciitilde}\ipa{x}\\
\ipa{ts} \change\ \{\ipa{ts}\raisebox{-0.6ex}{\textasciitilde}\ipa{tsH,s}\}\\
\ipa{h} \change\ \ipa{j} / _E\\
\ipa{h} \change\ \ipa{H}

\subsection{Kx'a}

\subsubsection{Proto-Kx'a to \ipa{\textdoublebarpipe}Hoan}{\it Pogostick Man}, from Heine, Bernd and Henry Honken (2010), ``The Kx'a family: A New Khoisan Genealogy"

``Something about word-initial glottal stops''\\
\O\ \change\ \ipa{a} / \ipa{o}_\ipa{m}\\
\ipa{a} \change\ \O\ / _\ipa{e} (sporadic)\\
\ipa{o} \change\ \O\ / \ipa{u}_\\
\ipa{u} \change\ \O\ / \ipa{o}_\\
\ipa{iaO} \change\ \ipa{iu}\\
\ipa{o} \change\ \O\ / \ipa{a}(C)_\\
V\ipa{n N} \change\ V[+nas] \O\ / _\#\\
\ipa{t d s} \change\ \{\ipa{c,tS}\} \ipa{\textbardotlessj\ S}\\
\ipa{!!} \change\ \ipa{||}\\
\ipa{\super n}Q\ipa{\super h} \t{Q\textipa{\;G}} \change\ Q\ipa{\super h} Q

\subsubsection{Proto-Kx'a to Northwestern !Xun}{\it Pogostick Man}, from Heine, Bernd and Henry Honken (2010), ``The Kx'a family: A New Khoisan Genealogy"

\ipa{a} \change\ \O\ / \#_\ipa{m}\\
\ipa{ui} \change\ \ipa{o} (?)\\
\ipa{i} \change\ \O\ / V_\\
\ipa{a} \change\ \O\ / _\ipa{e} (sporadic)\\
\ipa{u} \change\ \O\ / _\ipa{o}\\
\ipa{o} \change\ \O\ / _\ipa{u}\\
\ipa{o} \change\ \ipa{a} / _C\ipa{a}\\
\ipa{a} \change\ \O\ / _(C)\ipa{o}\\
``Some weird stuff with vowel pharyngealization/glottalization; some of the pharyngealized proto-vowels stayed that way, others glottalized"\\
\ipa{Q} \change\ \ipa{P} / _\ipa{m}\\
\ipa{n} \change\ \O\ / _\#\\
\ipa{ts(') s} \change\ \ipa{tS(') S}\\
\ipa{!! \!o} \change\ \ipa{|| |}\\
\ipa{\textdoublebarpipe} \change\ \ipa{!!} (dialectal)

\subsubsection{Proto-Kx'a to Southeastern !Xun}{\it Pogostick Man}, from Heine, Bernd and Henry Honken (2010), ``The Kx'a family: A New Khoisan Genealogy"

\ipa{a} \change\ \O\ / \#_\ipa{m}\\
\ipa{ui} \change\ \ipa{o} (?)\\
\ipa{i} \change\ \O\ / V_\\
\ipa{E O} \change\ \ipa{i u}\\
\ipa{u} \change\ \O\ / _\ipa{o}\\
\ipa{o} \change\ \O\ / _\ipa{u}\\
\ipa{o} \change\ \ipa{a} / _C\ipa{a}\\
\ipa{a} \change\ \O\ / _(C)\ipa{o}\\
``Some weird stuff with vowel pharyngealization/glottalization; some of the pharyngealized proto-vowels stayed that way, others glottalized"\\
\ipa{Qm} \change\ \{\ipa{b,\!b}\}\\
\ipa{n} \change\ \O\ / _\#\\
\ipa{ts(') s} \change\ \ipa{tS(') S}\\
\ipa{!! \!o} \change\ \ipa{|| |}\\
\ipa{P} \change\ \O\ / _\ipa{n}Q\\
\ipa{\super n}Q\ipa{\super h} \{\ipa{\t*{|\super hq},\t*{|\super h\;G}}\} \t{Q\textipa{q}} \change\ \ipa{\super n}Q\ipa{(\super h) \super n|\super h} Q[+voiced]

\clearpage

\section{Lakes Plain}

\tab Clouse (1993) reconstructs the following phonological inventory for Proto-Lakes Plain:

\begin{center}\begin{tabular}{c | c c c}
& Bilabial & Alveolar & Velar\\ \hline
Stop & \ipa{p b} & \ipa{t d} & \ipa{k}\end{tabular}\end{center}

\begin{center}\begin{tabular}{c | c c c}
& Front & Central & Back\\ \hline
High & \ipa{i} & & \ipa{u}\\
Mid & \ipa{e} & & \ipa{o}\\
Low & & \ipa{a}\end{tabular}\end{center}

\tab Additionally, *\ipa{R} is of uncertain reconstruction and is most likely an allophone of *\ipa{d}.

\tab For the following sound changes, a circumflexed vowel refers to an ``extra-high'' or ``fricativized'' vowel. There are a few cases where I may have either missed, misread, or put in an extraneous sound change to extra-high vowels; many of these were due to my perception of changes involving extra-high vowels being listed without a change creating them.

\tab (From Clouse, Duane (1993), ``Languages of the Western Lakes Plains". \textit{IRIAN: Bulletin of Irian Jaya} XXI:1 -- 17)

\subsection{Proto-Lakes Plain to Proto-Far West}{\it Pogostick Man}, from Clouse, Duane (1993), ``Languages of the Western Lakes Plains". \textit{IRIAN: Bulletin of Irian Jaya} XXI:1 -- 17

\ipa{R} \change\ \O\\
\ipa{ku} \change \O\ / \#_\\
\ipa{d} \change\ \ipa{R} / V_V\\
\O\ \change\ echo vowel / C_CV\\
CV \change\ \O\ / _\# (possibly only when CV_\#, possibly sporadic)\\
\ipa{e o} \change\ \ipa{E O} (?)

\subsubsection{Proto-Far West to Awera}{\it Pogostick Man}, from Clouse, Duane (1993), ``Languages of the Western Lakes Plains". \textit{IRIAN: Bulletin of Irian Jaya} XXI:1 -- 17

\ipa{k} \change\ \ipa{G} / V_V\\
\ipa{b} \change\ \ipa{B}\raisebox{-0.6ex}{\textasciitilde}\ipa{m} / \#_V[-high]\\
\ipa{b d g} \change\ \ipa{B}\raisebox{-0.6ex}{\textasciitilde}\ipa{w} \ipa{R}\raisebox{-0.6ex}{\textasciitilde}\O\ \ipa{G} / V_V\\
\ipa{ti} \change\ \ipa{s} / \#_V\\
\ipa{t d} \change\ \ipa{t}\raisebox{-0.6ex}{\textasciitilde}\ipa{R}\raisebox{-0.6ex}{\textasciitilde}\ipa{n} \ipa{n} / \#_\\
\ipa{iiE} V\ipa{diE} \change\ \ipa{ijE Be}

\subsubsection{Proto-Far West to Saponi}{\it Pogostick Man}, from Clouse, Duane (1993), ``Languages of the Western Lakes Plains". \textit{IRIAN: Bulletin of Irian Jaya} XXI:1 -- 17

\ipa{p d} \change\ \ipa{p}\raisebox{-0.6ex}{\textasciitilde}\ipa{f n} / \#_\\
\ipa{ti} \change\ \ipa{s} / \#_V\\
\ipa{b} \change\ \ipa{B}\raisebox{-0.6ex}{\textasciitilde}\ipa{m} / \#_V[+ low]\\
\ipa{p b d k} \change\ \ipa{p}\raisebox{-0.6ex}{\textasciitilde}\ipa{f w R g}\raisebox{-0.6ex}{\textasciitilde}\ipa{G} / V_V\\
\ipa{iiE} V\ipa{diE} \change\ \ipa{dzE RE}

\subsubsection{Proto-Far West to Rasawa}{\it Pogostick Man}, from Clouse, Duane (1993), ``Languages of the Western Lakes Plains". \textit{IRIAN: Bulletin of Irian Jaya} XXI:1 -- 17

\ipa{p b d k} \change\ \ipa{F B R x}\raisebox{-0.6ex}{\textasciitilde}\ipa{k} / V_V\\
\ipa{ti} \change\ \ipa{s} / \#_V\\
\ipa{b} \change\ \ipa{B}\raisebox{-0.6ex}{\textasciitilde}\ipa{m} / \#_V[+ low]\\
\ipa{iiE} V\ipa{diE} \change\ \ipa{ijE Bie}

\subsection{Proto-Lakes Plain to Proto-Tariku}{\it Pogostick Man}, from Clouse, Duane (1993), ``Languages of the Western Lakes Plains". \textit{IRIAN: Bulletin of Irian Jaya} XXI:1 -- 17

\ipa{p d} \change\ \ipa{F R} / V_V

\subsubsection{Proto-Tariku to Proto-Central Tariku}{\it Pogostick Man}, from Clouse, Duane (1993), ``Languages of the Western Lakes Plains". \textit{IRIAN: Bulletin of Irian Jaya} XXI:1 -- 17

\O\ \change\ echo vowel / C_CV\\
\ipa{ku} \change\ \ipa{b}\\
\ipa{p k} \change\ \ipa{F} \O\raisebox{-0.6ex}{\textasciitilde}\ipa{k}\\
\ipa{ti} \change\ \ipa{s}\raisebox{-0.6ex}{\textasciitilde}\ipa{ti} / _V\\
\ipa{d} \change\ \ipa{R}\raisebox{-0.6ex}{\textasciitilde}\ipa{d} / V_V\\
\ipa{i}C \ipa{u}C \change\ \ipa{\^\i\ \^u} / _\{C,\#\}\\
\ipa{a} \change\ \ipa{e} (?)\\
\ipa{e o} \change\ \ipa{E O}

\paragraph{Proto-Central Tariku to Edopi}{\it Pogostick Man}, from Clouse, Duane (1993), ``Languages of the Western Lakes Plains". \textit{IRIAN: Bulletin of Irian Jaya} XXI:1 -- 17

C \change\ \O\ / _\#\\
\ipa{F} \change\ \ipa{h}\\
\ipa{b k} \change\ \ipa{m}\raisebox{-0.6ex}{\textasciitilde}\ipa{b} \O\ / \#_\\
\ipa{d} \change\ \ipa{d}\raisebox{-0.6ex}{\textasciitilde}\ipa{n}\raisebox{-0.6ex}{\textasciitilde}\ipa{l} / \#_\ipa{a}\\
\ipa{d} \change\ \ipa{dz} / _\ipa{i}\\
\ipa{s} \change\ \ipa{s}\raisebox{-0.6ex}{\textasciitilde}\ipa{t}\\
\O\ \change\ \ipa{dz} / _\ipa{\^\i}\\
\ipa{E} \change\ \ipa{e}

\paragraph{Proto-Central Tariku to Iau}{\it Pogostick Man}, from Clouse, Duane (1993), ``Languages of the Western Lakes Plains". \textit{IRIAN: Bulletin of Irian Jaya} XXI:1 -- 17

CV \change\ \O\ / _\#\\
\ipa{F} \change\ \ipa{F}\raisebox{-0.6ex}{\textasciitilde}\ipa{h}\\
\ipa{b k} \change\ \ipa{m}\raisebox{-0.6ex}{\textasciitilde}\ipa{b} \O\ / \#_\\
\ipa{d} \change\ \ipa{d}\raisebox{-0.6ex}{\textasciitilde}\ipa{l}\raisebox{-0.6ex}{\textasciitilde}\ipa{n} / \#_\ipa{a}\\
\ipa{R} \change\ \O\ / V_V\\
``Some vowel coalescence takes place following the above; the author notes that the vowels often take on the tonal characteristics of the absorbed vowel"\\
\ipa{au} \change\ \ipa{O}

\subsubsection{Proto-Tariku to Proto-East Tariku}{\it Pogostick Man}, from Clouse, Duane (1993), ``Languages of the Western Lakes Plains". \textit{IRIAN: Bulletin of Irian Jaya} XXI:1 -- 17

\ipa{ti} \change\ \ipa{s}\raisebox{-0.6ex}{\textasciitilde}\ipa{ti} / _V\\
\ipa{R} \change\ \ipa{R}\raisebox{-0.6ex}{\textasciitilde}\O\ / V_V\\
\ipa{a} \change\ \ipa{e} (?)\\
\ipa{e} \change\ \ipa{E}

\paragraph{Proto-East Tariku to Biritai}{\it Pogostick Man}, from Clouse, Duane (1993), ``Languages of the Western Lakes Plains". \textit{IRIAN: Bulletin of Irian Jaya} XXI:1 -- 17

\ipa{p} \change\ \ipa{h}\raisebox{-0.6ex}{\textasciitilde}\ipa{F}\\
C \change\ \O\ / _\#\\
V \change\ \O\ / \ipa{di}_\\
\ipa{i}C \ipa{u}C \change\ \ipa{\^\i\ \^u} / _\{C,\#\}\\
\O\ \change\ \ipa{dz} / \ipa{\^\i}_V\\
\ipa{ku} \change\ \ipa{b}\\
\ipa{k} \change\ \O\ / _\^{V}

\paragraph{Proto-East Tariku to Doutai}{\it Pogostick Man}, from Clouse, Duane (1993), ``Languages of the Western Lakes Plains". \textit{IRIAN: Bulletin of Irian Jaya} XXI:1 -- 17

\ipa{p} \change\ \ipa{p}\raisebox{-0.6ex}{\textasciitilde}\ipa{F}\\
C \change\ \O\ / _\#\\
\ipa{di} \change\ \ipa{dz} / _V\\
\ipa{i}C \ipa{u}C \change\ \ipa{\^\i\ \^u} / _\{C,\#\}\\
V \change\ \ipa{dz} / _\^{V}\\
\ipa{R} \change\ \O\ / ! _C

\paragraph{Proto-East Tariku to Eritai}{\it Pogostick Man}, from Clouse, Duane (1993), ``Languages of the Western Lakes Plains". \textit{IRIAN: Bulletin of Irian Jaya} XXI:1 -- 17

\ipa{p} \change\ \ipa{p}\raisebox{-0.6ex}{\textasciitilde}\ipa{h}\\
\ipa{b} C \change\ \O\ \ipa{d} / _\#\\
\ipa{di}V \change\ \ipa{dz}\\
\ipa{i}C \ipa{u}C \change\ \ipa{\^\i\ \^u} / _\{C,\#\}\\
\O\ \change\ \ipa{dz} \change\ \ipa{\^\i}_V\\
\ipa{i}C \change\ \ipa{i}C\raisebox{-0.6ex}{\textasciitilde}\ipa{i} / _\{C,\#\} (not sure how this plays in with the change mentioned earlier about extra-high vowels; I must have misread something)

\paragraph{Proto-East Tariku to Kai}{\it Pogostick Man}, from Clouse, Duane (1993), ``Languages of the Western Lakes Plains". \textit{IRIAN: Bulletin of Irian Jaya} XXI:1 -- 17

\ipa{p} \change\ \ipa{F}\\
C \change\ \O\ / _\#\\
V \change\ \O\ / \ipa{di}_\\
\ipa{i}C \ipa{u}C \change\ \ipa{\^\i\ \^u} / _\{C,\#\}\\
\O\ \change\ \ipa{dz} / \ipa{\^\i}_V

\paragraph{Proto-East Tariku to Obokuitai}{\it Pogostick Man}, from Clouse, Duane (1993), ``Languages of the Western Lakes Plains". \textit{IRIAN: Bulletin of Irian Jaya} XXI:1 -- 17

\ipa{p} \change\ \ipa{F}\raisebox{-0.6ex}{\textasciitilde}\ipa{h}\\
\ipa{b} C \change\ \ipa{b\textcorner\ g\textcorner} / _\#\\
V \change\ \O\ / \ipa{di}_\\
\ipa{i}C \ipa{u}C \change\ \ipa{\^\i\ \^u} / _\{C,\#\}\\
\O\ \change\ \ipa{dz} / \ipa{\^\i}V

\paragraph{Proto-East Tariku to Sikaritai}{\it Pogostick Man}, from Clouse, Duane (1993), ``Languages of the Western Lakes Plains". \textit{IRIAN: Bulletin of Irian Jaya} XXI:1 -- 17

\ipa{p} \change\ \ipa{p}\raisebox{-0.6ex}{\textasciitilde}\ipa{h}\\
\ipa{b} C \change\ \ipa{b}\raisebox{-0.6ex}{\textasciitilde}\O\ \{\ipa{d,g}\} / _\#\\
\ipa{di}V \change\ \ipa{dz}\\
\ipa{i}C \ipa{u}C \change\ \ipa{\^\i\ \^u} / _\{C,\#\}\\
\O\ \change\ \ipa{dz} / \ipa{\^\i}_V\\
\ipa{ik} \change\ \ipa{g} / \{\ipa{s,k,p}\}_ ?

\paragraph{Proto-East Tariku to Waritai}{\it Pogostick Man}, from Clouse, Duane (1993), ``Languages of the Western Lakes Plains". \textit{IRIAN: Bulletin of Irian Jaya} XXI:1 -- 17

\ipa{p d} \change\ \ipa{p}\raisebox{-0.6ex}{\textasciitilde}\ipa{Fs d}\raisebox{-0.6ex}{\textasciitilde}\ipa{t}\\
V \change\ \O\ / \ipa{di}_\\
C \change\ \O\ / _\#\\
\ipa{i}C \ipa{u}C \change\ \ipa{\^\i\ \^u} / _\{C,\#\}\\
\ipa{R} \change\ \O\ / ! C_\\
\ipa{k} \change\ \O\ / _\^{V}\\
\O\ \change\ \ipa{dz} / \ipa{\^\i}_V\\
\ipa{ik} \change\ \ipa{g} / \{\ipa{s,p}\}_ ?

\subsubsection{Proto-Tariku to Proto-West Tariku}{\it Pogostick Man}, from Clouse, Duane (1993), ``Languages of the Western Lakes Plains". \textit{IRIAN: Bulletin of Irian Jaya} XXI:1 -- 17

\ipa{p} \change\ \ipa{F}\\
\ipa{R} \change\ \ipa{R}\raisebox{-0.6ex}{\textasciitilde}\O\ / V_V\\
\ipa{k} \change\ \ipa{k}\raisebox{-0.6ex}{\textasciitilde}\O\\
\ipa{i}C \ipa{u}C \change\ \ipa{\^\i\ \^u} / _\{C,\#\}

\paragraph{Proto-West Tariku to Deirate}{\it Pogostick Man}, from Clouse, Duane (1993), ``Languages of the Western Lakes Plains". \textit{IRIAN: Bulletin of Irian Jaya} XXI:1 -- 17

\ipa{p} \change\ \ipa{F}\raisebox{-0.6ex}{\textasciitilde}\ipa{h}\\
\ipa{b d k} \change\ \ipa{b}\raisebox{-0.6ex}{\textasciitilde}\ipa{B R}\raisebox{-0.6ex}{\textasciitilde}\ipa{l}\raisebox{-0.6ex}{\textasciitilde}\O\ \ipa{k}\raisebox{-0.6ex}{\textasciitilde}\ipa{x}\raisebox{-0.6ex}{\textasciitilde}\ipa{g}\raisebox{-0.6ex}{\textasciitilde}\ipa{G} / V_V\\
\ipa{b} \change\ \ipa{m} / \#_\ipa{a}\\
\ipa{b} \change\ \ipa{b}\raisebox{-0.6ex}{\textasciitilde}\ipa{\super mb}\\
\ipa{ti di} \change\ \ipa{s dz} / _V

\paragraph{Proto-West Tariku to Faia}{\it Pogostick Man}, from Clouse, Duane (1993), ``Languages of the Western Lakes Plains". \textit{IRIAN: Bulletin of Irian Jaya} XXI:1 -- 17

\ipa{p} \change\ \ipa{F}\raisebox{-0.6ex}{\textasciitilde}\ipa{h}\\
\ipa{b d k} \change\ \ipa{b}\raisebox{-0.6ex}{\textasciitilde}\ipa{B R}\raisebox{-0.6ex}{\textasciitilde}\O\ \ipa{k}\raisebox{-0.6ex}{\textasciitilde}\ipa{x}\raisebox{-0.6ex}{\textasciitilde}\ipa{g}\raisebox{-0.6ex}{\textasciitilde}\ipa{G} / V_V\\
\ipa{b d} \change\ \ipa{m n} / \#_\ipa{a}\\
\ipa{ti} \change\ \ipa{s} / _V

\paragraph{Proto-West Tariku to Fayu}{\it Pogostick Man}, from Clouse, Duane (1993), ``Languages of the Western Lakes Plains". \textit{IRIAN: Bulletin of Irian Jaya} XXI:1 -- 17

\ipa{p} \change\ \ipa{F}\raisebox{-0.6ex}{\textasciitilde}\ipa{h}\\
\ipa{b d k} \change\ \ipa{b}\raisebox{-0.6ex}{\textasciitilde}\ipa{B R}\raisebox{-0.6ex}{\textasciitilde}\O\ \ipa{k}\raisebox{-0.6ex}{\textasciitilde}\ipa{x}\raisebox{-0.6ex}{\textasciitilde}\ipa{g}\raisebox{-0.6ex}{\textasciitilde}\ipa{G} / V_V\\
\ipa{b d} \change\ \ipa{m n} / \#_\ipa{a}\\
\ipa{b d} \change\ \ipa{b}\raisebox{-0.6ex}{\textasciitilde}\ipa{\super mb d}\raisebox{-0.6ex}{\textasciitilde}\ipa{\super nd}\\
\ipa{ti di} \change\ \ipa{s dz} / _V

\paragraph{Proto-West Tariku to Kirikiri}{\it Pogostick Man}, from Clouse, Duane (1993), ``Languages of the Western Lakes Plains". \textit{IRIAN: Bulletin of Irian Jaya} XXI:1 -- 17

\ipa{p} \change\ \ipa{F}\raisebox{-0.6ex}{\textasciitilde}\ipa{h}\\
\ipa{b d k} \change\ \ipa{b}\raisebox{-0.6ex}{\textasciitilde}\ipa{B R}\raisebox{-0.6ex}{\textasciitilde}\ipa{l}\raisebox{-0.6ex}{\textasciitilde}\O\ \ipa{k}\raisebox{-0.6ex}{\textasciitilde}\ipa{x}\raisebox{-0.6ex}{\textasciitilde}\ipa{g}\raisebox{-0.6ex}{\textasciitilde}\ipa{G} / V_V\\
\ipa{b d} \change\ \ipa{m n} / \#_\ipa{a}\\
\ipa{b d} \change\ \ipa{b}\raisebox{-0.6ex}{\textasciitilde}\ipa{\super mb d}\raisebox{-0.6ex}{\textasciitilde}\ipa{\super nd}\\
\ipa{ti} \change\ \ipa{s} / _V

\paragraph{Proto-West Tariku to Sehudate}{\it Pogostick Man}, from Clouse, Duane (1993), ``Languages of the Western Lakes Plains". \textit{IRIAN: Bulletin of Irian Jaya} XXI:1 -- 17

\ipa{p} \change\ \ipa{F}\raisebox{-0.6ex}{\textasciitilde}\ipa{h}\\
\ipa{b d k} \change\ \ipa{b}\raisebox{-0.6ex}{\textasciitilde}\ipa{B R}\raisebox{-0.6ex}{\textasciitilde}\O\ \ipa{k}\raisebox{-0.6ex}{\textasciitilde}\ipa{x}\raisebox{-0.6ex}{\textasciitilde}\ipa{g}\raisebox{-0.6ex}{\textasciitilde}\ipa{G} / V_V\\
\ipa{b} \change\ \ipa{m} / \#_\ipa{a}\\
\ipa{b} \change\ \ipa{b}\raisebox{-0.6ex}{\textasciitilde}\ipa{\super mb}\\
\ipa{ti di} \change\ \ipa{s dz} / _V

\paragraph{Proto-West Tariku to Tause}{\it Pogostick Man}, from Clouse, Duane (1993), ``Languages of the Western Lakes Plains". \textit{IRIAN: Bulletin of Irian Jaya} XXI:1 -- 17

\ipa{p} \change\ \ipa{F}\raisebox{-0.6ex}{\textasciitilde}\ipa{h}\\
\ipa{b d k} \change\ \ipa{b}\raisebox{-0.6ex}{\textasciitilde}\ipa{B R}\raisebox{-0.6ex}{\textasciitilde}\O\ \ipa{k}\raisebox{-0.6ex}{\textasciitilde}\ipa{x}\raisebox{-0.6ex}{\textasciitilde}\ipa{g}\raisebox{-0.6ex}{\textasciitilde}\ipa{G} / V_V\\
\ipa{b} \change\ \ipa{m} / \#_\ipa{a}\\
\ipa{b d} \change\ \ipa{b}\raisebox{-0.6ex}{\textasciitilde}\ipa{\super mb d}\raisebox{-0.6ex}{\textasciitilde}\ipa{\super nd}\\
\ipa{ti di} \change\ \ipa{s j} / _V

\paragraph{Proto-West Tariku to Weirate}{\it Pogostick Man}, from Clouse, Duane (1993), ``Languages of the Western Lakes Plains". \textit{IRIAN: Bulletin of Irian Jaya} XXI:1 -- 17

\ipa{p} \change\ \ipa{F}\raisebox{-0.6ex}{\textasciitilde}\ipa{h}\\
\ipa{b d k} \change\ \ipa{b}\raisebox{-0.6ex}{\textasciitilde}\ipa{B R}\raisebox{-0.6ex}{\textasciitilde}\ipa{l}\raisebox{-0.6ex}{\textasciitilde}\O\ \ipa{k}\raisebox{-0.6ex}{\textasciitilde}\ipa{x}\raisebox{-0.6ex}{\textasciitilde}\ipa{g}\raisebox{-0.6ex}{\textasciitilde}\ipa{G} / V_V\\
\ipa{b d} \change\ \ipa{m n} / \#_\ipa{a}\\
\ipa{b d} \change\ \ipa{b}\raisebox{-0.6ex}{\textasciitilde}\ipa{\super mb d}\raisebox{-0.6ex}{\textasciitilde}\ipa{\super nd}\\
\ipa{ti di} \change\ \ipa{s dz} / _V

\clearpage

\section{Macro-Arawakan}\tab Dixon (2004) gives the following reconstruction for Proto-Araw\'{a}:

\begin{center}
\begin{tabular}{c | c c c c c}
& Bilabial & Coronal & Postalveolar & Velar & Glottal \\ \hline
Nasal & \ipa{m} & \ipa{n}\\
Plosive & \ipa{p p\super h b \!b} & \ipa{t t\super h d \!d} & & \ipa{k k\super h g \!g} & \ipa{P}\\
Fricative & & \ipa{s} & & & \ipa{h}\\
Affricate & & \ipa{ts ts\super h dz} & \ipa{tS}\\
Liquid & & \ipa{r}\end{tabular}

\begin{tabular}{c | c c c}
& Front & Central & Back\\ \hline
High & \ipa{i}\\
Mid & \ipa{e} & & \ipa{o}\\
Low & & \ipa{a}\end{tabular}
\end{center}

\tab Dixon states ``[i]t is likely that, as in modern languages, {\it *o} ranged over [u] and [o]". Further, he notes that ``{\it *w} could have been a voiced bilabial fricative [\ipa{B}], or a labial-velar semivowel [w] (or could have had both as allophones), and {\it *j} could have been a voiced alveolar affricate [dz], a voiced postalveolar affricate [\ipa{dZ}], or a voiced palatal stop [\ipa{\textbardotlessj}]".

\tab Dixon also hypothesizes that there was an Araw\'{a}n language that eventually died out after gaining substrate status; he has attempted to identify sound changes from Proto-Araw\'{a} to this hypothetical language. While most likely incomplete his findings are presented below along with those of the other languages.

\tab (From Dixon, R.M.W. (2004), ``Proto-Araw\'{a} Phonology". {\it Anthropological Linguistics} 46(1):1 -- 83)

\subsection{Proto-Araw\'{a} to Araw\'{a}}{\it Pogostick Man}, from Dixon, R.M.W. (2004), ``Proto-Araw\'{a} Phonology". {\it Anthropological Linguistics} 46(1):1 -- 83

\ipa{p} \change\ \ipa{f}\\
\ipa{p\super h} \change\ \ipa{p} / \#_\\
\ipa{p\super h} \change\ \ipa{F} / else\\
\ipa{\!d} \change\ \ipa{t} / \#_\\
\ipa{dz} \change\ \ipa{s} / medially\\
\ipa{ts\super h} \change\ \ipa{s}

\subsection{Proto-Araw\'{a} to Banaw\'{a}-Jamamadi}{\it chris_notts \& Pogostick Man}, the latter citing Dixon, R.M.W. (2004), ``Proto-Araw\'{a} Phonology". {\it Anthropological Linguistics} 46(1):1 -- 83

\ipa{\!b} \change\ \ipa{b} / \#_\\
\ipa{\!b} \change\ \ipa{F} / medially\\
\ipa{p(\super h)} \change\ \ipa{F}\\
\{\ipa{t\super h,\!d}\} \change\ \ipa{t}\\
\ipa{k\super h g} \change\ \ipa{k w}\\
\{\ipa{ts(\super h),tS}\} \ipa{dz} \change\ \ipa{s \textbardotlessj}\\
\ipa{P} \change\ \O

\subsection{Proto-Araw\'{a} to Hypothetical Araw\'{a}n Substrate}{\it Pogostick Man}, from Dixon, R.M.W. (2004), ``Proto-Araw\'{a} Phonology". {\it Anthropological Linguistics} 46(1):1 -- 83

\ipa{e} \change\ \ipa{a} / medially\\
\ipa{e} \change\ \ipa{i} / \#_\\
\ipa{p} \change\ \ipa{P}\\
\ipa{p\super h} \change\ \ipa{h} / medially\\
\ipa{dz} \change\ \ipa{s}

\subsection{Proto-Araw\'{a} to Jarawara}{\it chris_notts \& Pogostick Man}, the latter citing Dixon, R.M.W. (2004), ``Proto-Araw\'{a} Phonology". {\it Anthropological Linguistics} 46(1):1 -- 83

\ipa{\!b} \change\ \ipa{b} / \#_
\ipa{\!b} \change\ \ipa{f} / medially\\
\ipa{\!d} \change\ \ipa{t}\\
\ipa{p(\super h)} \{\ipa{t\super h,d}\} \change\ \ipa{F t}\\
\ipa{k\super h g} \change\ \ipa{k w}\\
\ipa{dz} \change\ \ipa{\textbardotlessj}\\
\{\ipa{ts(\super h),tS}\} \change\ \ipa{s}\\
\ipa{P} \change\ \O

\subsection{Proto-Araw\'{a} to Kul\'{i}na-Den\'{i}}{\it Pogostick Man}, from Dixon, R.M.W. (2004), ``Proto-Araw\'{a} Phonology". {\it Anthropological Linguistics} 46(1):1 -- 83

\ipa{\!b} \change\ \ipa{b} / \#_\\
\ipa{\!b} \change\ \ipa{p} / medially\\
\ipa{\!d} \change\ \ipa{t}\\
\ipa{g} \change\ \ipa{w}\\
\ipa{tS} \change\ \ipa{S} (?)\\
\ipa{P} \change\ \O\ / \#_

\subsection{Proto-Araw\'{a} to Sorowah\'{a}}{\it Pogostick Man}, from Dixon, R.M.W. (2004), ``Proto-Araw\'{a} Phonology". {\it Anthropological Linguistics} 46(1):1 -- 83

\ipa{e} \change\ \ipa{1} / _\#\\
\ipa{e} \change\ \ipa{a} / else\\
\{\ipa{p(\super h),\!b}\} \change\ \ipa{h}\\
\ipa{\!d} \change\ \ipa{d} / \#_\\
\{\ipa{t\super h,\!d} \change\ \ipa{t}\\
\ipa{k\super h} \change\ \ipa{k}\\
\ipa{ts(\super h)} \change\ \ipa{s}
\ipa{P} \change\ \O\ / \#_, possibly everywhere?

\subsection{Proto-Araw\'{a} to Paumar\'{i}}{\it Pogostick Man}, from Dixon, R.M.W. (2004), ``Proto-Araw\'{a} Phonology". {\it Anthropological Linguistics} 46(1):1 -- 83

\ipa{e} \change\ \ipa{a} / medially\\
\ipa{e} \change\ \{\ipa{a,i}\} / \#_\\
\ipa{p\super h} \change\ \ipa{p} / \#_\\
\ipa{p\super h t} \change\ \ipa{F P} / medial\\
\ipa{ts(\super h)} \change\ \ipa{s}

\clearpage

\section{Macro-Chibchan}

\subsection{Lenmichian}The following inventory for Proto-Lenmichian is posited by Constenla (2005).

\begin{center}\begin{tabular}{c | c c c c}
& Bilabial & Alveolar & Velar & Glottal \\ \hline
Stop & \ipa{b} & \ipa{d t} & \ipa{k} & \ipa{P}\\
Affricate & & \ipa{ts}\\
Fricative & & \ipa{s} & & \ipa{h}\\
Approximant & \ipa{w} & \ipa{R}\\
Lateral & & \ipa{l}\end{tabular}

\begin{tabular}{c | c c c}
& Front & Central & Back\\ \hline
High & \ipa{i} & & \ipa{u}\\
Mid & \ipa{e} & & \ipa{o}\\
Low & & \ipa{a}\end{tabular}\end{center}

(From Wikipedia contributors (2015), ``Macro-Chibchan languages''. {\it Wikipedia, the Free Encyclopedia}. \textless\url{https://en.wikipedia.org/w/index.php?title=Macro-Chibchan_languages&oldid=672637970}\textgreater, presumably citing Constenla Uma�a, Adolfo (2005), ``?`Existe relaci\'{o}n geneal\'{o}gica entre las lenguas misumalpas y las chibchenses?''. {\it Estudios de Ling��stica Chibcha} 24:7 -- 85)

\subsubsection{Proto-Lenmichian to Proto-Chibchan}{\it Pogostick Man}, from Wikipedia contributors (2015), ``Macro-Chibchan languages''. {\it Wikipedia, the Free Encyclopedia}. \textless\url{https://en.wikipedia.org/w/index.php?title=Macro-Chibchan_languages&oldid=672637970}\textgreater, presumably citing Constenla Uma�a, Adolfo (2005), ``?`Existe relaci\'{o}n geneal\'{o}gica entre las lenguas misumalpas y las chibchenses?''. {\it Estudios de Ling��stica Chibcha} 24:7 -- 85

\ipa{l} \change\ \ipa{R}\\
\ipa{w} \change\ \O

\paragraph{Chibchan}The following phonemic inventory is adapted from Wheeler (1972).

\begin{center}
\begin{tabular}{c | c c c c c}
& Labial & Alveolar & Palatal & Velar & Glottal\\ \hline
Nasal & \ipa{m} & \ipa{n}\\
Stop & \ipa{p b} & \ipa{t d} & & \ipa{k k\super w g g\super w}\\
Affricate & & \ipa{ts} & \ipa{tS}\\
Fricative & & \ipa{s} & & \ipa{x} & \ipa{h}\\
Glide & \ipa{w} & & & \ipa{j}\end{tabular}

\begin{tabular}{c | c c c}
& Front & Central & Back\\ \hline
High & \ipa{i} & & \ipa{u}\\
Mid & \ipa{e} & & \ipa{o}\\
Low & & \ipa{a}\end{tabular}
\end{center}

Information in this section may be missing or incomplete, as I found the source document using Google Books and several pages were not available in the preview.

(From Wheeler, Alva (1972), ``Proto-Chibchan". In Matteson, Esther, ed., {\it Comparative Studies in Amerindian Languages} 93 -- 108)

\subparagraph{Proto-Chibchan to Arhuaco}{\it Pogostick Man}, from Wheeler, Alva (1972), ``Proto-Chibchan". In Matteson, Esther, ed., {\it Comparative Studies in Amerindian Languages} 93 -- 108

\ipa{s} \change\ \ipa{kh} / _(V)\ipa{k}\\
\ipa{s d} \change\ \ipa{z r} / _(V)\ipa{j}\\
\ipa{ts h} \change\ \ipa{s} \O\\
\O\ \change\ \ipa{u} / \ipa{w}_V\\
\ipa{j} \change\ \O\ / \ipa{i}_V\\
\ipa{j} \change\ \{\ipa{j},\O\}\\
\ipa{e} \change\ \ipa{a}\\
\ipa{i} \change\ \O\ / \ipa{a}_\\
\ipa{ia} \change\ \ipa{@} (``unspecified'')\\
\ipa{i} \change\ \O\ / C"V(C)_\\
\ipa{i} \change\ \O\ / _(C)"V

\subparagraph{Proto-Chibchan to Chibcha}{\it Pogostick Man}, from Wheeler, Alva (1972), ``Proto-Chibchan". In Matteson, Esther, ed., {\it Comparative Studies in Amerindian Languages} 93 -- 108

\ipa{b} \change\ \ipa{p} / _V\ipa{k\super w}\\
V\ipa{s} \change\ \ipa{h} / \ipa{n}_\\
V \change\ \O\ / \ipa{s}_\ipa{j} (sometimes)\\
\ipa{s} \change\ \ipa{h} / V_V\\
\ipa{ts} \change\ \{\ipa{s,z}\}\\
\ipa{w} \change\ \{\ipa{w,}\O\}\\
\ipa{j} \change\ \O\\
\ipa{n} \change\ \{\ipa{n,}\O\}\\
\ipa{a} \change\ \O\ / _\ipa{i}, when unstressed\\
\ipa{a} \change\ \ipa{i} / _C\ipa{i}\\
\ipa{a} \change\ \O\ / _\ipa{u}\\
\ipa{u} \change\ \ipa{o} / _C\ipa{a}\\
\ipa{u} \change\ \O\ / _\ipa{a}\\
\ipa{i} \change\ \ipa{u} / _(C)\ipa{u}\\
\ipa{i} \change\ \ipa{a} / _(C)\ipa{a}\\
\ipa{i} \change\ \ipa{e} / _(C)\ipa{e}\\
\ipa{a \{e,i\}} \change\ \ipa{i} \O\ / C"V(C)_\\
\ipa{i} \change\ \O\ / _(C)"V\\
\ipa{a} \change\ \O\ / _"V

\subparagraph{Proto-Chibchan to Kogi}{\it Pogostick Man}, from Wheeler, Alva (1972), ``Proto-Chibchan". In Matteson, Esther, ed., {\it Comparative Studies in Amerindian Languages} 93 -- 108

\ipa{s} \change\ \O\ / \#_\\
\ipa{s} \change\ \{\ipa{S,tS}\} / _\ipa{i}\\
\ipa{s} \change\ \ipa{h} / _V\ipa{n}\\
\ipa{n}V\ipa{s} \ipa{d}V\ipa{s} \change\ \ipa{n}(V(\ipa{s})) \ipa{Z}(V\ipa{Z})\\
\ipa{s h} \change\ \{\ipa{s,tS}\} \{\ipa{h},\O\}\\
\ipa{j} \change\ \ipa{i} / C(V)_\\
\ipa{n j} \change\ \{\ipa{n},\O\} \{\ipa{j},\O\}\\
\ipa{e} \change\ \ipa{a}\\
\ipa{i} \change\ \O\ / \ipa{a}_ (sometimes)\\
\ipa{u} \change\ \ipa{w} / _\ipa{a}

\subparagraph{Proto-Chibchan to Marocacero}{\it Pogostick Man}, from Wheeler, Alva (1972), ``Proto-Chibchan". In Matteson, Esther, ed., {\it Comparative Studies in Amerindian Languages} 93 -- 108

\ipa{s} \change\ \{\ipa{ts,S}\} / _\ipa{i}\\
\ipa{s} \change\ \{\ipa{s,ts}\}\\
\ipa{d}(V)\ipa{j} \ipa{s}(V)\ipa{j} \change\ \ipa{l\super j dZ}\\
\ipa{d} \change\ \ipa{l}\\
\ipa{a} \change\ \O\ / _\ipa{i}\\
\ipa{e} \change\ \ipa{a}\\
\ipa{i} \change\ \ipa{@} / \ipa{o}C_\\
\ipa{i} \change\ \O\ / C"V(C)_\\
\ipa{i} \change\ \O\ / _(C)"V

\subparagraph{Proto-Chibchan to Motil\'on}{\it Pogostick Man}, from Wheeler, Alva (1972), ``Proto-Chibchan". In Matteson, Esther, ed., {\it Comparative Studies in Amerindian Languages} 93 -- 108

\ipa{g} \change\ \O\\
\ipa{s} \change\ \{\ipa{s,x,tS}\raisebox{-0.6ex}{\textasciitilde}\ipa{S}\}\\
\ipa{s} \change\ \{\ipa{S,tS}\} / _\ipa{i}\\
\ipa{s} \change\ \ipa{x} / \ipa{i}_\\
\ipa{h} \change\ \O\\
\{\ipa{w,m}\} \change\ \ipa{b}\\
\ipa{s}(V)\ipa{d} \change\ \ipa{d\super j}\\
\ipa{d} \change\ \O\ / V_\ipa{s}\\
\ipa{e} \change\ \ipa{a}\\
\ipa{i} \change\ \O\ / \ipa{a}_ (usually; sometimes the diphthong is retained or \change\ \ipa{aj})\\
\ipa{a} \change\ \ipa{i} / \ipa{i}C_\\
\ipa{u} \change\ \O\ / \ipa{a}_\\
\ipa{u} \change\ \O\ / _\ipa{a}\\
\{\ipa{ue,aja}\} \change\ \ipa{@}\\
\{\ipa{u,i}\} \change\ \O\ / C"V(C)_\\
\ipa{i} \change\ \O\ / _(C)"V\\
\ipa{e} \change\ \ipa{u} / \ipa{u}C_\\
\ipa{e} \change\ \O\ / \ipa{u}_C

\subparagraph{Proto-Chibchan to Tunebo}{\it Pogostick Man}, from Wheeler, Alva (1972), ``Proto-Chibchan". In Matteson, Esther, ed., {\it Comparative Studies in Amerindian Languages} 93 -- 108

\ipa{g g\super w} \change\ \O\ \ipa{b}\\
\{\ipa{d}(V)\ipa{s,n}(V)\ipa{j}\} \ipa{n}(V)\ipa{s} \change\ \ipa{r} \{V\ipa{s}V,\ipa{r}\}\\
\ipa{d}(V)\ipa{j} \change\ \ipa{r}(V)\\
\ipa{d} \change\ \ipa{r}\\
\ipa{s} \change\ \ipa{h} / _(V)C\\
\ipa{ts tS} \change\ \ipa{tS s}\\
\ipa{m n h j} \change\ \ipa{b \{n,r,}\O\} \{\ipa{h},\O\} \{\ipa{j},\O\}\\
\ipa{e} \change\ \ipa{a}\\
\ipa{ai} \change\ \ipa{e} / if the *\ipa{a} is not stressed\\
\ipa{a} \change\ \O\ / _\ipa{u}\\
\ipa{i} \change\ \ipa{a} / _C\ipa{a}\\
\{\ipa{a,i}\} \change\ \O\ / C"V(C)_\\
\ipa{i} \change\ \O\ / _(C)"V\\
\ipa{e} \change\ \O\ / \ipa{u}_C\\
\ipa{e} \change\ \ipa{i} / ``conditioning undetermined''

\subsubsection{Proto-Lenmichian to Proto-Lencan}{\it Pogostick Man}, from Wikipedia contributors (2015), ``Macro-Chibchan languages''. {\it Wikipedia, the Free Encyclopedia}. \textless\url{https://en.wikipedia.org/w/index.php?title=Macro-Chibchan_languages&oldid=672637970}\textgreater, presumably citing Constenla Uma�a, Adolfo (2005), ``?`Existe relaci\'{o}n geneal\'{o}gica entre las lenguas misumalpas y las chibchenses?''. {\it Estudios de Ling��stica Chibcha} 24:7 -- 85

\ipa{b d} \change\ \ipa{m n} / _V[+ nasal] (I'm inferring this from the statement that ``[t]here are also a series\textellipsis of nasal vowels'')\\
\ipa{b d} \change\ \ipa{p l}\\
\{\ipa{P,h}\} \change\ \O\\
\{\ipa{s,ts}\} \change\ \ipa{ts'}\\
\ipa{o a} \change\ \ipa{\{u,o\} \{a,e\}}

\subsubsection{Proto-Lenmichian to Proto-Misumalpan}{\it Pogostick Man}, from Wikipedia contributors (2015), ``Macro-Chibchan languages''. {\it Wikipedia, the Free Encyclopedia}. \textless\url{https://en.wikipedia.org/w/index.php?title=Macro-Chibchan_languages&oldid=672637970}\textgreater, presumably citing Constenla Uma�a, Adolfo (2005), ``?`Existe relaci\'{o}n geneal\'{o}gica entre las lenguas misumalpas y las chibchenses?''. {\it Estudios de Ling��stica Chibcha} 24:7 -- 85

\ipa{b d} \change\ \ipa{m n} / _V[+ nasal] (I'm inferring this from the statement that ``[t]here are also a series\textellipsis of nasal vowels'')\\
\ipa{b d} \change\ \ipa{\{b,p\} l}\\
\{\ipa{P,h}\} \change\ \O\\
\ipa{ts} \change\ \ipa{s}

\clearpage

\section{Macro-Pama-Nyungan}\tab Wikipedia gives the following reconstruction for the phonological inventory of Proto-Pama-Nyungan, citing Alpher (2004). The particulars of the presentation have been modified somewhat from that presented in the article.

\begin{center}\begin{tabular}{c | c c c c c}
& Bilabial & Alveolar & Retroflex & Palatal & Velar \\ \hline
Plosive & \ipa{p} & \ipa{t} & \ipa{\:t} & \ipa{c (c\super j?)} & \ipa{k}\\
Nasal & \ipa{m} & \ipa{n} & \ipa{\:n} & \ipa{\textltailn} & \ipa{N}\\
Rhotic & & \ipa{r} & \ipa{\:r}\\
Lateral & & \ipa{l} & \ipa{\:l} & \ipa{L}\\
Semivowel & \ipa{w} & & & \ipa{j}
\end{tabular}\end{center}

\begin{center}\begin{tabular}{c | c c c}
& Front & Central & Back \\ \hline
High & \ipa{i i:} & & \ipa{u u:} \\
Low & & \ipa{a a:}
\end{tabular}\end{center}

\tab (From Wikipedia contributors (2014), ``Pama--Nyungan languages". {\it Wikipedia, the Free Encyclopedia}. \textless\url{https://en.wikipedia.org/w/index.php?title=Pama%E2%80%93Nyungan_languages&oldid=605755580}\textgreater, presumably citing Alpher, Barry (2004), ``Proto-Pama-Nyungan etyma". In Bowern, Claire, and Harold Koch (eds.), {\it Australian Languages: Classification and the Comparative Method})

\subsection{Paman}\tab {\it NB: These changes are most likely largely incomplete, especially for languages with fewer changes shown.}

\subsubsection{Proto-Paman to Aritin\ipa{N}iti\ipa{G}}{\it Pogostick Man}, from Hale, Kenneth (1964), ``Classification of Northern Paman Languages, Cape York Peninsula, Australia: A Research Report". {\it Oceanic Linguistics} Vol. 3, No. 2, 248 -- 265

C \change\ \O\ / \#_\\
\ipa{i}[-long]C \change\ C\ipa{j} / \#_ ! _\ipa{i}\\
\ipa{u}[-long]C \change\ C\ipa{w} / \#_\\
\ipa{a}[-long]C \change\ C\ipa{a} / \#_ (! _\ipa{a}?)\\
\ipa{u i} \change\ \ipa{w j} / \ipa{a}_ when this \ipa{a} is a result of metathesis (?)\\
V\ipa{:} \change\ V[-long] / in \#U

\subsubsection{Proto-Paman to Aw\ipa{N}tim}{\it Pogostick Man}, from Hale, Kenneth (1964), ``Classification of Northern Paman Languages, Cape York Peninsula, Australia: A Research Report". {\it Oceanic Linguistics} Vol. 3, No. 2, 248 -- 265

\ipa{i}C \ipa{a}C \ipa{u}C \change\ C\ipa{j} C\ipa{a} C\ipa{w} / \#_ ! before an identical vowel\\
\ipa{u i} \change\ \ipa{w j} / \ipa{a}_ when this \ipa{a} is a result of the preceding metathesis\\
(N)S \change\ F / \#(C)V_\\
\O\ \change\ \ipa{j} / \#(C)\ipa{i:}(C)_V ! _\ipa{i}\\
\O\ \change\ \ipa{w} / \#(C)\ipa{u:}(C)_V ! _\ipa{u}\\
\O\ \change\ \ipa{a} / \#(C)\ipa{a:}(C)_V ! _\ipa{a}\\
C \change\ \O\ / \#_\\
V\ipa{:} \change\ \ipa{@} / in \#U

\subsubsection{Proto-Paman to Lin\ipa{N}iti\ipa{G}}{\it Pogostick Man}, from Hale, Kenneth (1964), ``Classification of Northern Paman Languages, Cape York Peninsula, Australia: A Research Report". {\it Oceanic Linguistics} Vol. 3, No. 2, 248 -- 265

(N)S \change\ F / \#(C)V_\\
N \change\ \O\ / \#NV_SV\\
C \change\ \O\ / \#_\\
V\ipa{:} V[-long] \change\ V[-long] \O\ / in \#U

\subsubsection{Proto-Paman to Mbiywom}{\it Pogostick Man}, from Hale, Kenneth (1964), ``Classification of Northern Paman Languages, Cape York Peninsula, Australia: A Research Report". {\it Oceanic Linguistics} Vol. 3, No. 2, 248 -- 265

C \change\ \O\ / \#_\\
\ipa{i}[-long]C \change\ C\ipa{j} / \#_ ! _\ipa{i}\\
\ipa{u}[-long]C \change\ C\ipa{w} / \#_\\
\ipa{a}[-long]C \change\ C\ipa{a} / \#_ (! _\ipa{a}?)\\
\ipa{u i} \change\ \ipa{w j} / \ipa{a}_ when this \ipa{a} is a result of metathesis (?)\\
V\ipa{:} \change\ V[-long] / in \#U

\subsubsection{Proto-Paman to Mpalican}{\it Pogostick Man}, from Hale, Kenneth (1964), ``Classification of Northern Paman Languages, Cape York Peninsula, Australia: A Research Report". {\it Oceanic Linguistics} Vol. 3, No. 2, 248 -- 265

NVS \change\ V\ipa{\super n}S / \#_\\
C \change\ \O\ / \#_\\
V\ipa{:} \change\ V[-long] / in \#U

\subsubsection{Proto-Paman to Ura\ipa{D}i}{\it Pogostick Man}, from Hale, Kenneth (1964), ``Classification of Northern Paman Languages, Cape York Peninsula, Australia: A Research Report". {\it Oceanic Linguistics} Vol. 3, No. 2, 248 -- 265

(N)S \change\ F / \#(C)V_\\
V\ipa{:} \change\ V[-long] / in \#U

\subsubsection{Proto-Paman to Yinwum}{\it Pogostick Man}, from Hale, Kenneth (1964), ``Classification of Northern Paman Languages, Cape York Peninsula, Australia: A Research Report". {\it Oceanic Linguistics} Vol. 3, No. 2, 248 -- 265

\ipa{a} \change\ \ipa{i} / \#C[+palatal]V[-long]C_\\
\ipa{i} \change\ \ipa{e} / \#(C)\ipa{a}C[-palatal]_\\
\#"UU \change\ \#U"U\\
NVS \change\ V\ipa{\super n}S / \#_\\
C \change\ \O\ / \#_\\
\O\ \change\ \ipa{j} / \#\ipa{i:}(C)_\ipa{a}\\
\O\ \change\ \ipa{w} / \#\ipa{u:}(C)_\ipa{a}\\
V\ipa{:} \change\ V[-long] / in \#U

\clearpage

\section{Macro-Panoan}

\subsection{Tacanan}Ritchie (1968) gives the following phonology for Proto-Tacanan. The alveolopalatal series is inferred from the notation and may be incorrect.

\begin{center}
\begin{tabular}{c | c c c c c c}
& Bilabial & Alveolar & Alveolopalatal & Palatal & Velar & Glottal\\ \hline
Nasal & \ipa{m} & \ipa{n}\\
Stop & \ipa{p b} & \ipa{t d} & & & \ipa{k} & \ipa{P}\\
Affricate & & \ipa{ts} & \ipa{tC} & \ipa{tS}\\
Fricative & & \ipa{s} & \ipa{C} & \ipa{S}\\
Approximant & \ipa{w} & \ipa{r} & \ipa{*\'{r}} & \ipa{j}
\end{tabular}

\begin{tabular}{c | c c c}
& Front & Central & Back\\ \hline
High & \ipa{i}\\
Mid & \ipa{e} & & \ipa{o}\\
Low & & \ipa{a}
\end{tabular}
\end{center}

(From Key, Mary Ritchie (1968), {\it Comparative Tacanan Phonology with Cavine\~{n}a Phonology and Notes on Pano-Tacanan Relationship})

\subsubsection{Proto-Tacanan to Cavine\~{n}a}{\it Pogostick Man}, from Key, Mary Ritchie (1968), {\it Comparative Tacanan Phonology with Cavine\~{n}a Phonology and Notes on Pano-Tacanan Relationship}

\ipa{k} \change\ \ipa{k\super w}\\
\ipa{\{C,tC\}} \change\ \ipa{h}\\
\ipa{x} \change\ \ipa{k}\\
\ipa{*\'{r}} \change\ \ipa{r}

\subsubsection{Proto-Tacanan to Chama}{\it Pogostick Man}, from Key, Mary Ritchie (1968), {\it Comparative Tacanan Phonology with Cavine\~{n}a Phonology and Notes on Pano-Tacanan Relationship}

\ipa{b d} \change\ \ipa{\!b \!d}\\
\ipa{\!d} \change\ \{\ipa{P},\O\} / ! \#_\\
\ipa{t k} \change\ \ipa{k k\super w}\\
\ipa{tS} \change\ \ipa{s} / _\ipa{i}\\
\ipa{tC} \change\ \ipa{S}\\
\ipa{s C} \change\ \ipa{D h}\\
\ipa{n} \change\ \ipa{\textltailn} / \ipa{i}_\{\ipa{o,a}\} (the former is conjectured)\\
\ipa{\{r,*\'{r}\}} \change\ \{\O,\ipa{w,j}\}

\subsubsection{Proto-Tacanan to Reyesano}{\it Pogostick Man}, from Key, Mary Ritchie (1968), {\it Comparative Tacanan Phonology with Cavine\~{n}a Phonology and Notes on Pano-Tacanan Relationship}

\ipa{k} \change\ \ipa{k\super w}\\
\ipa{b d} \change\ \ipa{\super mb \super ndz}\\
\ipa{ts} \change\ \ipa{tS} / \#_\\
\ipa{tS C} \change\ \ipa{ts S}\\
\ipa{C x} \change\ \ipa{D h}\\
\ipa{w} \change\ \ipa{B} / _E ?\\
\ipa{*\'{r}} \change\ \O\\
\ipa{j} \change\ \ipa{tS} / V_V

\subsubsection{Proto-Tacanan to Tacana}{\it Pogostick Man}, from Key, Mary Ritchie (1968), {\it Comparative Tacanan Phonology with Cavine\~{n}a Phonology and Notes on Pano-Tacanan Relationship}

\ipa{k} \change\ \ipa{k\super w} / _\ipa{a}\\
\ipa{k} \change\ \ipa{k\super w} / \#_\ipa{i}\\
\O\ \change\ \ipa{j} / \ipa{k}_\ipa{e}\\
\ipa{tC} \change\ \ipa{d\textctz}\\
\ipa{tS} \change\ \ipa{ts} / _E\\
\ipa{C} \change\ \ipa{s}\\
\ipa{x} \change\ \ipa{h} / \#_\\
\ipa{x} \change\ \{\ipa{h,}\O\}\\
\ipa{w} \change\ \ipa{B} / _E\\
\ipa{j} \change\ \ipa{tS} / V_V\\
\ipa{*\'{r}} \change\ \O

\clearpage

\section{Mande}

Dwyer (1987/1988) gives the following consonants for Proto-Mande.

\begin{center}
\begin{tabular}{c | c c c c c}
& Labial-Velar & Bilabial & Alveolar & Palatal & Velar\\ \hline
Nasal & & \ipa{m} & \ipa{n} & & \ipa{N}\\
Stop & \ipa{kp kp' gb} & \ipa{p b} & \ipa{t t' d} & & \ipa{k k' g}\\
Fricative & & \ipa{f} & \ipa{s z}\\
Approximant & & \ipa{l} & \ipa{j} & \ipa{w}
\end{tabular}
\end{center}

(From Dwyer, David J. (1987/1988), ``Towards Proto Mande Morphology". {\it Mandekan: Bulletin semestriel d'\'{e}tudes linguistiques} 14/15:139 -- 152)

\subsection{Proto-Mande to Bobo}{\it Pogostick Man}, from Dwyer, David J. (1987/1988), ``Towards Proto Mande Morphology". {\it Mandekan: Bulletin semestriel d'\'{e}tudes linguistiques} 14/15:139 -- 152

\tab {\it NB: These changes only deal with consonants.}

\ipa{p} \change\ \ipa{f}\\
\ipa{t' \{k',g\}} \change\ \ipa{t k}\\
\ipa{l} \change\ \ipa{d}\\
\ipa{z} \change\ \ipa{\{s,j\}}\\
\ipa{w} \change\ \ipa{g}\\
\ipa{N} \change\ \ipa{\textltailn}\\
\ipa{kp gb} \change\ \ipa{k gw}

\subsection{Proto-Mande to Busa}{\it Pogostick Man}, from Dwyer, David J. (1987/1988), ``Towards Proto Mande Morphology". {\it Mandekan: Bulletin semestriel d'\'{e}tudes linguistiques} 14/15:139 -- 152

\tab {\it NB: These changes only deal with consonants.}

\ipa{t' k'} \change\ \ipa{t k}\\
\ipa{d} \change\ \ipa{l}\\
\ipa{N} \change\ \ipa{\textltailn}\\
\ipa{kp kp'} \change\ \ipa{\{kp,k\} gb}\\
\ipa{j} \change\ \ipa{i}

\subsection{Proto-Mande to Dan}{\it Pogostick Man}, from Dwyer, David J. (1987/1988), ``Towards Proto Mande Morphology". {\it Mandekan: Bulletin semestriel d'\'{e}tudes linguistiques} 14/15:139 -- 152

\tab {\it NB: These changes only deal with consonants.}

\ipa{t' k'} \change\ \ipa{t k}\\
\ipa{d} \change\ \ipa{l}\\
\ipa{N} \change\ \ipa{\textltailn}\\
\ipa{kp'} \change\ \ipa{gb}

\subsection{Proto-Mande to Guro}{\it Pogostick Man}, from Dwyer, David J. (1987/1988), ``Towards Proto Mande Morphology". {\it Mandekan: Bulletin semestriel d'\'{e}tudes linguistiques} 14/15:139 -- 152

\tab {\it NB: These changes only deal with consonants.}

\ipa{p f} \change\ \ipa{f v}\\
\ipa{t' k'} \change\ \ipa{t k}\\
\ipa{d} \change\ \ipa{l}\\
\ipa{N} \change\ \ipa{\textltailn}\\
\ipa{kp \{kp',gb\}} \change\ \ipa{p b}

\subsection{Proto-Mande to Kono-Vai}{\it Pogostick Man}, from Dwyer, David J. (1987/1988), ``Towards Proto Mande Morphology". {\it Mandekan: Bulletin semestriel d'\'{e}tudes linguistiques} 14/15:139 -- 152

\tab {\it NB: These changes only deal with consonants.}

\ipa{p} \change\ \ipa{f}\\
\ipa{l} \change\ \ipa{d}\\
\ipa{z} \change\ \ipa{s}\\
\ipa{\{g,w,kp\} kp' gb} \change\ \ipa{k kp b}\\
\ipa{t' k'} \change\ \ipa{t k}\\
\ipa{N} \change\ \ipa{\textltailn}

\subsection{Proto-Mande to Southwest Mande}{\it Pogostick Man}, from Dwyer, David J. (1987/1988), ``Towards Proto Mande Morphology". {\it Mandekan: Bulletin semestriel d'\'{e}tudes linguistiques} 14/15:139 -- 152

\tab {\it NB: These changes only deal with consonants.}

\ipa{f} \change\ \ipa{p}\\
\ipa{\{t',d\} \{k',g\}} \change\ \ipa{l k}\\
\ipa{z} \change\ \ipa{s}\\
\ipa{w} \change\ \ipa{g}\\
\ipa{N} \change\ \ipa{\textltailn}\\
\ipa{kp kp' gb} \change\ \{\ipa{k},B\} \ipa{kp} B (it's unclear what this $\langle$B$\rangle$ is)

\subsection{Proto-Mande to Mandekan}{\it Pogostick Man}, from Dwyer, David J. (1987/1988), ``Towards Proto Mande Morphology". {\it Mandekan: Bulletin semestriel d'\'{e}tudes linguistiques} 14/15:139 -- 152

\tab {\it NB: These changes only deal with consonants.}

\ipa{p} \change\ \ipa{f}\\
\ipa{l} \change\ \ipa{d}\\
\ipa{z} \change\ \ipa{s}\\
\ipa{\{k,g,kp\} \{kp',gb\}} \change\ \ipa{s b}\\
\ipa{t' k'} \change\ \ipa{t k}\\
\ipa{N} \change\ \ipa{\textltailn}\\
\ipa{w j} \change\ \ipa{k dZ}

\subsection{Proto-Mande to Mano}{\it Pogostick Man}, from Dwyer, David J. (1987/1988), ``Towards Proto Mande Morphology". {\it Mandekan: Bulletin semestriel d'\'{e}tudes linguistiques} 14/15:139 -- 152

\tab {\it NB: These changes only deal with consonants.}

\ipa{f} \change\ \ipa{v}\\
\ipa{t' k'} \change\ \ipa{t k}\\
\ipa{d} \change\ \ipa{l}\\
\ipa{N} \change\ \ipa{\textltailn}\\
\ipa{kp'} \change\ \ipa{gb}

\subsection{Proto-Mande to Mwa}{\it Pogostick Man}, from Dwyer, David J. (1987/1988), ``Towards Proto Mande Morphology". {\it Mandekan: Bulletin semestriel d'\'{e}tudes linguistiques} 14/15:139 -- 152

\tab {\it NB: These changes only deal with consonants.}

\ipa{f} \change\ \ipa{v}\\
\ipa{t' k'} \change\ \ipa{t k}\\
\ipa{d} \change\ \ipa{l}\\
\ipa{N} \change\ \ipa{\textltailn}\\
\ipa{kp'} \change\ \ipa{gb}

\subsection{Proto-Mande to San}{\it Pogostick Man}, from Dwyer, David J. (1987/1988), ``Towards Proto Mande Morphology". {\it Mandekan: Bulletin semestriel d'\'{e}tudes linguistiques} 14/15:139 -- 152

\tab {\it NB: These changes only deal with consonants.}

\ipa{t' k'} \change\ \ipa{t k}\\
\ipa{d} \change\ \ipa{l}\\
\ipa{N} \change\ \ipa{\textltailn}\\
\ipa{kp kp'} \change\ \ipa{k b}

\subsection{Proto-Mande to Sembla}{\it Pogostick Man}, from Dwyer, David J. (1987/1988), ``Towards Proto Mande Morphology". {\it Mandekan: Bulletin semestriel d'\'{e}tudes linguistiques} 14/15:139 -- 152

\tab {\it NB: These changes only deal with consonants.}

\ipa{p} \change\ \ipa{f}\\
\ipa{f} \change\ \ipa{d} (yes, really)\\
\ipa{t' \{k',g,w\}} \change\ \ipa{\{t,d\} k}\\
\ipa{l} \change\ \ipa{d}\\
\ipa{z} \change\ \ipa{s}\\
\ipa{gb} \change\ \ipa{b}\\
\ipa{j} \change\ \ipa{dZ}\\
\ipa{N} \change\ \ipa{\textltailn}

\subsection{Proto-Mande to Soninka}{\it Pogostick Man}, from Dwyer, David J. (1987/1988), ``Towards Proto Mande Morphology". {\it Mandekan: Bulletin semestriel d'\'{e}tudes linguistiques} 14/15:139 -- 152

\tab {\it NB: These changes only deal with consonants.}

\ipa{p} \change\ \ipa{f}\\
\ipa{t' \{k',g\}} \change\ \ipa{\{t,d\} k}\\
\ipa{z} \change\ \ipa{j} ?\\
\ipa{\{w,N\}} \change\ \ipa{j}\\
\ipa{kp gb} \change\ \ipa{k b}

\subsection{Proto-Mande to Susu}{\it Pogostick Man}, from Dwyer, David J. (1987/1988), ``Towards Proto Mande Morphology". {\it Mandekan: Bulletin semestriel d'\'{e}tudes linguistiques} 14/15:139 -- 152

\tab {\it NB: These changes only deal with consonants.}

\ipa{p} \change\ \ipa{f}\\
\ipa{t'} \change\ \ipa{t}\\
\ipa{l} \change\ \ipa{d}\\
\ipa{z} \change\ \ipa{s}\\
\ipa{w} \change\ \ipa{x}\\
\ipa{N} \change\ \ipa{j}\\
\ipa{kp gb} \change\ \ipa{k b}

\clearpage

\section{Mayan}\tab Wikipedia gives the following for the Proto-Mayan phonology:

\begin{center}\begin{tabular}{c | c c c c c c}
& Bilabial & Alveolar & Palatal & Velar & Uvular & Glottal\\ \hline
Nasal & \ipa{m} & \ipa{n} & & \ipa{N}\\
Plosive & \ipa{p \!b} & \ipa{t t'} & \ipa{t\super j t\super j'} & \ipa{k k'} & \ipa{q q'} & \ipa{P}\\
Fricative & & \ipa{s} & \ipa{S} & & \ipa{X} & \ipa{h}\\
Affricate & & \ipa{ts ts'} & \ipa{tS tS'}\\
Liquid & & \ipa{l r}\\
Glide & & & \ipa{j} & \ipa{w}\end{tabular}

\begin{tabular}{c | c c c}
& Front & Central & Back\\ \hline
High & \ipa{i i:} & & \ipa{u u:}\\
Mid & \ipa{e e:} & & \ipa{o o:}\\
Low & & \ipa{a a:}\end{tabular}\end{center}

\tab (From Wikipedia contributors (2013), ``Mayan languages". {\it Wikipedia, the Free Encyclopedia}. \textless\url{https://en.wikipedia.org/w/index.php?title=Mayan_languages&oldid=583331877}\textgreater)

\subsection{Proto-Mayan to Ch'olan}{\it Pogostick Man}, from Wikipedia contributors (2013), ``Proto-Mayan language". {\it Wikipedia, the Free Encyclopedia}. \textless\url{https://en.wikipedia.org/w/index.php?title=Proto-Mayan_language&oldid=571518268}\textgreater

\ipa{q(')} \change\ \ipa{k(')}\\
\ipa{N} \change\ \ipa{n}\\
\ipa{a: e: o:} \change\ \ipa{1 i u}\\
\ipa{t\super j(')} \change\ \ipa{t(')}\\
\ipa{r} \change\ \ipa{j}\\
V\ipa{:} \change\ V[-long]

\subsection{Proto-Mayan to Chujean}{\it Pogostick Man}, from Wikipedia contributors (2013), ``Proto-Mayan language". {\it Wikipedia, the Free Encyclopedia}. \textless\url{https://en.wikipedia.org/w/index.php?title=Proto-Mayan_language&oldid=571518268}\textgreater

\ipa{N} \change\ \ipa{n}\\
\ipa{t\super j(')} \change\ \ipa{t(')}\\
\ipa{r} \change\ \ipa{j}\\
V\ipa{:} \change\ V[-long]

\subsection{Proto-Mayan to Huastecan}{\it Pogostick Man}, from Wikipedia contributors (2013), ``Proto-Mayan language". {\it Wikipedia, the Free Encyclopedia}. \textless\url{https://en.wikipedia.org/w/index.php?title=Proto-Mayan_language&oldid=571518268}\textgreater

\ipa{w} \change\ \ipa{b}\\
\ipa{h} \change\ \ipa{w} / _\{\ipa{o,u}\}\\
\ipa{q(')} \change\ \ipa{k(')}\\
\ipa{N} \change\ \ipa{h}\\
\ipa{k}V[+round]C[+glide] \change\ \ipa{k\super w}

\subsection{Proto-Mayan to Ixilean}{\it Pogostick Man}, from Wikipedia contributors (2013), ``Proto-Mayan language". {\it Wikipedia, the Free Encyclopedia}. \textless\url{https://en.wikipedia.org/w/index.php?title=Proto-Mayan_language&oldid=571518268}\textgreater

\ipa{N} \change\ \ipa{x}\\
\ipa{t} \change\ \ipa{tS}\\
CV\ipa{P}VC \change\ CV\ipa{P}C\\
\ipa{r} \change\ \{\ipa{t,j}\}\\
\ipa{tS} \change\ \ipa{\:t\:s}

\subsection{Proto-Mayan to Kaqchikel-Tz'utujil}{\it Pogostick Man}, from Wikipedia contributors (2013), ``Proto-Mayan language". {\it Wikipedia, the Free Encyclopedia}. \textless\url{https://en.wikipedia.org/w/index.php?title=Proto-Mayan_language&oldid=571518268}\textgreater

\ipa{N} \change\ \ipa{x}\\
\ipa{h} \change\ \ipa{j} / _\#\\
CV\ipa{P}VC \change\ CV\ipa{P}C\\
\ipa{\!b w} \change\ \ipa{P j} / VCV_\#\\
\ipa{t\super j(')} \change\ \ipa{tS(')}\\
V\ipa{:} \change\ V[-long]\\
``Kaqchikel retains a centralized lax schwa-like vowel as a reflex of Proto-Mayan [a]"

\subsection{Proto-Mayan to Core K'iche'}{\it Pogostick Man}, from Wikipedia contributors (2013), ``Proto-Mayan language". {\it Wikipedia, the Free Encyclopedia}. \textless\url{https://en.wikipedia.org/w/index.php?title=Proto-Mayan_language&oldid=571518268}\textgreater

\ipa{N} \change\ \ipa{x}\\
CV\ipa{P}VC \change\ CV\ipa{P}C\\
\ipa{t\super j(')} \change\ \ipa{tS(')}

\subsection{Proto-Mayan to Mamean}{\it Pogostick Man}, from Wikipedia contributors (2013), ``Proto-Mayan language". {\it Wikipedia, the Free Encyclopedia}. \textless\url{https://en.wikipedia.org/w/index.php?title=Proto-Mayan_language&oldid=571518268}\textgreater

\ipa{N} \change\ \ipa{x}\\
\ipa{t} \change\ \ipa{tS}\\
CV\ipa{P}VC \change\ CV\ipa{P}C\\
\ipa{r} \change\ \{\ipa{t,j}\}\\
\ipa{tS} \change\ \ipa{\:t\:s}\\
\ipa{t\super j(')} \change\ \ipa{t(')}\\
\ipa{t\super j(')} \change\ \ipa{ts(')}\\
\ipa{r t tS S} \change\ \ipa{t tS \:t\:s \:s}

\subsection{Proto-Mayan to Q'anjob'alan}{\it Pogostick Man}, from Wikipedia contributors (2013), ``Proto-Mayan language". {\it Wikipedia, the Free Encyclopedia}. \textless\url{https://en.wikipedia.org/w/index.php?title=Proto-Mayan_language&oldid=571518268}\textgreater

\ipa{q(')} \change\ \ipa{k(')}\\
\ipa{N} \change\ \ipa{n}\\
\ipa{r} \change\ \ipa{j}\\
V\ipa{:} \change\ V[-long]

\subsection{Proto-Mayan to Tzeltalan}{\it Pogostick Man}, from Wikipedia contributors (2013), ``Proto-Mayan language". {\it Wikipedia, the Free Encyclopedia}. \textless\url{https://en.wikipedia.org/w/index.php?title=Proto-Mayan_language&oldid=571518268}\textgreater

\ipa{q(')} \change\ \ipa{k(')}\\
\ipa{N} \change\ \ipa{n}\\
\ipa{a: e: o:} \change\ \ipa{1 i u}

\subsection{Proto-Mayan to Yucatecan}{\it Pogostick Man}, from Wikipedia contributors (2013), ``Proto-Mayan language". {\it Wikipedia, the Free Encyclopedia}. \textless\url{https://en.wikipedia.org/w/index.php?title=Proto-Mayan_language&oldid=571518268}\textgreater

\ipa{q(')} \change\ \ipa{k(')}\\
\ipa{N} \change\ \ipa{n}\\
\ipa{a:} \change\ \ipa{1}\\
\ipa{t} \change\ \ipa{tS} / _\#\\
\ipa{t\super j(')} \change\ \ipa{tS(')}\\
``[V]owel length and [h] and [\ipa{P}]" have converted into a tone distinction

\clearpage

\section{Muskogean}The following Proto-Muskogean phonemic inventory is adapted from Wikipedia contributors (2016), citing Booker (2005).

\begin{center}
\begin{tabular}{c | c c c c c}
& Labial & Alveolar & Palatal & Velar & Rounded Velar\\ \hline
Nasal & \ipa{m} & \ipa{n}\\
Stop & \ipa{p} & \ipa{t} & & \ipa{k} & \ipa{k\super w}\\
Affricate & & \ipa{ts} & \ipa{tS}\\
Fricative & & \ipa{s} & \ipa{S} & \ipa{x} & \ipa{x\super w}\\
Lateral Fricative & & \ipa{\textbeltl}\\
Approximant & & \ipa{l} & \ipa{j} & & \ipa{w}\\
Unknown & & \ipa{T}
\end{tabular}
\end{center}

In addition, Booker (2005) posits two phonemes of unknown value. These phonemes dropped out in all positions in Eastern Muskogean, and only survived in the final syllable in Western Muskogean, where they yielded a glottal stop (/\ipa{P}/) and a glottal fricative (/\ipa{h}/) before developing further in the respective languages. I have termed the progenitor phonemes ``weak'' (namely C$_1$[+ weak] and C$_2$[+ weak]). (I tentatively hypothesize that these were *\ipa{P} *\ipa{h}, respectively, but am not sure.)

I would like to take the unusual step of asking for help. I had to go to the library to find Booker's paper, and in my notes I failed to write down the languages for which the following sound changes occurred:

\ipa{l} \change\ \ipa{j} / \ipa{a}_\ipa{i}\\
\ipa{k} \change\ \O\ / V_C ! penult\\
V \change\ \^{V}\ipa{:} / _C\ipa{ko}, \ipa{ko} lost?

If anyone has a copy of Booker's paper and can double-check, please contact me via one of the methods listed in the appropriate section.

(From Wikipedia contributors (2016), ``Muskogean languages". {\it Wikipedia, the Free Encyclopedia}. \textless\url{https://en.wikipedia.org/w/index.php?title=Muskogean_languages&oldid=704652062}\textgreater, citing Booker, Karen (2005), ``Muskogean Historical Phonology". In Hardy, Heather Kay, and Janine Scancarelli, eds., {\it Native Languages of the Southeastern United States} 246 -- 298; and Booker, Karen (2005), ``Muskogean Historical Phonology". In Hardy, Heather Kay, and Janine Scancarelli, eds., {\it Native Languages of the Southeastern United States} 246 -- 298)

\subsection{Proto-Muskogean to Proto-Eastern Muskogean}{\it Pogostick Man}, from Booker, Karen (2005), ``Muskogean Historical Phonology". In Hardy, Heather Kay, and Janine Scancarelli, eds., {\it Native Languages of the Southeastern United States} 246 -- 298

\ipa{T} \change\ \ipa{\textbeltl}\\
\ipa{x\super w} \change\ \ipa{f}\\
VC[+ weak] \change\ \O\ / _V\#

\subsubsection{Proto-Eastern Muskogean to Alabama}{\it Pogostick Man}, from Booker, Karen (2005), ``Muskogean Historical Phonology". In Hardy, Heather Kay, and Janine Scancarelli, eds., {\it Native Languages of the Southeastern United States} 246 -- 298

\ipa{S} \change\ \ipa{ts}\\
\ipa{k}V \change\ \O\ / V_\#\\
\ipa{k} \change\ \O\\
\ipa{ts x} \change\ \ipa{s h} / _C

\subsubsection{Proto-Eastern Muskogean to Creek}{\it Pogostick Man}, from Booker, Karen (2005), ``Muskogean Historical Phonology". In Hardy, Heather Kay, and Janine Scancarelli, eds., {\it Native Languages of the Southeastern United States} 246 -- 298

V$_1$\ipa{k}V$_2$ \change\ V$_2$\ipa{:} / \#((C)V(C))(C)_\#\\
V$_1$\ipa{k} \change\ \O\ / _\#\\
\ipa{S} \change\ \ipa{ts}\\
\ipa{k\super w} \change\ \ipa{k} / \#_\\
\ipa{k\super w} \change\ \ipa{b}\\
S \change\ S[+ voice] / V_V\\
\ipa{h} \change\ \ipa{x} / _\%\\
V$_0$\ipa{x}V$_0$ \change\ V$_0$\ipa{:}\\
Initial vowels lost?\\
\ipa{x} \change\ \ipa{w} / \ipa{a}_\ipa{o}\\
\ipa{x} \change\ \ipa{h}\\
\ipa{m} \change\ \ipa{N} / _\ipa{k}\\
\ipa{ts} and \ipa{t} alternate before \ipa{k}\\
\ipa{kl} \change\ \ipa{k:}\\
C\ipa{:} \change\ C[- long]

\subsubsection{Proto-Eastern Muskogean to Hitchiti}{\it Pogostick Man}, from Booker, Karen (2005), ``Muskogean Historical Phonology". In Hardy, Heather Kay, and Janine Scancarelli, eds., {\it Native Languages of the Southeastern United States} 246 -- 298

\ipa{S} \change\ \ipa{ts}\\
V \change\ \O\ / V\ipa{k}_\#\\
\ipa{x} \change\ \ipa{j} / V$_0$_V$_0$\\
\ipa{x} \change\ \ipa{h}

\subsubsection{Proto-Eastern Muskogean to Korasati}{\it Pogostick Man}, from Booker, Karen (2005), ``Muskogean Historical Phonology". In Hardy, Heather Kay, and Janine Scancarelli, eds., {\it Native Languages of the Southeastern United States} 246 -- 298

\ipa{S} \change\ \ipa{ts}\\
V\ipa{k}V \change\ "V / _\#\\
\ipa{k} \change\ \O\\
\ipa{ts} \change\ \ipa{s} / _C\\
\ipa{nt} \change\ \ipa{t:}\\
\ipa{x} \change\ \ipa{h}

\subsubsection{Proto-Eastern Muskogean to Mikasuri}{\it Pogostick Man}, from Booker, Karen (2005), ``Muskogean Historical Phonology". In Hardy, Heather Kay, and Janine Scancarelli, eds., {\it Native Languages of the Southeastern United States} 246 -- 298

V \change\ \O\ / V\ipa{k}_\#\\
\ipa{x} \change\ \ipa{j} / V$_0$_V$_0$\\
\ipa{tS} \change\ \ipa{s} / _C ! _\ipa{k}\\
\ipa{S x} \change\ \ipa{ts h}

\subsubsection{Proto-Eastern Muskogean to Seminole}{\it Pogostick Man}, from Booker, Karen (2005), ``Muskogean Historical Phonology". In Hardy, Heather Kay, and Janine Scancarelli, eds., {\it Native Languages of the Southeastern United States} 246 -- 298

\ipa{S x} \change\ \ipa{ts h}\\
\ipa{tl} \change\ \ipa{t:}

\subsection{Proto-Muskogean to Proto-Western Muskogean}{\it Pogostick Man}, from Booker, Karen (2005), ``Muskogean Historical Phonology". In Hardy, Heather Kay, and Janine Scancarelli, eds., {\it Native Languages of the Southeastern United States} 246 -- 298

\ipa{ts tS} \change\ \ipa{s ts}\\
\ipa{T} \change\ \ipa{n}\\
\ipa{s} \change\ \ipa{S}\\
\ipa{x} \change\ \ipa{h}\\
C$_1$[+ weak] C$_2$[+ weak] \change\ \ipa{P h} / V_V\#\\
V \change\ \O\ / V\{\ipa{k,P,h}\}_\#\\
\ipa{x\super w} \change\ \ipa{h} / \%_\{\ipa{o,i}\}(C)\#\\
\ipa{a} \change\ \ipa{o} / \ipa{x\super w}_\#\\
\ipa{oj aj} \change\ \ipa{i: \{a:,i:\}}\\
\ipa{i} \change\ \O\ / \#(C)V(C)(C)V(C)(C)_\# (sporadic in the case of other vowels)\\
\ipa{tl st} \change\ \ipa{l: t:}

\subsubsection{Proto-Western Muskogean to Chickasaw}{\it Pogostick Man}, from Booker, Karen (2005), ``Muskogean Historical Phonology". In Hardy, Heather Kay, and Janine Scancarelli, eds., {\it Native Languages of the Southeastern United States} 246 -- 298

\ipa{h} \change\ \O\ / _\#\\
\ipa{aw} \change\ \ipa{o}\\
\ipa{x\super w\textbeltl} \change\ \ipa{\textbeltl:}

\subsubsection{Proto-Western Muskogean to Choctaw}{\it Pogostick Man}, from Booker, Karen (2005), ``Muskogean Historical Phonology". In Hardy, Heather Kay, and Janine Scancarelli, eds., {\it Native Languages of the Southeastern United States} 246 -- 298

\ipa{P} \change\ \O / _\#\\
\ipa{x\super w} \change\ \ipa{h} / V_V\\
\ipa{\textbeltl h} \change\ \ipa{\textbeltl:}\\
\ipa{a} \change\ \ipa{o} / _\ipa{w}\\
\ipa{p} \change\ \ipa{k} / _C

\clearpage

\section{Na-Dene}Note that the changes from Proto-Na-Dene and Proto-Athabaskan-Eyak deal only with obstruents.

\subsection{Proto-Na-Dene to Proto-Athabaskan-Eyak}{\it Pogostick Man}, from Wikipedia contributors (2015), ``Na-Dene languages''. {\it Wikipedia, the Free Encyclopedia}. \textless\url{https://en.wikipedia.org/w/index.php?title=Na-Dene_languages&oldid=666126262}\textgreater

\ipa{k\super j k\super j' g\super j x\super j} \change\ \ipa{ts ts' dz s}\\
\{\ipa{s,S}\} \change\ \O\ / _\ipa{x}

\subsubsection{Proto-Athabaskan-Eyak to Proto-Athabaskan}{\it Pogostick Man}, from Wikipedia contributors (2015), ``Na-Dene languages''. {\it Wikipedia, the Free Encyclopedia}. \textless\url{https://en.wikipedia.org/w/index.php?title=Na-Dene_languages&oldid=666126262}\textgreater

\ipa{\textbeltl} \change\ \{\ipa{\textbeltl,l}\}\\
\ipa{S} \change\ \{\ipa{\:s,\:z}\}\\
\{\ipa{s,dz}\} \change\ \{\ipa{s,z}\}\\
\ipa{k k' k\super w k\super w' g g\super w} \change\ \ipa{k\super j k\super j' t\:s t\:s' g\super j \:d\:z}\\
\ipa{x x\super w} \change\ \{\ipa{x\super j,j}\} \{\ipa{\:s,\:z}\}\\
Q\super w \change\ Q\super w \change\ Q ?\\
\ipa{X(\super w)} \change\ \{\ipa{X,K}\}

\paragraph{Athabaskan}\tab Wikipedia gives the following reconstructions, adapted from Cook (1981), Krauss \&\hspace{0pt} Golla (1981), Krauss \&\hspace{0pt} Leer (1981), and Cook \&\hspace{0pt} Rice (1981) for the consonants and from Leer (2005:284) for the vowels; the vowel phonemes in parentheses are reduced.

\begin{center}\begin{tabular}{c | c c c c c c}
& Bilabial & Alveolar & Postalveolar & Velar & Uvular & Glottal \\ \hline
Nasal & \ipa{m} & \ipa{n} & \ipa{\textltailn} & & & \\
Plosive & & \ipa{t t\super h t'} & & \ipa{k k\super h k'} & \ipa{q q\super w q\super h q\super w\super h q' q\super w'} & \textipa{P} \\
Fricative & & \ipa{s z} & \textipa{S S\super w Z Z\super w} & \textipa{x G} & \textipa{X X\super w K K\super w} & \ipa{h} \\
Lat. Fric. & & \textipa{\textbeltl} \textipa{\textlyoghlig}\raisebox{-0.7ex}{\textasciitilde}\ipa{l} & & & & \\
Affricate & & \textipa{\t*{ts} \t*{ts}\super h \t*{ts}'} & \textipa{\t*{tS} \t*{tS}\super w \t*{tS}\super h \t*{tS}\super w\super h \t*{tS}' \t*{tS}\super w'} & & & \\
Lat. Aff. & & \textipa{\t*{t\textbeltl}} \textipa{\t*{t\textbeltl}\super h} \textipa{\t*{t\textbeltl}}' & & & & \\
Approximant & & & \ipa{j} & & \ipa{w} & \end{tabular}\end{center}

\begin{center}\begin{tabular}{c | c c}
& Front & Back \\ \hline
High & \textipa{i:} & \textipa{u:} \\
Mid & (\textipa{@}) & (\textipa{U}) \\
Low & \textipa{e:} & (\textipa{A}) \textipa{A:} \end{tabular}\end{center}

\tab In addition, though it is not encountered in these changes, there is a phoneme that crops up in forms of the first-person singular pronoun which has various reflexes in many Athabaskan languages; Krauss (1976b) represents it as *\$. Leer transcribed it as *\v{s}\super y in 2005:284 but in 2008 opted to use the *\$\hspace{0pt} transcription.

\tab The great majority of changes in this section are for the respective \textit{series} of consonants, not for individual ones; therefore, changes specific to single consonants are marked so, and the reader should assume that unless explicitly stated, all of the following changes apply to the entire consonantal series. At the recommendation of Jan Strasser, the following conventions will be used to refer to the series; these are based on the abbreviations Whimemsz gave on the original Correspondence Library page, derived from the voiceless members of each series:
\begin{enumerate}
\item T, dental stops
\item T\L, laterals
\item TS, dental affricates and fricatives
\item T\v{S}, palatals
\item T\v{S}\super w, labialized palatals
\item K, front (palatalized) velars
\item Q, uvulars
\item Q\super w, labialized uvulars
\end{enumerate}
\tab Whimemsz was unsure of the abbreviation of the glottal series. In addition, there also exist a series of (inter-?)dentals, abbreviated TH, and one of retroflexes, abbreviated T\d{S}. Changes marked with an asterisk, *, apply to the individual phone(me)s.

\tab (From Whimemsz's statements from the TCL thread and Wiki, and from Wikipedia contributors (2011), \textquotedblleft Athabaskan languages". \textit{Wikipedia, The Free Encyclopedia}. \textless\url{http://en.wikipedia.org/w/index.php?title=Athabaskan_languages&oldid=454112398}\textgreater)

\subparagraph{Proto-Athabaskan to Ahtna}{\it Whimemsz}, from Krauss, Michael and Victor Golla (1981), ``Northern Athapaskan Languages". {\it Handbook of North American Indians}, Vol. 6 (Subarctic), 67 -- 85

\{T\v{S},T\v{S}\super w\} \change\ TS\\
K \change\ T\v{S} / in Mentasta Ahtna\\
\{\ipa{S(\super w),x}\} \change\ \ipa{s}\\
\ipa{A @ U} \change\ \ipa{a e o}\\
V\ipa{\super P} \change\ V[-glottalized]

\subparagraph{Proto-Athabaskan to Babine}{\it Whimemsz}, from Krauss, Michael and Victor Golla (1981), ``Northern Athapaskan Languages". {\it Handbook of North American Indians}, Vol. 6 (Subarctic), 67 -- 85

\{T\v{S},T\v{S}\super w\} \change\ TS\\
\ipa{u A U} \change\ \{\ipa{o,u}\} \ipa{@ u}\\
V\ipa{\super P} \change\ V[-glottalized]

\subparagraph{Proto-Athabaskan to Beaver}{\it Whimemsz}, from Krauss, Michael and Victor Golla (1981), ``Northern Athapaskan Languages". {\it Handbook of North American Indians}, Vol. 6 (Subarctic), 67 -- 85

TS \change\ TH (most often back to 3, however)\\
\{T\v{S},T\v{S}\super w\} \change\ TS\\
K Q \change\ T\v{S} K\\
T \change\ T\v{S} / _\{\ipa{i,e,u}\}, in the British Columbian dialect\\
\{\ipa{n,\textltailn}\} \change\ \ipa{d} / \$_V[-nas] (\change\ \ipa{dZ} in the British Columbian dialect)\\
\ipa{A} \change\ \ipa{@}\\
V\ipa{\super P} \change\ V[+high tone]

\subparagraph{Proto-Athabaskan to Chilcotin}{\it Whimemsz}, from Krauss, Michael and Victor Golla (1981), ``Northern Athapaskan Languages". {\it Handbook of North American Indians}, Vol. 6 (Subarctic), 67 -- 85

TS series desibilantizes\\
V \change\ \{V\ipa{\super P},V[+RTR]\}\\
\{T\v{S},T\v{S}\super w\} K \change\ TS T\v{S}\\
The Q series incompletely moves to the K series, the latter being more common\\
\ipa{e} \{\ipa{A,@}\} \change\ \ipa{i} \{\ipa{e,I}\}\\
V\ipa{\super P} \change\ V[+high tone]

\subparagraph{Proto-Athabaskan to Chipewyan}{\it Whimemsz}, from Krauss, Michael and Victor Golla (1981), ``Northern Athapaskan Languages". {\it Handbook of North American Indians}, Vol. 6 (Subarctic), 67 -- 85

TS \{T\v{S},T\v{S}\super w\} K Q \change\ TH TS T\v{S} K\\
\ipa{t} \change\ \ipa{k} (not for all speakers)\\
\{A,O'\} \change\ F / _\$\\
\ipa{A U} \change\ \ipa{a o}\\
V\ipa{\super P} \change\ V[+high tone]

\subparagraph{Proto-Athabaskan to Dakelh}{\it Whimemsz}, from Krauss, Michael and Victor Golla (1981), ``Northern Athapaskan Languages". {\it Handbook of North American Indians}, Vol. 6 (Subarctic), 67 -- 85

\{T\v{S},T\v{S}\super w\} K Q \change\ TS T\v{S} K\\
\ipa{u} \{\ipa{A,U}\} \change\ \{\ipa{o,u}\} \ipa{@}\\
V\ipa{\super P} \change\ V[-glottalized]

\subparagraph{Proto-Athabaskan to Deg Hit'an}{\it Whimemsz}, from Krauss, Michael and Victor Golla (1981), ``Northern Athapaskan Languages". {\it Handbook of North American Indians}, Vol. 6 (Subarctic), 67 -- 85

TS T\v{S} \change\ TH TS\\
T\v{S}\super w \change\ T\d{S} (\change\ TS in Kuskokwim dialect)\\
K \change\ \{K,T\v{S}\}\\
\ipa{w} \change\ \ipa{v} (\change\ \ipa{w} in Shageluk dialect)\\
\ipa{\textltailn} \change\ \ipa{N}\\
R F \change\ R[-voiced] F[-voiced] / _\# in suffixes\\
\ipa{e u a} \{\ipa{A,U}\} \change\ \ipa{a i u @}\\
C' \change\ C / _\$\\
V\ipa{\super P} \change\ V[-glottalized]

\subparagraph{Proto-Athabaskan to Dena'ina}{\it Whimemsz}, from Krauss, Michael and Victor Golla (1981), ``Northern Athapaskan Languages". {\it Handbook of North American Indians}, Vol. 6 (Subarctic), 67 -- 85

\{T\v{S},T\v{S}\super w\} \change\ T\v{S} (\change\ TS in Upper Inlet dialect)\\
\ipa{e a u} \{\ipa{@,U}\} \change\ \ipa{a u i @}\\
\{\ipa{S(\super w),x} \{\ipa{z,Z(\super w),G}\} \change\ \ipa{s j}\\
V\ipa{\super P} \change\ V[-glottalized]

\subparagraph{Proto-Athabaskan to Dogrib}{\it Whimemsz}, from Krauss, Michael and Victor Golla (1981), ``Northern Athapaskan Languages". {\it Handbook of North American Indians}, Vol. 6 (Subarctic), 67 -- 85

C \change\ \ipa{h} / _\$\\
\ipa{A @ U u} \change\ \ipa{a e o i}\\
\{T\v{S},T\v{S}\super w\} K Q \change\ TS T\v{S} K\\
\ipa{ts ts\super h ts' s z} \change\ \ipa{k\super w k\super w\super h k\super w' \textturnw\ w}

\subparagraph{Proto-Athabaskan to Easter Gwich'in}{\it Whimemsz}, from Krauss, Michael and Victor Golla (1981), ``Northern Athapaskan Languages". {\it Handbook of North American Indians}, Vol. 6 (Subarctic), 67 -- 85

\tab {\it NB: Here, $\langle${\bf TS}$\rangle$ represents a sound that Whimemsz says ``is between'' the TH and TS series POA-wise.}

TS K \change\ T\v{S} {\bf TS} / _E\\
TS K \change\ TH T\v{S} / else\\
T\v{S} T\v{S}\super w Q \change\ TS T\d{S} K\\
\ipa{j w} \change\ \ipa{Z v}\\
\{\ipa{n,\textltailn}\} \change\ \ipa{\super ndZ} / _E[-nas]\\
\{\ipa{n,\textltailn}\} \change\ \ipa{\super nd} / _V[-nas]\\
\{\ipa{i,e}\} \change\ \{\ipa{i,ja}\} (this latter due to the loss of final consonants within the stem)\\
\ipa{a u} \{\ipa{A,@}\} \ipa{U} \change\ \{\ipa{i,e}\} \ipa{ju a o}\\
V\ipa{\super P} \change\ V[+low tone]\\
``An `extensive reduction' of stem-final consonants; however, reflexes of final *-\ipa{\textltailn} and *-n after PA *a and *e are kept distinct''

\subparagraph{Proto-Athabaskan to Han}{\it Whimemsz}, from Krauss, Michael and Victor Golla (1981), ``Northern Athapaskan Languages". {\it Handbook of North American Indians}, Vol. 6 (Subarctic), 67 -- 85

TS T\v{S} T\v{S}\super w K Q \change\ TH TS T\d{S} T\v{S} K\\
Occasional palatalization in front of high vowels\\
\ipa{n} \change\ \ipa{(\super n)d} / \$_V[-nas]\\
\ipa{j} \change\ \ipa{Z} / \$_\\
\ipa{a A @ U} \change\ \ipa{\ae\ a} \{\ipa{@,\"{e}}\} \ipa{o}\\
Acquisition of vowel length, but how this occurs is not described\\
V\ipa{\super P} \change\ V[+low tone]\\
Majority of stem-final consonants lost; the only stem-finals permitted in comtemporary Han are /\ipa{t k w j r n h P}/, with the addition of /\ipa{l}/ in Dawson Han

\subparagraph{Proto-Athabaskan to Holikachuk}{\it Whimemsz}, from Krauss, Michael and Victor Golla (1981), ``Northern Athapaskan Languages". {\it Handbook of North American Indians}, Vol. 6 (Subarctic), 67 -- 85

TS \change\ TH\\
\ipa{e} \change\ \ipa{a} / in prefixes\\
\ipa{i e a u} \{\ipa{A,U}\} \change\ \ipa{e a O o \u{u}}\\
TS \{T\v{S},T\v{S}\super w\} \change\ T\L\ TS\\
C' \change\ C / _\$\\
V\ipa{\super P} \change\ V[+low tone]\\
\ipa{w} (\change\ \ipa{b}?) \change\ \ipa{m}\\
\ipa{@} \change\ \O\ / \{R,F\}_\#

\subparagraph{Proto-Athabaskan to Hupa}{\it Pogostick Man}, from Sapir, Edward (1936), ''Reflexes of Proto-Athabaskan in Several Languages (Hupa, Navaho, Chipewyan, Sarcee)"

\tab {\it NB: First, part of the list of correspondences was cut off; second, it is sometimes difficult to read Sapir's handwriting; and third, I'm hoping I made the correct inferences about his notation.}

\ipa{h} \change\ \{\ipa{h},\O\}\\
\ipa{q \;R} \change\ \ipa{x w}\\
\ipa{\{q\super w,x\super w\} q\super w' \;G\super w G\super w} \change\ \ipa{x(\super w) q(\super w)' \;G(\super w) w}\\
\ipa{z} \change\ \ipa{s}\\
\{\ipa{S,Z}\} \change\ \ipa{w}\\
\ipa{x\super j} \change\ \ipa{w}

\subparagraph{Proto-Athabaskan to Lower Koyukon}{\it Whimemsz}, from Krauss, Michael and Victor Golla (1981), ``Northern Athapaskan Languages". {\it Handbook of North American Indians}, Vol. 6 (Subarctic), 67 -- 85

\ipa{e a A U} \change\ \ipa{a o \u{o} \u{u}}\\
TS \{T\v{S},T\v{S}\super w\} \change\ T\L\ TS\\
C' \change\ C / _\$\\
V\ipa{\super P} \change\ V[+low tone]\\
\ipa{w} (\change\ \ipa{b}?) \change\ \ipa{m}\\
\ipa{@} \change\ \O\ / \{R,F\}_\#\\

\subparagraph{Proto-Athabaskan to Upper Koyukon}{\it Whimemsz}, from Krauss, Michael and Victor Golla (1981), ``Northern Athapaskan Languages". {\it Handbook of North American Indians}, Vol. 6 (Subarctic), 67 -- 85

\ipa{e a A U} \change\ \ipa{a o \u{o} \u{u}}\\
TS \{T\v{S},T\v{S}\super w\} \change\ T\L\ TS\\
Stem-final/suffixal consonant clusters lost in Minchumina-Bearpaw Upper Koyukon\\
K \change\ T\v{S}\\
C' \change\ C / _\$\\
V\ipa{\super P} \change\ V[+low tone] \change\ V[-tone]\\
\ipa{w} \change\ \ipa{m} / _V\ipa{n} (sporadic)\\
\ipa{w} \change\ \ipa{b}

\subparagraph{Proto-Athabaskan to Upper Kuskokwim Kolchan}{\it Whimemsz}, from Krauss, Michael and Victor Golla (1981), ``Northern Athapaskan Languages". {\it Handbook of North American Indians}, Vol. 6 (Subarctic), 67 -- 85

T\v{S} T\v{S}\super w K Q \change\ TS T\d{S} T\v{S} K\\
\ipa{e a} \{\ipa{A,U}\} \change\ \ipa{a o \u{u}}\\
V\ipa{\super P} \change\ V[-glottalized]\\

\subparagraph{Proto-Athabaskan to Sarcee}{\it Whimemsz}, from Krauss, Michael and Victor Golla (1981), ``Northern Athapaskan Languages". {\it Handbook of North American Indians}, Vol. 6 (Subarctic), 67 -- 85

\{T\v{S},T\v{S}\super w\} K Q \change\ TS T\v{S} K\\
\ipa{\'{A} \'{@} \'{U}} \change\ \ipa{\={A} \={@} \={U}}\\
\{\ipa{e,@}\} \ipa{U} \change\ \ipa{A u}\\
V\ipa{\super P} \change\ V[+low tone]

\subparagraph{Proto-Athabaskan to Sekani}{\it Whimemsz}, from Krauss, Michael and Victor Golla (1981), ``Northern Athapaskan Languages". {\it Handbook of North American Indians}, Vol. 6 (Subarctic), 67 -- 85

TS \change\ TH (\change\ TS again in some areas)\\
\{T\v{S},T\v{S}\super w\} K Q \change\ TS T\v{S} Q\\
T \change\ T\v{S} / _\{\ipa{i,e,u}\} ! in Ware Sekani\\
\ipa{U} \change\ \ipa{o}\\
V\ipa{\super P} \change\ V[+high tone]

\subparagraph{Proto-Athabaskan to Proto-Southern Athabaskan}{\it Pogostick Man}, from Hoijer, Harry (1938), ``The Southern Athapaskan Languages". {\it American Anthropologist} 40:75 -- 87

K \change\ TS\\
\ipa{m} \change\ \{\ipa{m,b}\} (\change\ \ipa{b} seems more common)\\
\ipa{G} \change\ \ipa{h} / in prefixes relating to word derivation\\
V\ipa{n}C \change\ V[+ nas]C / _\#, unless C = \ipa{P}\\
\ipa{t n x} \change\ \ipa{d n h} / in prefixes relating to word derivation

\subparagraph{Proto-Southern Athabaskan to Proto-Eastern Southern Athabaskan}{\it Pogostick Man}, from Hoijer, Harry (1938), ``The Southern Athapaskan Languages". {\it American Anthropologist} 40:75 -- 87

\ipa{t} \change\ \ipa{k}\\
\ipa{\{s,z\}(P) \{S,Z\}(P) \{\textbeltl,\textlyoghlig\}(P)} \change\ \ipa{s S \textbeltl} / _\#

\subparagraph{Proto-Eastern Southern Athabaskan to Kiowa Apache}{\it Pogostick Man}, from Hoijer, Harry (1938), ``The Southern Athapaskan Languages". {\it American Anthropologist} 40:75 -- 87

\ipa{n} \change\ \ipa{d}\\
\ipa{d} \change\ \O\ / _\#\\
\ipa{k} \change\ \ipa{tS} / _E\\
\ipa{\{x\super j,j\}P \{x,G\}P} \change\ \O\ \ipa{h(P)} / _\#\\
V\ipa{nP} V\ipa{n}C \change\ V[+ nas] V[+ nas]C / _\#\\
\ipa{x G} \change\ \ipa{h} \O\ / _\#\\
\ipa{\{d,j\}} \change\ \ipa{j} / \O\ / _\#\\
\ipa{x\super j j} \change\ \ipa{S Z}

\subparagraph{Proto-Eastern Southern Athabaskan to Jicarilla}{\it Pogostick Man}, from Hoijer, Harry (1938), ``The Southern Athapaskan Languages". {\it American Anthropologist} 40:75 -- 87

\ipa{d} \change\ \ipa{P} / _\#\\
\ipa{x\super j xP j GP} \change\ \ipa{h P} \O\ \ipa{P} / E_\#\\
\ipa{x\super j xP j gP} \change\ \ipa{ih iP i iP} / _\#\\
\ipa{x\super jP jP} \change\ \ipa{h P} / _\#\\
\ipa{x G} / \ipa{h} \O\ / _\#\\
\ipa{n} \change\ \ipa{\super nd}\\
\ipa{x\super j} \change\ \ipa{s}\\
\ipa{j} \change\ \ipa{G} / _E\\
V\ipa{nP} V\ipa{n}C \change\ V[+ nas] V[+ nas]C / _\#

\subparagraph{Proto-Eastern Southern Athabaskan to Lipan}{\it Pogostick Man}, from Hoijer, Harry (1938), ``The Southern Athapaskan Languages". {\it American Anthropologist} 40:75 -- 87

\ipa{d} \change\ \O\ / _\#\\
\ipa{x\super j j} \change\ \ipa{S} \O\ / _\#\\
\ipa{\{x\super j,j\}P} \change\ \O\ / _\#\\
\{\ipa{x,G}\} \change\ \O\ / _\ipa{(P)}\#\\
V\ipa{nP} V\ipa{n}C \change\ V[+ nas] V[+ nas] / _\#\\
\ipa{n} \change\ \ipa{\super nd}\\
\ipa{x\super j} \change\ \ipa{s}\\
\ipa{j} \change\ \ipa{G} / _E

\subparagraph{Proto-Southern Athabaskan to Proto-Western Southern Athabaskan}{\it Pogostick Man}, from Hoijer, Harry (1938), ``The Southern Athapaskan Languages". {\it American Anthropologist} 40:75 -- 87

\ipa{j} \change\ \ipa{G} / _E

\subparagraph{Proto-Western Southern Athabaskan to Chiricahua}{\it Pogostick Man}, from Hoijer, Harry (1938), ``The Southern Athapaskan Languages". {\it American Anthropologist} 40:75 -- 87

\ipa{d} \change\ \O\ / _\#\\
\ipa{x\super j j} \change\ \ipa{S} \O\ / _\#\\
\ipa{\{x\super j,j\}P} \change\ \O\ / _\#\\
\ipa{\{x,G\}} \change\ \O\ / _(\ipa{P})\#\\
V\ipa{nP} \change\ V[+ nas] / _\#\\
\ipa{n} \change\ \ipa{\super nd}\\
\ipa{\{s,z\}(P) \{S,Z\}(P) \{\textbeltl,\textlyoghlig\}(P)} \change\ \ipa{s S \textbeltl} / _\#

\subparagraph{Proto-Western Southern Athabaskan to Mescalero}{\it Pogostick Man}, from Hoijer, Harry (1938), ``The Southern Athapaskan Languages". {\it American Anthropologist} 40:75 -- 87

\ipa{d} \change\ \O\\
\ipa{x\super j j} \change\ \ipa{S} \O\ / _\#\\
\ipa{\{x\super j,j\}P} \change\ \O\ / _\#\\
\ipa{\{x,G\}} \change\ \O\ / _(\ipa{P})\#\\
V\ipa{nP} \change\ V[+ nas] / _\#\\
\ipa{n} \change\ \ipa{\super nd}\\
\ipa{P} \change\ \O\ / \{\ipa{s,S,\textbeltl}\}_\#\\
\ipa{z(P) Z(P) \textlyoghlig(P)} \change\ \ipa{dz dZ d\textlyoghlig} / _\#

\subparagraph{Proto-Western Southern Athabaskan to Navajo}{\it Pogostick Man}, from Hoijer, Harry (1938), ``The Southern Athapaskan Languages". {\it American Anthropologist} 40:75 -- 87

\{\ipa{G,h}\} \change\ \ipa{j} / in prefixes related to word derivation\\
\ipa{x\super j(P) j(P)} \change\ \ipa{h P} / _\#\\
\ipa{x G} \change\ \ipa{h P} / _\#\\
\ipa{\{x,G\}P} \change\ \ipa{\{P,g\}} (Hoijer notes a reflex ``-g-'')\\
\ipa{x\super j} \change\ \ipa{s}\\
\ipa{j} \change\ \ipa{G} / _E\\
\ipa{\{s,z\}(P) \{S,Z\}(P) \{\textbeltl,\textlyoghlig\}(P)} \change\ \ipa{s S \textbeltl} / _\#

\subparagraph{Proto-Western Southern Athabaskan to San Carlos}{\it Pogostick Man}, from Hoijer, Harry (1938), ``The Southern Athapaskan Languages". {\it American Anthropologist} 40:75 -- 87

\ipa{x(\super j) \{j,G\}} \change\ \ipa{h} \O\ / _\#\\
\ipa{x\super jP jP} \change\ \ipa{h P} / _\#\\
\ipa{\{x,G\}P} \change\ \ipa{g}\\
V\ipa{n\super P} \change\ V[+ nas] / _\#\\
\ipa{n} \change\ \ipa{\super nd}\\
\ipa{\{s,z\}(P) \{S,Z\}(P) \{\textbeltl,\textlyoghlig\}(P)} \change\ \ipa{s S \textbeltl} / _\#

\subparagraph{Proto-Athabaskan to Bearlake Slavey-Hare}{\it Whimemsz}, from Krauss, Michael and Victor Golla (1981), ``Northern Athapaskan Languages". {\it Handbook of North American Indians}, Vol. 6 (Subarctic), 67 -- 85

C \change\ \{\ipa{h,P}\} / _\#\\
\ipa{A @ U} \change\ \ipa{a E o}\\
\{T\v{S},T\v{S}\super w\} K Q \change\ TS T\v{S} K\\
\ipa{ts ts\super h ts' s z} \change\ \ipa{k\super w k\super w\super h k\super w' \textturnw\ w}

\subparagraph{Proto-Athabaskan to Hare Slavey-Hare}{\it Whimemsz}, from Krauss, Michael and Victor Golla (1981), ``Northern Athapaskan Languages". {\it Handbook of North American Indians}, Vol. 6 (Subarctic), 67 -- 85

\ipa{t\textbeltl\super h tS(\super w)\super h k\super h} \change\ \ipa{\textbeltl\ s S}\\
\{T\v{S},T\v{S}\super w\} K \change\ TS T\v{S} (with exceptions)\\
Q \change\ K \\
\O\ \change\ \ipa{j} / _\ipa{e}\\
\ipa{\textbeltl} \change\ \ipa{l}\\
\ipa{ts ts\super h ts'} \{\ipa{s,z}\} \change\ \{\ipa{k\super w,p}\} \ipa{f w\super P w}

\subparagraph{Proto-Athabaskan to Mountain Slavey-Hare}{\it Whimemsz}, from Krauss, Michael and Victor Golla (1981), ``Northern Athapaskan Languages". {\it Handbook of North American Indians}, Vol. 6 (Subarctic), 67 -- 85

C \change\ \{\ipa{h,P}\} / _\#\\
\ipa{A @ U} \change\ \ipa{a e o}\\
\{T\v{S},T\v{S}\super w\} K Q \change\ TS T\v{S} K\\
\ipa{ts ts\super h ts' s z} \change\ \ipa{p p\super h p' f v}

\subparagraph{Proto-Athabaskan to Slavey Slavey-Hare}{\it Whimemsz}, from Krauss, Michael and Victor Golla (1981), ``Northern Athapaskan Languages". {\it Handbook of North American Indians}, Vol. 6 (Subarctic), 67 -- 85

C \change\ \{\ipa{h,P}\} / _\#\\
\ipa{A @ U} \change\ \ipa{a e o}\\
\{T\v{S},T\v{S}\super w\} K Q \change\ TS T\v{S} K

\subparagraph{Proto-Athabaskan to Tahltan-Kaska-Tagish}{\it Whimemsz}, from Krauss, Michael and Victor Golla (1981), ``Northern Athapaskan Languages". {\it Handbook of North American Indians}, Vol. 6 (Subarctic), 67 -- 85

\ipa{A @ U} \change\ \ipa{a} \{\ipa{i,e}\} \ipa{u}\\
C' \change\ C / _\$\\
Q \change\ K

\subparagraph{Tahltan-Kaska-Tagish to Kaska}{\it Whimemsz}, from Krauss, Michael and Victor Golla (1981), ``Northern Athapaskan Languages". {\it Handbook of North American Indians}, Vol. 6 (Subarctic), 67 -- 85

V\ipa{\super P} \change\ V[+high tone]\\
TS \{T\v{S},T\v{S}\super w\} \change\ TH TS\\
K \change\ T\v{S} (although /\ipa{x\super j}/ stays as such in a few dialects)

\subparagraph{Tahltan-Kaska-Tagish to Tagish}{\it Whimemsz}, from Krauss, Michael and Victor Golla (1981), ``Northern Athapaskan Languages". {\it Handbook of North American Indians}, Vol. 6 (Subarctic), 67 -- 85

\{T\v{S},T\v{S}\super w\} TS \change\ TS TS\ipa{\super j}

\subparagraph{Tahltan-Kaska-Tagish to Tahltan}{\it Whimemsz}, from Krauss, Michael and Victor Golla (1981), ``Northern Athapaskan Languages". {\it Handbook of North American Indians}, Vol. 6 (Subarctic), 67 -- 85

\{T\v{S}\super w,K\} \change\ T\v{S} (although /\ipa{x\super j}/ stays as such in a few dialects)

\subparagraph{Proto-Athabaskan to Tanacross}{\it Whimemsz}, from Krauss, Michael and Victor Golla (1981), ``Northern Athapaskan Languages". {\it Handbook of North American Indians}, Vol. 6 (Subarctic), 67 -- 85

K \change\ T\v{S} / ! _\$\\
TS T\v{S} T\v{S}\super w Q \change\ TH TS T\d{S} K\\
S' \change\ S / _\$\\
V\ipa{\super P} \change\ V[+high tone]\\
Acquisition of phonemic length in some unreduced vowels, though exactly how is not explored\\
V\ipa{\textltailn} \change\ V[+nas]\\
\{\ipa{n,\textltailn}\} \change\ \ipa{\super nd}\\
F[+voiced] \change\ F[-voiced] / _\$\\
\ipa{S} \change\ \ipa{h} / in the ``1sg subject prefix''\\
\ipa{\textbeltl} \change\ \ipa{h} / in the grammatical classifier\\
\ipa{A @ U} \change\ \{\ipa{\u{\ae},\u{a}}\} \ipa{\u{\ae} \u{o}}

\subparagraph{Proto-Athabaskan to Lower Tanana}{\it Whimemsz}, from Krauss, Michael and Victor Golla (1981), ``Northern Athapaskan Languages". {\it Handbook of North American Indians}, Vol. 6 (Subarctic), 67 -- 85

K \change\ T\v{S} / ! _\$\\
TS T\v{S} T\v{S}\super W Q \change\ TH TS T\d{S} K\\
S' \change\ S / _\$\\
V\ipa{\super P} \change\ V[+low tone] {\it (``since then partially neutralized in noun and verb stems, but `still clear in verbal prefixes'")}\\
\ipa{e a} \{\ipa{A,U}\} \change\ \ipa{\ae\ O \u{u}}

\subparagraph{Proto-Athabaskan to Upper Tanana}{\it Whimemsz}, from Krauss, Michael and Victor Golla (1981), ``Northern Athapaskan Languages". {\it Handbook of North American Indians}, Vol. 6 (Subarctic), 67 -- 85

K \change\ T\v{S} / ! _\$\\
TS T\v{S} T\v{S}\super w Q \change\ TH TS T\d{S} K\\
S' \change\ S / _\$\\
Acquisition of phonemic length in some unreduced vowels, but this is not explored\\
V\ipa{\textltailn} \change\ V[+nas]\\
\{\ipa{n,\textltailn}\} \change\ \ipa{\super nd}\\
\{A,F\} \change\ \O\ / _\$; diphthongs sometimes lengthen in comparison\\
\ipa{S} \change\ \ipa{h} / in the ``1sg subject prefix''\\
\ipa{\textbeltl} \change\ \ipa{h} / in the grammatical classifier\\
\{\ipa{a,A}\} \ipa{e i u U} \change\ \ipa{e(a)} \{\ipa{i,ea}\} \ipa{ju} \{\ipa{a,1}\} \ipa{o}\\
\ipa{@} \change\ \{\ipa{a,1}\} (\change\ \ipa{\o} in the Northway dialect)\\
V\ipa{\super P} \change\ V[+low tone] (\change\ V[-tone] in ``young speakers by 1980'')

\subparagraph{Proto-Athabaskan to Tsetsaut}{\it Whimemsz}, from Krauss, Michael and Victor Golla (1981), ``Northern Athapaskan Languages". {\it Handbook of North American Indians}, Vol. 6 (Subarctic), 67 -- 85

\tab {\it NB: Whimemsz indicates that the following are to be taken with a grain of salt, as not all of the correspondences are clear due to a lack of detailed sources.}

T\v{S} \change\ TS\\
Series T\v{S}\super w apparently moved its POA to the labiodental or bilabial position\\
K Q \change\ T\v{S} K\\
C \change\ \O\ / _\$ in many cases\\
\ipa{U} \change\ \ipa{o}

\subparagraph{Proto-Athabaskan to Northern Tutchone}{\it Whimemsz}, from Krauss, Michael and Victor Golla (1981), ``Northern Athapaskan Languages". {\it Handbook of North American Indians}, Vol. 6 (Subarctic), 67 -- 85

TS \{T\v{S},T\v{S}\super w\} \change\ TH TS\\
K \change\ T\v{S} (although /\ipa{x\super j}/ remained as such in a few dialects)\\
Q \change\ K\\
\{\ipa{A,@,U}\} \change\ \{\ipa{a,o}\}\\
Most stem-final consonants lost, though some plain and labialized palatal reflexes have developed differently\\
V\ipa{\super P} \change\ V[+high tone]\\
Acquisition of nasalized vowels and diphthongs\\
/\ipa{o}/ somehow develops

\subparagraph{Proto-Athabaskan to Southern Tutchone}{\it Whimemsz}, from Krauss, Michael and Victor Golla (1981), ``Northern Athapaskan Languages". {\it Handbook of North American Indians}, Vol. 6 (Subarctic), 67 -- 85

TS \{T\v{S},T\v{S}\super w\} \change\ TH TS\\
K \change\ T\v{S} (although /\ipa{x\super j}/ remained as such in a few dialects)\\
Q \change\ K\\
\ipa{e a} \change\ \ipa{i e}\\
\{\ipa{A,@,U}\} \change\ \{\ipa{a,o}\}\\
Most stem-final consonants lost, though some plain and labialized palatal reflexes have developed differently\\
V\ipa{\super P} \change\ V[+low tone]\\
Acquisition of nasalized vowels and diphthongs\\
/\ipa{1}/ somehow develops\\
A \change\ F (some slight POA changes; alveolars become dentals, for instance)

\subsubsection{Proto-Athabaskan-Eyak to Eyak}{\it Pogostick Man}, from Wikipedia contributors (2015), ``Na-Dene languages''. {\it Wikipedia, the Free Encyclopedia}. \textless\url{https://en.wikipedia.org/w/index.php?title=Na-Dene_languages&oldid=666126262}\textgreater

\ipa{k\super j k\super j' g\super j x\super j} \change\ \ipa{ts ts' dz \{s,S\}}\\
K\super w \change\ K\\
\ipa{q\super w q\super w' \;G\super w} \change\ \ipa{q q' \;G}\\
\ipa{s} \change\ \O\ / _\ipa{x}\\
\ipa{x} \change\ \O\ / \ipa{S}_\\
\$\ \change\ \ipa{x\super w} \change\ \{\ipa{x,s}\}

\subsection{Proto-Na-Dene to Tlingit}{\it Pogostick Man}, from Wikipedia contributors (2015), ``Na-Dene languages''. {\it Wikipedia, the Free Encyclopedia}. \textless\url{https://en.wikipedia.org/w/index.php?title=Na-Dene_languages&oldid=666126262}\textgreater

\tab {\it NB: Where a colon appears, forms to the left are the typical forms and forms to the right are ``l-assimilated''.}

\ipa{s} \change\ \ipa{s} : \ipa{\textbeltl}\\
\ipa{ts} \change\ \ipa{ts} : \ipa{t\textbeltl}\\
\ipa{ts'} \change\ \{\ipa{s',ts'}\} : \{\textipa{\textbeltl,t\textbeltl'}\}\\
\ipa{S} \change\ \{\ipa{S,s}\} : \ipa{\textbeltl}\\
\ipa{tS} \change\ \{\ipa{tS,ts}\} : \ipa{t\textbeltl}\\
\ipa{tS'} \change\ \{\ipa{s',tS'}\} : \ipa{t\textbeltl'}\\
Somethings going on with the velars and uvulars; apparently, both the rounded and unrounded consonants have reflexes that may or may not be rounded\\
\ipa{k\super j k\super j'} \change\ \{\ipa{k,S}\} \ipa{k'}\\
\ipa{x\super j} \change\ \ipa{x}\\
\ipa{k(\super w)'} \change\ \ipa{\{x,k\}(\super w)'}\\
\ipa{x(\super w)} \change\ \ipa{x}\\
\ipa{q' q\super w'} \change\ \ipa{X(\super w)' \{X',q(\super w)'\}}
\ipa{x} \change\ \O\ / \{\ipa{s,S}\}_\\
\$\ \change\ \ipa{X}

\clearpage

\section{Niger-Congo}\tab Hedinger (1987) reconstructs the following consonant inventory for Pre-Proto-Bantu:

\begin{center}\begin{tabular}{c | c c c c}
& Labial & Alveolar & Palatal & Velar\\ \hline
Lenis nasal & '\ipa{m} & '\ipa{n} & '\ipa{\textltailn}\\
Fortis nasal & \ipa{m} & \ipa{n} & \ipa{\textltailn} & \ipa{N}\\
Lenis stop & '\ipa{p} & '\ipa{t} '\ipa{d} & '\ipa{\textbardotlessj} & '\ipa{k} '\ipa{g}\\
Fortis stop & \ipa{p b} & \ipa{t d} & \ipa{c \textbardotlessj} & \ipa{k g}\\
Unknown (stop?) & & \ipa{d}$_2$
\end{tabular}\end{center}

\begin{center}\begin{tabular}{c | c c c}
& Front & Central & Back \\ \hline
High & \ipa{i} & & \ipa{u}\\
Mid-high & \ipa{e} & & \ipa{o}\\
Mid-low & \ipa{E} & & \ipa{O}\\
Low & & \ipa{a}
\end{tabular}\end{center}

\tab *'\ipa{p} *'\ipa{c} *'\ipa{\textbardotlessj} *'\ipa{g} appear confined to C$_1$ position; *\ipa{N}, to C$_2$ position.

\tab Hedinger also considers the Manenguba languages (and possibly the Mbo languages in general) as sharing a common ancestor with Proto-Bantu instead of being descended from it, although the author seems to use the abbreviation ``PM" to refer to Proto-Manenguba.

\tab Due to the scarcity of available resources on Niger-Congo historical phonology, there will likely be many overlaps or contradictions in the available data, maybe more so than in other sections, even Indo-European. What is included in the Index is what is available.

\tab (From Hedinger, Robert (1987), {\it The Manenguba Languages (Bantu A.15, Mbo Cluster) of Cameroon})

\subsection{Proto-Potou-Akanic-Bantu to Proto-Bantu}{\it Pogostick Man}, from Stewart, John M. (2002), ``The potential of proto-Potou-Akanic-Bantu as a pilot Proto-Niger-Congo, and the reconstructions updated". JALL 23:197 -- 224

\tab {\it NB: For at least the first batch of sound changes herein, the sound changes applying to those consonants in \#U will also apply in U$_2$ under the following conditions, as reported by Stewart (2002): If V$_2$ = V$_1$ (vowel nasality does not necessarily have to be the same, however), changes affecting the vowels will also affect V$_2$. If C is an approximant, changes involving a nasalized V$_1$ will also affect C$_2$ and V$_2$.}

\ipa{\~{V}} \change\ \ipa{\~{l}} / \#_\\
\ipa{u \~{u}} \change\ \ipa{i \~{\i}} / \#R[-labial]_\\
\ipa{\~{V} \~{I}} \change\ \ipa{\~{u} \~{\i}} / \#N[-labial]_\\
\ipa{i \~{\i}} V[-round] \change\ \ipa{uI u\~{I}} V[+round] / C\super w_\\
\ipa{\~{V}} \change\ \ipa{V} / \#R[-labial]_\\
\ipa{\!j} C \change\ \ipa{c} R / \#_\\
\ipa{\!g\super w} \change\ \ipa{w} / \#_V[-nas]\\
\ipa{\!g\super w} \change\ \ipa{\~{w}} / \#_V[+nas]\\
V[+nas] \change\ V[-nas] / \#S[+voiced]_\\
\ipa{\~{I}} \change\ \ipa{\~{E}} / \#(C)V_C\\
\ipa{\textturnmrleg\ \~{\textturnmrleg}} \change\ \ipa{j \~{\j}} / \#C_\\
\ipa{I \~{I}} \change\ \ipa{i \~{\i}} / \#(C)V[-high]C_\\
\ipa{\~{V}} \change\ \ipa{m} / \#(C)V_\\
O[+nas -voiced] \change\ O[-nas] / \#(C)VC_

\subsubsection{Pre-Proto-Bantu to Proto-Bantu}{\it Pogostick Man}, from Hedinger, Robert (1987), {\it The Manenguba Languages (Bantu A.15, Mbo Cluster) of Cameroon}

S[+ lenis] N[+ lenis] \change\ S[- lenis] N[- lenis] / in C$_1$ position\\
N[+ lenis] N[-lenis] \change\ N[- lenis] \ipa{\super n}S / in C$_2$ position\\
\ipa{d}$_2$ \change\ \ipa{d} / in C$_2$ position

\paragraph{Proto-Bantu to Sebirwa}{\it Pogostick Man}, from Chebanne, A. (2000), ``The Sebirwa language: a synchronic and diachronic account". {\it Pula: Botswana Journal of African Studies} 14(2)

\ipa{i u} VS \change\ \ipa{j w} A / _V[+high +ATR]\\
S \change\ S\super h \change\ A\super h\\
V[+high +ATR] \change\ V[+high -ATR]\\
NC \change\ C[-voiced] / \#_ (in nouns)\\
NC \change\ N[+same POA]C / \#_ (in verbs)\\
\ipa{t d l} \change\ \{\ipa{\:t,t\super j}\} \{\ipa{\:d,d\super j}\} \{\ipa{\:l,l\super j,L}\} (The paper is a bit unclear as to which is meant, as the transcription and the textual aspects of the paper seem to disagree here)\\
\ipa{p t d c \textbardotlessj\ k g} \change\ \ipa{F \*r} \{\ipa{d,l}\} \ipa{t\super h} \O\ \ipa{h} \{\O,\ipa{g}\}\\
\O\ \change\ \ipa{g} / \#\ipa{n}_V (in verbs)\\
\ipa{l} \change\ \ipa{d} / \ipa{n}_

\paragraph{Proto-Bantu to Tswana}{\it Whimemsz}, from Creissels, Dennis (1999), ``Remarks on the Sound Corresponences between Proto-Bantu and Tswana (S.31), with Particular Attention to Problems Involving *j (or *y), *\ipa{\v*j} and Sequences *NC". {\it Bantu Historical Linguistics: Theoretical and Empirical Perspectives}, ed. Jean-Marie Hombert and Larry M. Hyman

\{\ipa{p,t,tS,k}\} \{\ipa{mp,nt,\textltailn tS,nk}\} \{\ipa{(m)b,(n)d,(\textltailn)dZ,(n)g}\} \{\ipa{m,n}\} \change\ \ipa{s ts\super h ts \textltailn} / _\ipa{i}V\\
\{\ipa{tS,k}\} \{\ipa{ntS,nk,r}\} \{\ipa{\textltailn dZ,ng}\} \ipa{mp mb p b} \{\ipa{d,l}\} \ipa{m n} \change\ \ipa{s ts\super h ts tS\super h(w) tS(w) S(w) dZ(w) dZ Nw \textltailn} / _\{\ipa{I,e}\}V\\
\{\ipa{p,t,tS,k}\} \{\ipa{mp,nt,\textltailn tS,nk}\} \{\ipa{(m)b,(n)d,(\textltailn)dZ,(n)g}\} \{\ipa{m,n,\textltailn}\} \change\ \ipa{sw ts\super hw tsw \textltailn w} / _\ipa{i}V\\
\ipa{mp mb p b m} \change\ \ipa{tS\super h(w) tS(w) S(w) dZ(w) Nw} / _\{\ipa{U,o}\}V\\
``In these cases, the initial vowel of the sequence drops following the consonant change"; Whimemsz doesn't specify if all V$_1$V$_2$ sequences drop the V$_1$\\
\{\ipa{tS,k}\} \{\ipa{\textltailn tS,nk}\} \{\ipa{ndZ,ng}\} \change\ \ipa{s ts\super h ts} / _\{\ipa{i,I,e}\}\\
\ipa{nk k} \change\ \ipa{k\super h h} / _\ipa{u}\\
\ipa{mp nt \textltailn tS nk mp nd \textltailn dZ ng} \change\ \ipa{p\super h t\super h t\textbeltl\super h q\super h p t t\textbeltl\ k}\\
\ipa{p t tS k} \{\ipa{dZ,g}\} \ipa{\textltailn} \change\ \ipa{h r t\textbeltl\super h X} \O\ \ipa{n}

\paragraph{Sam}

\subparagraph{Proto-Sam to Amu}{\it Pogostick Man}, from Nurse, Derek (1985), ``Dentality, Areal Features, and Phonological Change in Northeastern Bantu". In {\it Studies of African Linguistics} 16(3):243 -- 279

\tab {\it NB: Due to the source, only changes creating dental consonants are considered here.}

\ipa{nt\c{c} t\c{c}} \change\ \ipa{(\|[n)\|[t\super h \|[t} \\ %]]]
\{\ipa{ndj,nz}\} \change\ \ipa{\|[n\|[d} %]]

\subparagraph{Proto-Sam to Bajuni}{\it Pogostick Man}, from Nurse, Derek (1985), ``Dentality, Areal Features, and Phonological Change in Northeastern Bantu". In {\it Studies of African Linguistics} 16(3):243 -- 279

\tab {\it NB: Due to the source, only changes creating dental consonants are considered here.}

\ipa{nt\c{c} t\c{c}} \change\ \ipa{(\|[n)\|[t\super h \|[t} \\ %]]]
\{\ipa{ndj,nz}\} \ipa{\textltailn z} \change\ \ipa{\|[n\|[d \|[n D} %]]]

\subparagraph{Proto-Sam to Mwiini}{\it Pogostick Man}, from Nurse, Derek (1985), ``Dentality, Areal Features, and Phonological Change in Northeastern Bantu". In {\it Studies of African Linguistics} 16(3):243 -- 279

\tab {\it NB: Due to the source, only a few changes, mostly concerning creating dental consonants, are considered here.}

\ipa{nt\c{c} t\c{c}} \change\ \ipa{(\|[n)\|[t\super h \|[t} \\ %]]]
\ipa{ndj} \change\ \ipa{\|[n\|[d} \\ %]]
\ipa{\textltailn} \change\ \{\ipa{\textltailn,\|[n}\}%]

\subparagraph{Proto-Sam to Siu-Pate}{\it Pogostick Man}, from Nurse, Derek (1985), ``Dentality, Areal Features, and Phonological Change in Northeastern Bantu". In {\it Studies of African Linguistics} 16(3):243 -- 279

\tab {\it NB: Due to the source, only changes creating dental consonants are considered here.}

\ipa{nt\c{c} t\c{c}} \change\ \ipa{(\|[n)\|[t\super h \|[t} \\ %]]]
\{\ipa{ndj,nz}\} \ipa{z} \change\ \ipa{\|[n\|[d D} %]]

\subparagraph{Proto-Sam to Proto-Aweera}{\it Pogostick Man}, from Nurse, Derek (1985), ``Dentality, Areal Features, and Phonological Change in Northeastern Bantu". In {\it Studies of African Linguistics} 16(3):243 -- 279

\tab {\it NB: Due to the source, only a few changes are considered here.}

\ipa{nz z c} \change\ \ipa{\textltailn \|[d S} %]

\subparagraph{Proto-Sam to Lower Pokomo}{\it Pogostick Man}, from Nurse, Derek (1985), ``Dentality, Areal Features, and Phonological Change in Northeastern Bantu". In {\it Studies of African Linguistics} 16(3):243 -- 279

\tab {\it NB: Due to the source, only a change creating a dental consonant is considered here.}

\ipa{l} \change\ \ipa{\|[d} %]

\subsubsection{Pre-Proto-Bantu to Proto-Manenguba}{\it Pogostick Man}, from Hedinger, Robert (1987), {\it The Manenguba Languages (Bantu A.15, Mbo Cluster) of Cameroon}

\tab {\it NB: In Hedinger's notation, an apostrophe indicates a lenis consonant in Pre-Proto-Bantu.}

*\ipa{\textbardotlessj} may have turned into one of \{\ipa{c,(n)z}\}?\\
'\ipa{p} '\ipa{t} '\ipa{d}/\ipa{d}$_2$ \ipa{c} \{'\ipa{\textbardotlessj}\} '\ipa{k g} \change\ \ipa{f l \textbardotlessj\ s} \O\ \{\ipa{w},\O\} \{\ipa{k,w}\} / in C$_1$ position\\
N[+ lenis] \change\ N[- lenis] / in C$_1$ position\\
\ipa{p t} '\ipa{t} \{'\ipa{d,d}$_2$\} \ipa{c k} '\ipa{k} \change\ \ipa{b d l} \{\ipa{l},\O\} \ipa{\textbardotlessj\ g} \O\ / in C$_2$ position\\
'\ipa{m m} '\ipa{n n} \{'\ipa{\textltailn,\textltailn}\} \ipa{N} \change\ \ipa{m} \{\ipa{\super mb,m}\} \ipa{n} \{\ipa{\super nd,n}\} \ipa{\textltailn} \{\ipa{\super Ng,N}\} / in C$_2$ position\\
\{\ipa{u,o}\} \{\ipa{E,e,i}\} \change\ \ipa{w j} / C_\ipa{a} in noun roots\\
\{\ipa{u,o}\} \{\ipa{E,e,i}\} \change\ \ipa{w j} / C_(\ipa{a}) in verb roots\\
\{\ipa{u,o}\} \{\ipa{E,e,i}\} \change\ \ipa{w j} / C_\$V in noun class prefixes\\
\ipa{f} \change\ \ipa{h} (perhaps not in all languages?)

\subsection{Proto-Potou-Akanic-Bantu to Proto-Potou-Akanic}{\it Pogostick Man}, from Stewart, John M. (2002), ``The potential of proto-Potou-Akanic-Bantu as a pilot Proto-Niger-Congo, and the reconstructions updated". JALL 23:197 -- 224

C \change\ J[+nas] / \#(C)V[+nas]_\\
\ipa{\textturnmrleg \~{\textturnmrleg}} \change \ipa{l \~{l}} / \#(C)V_\\
C \change\ J / \#(C)V_\\
\ipa{I}(C\ipa{I}) \change\ \ipa{e}(C\ipa{i}) / \ipa{t}_; ``vowel nasalizations are retained either way on each"\\
\ipa{\textbardotlessj\ \!j j c \!g\super w} \change\ \ipa{c \textbardotlessj\ \!j t \r*{\!g}p}\\
(N)V$_1$[+mid +nas](\~{l}) \change\ CV[-nas]\ipa{n}\\
V[+nas](\ipa{\~{V},\~{l}}) \change\ V[-nas](\ipa{m,n}) / \#J[+voiced]_

\subsubsection{Proto-Potou-Akanic to Proto-Akanic}{\it Pogostick Man}, from Stewart, John M. (2002), ``The potential of proto-Potou-Akanic-Bantu as a pilot Proto-Niger-Congo, and the reconstructions updated". JALL 23:197 -- 224

V[+high +ATR](C(V[+high -ATR])) \change\ \#(C)V[-high +ATR](CV[+high +ATR]) / \#J[+dorsal -voiced]_\\
\ipa{E} \change\ \ipa{ia} / \#(C)_\\
R[-voiced] R[+voiced] W \change\ Z[-voiced] O[-voiced] F / \#_\\
\ipa{\textturnmrleg\ \~{\textturnmrleg} \~{w}} \change\ \ipa{h \~{h} \~{h}\super w} / \#_\\
\ipa{\~{h} h\super w} \change\ \ipa{C C\super w}\\
\ipa{h} \change\ \ipa{\textltailn\~{\i}} / \#_\ipa{\~{a}}\\
\ipa{h} \change\ \ipa{w} / \#_\\
\ipa{t} \change\ \ipa{c} / \#_V[-nas]

\paragraph{Proto-Akanic to Akan}{\it Pogostick Man}, from Stewart, John M. (2002), ``The potential of proto-Potou-Akanic-Bantu as a pilot Proto-Niger-Congo, and the reconstructions updated". JALL 23:197 -- 224

\ipa{l \~{l}} \change\ \ipa{j \~{\j}} / \#_\\
C[+dorsal] \change\ C\super w / _V[+round]\\
V \change\ V[-round] / \#C[+dorsal]_C[-labial]\\
\ipa{j}V[+nas] \change\ \ipa{j}V[-nas] / \#_\\
\ipa{f} \change\ \ipa{j} / \#_V[-nas]\\
\ipa{f} \change\ \ipa{\~{\j}} / \#_V[+nas]\\
\{\ipa{p,\~{V}}\} \ipa{c \t*{kp}} \change\ \ipa{f s p} / \#_\\
\ipa{n} \change\ \ipa{N} / \#(C)V_\\
N \change\ S (I'm not sure what's going on here in the paper, but here it is presented anyway for your enjoyment)\\
V \change\ \O\ / \#(C)VC[-coronal]_\\
\ipa{V l} \change\ \ipa{w \*r} / \#(C)V_\\
V[+high] \change\ V[+nas] / \#(C)_N\\
\ipa{i} \change\ \O\ / \#C_\ipa{a}\\
\ipa{\~{\i}} \change\ \O\ / \#_\ipa{\~{a}}

\subsection{Volta-Congo}

\subsubsection{Volta-Niger}

\paragraph{Gbe}

\subparagraph{Proto-Gbe to Aj\'{a}}{\it Pogostick Man}, from Capo, Hounkpati B.C. (1991), {\it A Comparative Phonology of Gbe}

\ipa{E \~E} \change\ \ipa{e \~e}\\
\ipa{\{o,O\} \{E,e\}} \change\ \ipa{u i} / _\ipa{i}\\
V[+ nas - high] \change\ [+ high] / _\ipa{i}\\
\ipa{j}\{\ipa{\~a,\~e}\} \change\ \ipa{\{4,\~4\}} / _E\\
\ipa{j} \change\ \ipa{\textltailn} / _V[+ nas]\\
\ipa{K} \change\ \ipa{j} / _\ipa{i}\\
\ipa{X K} \change\ \ipa{s z} / _\{\ipa{i,j}\}\\
\ipa{X\super w h\super w} \change\ \ipa{w p}

\subparagraph{Proto-Aj\'{a} to Hwe}{\it Pogostick Man}, from Capo, Hounkpati B.C. (1991), {\it A Comparative Phonology of Gbe}

\ipa{t d} \change\ \ipa{tS dZ} / _\{\ipa{u,i}\}

\subparagraph{Proto-Gbe to Proto-Fon}{\it Pogostick Man}, from Capo, Hounkpati B.C. (1991), {\it A Comparative Phonology of Gbe}

\ipa{\{ts,t\super h\} \{dz,d\super H\}} \change\ \ipa{s z}\\
\ipa{\{t,k\}j \{d,g\}j} \change\ \ipa{tS dZ}\\
\ipa{h\super w} \change\ \ipa{K\super w}\\
\ipa{oi Oi \~Oi \{a,E\}i \{\~E,\~e\}i ei} \change\ \ipa{oe OE \~O\~E EE \~E\~E ee}\\
\ipa{\~ai} \change\ \ipa{\~E\~E}

\subparagraph{Proto-Gbe to Proto-Gen}{\it Pogostick Man}, from Capo, Hounkpati B.C. (1991), {\it A Comparative Phonology of Gbe}

\ipa{X\super w} \change\ \ipa{p}\\
\ipa{ts dz} \change\ \ipa{s z}\\
\ipa{\{t,k\}j \{d,g\}j} \change\ \ipa{tS dZ}\\
\ipa{t\super h d\super H} \change\ \ipa{t d}\\
\ipa{h\super w} \change\ \ipa{\{w,K\super w\}}\\
\ipa{w} \change\ \ipa{N}\\
\ipa{E \~E} \change\ \ipa{e \~e}\\
\ipa{j} \change\ \ipa{\textltailn} / _V[+ nas]

\subparagraph{Proto-Gbe to Proto-Phla-Pher\'{a}}{\it Pogostick Man}, from Capo, Hounkpati B.C. (1991), {\it A Comparative Phonology of Gbe}

\ipa{\{ts,t\super h\} \{dz,d\super H\}} \change\ \ipa{s z}\\
\ipa{h\super w} \change\ \ipa{K\super w}

\subparagraph{Proto-Phla-Pher\'{a} to Alada}{\it Pogostick Man}, from Capo, Hounkpati B.C. (1991), {\it A Comparative Phonology of Gbe}

\ipa{t d} \change\ \ipa{S Z} / _\ipa{j}\\
\ipa{j} \change\ \O\ / \{\ipa{S,Z}\}_\\
\ipa{k g} \change\ \ipa{S Z} / _\ipa{i}

\subparagraph{Proto-Gbe to Proto-Vhe}{\it Pogostick Man}, from Capo, Hounkpati B.C. (1991), {\it A Comparative Phonology of Gbe}

\ipa{X\super w K\super w} \change\ \ipa{F B}\\
\ipa{\{E,e\} \{~E,\~e\}} \change\ \ipa{@ \~@}\\
\ipa{h\super w} \change\ \ipa{w}\\
\ipa{w} \change\ \ipa{G} / _\{\ipa{a},E\}\\
\ipa{w} \change\ \ipa{\{w,N\}}\\
\ipa{j} \change\ \ipa{\textltailn} / _V[+ nas]

\subparagraph{Proto-Vhe to Ad\'{a}ngbe}{\it Pogostick Man}, from Capo, Hounkpati B.C. (1991), {\it A Comparative Phonology of Gbe}

V \change\ V[+ nas] / N_\\
V[+ nas] \change\ V[- nas] / C_ ! C = N\\
\ipa{ts dz} \change\ \ipa{s z}\\
\ipa{k g} \change\ \ipa{tS dZ} / _\ipa{i}

\subparagraph{Proto-Vhe to Av\'{e}no}{\it Pogostick Man}, from Capo, Hounkpati B.C. (1991), {\it A Comparative Phonology of Gbe}

\ipa{@} \change\ \ipa{e} / _\{\ipa{i,j}\}\\
\ipa{t d} \change\ \ipa{tS dZ} / _\{\ipa{u,i}\}\\
\ipa{k g s} \change\ \ipa{ts dz S} / _\ipa{i}\\
\ipa{t d} \change\ \ipa{ts dz} / _\ipa{j}\\

\subparagraph{Proto-Vhe to Awalan}{\it Pogostick Man}, from Capo, Hounkpati B.C. (1991), {\it A Comparative Phonology of Gbe}

\ipa{a} \change\ \ipa{e} / _\ipa{\{i,j\}}\\
\ipa{\{o,O\} @} \change\ \ipa{u i} / _\ipa{i}\\
\ipa{t d} \change\ \ipa{tS dZ} / _\{\ipa{u,i}\}\\
\ipa{s \{k,ts\} \{g,dz\}} \change\ \ipa{S tS dZ} / _\ipa{i}\\
\ipa{X K} \change\ \ipa{S \{Z,j\}} / _\{\ipa{u,i,j}\}

\subparagraph{Proto-Vhe to Kp\'{a}ndo}{\it Pogostick Man}, from Capo, Hounkpati B.C. (1991), {\it A Comparative Phonology of Gbe}

\ipa{@ \~@} \change\ \ipa{E \~E}\\
\ipa{\{t,k\} \{d,g\}} \change\ \ipa{ts dz} / _\ipa{i}\\
\ipa{t d} \change\ \ipa{tS dZ} / _\ipa{j}\\
\ipa{j} \change\ \O\ / \{\ipa{ts,dz}\}_\\
V \change\ [+ round] / \ipa{w}_

\subparagraph{Proto-Vhe to Pec\'{\i}}{\it Pogostick Man}, from Capo, Hounkpati B.C. (1991), {\it A Comparative Phonology of Gbe}

\ipa{@ \~@} \change\ \ipa{E \~E}\\
\ipa{\{k,ts\} \{g,dz\}} \change\ \ipa{tS dZ} / _\ipa{i}\\
V \change\ [+ round] / \ipa{w}_

\subparagraph{Proto-Vhe to T\d{\ipa{O}}wun}{\it Pogostick Man}, from Capo, Hounkpati B.C. (1991), {\it A Comparative Phonology of Gbe}

\ipa{@} \change\ \ipa{e} / _\{\ipa{i,j}\}\\
\ipa{n} \change\ \ipa{N} / _\ipa{\~u}\\
\ipa{k g} \change\ \ipa{tS dZ} / _\ipa{i}

\subparagraph{Proto-Vhe to Wac\'{\i}}{\it Pogostick Man}, from Capo, Hounkpati B.C. (1991), {\it A Comparative Phonology of Gbe}

\ipa{@} \change\ \ipa{e} / _\{\ipa{i,j}\}\\
\ipa{\{k,ts\} \{g,dz\}} \change\ \ipa{tS dZ} / _\ipa{i}\\
V \change\ [+ round] / \ipa{w}_

\clearpage

\section{Nyulnyulan}The following phonemic inventory for Proto-Nyulnyulan is adapted from Bowern (2004).

\begin{center}
\begin{tabular}{c | c c c c c}
& Labial & Alveolar & Retroflex & Palatal & Velar \\ \hline
Nasal & \ipa{m} & \ipa{n} & \ipa{\:n} & \ipa{\textltailn} & \ipa{N}\\
Stop & \ipa{b} & \ipa{d} & \ipa{\:d} & \ipa{\textbardotlessj} & \ipa{g}\\
Rhotic & & \ipa{r} & \ipa{\:r}\\
Lateral & & \ipa{l} & \ipa{\:l} & \ipa{L}\\
Glide & \ipa{w} & & & \ipa{j}
\end{tabular}

\begin{tabular}{c | c c c}
& Front & Central & Back\\ \hline
High & \ipa{i i:} & & \ipa{u u:}\\
Low & & \ipa{a a:}\end{tabular}
\end{center}

(From Bowern, Claire Louise (2004), ``Bardi Verb Morphology in Historical Perspective")

\subsection{Proto-Nyulnyulan to Bardi}{\it Pogostick Man}, from Bowern, Claire Louise (2004), ``Bardi Verb Morphology in Historical Perspective"

\{\ipa{w,j}\} \change\ \O\ / \#_\\
\{\ipa{w,j}\} \change\ \O\ / V$_0$_V$_0$\\
\ipa{awu} \change\ \ipa{o}\\
\ipa{aji} \change\ \ipa{i:} / when unstressed\\
\ipa{i\{w,j\}} \change\ \O\ / _\ipa{a}, when unstressed\\
\ipa{i} \change\ \ipa{u} / _\ipa{ju}\\
\ipa{u} \change\ \ipa{i} / _\ipa{j}\\
\ipa{j} \change\ \O\ / \ipa{i}_\\
\ipa{\textbardotlessj} \change\ \ipa{j} / V_V\\
\ipa{\textltailn} \change\ \O\ / _\#\\
\ipa{ubu aba} \change\ \ipa{u: a:} / when stressed\\
\ipa{ib} \change\ \O\ / _\ipa{i}, when unstressed\\
\ipa{b} \change\ \ipa{w} / \ipa{a}_\ipa{u}\\
\ipa{agu} \change\ \ipa{o}\\
\ipa{i(:)b ik} \change\ \ipa{iw ij} / _\ipa{a}\\
V$_0$ \change\ \O\ / V(C)(C)V$_0$(C)(C)_\# (with some exceptions)\\
Some vowel deletions, the conditioning of which the author does not elaborate upon\\
V \change\ V\ipa{:} / when stressed ?

\clearpage

\section{Oto-Manguean}\tab Rensch (1977) reconstructs Proto-Oto-Manguean as having had the following phonemic inventory:

\begin{center}
\begin{tabular}{c | c c c c}
& Alveolar & Palatal & Velar & Laryngeal\\ \hline
Nasal & \ipa{n}\\
Plosive & \ipa{t} & & \ipa{k k\super w} & \ipa{P}\\
Fricative & \ipa{s} & & & \ipa{h}\\
Liquid & & Y & \ipa{w}\end{tabular}

\begin{tabular}{c | c c}
& Front & Back\\ \hline
High & \ipa{i} & \ipa{u}\\
Low & \ipa{e} & \ipa{a}\end{tabular}
\end{center}

\tab Vowels could have had one of four tones, the first of which is denoted as a high tone.

\tab (From Rensch, Calvin R. (1977), ``Classification of the Otomanguean Languages and the Position of Tlapanec". {\it Summer Institute of Linguistics Publications in Linguistics} 55:53 -- 108) %TO DO: MORE CORRESPONDENCES

\subsection{Chatino}\tab Unless otherwise noted specifically, for Chatino correspondences, assume vowels may be either long or short.

\subsubsection{Proto-Chatino to Papabuco Chatino}{\it Pogostick Man}, from Upson, B.W., and Robert E. Longacre (1965), ``Proto-Chatino Phonology''. International Journal of American Linguistics 31(4):312 -- 322

\ipa{t} \change\ \ipa{R} / _\{\ipa{u,\~{e}}\} when unstressed\\
\ipa{t} \change\ \ipa{R} / _"\ipa{a}\\
\ipa{t} \change\ \ipa{tS} / _"\{\ipa{e,iP}\}\\
\ipa{t} \change\ \ipa{S} / \ipa{i}_\ipa{i}\\
\ipa{t} \change\ \ipa{s} / ! ``in cluster with \v{s}" (presumably [\ipa{S}])\\
\ipa{t\super j} \change\ \ipa{s} / _\ipa{u}\\
\ipa{t\super j} \change\ \ipa{tS} / else\\
\ipa{k} \change\ \ipa{g} / \ipa{a}_\ipa{a}\\
\ipa{k\super w} \change\ \ipa{R} / \#_\ipa{ek}\\
\ipa{k\super w} \change\ \ipa{b} / else\\
\ipa{P} \change\ \O\ (?)\\
\{\ipa{c,tS}\} \change\ \ipa{S}\\
\ipa{s} \change\ \ipa{tS} (in certain cases? Not a lot of data available on this one)\\
\ipa{l} \change\ \{\ipa{l\super j,n}\} ``under obscure conditions"\\
\ipa{n} \change\ \ipa{n\super j}\\
\ipa{n\super j} \change\ \ipa{l} / \#_\ipa{i}\\
\ipa{h} \change\ \ipa{d} / _\ipa{a} (\textellipsis lolwut)\\
\ipa{h} \change\ \ipa{t} / else (\textellipsis again, lolwut)\\
\ipa{j} \change\ \ipa{n\super j} ``under obscure conditions (PC morphophonemics?)"\\
\ipa{i} \change\ \ipa{e} / \ipa{tS}_\\
\ipa{\~{\i}} \change\ \ipa{i}\\
\ipa{e} \change\ \ipa{a} / _\{\ipa{l,Pn}\} when unstressed\\
\ipa{e} \change\ \ipa{i} / \{\ipa{k\super j,nt}\}_ when stressed\\
\ipa{e} \change\ \ipa{i} / in a few data sets ``where obscure morphological developments (in the ultimate or penultimate syllable) have resulted in regressive assimilation of vowel quality"\\
\ipa{\~{e}} \change\ \ipa{a} / \ipa{t\super j}_\\
\ipa{\~{e}} \change\ \ipa{i} / \ipa{h}_\\
\ipa{\~{e}} \change\ \ipa{i} / in U[+long -stress]\\
\ipa{\~{e}} \change\ \ipa{e} / else\\
\ipa{a} \change\ \{\ipa{i,e}\} ``under special conditions"\\

\subsubsection{Proto-Chatino to Tataltepec Chatino}{\it Pogostick Man}, from Upson, B.W., and Robert E. Longacre (1965), ``Proto-Chatino Phonology''. International Journal of American Linguistics 31(4):312 -- 322

\ipa{t} \change\ \ipa{t\super j} / _\ipa{\~{o}} (\textellipsis again\textellipsis lolwut)\\
\ipa{t} \change\ \ipa{tS} / \#_\ipa{i\c{c}}\\
\ipa{t\super j} \change\ \ipa{tS} / _\ipa{i}[-long -stress]\\
\ipa{t\super j} \change\ \ipa{t} / _\ipa{a}\\
\ipa{k\super j} \change\ \ipa{t\super j}\\
\ipa{c} \change\ \ipa{tS} / \ipa{i}_\\
\ipa{tS} \change\ \ipa{c}\\
\ipa{s} \change\ \ipa{S} / E_ (? Not a lot of data available on this one)\\
\ipa{S} \change\ \ipa{s}\\
\ipa{l n} \change\ \ipa{l\super j n\super j} / \ipa{e}_ in U[-long -stress]\\
\ipa{\c{c}} \change\ \ipa{P} (? Not sure if I'm reading the phone(me)s right on this one)\\
\ipa{e} \change\ \ipa{a} / _\ipa{P} in U[-stress]\\
\ipa{et el en} \change\ \ipa{it\super j el\super j en\super j}\\
\ipa{e} \change\ \ipa{i} / _\ipa{j}\\
\ipa{E} \change\ \ipa{\~{\i}} / S_\#
\ipa{E} \change\ \ipa{i} / \ipa{n}_\#\\
\ipa{E} \change\ \ipa{e} / _\ipa{P}\#, in monosyllables

\subsubsection{Proto-Chatino to Yaitepec Chatino}{\it Pogostick Man}, from Upson, B.W., and Robert E. Longacre (1965), ``Proto-Chatino Phonology''. International Journal of American Linguistics 31(4):312 -- 322

\ipa{t\super j} \change\ \ipa{tj}\\
Some consonant disharmony involving reflexes of *\ipa{k}, *\ipa{k\super w}\\
\ipa{k} \change\ \ipa{tS} / _E\ipa{k(\super w)}\\
\ipa{k\super j} \change\ \ipa{k} / _\{\ipa{a,\~{a}}\}\\
\ipa{k\super j} \change\ \ipa{kj} / else\\
\ipa{k\super w} \change\ \ipa{w} / _\ipa{e:j}\\
\ipa{k\super w} \change\ \ipa{w} / _\ipa{ek}\\
\ipa{k\super w} \change\ \ipa{\textturnw} / \ipa{ku}_ (medial)\\
\ipa{k\super w} \change\ \ipa{kw} / else\\
\ipa{ts} \change\ \ipa{tS} / \#_\{\ipa{a,\~{o}}\}\ipa{P}\\
\ipa{ts} \change\ \ipa{tS} / V[+high]_\\
\ipa{ts} \change\ \ipa{S} / \#_\ipa{\~{e}}\\
\ipa{c} \change\ \ipa{ts}\\
\ipa{tS} \change\ \ipa{ts} / \ipa{a}_\\
\ipa{s} \change\ \ipa{S} / V_V\\
\ipa{s} \change\ \ipa{S} / if /\ipa{l}/ is present in the same syllable\\
\ipa{s} \change\ \ipa{tS} / \#_\ipa{a}\\
\ipa{S} \change\ \ipa{s} / _\{\ipa{ik,e,\~{\i},\~{e}}\}\\
\ipa{S} \change\ \ipa{ts} / _\ipa{i}, in monosyllables\\
\ipa{l\super j} \change\ \ipa{l} / \#_ in U[-long -stress]\\
\ipa{l\super j} \change\ \ipa{lj} / else\\
\ipa{n\super j} \change\ \ipa{j} / _\ipa{\~{a}} (with some exceptions?)\\
\ipa{n\super j} \change\ \ipa{nj} / else\\
\ipa{h} \change\ \ipa{P} / _\ipa{\~{a}}\\
\ipa{\c{c}} \change\ \ipa{hj}\\
\ipa{h\super w} \change\ \ipa{\textturnw}\\
\ipa{\~{e}} \change\ \ipa{\~{\i}} / \{\ipa{t,h}\}_\#\\
\ipa{\~{e}} \change\ \ipa{\~{\i}} / _\ipa{P}\#\\
\ipa{\~{e}} \change\ \ipa{\~{\i}} / \ipa{P}_\# ``in one case"

\subsubsection{Proto-Chatino to Zenzontepec Chatino}{\it Pogostick Man}, from Upson, B.W., and Robert E. Longacre (1965), ``Proto-Chatino Phonology''. International Journal of American Linguistics 31(4):312 -- 322

\tab {\it NB: This set is likely very incomplete.}

\ipa{k\super j} \change\ \ipa{tS}\\
\ipa{e} \change\ \ipa{i} / \ipa{l}_ when unstressed\\
\ipa{e} \change\ \ipa{i} / \ipa{k\super j}_ when stressed

\subsection{Proto-Oto-Manguean to Tlapanec}{\it Pogostick Man}, from Rensch, Calvin R. (1977), ``Classification of the Otomanguean Languages and the Position of Tlapanec". {\it Summer Institute of Linguistics Publications in Linguistics} 55:53 -- 108

\tab {\it NB: Y here refers to some sort of palatalizing element; H, to some laryngeal.}

\ipa{k\super w} \change\ \ipa{p}\\
\ipa{n} \change\ \{\ipa{n,\textltailn}\} (the latter ``under obscure conditions")\\
\ipa{j} \change\ \ipa{l}(V)\\
Y\ipa{t} Y\ipa{nt} Y\ipa{s} \change\ \ipa{tS dZ S}\\
\{\ipa{ns,nt}\} \ipa{nk\super w nj nw} \change\ \ipa{(n)d (m)b r m}\\
\ipa{nk} \change\ \{\ipa{g,N}\} (the latter ``under obscure conditions")\\
\{\ipa{in,en}\} \ipa{an} \change\ \ipa{a u}\\
\{\ipa{i}H\ipa{n,e}H\ipa{n,a}H\ipa{n}\} \ipa{u}H\ipa{n} \change\ \ipa{\~{a} \~{u}}\\
``No clearly distinct reflex of **un has been identified"; the author speculates that this most likely turned into /u/, but does not rule out /o/ as a reflex\\
\ipa{e} \change\ \ipa{i}\\
\ipa{P} \change\ \O\ / \#_\\
\ipa{h} \change\ \ipa{S} / _C[-voice] (? ``both h and \v{s} occur before nasals, so it is possible that \v{s} has a separate source in Proto Otomanguean")\\
\ipa{h} \change\ CV\ipa{P}V / _\# (or possibly in just any final syllable?)

\clearpage 

\section{Penutian}

\subsection{Utian}\tab Callaghan (1983, 1988) reconstructs the following inventory for Proto-Utian:

\begin{center}\begin{tabular}{c | c c c c c c c}
& Bilabial & Coronal & Retroflex & Palatal & Velar & Glottal \\ \hline
Nasal & \ipa{m} & \ipa{n}\\
Plosive & \ipa{p} & \ipa{\|[t} & \ipa{\:t} & \ipa{tS} & \ipa{k k\super w} & \ipa{P} \\ %]
Fricative & & \ipa{\|[s} & \ipa{\:s} & \ipa{S} & & \ipa{h}\\ %]
Resonant & & \ipa{l R} & \ipa{j} & \ipa{w}\end{tabular}

\begin{tabular}{c | c c c}
& Front & Central & Back\\ \hline
High & \ipa{i i:} & \ipa{11:} & \ipa{u u:}\\
Mid & \ipa{e e:} & & \ipa{o o:}\\
Low & & \ipa{a a:}\end{tabular}
\end{center}

\tab (From Callaghan, Catherine A. (1983), ``Proto-Utian Derivational Verb Morphology". {\it Proceedings of the 1982 Conference on Far Western American Indian Languages, Occasional Papers on Linguistics Number 11}; and Callaghan, Catherine A. (1988), ``Proto-Utian Stems" in {\it In Honor of Mary Haas})

\subsubsection{Proto-Utian to Proto-Costanoan (Ohlone)}{\it CatDoom}, from Callaghan, Catherine A. (1983), ``Proto-Utian Derivational Verb Morphology". {\it Proceedings of the 1982 Conference on Far Western American Indian Languages, Occasional Papers on Linguistics Number 11}; and Callaghan, Catherine A. (1988), ``Proto-Utian Stems" in {\it In Honor of Mary Haas}

\ipa{S} \change\ \ipa{h}\\
\ipa{i}\$C\ipa{i} \change\ \ipa{e}\$C\ipa{e} / _C ! _\ipa{R}\\
\ipa{1}\$C\ipa{1} \change\ \ipa{e}\$C\ipa{e} / _C ! _\ipa{R}\\
\ipa{e} \change\ \ipa{i} / _(C\textellipsis)\ipa{u}\\
\ipa{k} \change\ \ipa{\:s} / _\ipa{i}\\
\ipa{k} \change\ \ipa{\:s} / \ipa{i}_\\
\ipa{k} \change\ \ipa{\|[s} / _\{\ipa{1,u}\}\\%]\\
\ipa{k} \change\ \ipa{\|[s} / \{\ipa{1,u}\}_\\%]
\ipa{l} \change\ \ipa{R} / ! _\$ or \ipa{o}_\\
\ipa{tS} \change\ \ipa{\:s} / _\#\\
\ipa{o} \change\ \ipa{a} / ! \ipa{o}(C\textellipsis)_ or _(C\textellipsis)\{\ipa{o,i}\}\\
\ipa{1} \change\ \ipa{e} / CC_\#\\
\ipa{1} \change\ \{\ipa{e,i}\} / CC_\\
\ipa{1} \change\ \ipa{i}

\paragraph{Proto-Costanoan to Chochenyo}{\it CatDoom}, from Callaghan, Catherine A. (1983), ``Proto-Utian Derivational Verb Morphology". {\it Proceedings of the 1982 Conference on Far Western American Indian Languages, Occasional Papers on Linguistics Number 11}; and Callaghan, Catherine A. (1988), ``Proto-Utian Stems" in {\it In Honor of Mary Haas}

\ipa{t\super j \:s} \change\ \ipa{j S}\\
\ipa{k\super w} \change\ \ipa{k} / \#_\\
\ipa{k\super w} \change\ \ipa{w} / else\\
\ipa{l} \change\ \ipa{R} / V_V\\
\ipa{a} \change\ \ipa{e} / \ipa{il}_\\
\ipa{o} \change\ \ipa{u} / _(C\textellipsis)\ipa{i}

\paragraph{Proto-Utian to Proto-Miwok}{\it CatDoom}, from Callaghan, Catherine A. (1983), ``Proto-Utian Derivational Verb Morphology". {\it Proceedings of the 1982 Conference on Far Western American Indian Languages, Occasional Papers on Linguistics Number 11}; and Callaghan, Catherine A. (1988), ``Proto-Utian Stems" in {\it In Honor of Mary Haas}

\ipa{k\super w S} \change\ \ipa{w \:s}\\
\ipa{\:t} \change\ \ipa{tS} / _\ipa{e}\\
\ipa{\:t} \change\ \ipa{tS} / \ipa{e}_

\subparagraph{Proto-Miwok to Proto-Western Miwok}{\it CatDoom}, from Callaghan, Catherine A. (1983), ``Proto-Utian Derivational Verb Morphology". {\it Proceedings of the 1982 Conference on Far Western American Indian Languages, Occasional Papers on Linguistics Number 11}; and Callaghan, Catherine A. (1988), ``Proto-Utian Stems" in {\it In Honor of Mary Haas}

\ipa{\|[s} \change\ \ipa{\:s}\\%]
\ipa{\:t} \change\ \ipa{tS} / \{\ipa{a:,o:}\}_\\
\ipa{1} \change\ \{\ipa{u,i}\}

\paragraph{Proto-Costanoan to Mutsun}{\it CatDoom}, from Callaghan, Catherine A. (1983), ``Proto-Utian Derivational Verb Morphology". {\it Proceedings of the 1982 Conference on Far Western American Indian Languages, Occasional Papers on Linguistics Number 11}; and Callaghan, Catherine A. (1988), ``Proto-Utian Stems" in {\it In Honor of Mary Haas}

\ipa{\:s} \change\ \ipa{\|[s}\\%]
\ipa{\:t} \change\ \{\ipa{\:t,ts,tS}\} / _\{\ipa{j,R}\}\\
\ipa{k\super w} \change\ \ipa{k} / \#_\\
\ipa{k\super w} \change\ \{\ipa{k,w}\} / else\\
\ipa{l} \change\ \ipa{R} / V_V\\
\ipa{a} \change\ \ipa{e} / \ipa{il}_\\
\ipa{o} \change\ \ipa{u} / _(C\textellipsis)\ipa{i}

\paragraph{Proto-Costanoan to Rumsen}{\it CatDoom}, from Callaghan, Catherine A. (1983), ``Proto-Utian Derivational Verb Morphology". {\it Proceedings of the 1982 Conference on Far Western American Indian Languages, Occasional Papers on Linguistics Number 11}; and Callaghan, Catherine A. (1988), ``Proto-Utian Stems" in {\it In Honor of Mary Haas}

\ipa{t\super j} \change\ \ipa{tS}\\
\ipa{h} \change\ \{\ipa{h,x,P}\}\\
\ipa{\:t} \change\ \{\ipa{\:t,tS}\} / \{\ipa{a:,o:}\}_\\
\ipa{\:t} \change\ \{\ipa{\:t,tS}\} / \{\ipa{i,e,o}\}\$_\\
\ipa{\:t} \change\ \{\ipa{\:t,tS}\} / _\{\ipa{j,R}\}\\
\ipa{k\super w} \change\ \ipa{k} / \#_\\
\ipa{k\super w} \change\ \{\ipa{k,w}\} / else\\
\ipa{l} \change\ \ipa{R} / V_V\\
\ipa{a} \change\ \ipa{e} / \ipa{il}_\\
\ipa{i} \change\ \ipa{e} / _C(C)\ipa{o}C\\
\ipa{o} \change\ \ipa{u} / _(C\textellipsis)\ipa{i}

\subsection{Wintun}\tab Shepherd (2005) reconstructs the following inventory for Proto-Wintun:

\begin{center}\begin{tabular}{c | c c c c c c}
& Bilabial & Alveolar & Palatal & Velar & Uvular & Glottal \\ \hline
Nasal & \ipa{m} & \ipa{n}\\
Plosive & \ipa{p p\super h p' b} & \ipa{t t\super h t' d} & & \ipa{k k\super h k' g} & \ipa{q q\super h q'} & \ipa{P}\\
Fricative & & \ipa{s \textbeltl} & & \ipa{x} & \ipa{X} & \ipa{h}\\
Affricate & & \ipa{t\textbeltl'} & \ipa{tS tS\super h tS'}\\
Liquid & \ipa{w} & \ipa{r l} & \ipa{j}\end{tabular}

\begin{tabular}{c | c c c}
& Front & Central & Back \\ \hline
High & \ipa{i i:} & & \ipa{u u:}\\
Mid & \ipa{e e:} & & \ipa{o o:}\\
Low & & \ipa{a a:}\end{tabular}\end{center}

\tab Shepherd further notes that ``PW vowel length before continuants appears to be non-distinctive in many instances''.

\tab (From Shepherd, Alice (2005), ``Proto-Wintun". {\it UC Publications in Linguistics}. \textless\url{http://escholarship.org/uc/item/8dq1f3jj}\textgreater)

\subsubsection{Proto-Wintuan to Nomlaki}{\it Pogostick Man}, from Shepherd, Alice (2005), ``Proto-Wintun". {\it UC Publications in Linguistics}. \textless\url{http://escholarship.org/uc/item/8dq1f3jj}\textgreater

V\ipa{r}V \change\ \{V\ipa{:},M\}\\
\ipa{r} \change\ \ipa{j} / _\#\\
\ipa{tS\super h k\super h q\super h} \{\ipa{x,X}\} \change\ \ipa{tS k(\super h)} \{\ipa{k\super h,q\super h,X}\} \ipa{k\super h}

\subsubsection{Proto-Wintuan to Patwin}{\it Pogostick Man}, from Shepherd, Alice (2005), ``Proto-Wintun". {\it UC Publications in Linguistics}. \textless\url{http://escholarship.org/uc/item/8dq1f3jj}\textgreater

\ipa{tS tS\super h tS'} \change\ \ipa{t t\super h t'}\\
\ipa{k(\super h) k' q(\super h) q'} \change\ \ipa{tS(h) tS' k\super h k'}\\
\ipa{x X} \change\ \ipa{s h}

\subsubsection{Proto-Wintuan to South Patwin}{\it Pogostick Man}, from Shepherd, Alice (2005), ``Proto-Wintun". {\it UC Publications in Linguistics}. \textless\url{http://escholarship.org/uc/item/8dq1f3jj}\textgreater

\ipa{r} \change\ \{\ipa{r,j}\}\\
\ipa{tS tS\super h tS'} \change\ \ipa{t t\super h t'}\\
\ipa{k(\super h) k' q(\super h) q'} \change\ \ipa{tS(\super h) tS' k(\super h) k'}\\
\ipa{x} \change\ \ipa{s}\\
\ipa{X} \change\ \O\ (?)

\subsubsection{Proto-Wintuan to Wintu}{\it Pogostick Man}, from Shepherd, Alice (2005), ``Proto-Wintun". {\it UC Publications in Linguistics}. \textless\url{http://escholarship.org/uc/item/8dq1f3jj}\textgreater

\ipa{tS\super h} \change\ \ipa{tS}
\ipa{k\super h q\super h} \change\ \ipa{k X}

\subsection{Yokutsan}

\tab Whistler and Golla (1986) reconstruct the following phonological inventory for Proto-Yokuts:

\begin{center}\begin{tabular}{c | c c c c c c}
& Labial & Dental & Retroflex & Palatal & Velar & Glottal \\ \hline
Nasal & \ipa{m m\super P} & \ipa{n n\super P} & & & \ipa{N N\super P}\\
Stop & \ipa{p p\super h p'} & \ipa{t t\super h t'} & \ipa{\:t \:t\super h \:t'} & & \ipa{k k\super h k'} & \ipa{P}\\
Affricate & & \ipa{(ts) ts\super h ts'}\\
Fricative & & \ipa{s} & \ipa{\:s} & & \ipa{x} & \ipa{h}\\
Approximant & & \ipa{l l\super P} & & \ipa{j j\super P} & \ipa{w w\super P}
\end{tabular}\end{center}

\begin{center}\begin{tabular}{c | c c c}
& Front & Central & Back\\ \hline
High & \ipa{i i:} & \ipa{1 1:} & \ipa{u u:}\\
Mid & & & \ipa{o o:}\\
Low & & \ipa{a a:}\end{tabular}\end{center}

\tab It is further instructive to note some morphophonetic processes in Proto-Yokuts:
\begin{itemize}
\item S \change\ S\super h / _\{C,\#\} (also holds for affricates)
\item N\ipa{P} \change\ N\ipa{\super P}
\item \O\ \change\ \ipa{P} / V_V
\end{itemize}

\tab (From Whistler, Kenneth W., and Golla, Victor (1986), ``Proto-Yokuts Reconsidered''. \textit{International Journal of American Linguistics} Vol. 52, No. 4 (Oct. 1986))

\subsubsection{Proto-Yokuts to General Yokuts}{\it CatDoom}, from Whistler, Kenneth W., and Golla, Victor (1986), ``Proto-Yokuts Reconsidered''. \textit{International Journal of American Linguistics} Vol. 52, No. 4 (Oct. 1986)

\ipa{i: 1: u:} \change\ \ipa{e: @: o:} (this change sometimes did not occur)\\
\ipa{e: @:} \change\ \ipa{e @} (as a result of ablaut)\\
\ipa{o} \change\ \ipa{u} / _C\ipa{i}

\paragraph{General Yokuts to Buena Vista Yokuts}{\it CatDoom}, from Whistler, Kenneth W., and Golla, Victor (1986), ``Proto-Yokuts Reconsidered''. \textit{International Journal of American Linguistics} Vol. 52, No. 4 (Oct. 1986)

\ipa{t t\super h t'} \change\ \ipa{ts ts\super h ts'} / \#_ ``(in some words, conditioning factors unclear)"\\
V[+ high] \change\ \ipa{a} / V[+ high]C_(C)\#\\
\ipa{t\super h} \change\ \ipa{s} / \#_\ipa{u}

\paragraph{Buena Vista Yokuts to Hometwoli}{\it CatDoom}, from Whistler, Kenneth W., and Golla, Victor (1986), ``Proto-Yokuts Reconsidered''. \textit{International Journal of American Linguistics} Vol. 52, No. 4 (Oct. 1986)

\O\ \change\ \ipa{h} / V(\ipa{:})_, when stressed (only sometimes, ``particularly before consonants")

\paragraph{Buena Vista Yokuts to Tulamni}{\it CatDoom}, from Whistler, Kenneth W., and Golla, Victor (1986), ``Proto-Yokuts Reconsidered''. \textit{International Journal of American Linguistics} Vol. 52, No. 4 (Oct. 1986)

\ipa{1(:) @(:)} \change\ \ipa{i(:) e(:)}\\
V\ipa{P} \change\ V\ipa{:} / stressed

\subsubsection{Buena Vista Yokuts to Proto-Nim-Yokuts}{\it CatDoom}, from Whistler, Kenneth W., and Golla, Victor (1986), ``Proto-Yokuts Reconsidered''. \textit{International Journal of American Linguistics} Vol. 52, No. 4 (Oct. 1986)

\ipa{s} \change\ \ipa{S}\\
\ipa{ts ts\super h ts'} \change\ \ipa{tS tS\super h tS'}

\paragraph{Proto-Nim-Yokuts to Proto-Tule-Kaweah}{\it CatDoom}, from Whistler, Kenneth W., and Golla, Victor (1986), ``Proto-Yokuts Reconsidered''. \textit{International Journal of American Linguistics} Vol. 52, No. 4 (Oct. 1986)

\ipa{t t\super h t'} \change\ \ipa{tS tS\super h tS'} / \#_ ``(in some words, conditioning factors unclear)"\\
\ipa{l} \change\ \ipa{t}

\subparagraph{Proto-Tule-Kaweah to Wikchamni}{\it CatDoom}, from Whistler, Kenneth W., and Golla, Victor (1986), ``Proto-Yokuts Reconsidered''. \textit{International Journal of American Linguistics} Vol. 52, No. 4 (Oct. 1986)

\ipa{\:s} \change\ \ipa{s} ``(sometimes remains allophonically in word-initial position before back vowels, but not consistently)"

\subparagraph{Proto-Tule-Kaweah to Yawdanchi}{\it CatDoom}, from Whistler, Kenneth W., and Golla, Victor (1986), ``Proto-Yokuts Reconsidered''. \textit{International Journal of American Linguistics} Vol. 52, No. 4 (Oct. 1986)

``\ipa{S} may have merged with \ipa{\:s} in some positions"

\paragraph{Proto-Nim-Yokuts to Northern Yokuts}{\it CatDoom}, from Whistler, Kenneth W., and Golla, Victor (1986), ``Proto-Yokuts Reconsidered''. \textit{International Journal of American Linguistics} Vol. 52, No. 4 (Oct. 1986)

\ipa{1(:) @(:)} \change\ \ipa{i(:) e(:)}\\
\ipa{N} \change\ \ipa{n}

\subparagraph{Northern Yokuts to Gashowu}{\it CatDoom}, from Whistler, Kenneth W., and Golla, Victor (1986), ``Proto-Yokuts Reconsidered''. \textit{International Journal of American Linguistics} Vol. 52, No. 4 (Oct. 1986)

\ipa{p t \:t k} \change\ \ipa{b d \:d g}

\subparagraph{Northern Yokuts to Kings Valley Yokuts}{\it CatDoom}, from Whistler, Kenneth W., and Golla, Victor (1986), ``Proto-Yokuts Reconsidered''. \textit{International Journal of American Linguistics} Vol. 52, No. 4 (Oct. 1986)

\ipa{i} \change\ \ipa{u} / \ipa{u}C_

\subparagraph{Northern Yokuts to Valley Yokuts}{\it CatDoom}, from Whistler, Kenneth W., and Golla, Victor (1986), ``Proto-Yokuts Reconsidered''. \textit{International Journal of American Linguistics} Vol. 52, No. 4 (Oct. 1986)

``o-raising rule (\ipa{o} \textgreater\ \ipa{u} / _C\ipa{i}) ceases to be productive"

\subparagraph{Valley Yokuts to Chukchansi}{\it CatDoom}, from Whistler, Kenneth W., and Golla, Victor (1986), ``Proto-Yokuts Reconsidered''. \textit{International Journal of American Linguistics} Vol. 52, No. 4 (Oct. 1986)

/\ipa{s S \:s}/ may be a single alternating phoneme\\
\ipa{\:t \:t\super h \:t'} \change\ \ipa{tS tS\super h tS'}\\
\ipa{tS tS\super h tS'} \change\ \ipa{ts ts\super h ts'}

\subparagraph{Valley Yokuts to Tachi}{\it CatDoom}, from Whistler, Kenneth W., and Golla, Victor (1986), ``Proto-Yokuts Reconsidered''. \textit{International Journal of American Linguistics} Vol. 52, No. 4 (Oct. 1986)

\ipa{\:t \:t\super h \:t'} \change\ \ipa{\:t\:s \:t\:s\super h \:t\:s'} ``(\ipa{\:t'} remains unchanged in careful speech)"

\subparagraph{Valley Yokuts to Yawelmani}{\it CatDoom}, from Whistler, Kenneth W., and Golla, Victor (1986), ``Proto-Yokuts Reconsidered''. \textit{International Journal of American Linguistics} Vol. 52, No. 4 (Oct. 1986)

\ipa{S} \change\ \ipa{s}\\
\ipa{tS tS\super h tS'} \change\ \ipa{ts ts\super h ts'} (except in ``lexicalized diminutives'', where these go to \ipa{\:t\:s \:t\:s\super h \:t\:s'})

\subsubsection{Proto-Yokuts to Palewyami}{\it CatDoom}, from Whistler, Kenneth W., and Golla, Victor (1986), ``Proto-Yokuts Reconsidered''. \textit{International Journal of American Linguistics} Vol. 52, No. 4 (Oct. 1986)

\ipa{s} \change\ \ipa{S} / _\ipa{i}\\
\ipa{t t\super h t'} \change\ \ipa{ts ts\super h ts'} / \#_ ``(in some words; conditioning factors unclear)''\\
\ipa{ts ts\super h ts'} \change\ \ipa{tS tS\super h tS'} / _\ipa{i}\\
\ipa{1(:)} \change\ \ipa{i(:)}\\
\{\ipa{u,a}\} \change\ \ipa{e} / _CVC\#, when stressed (short only)\\
\ipa{i} \change\ \ipa{e} / _CVC\#, when stressed (! _H, short only)\\
V \change\ \ipa{e} / C"VC_\\
V \change\ \ipa{i} / C"\ipa{i}C_\\
V \change\ \ipa{u} / C"\ipa{u}C_\\
V \change\ \ipa{o} / C"\ipa{o}C_

\clearpage

\section{Quechumaran}\tab Orr and Longacre (1968) reconstruct Proto-Quechumaran as having the following inventory:

\begin{center}\begin{tabular}{c | c c c c c c c c}
& Bilabial & Alveolar & Postalveolar & Retroflex & Palatal & Velar & Uvular & Glottal\\ \hline
Nasal & \ipa{m} & \ipa{n} & & & \ipa{\textltailn}\\
Plosive & \ipa{p} & \ipa{t} & & & & \ipa{k} & \ipa{q} & \ipa{P}\\
Fricative & \ipa{F} & \ipa{s} & \ipa{S} & \ipa{\:s} & & & \ipa{X} & \ipa{h}\\
Affricate & & \ipa{ts} & \ipa{tS} & \ipa{\:t\:s}\\
Liquid & & \ipa{r l} & \ipa{L}\\
Semivowel & & & & & \ipa{j} & \ipa{w}\end{tabular}
\begin{tabular}{c | c c c}
& Front & Central & Back\\ \hline
High & \ipa{i} & & \ipa{u}\\
Low & & \ipa{a}\end{tabular}\end{center}

\tab (From Orr, Carolyn, and Robert E. Longacre (1968), ``Proto-Quechumaran". {\it Language} 44(3):528 -- 555)

\subsection{Proto-Quechumaran to Ayachuco}{\it Pogostick Man}, from Orr, Carolyn, and Robert E. Longacre (1968), ``Proto-Quechumaran". {\it Language} 44(3):528 -- 555

\ipa{p' t' tS' k' q'} \change\ \ipa{p t tS k q}\\
\ipa{h} \change\ \O\ / \{\ipa{p,t,k,q}\}_\\
\ipa{q} \change\ \ipa{X}\\
\ipa{X} \change\ \ipa{q} / \ipa{n}_\\
\ipa{ts(h) tSh} \change\ \ipa{tS s}\\
\ipa{\:t\:s} \change\ \ipa{s} / _K\\
\ipa{\:t\:s} \change\ \ipa{tS} / _V\\
\ipa{\:t\:s'} \change\ \ipa{tS'}\\
\ipa{F(',\super h)} \change\ \ipa{p}\\
\ipa{S \:s} \change\ \ipa{s h}

\subsection{Proto-Quechumaran to Bolivia}{\it Pogostick Man}, from Orr, Carolyn, and Robert E. Longacre (1968), ``Proto-Quechumaran". {\it Language} 44(3):528 -- 555

\ipa{qh} \change\ \ipa{h} / _\ipa{r}\\
\ipa{ph th kh qh} \change\ \ipa{p\super h t\super h k\super h q\super h}\\
\{\ipa{k,q}\} \change\ \ipa{h} / _\{C,\#\}\\
\ipa{ts} \{\ipa{tsh,tSh}\} \change\ \ipa{tS tS\super h}\\
\ipa{tS} \change\ \ipa{S} / _\ipa{q}\\
\ipa{tS'} \change\ \ipa{tS}\\
\ipa{\:t\:s} \change\ \ipa{s} / _K\\
\ipa{\:t\:s} \change\ \ipa{tS} / _V\\
\ipa{\:t\:s'} \change\ \ipa{tS'}\\
\ipa{F(') Fh} \change\ \ipa{p(') p\super h}\\
\ipa{S} \change\ \ipa{s}\\
\ipa{\:s} \change\ \ipa{h} / _\{\ipa{a,i}\}\\
\ipa{\:s} \change\ \O\ / _\ipa{u}\\
\ipa{X} \change\ \ipa{q\super h} / \#_\\
\ipa{j} \change\ \O\ / \ipa{i}_\{\ipa{a,u}\}\\
\ipa{j} \change\ \O\ / \ipa{u}_\ipa{L}

\subsection{Proto-Quechumaran to Cuzco}{\it Pogostick Man}, from Orr, Carolyn, and Robert E. Longacre (1968), ``Proto-Quechumaran". {\it Language} 44(3):528 -- 555

\ipa{h} \change\ \O\ / \ipa{nq}_\\
\ipa{h} \change\ \O\ / \#\ipa{q}_\{\ipa{i,u}\}\\
\ipa{ph} \change\ \ipa{p} / \#_VA\\
\ipa{ph} \change\ \ipa{p} / \#_\ipa{a}C[+sibilant]\\
\ipa{p'} \change\ \ipa{p}/ \#_C[+sibilant]\\
\ipa{ph} \change\ \ipa{p} / \ipa{a}_ (?)\\
\ipa{ph} \change\ \ipa{p\super h}\\
\ipa{kh} \change\ \ipa{k} / \ipa{r}_\\
\ipa{kh} \change\ \ipa{k} / \#_\ipa{a}\\
\ipa{kh} \change\ \ipa{k} / \#_ ``in a word with two back vowels''\\
\ipa{kh qh} \change\ \ipa{k\super h q\super h}\\
\ipa{ts tsh} \change\ \ipa{tS tS\super h}\\
\ipa{tS} \change\ \ipa{s} / _\ipa{q}\\
\ipa{tSh} \change\ \ipa{s} / _E\\
\{\ipa{tSh,tS'}\} \change\ \ipa{tS}\\
\ipa{\:t\:s} \change\ \ipa{s} / _K\\
\ipa{\:t\:s} \change\ \ipa{tS} / _V\\
\ipa{\:t\:s'} \change\ \ipa{tS} / \ipa{n}_\\
\ipa{\:t\:s'} \change\ \ipa{tS'}\\
\ipa{F(') Fh} \change\ \ipa{p(') p\super h}\\
\ipa{S \:s} \change\ \ipa{s h}\\
\ipa{X} \change\ \ipa{q\super h} / \#_

\subsection{Proto-Quechumaran to Huar\'{a}s}{\it Pogostick Man}, from Orr, Carolyn, and Robert E. Longacre (1968), ``Proto-Quechumaran". {\it Language} 44(3):528 -- 555

\ipa{p' t' tS' k' q'} \change\ \ipa{p t ts k q}\\
\ipa{h} \change\ \O\ / \ipa{p}_\\
\ipa{t\super h} \change\ \ipa{t}\\
\ipa{q} \change\ \ipa{\t{qX}}\\
\ipa{ts(h) tS} \change\ \ipa{tS ts}\\
\ipa{tS} \change\ \ipa{ts} / _\ipa{q}\\
\ipa{\:t\:s} \change\ \ipa{s} / _K\\
\ipa{\:t\:s} \change\ \ipa{tS} / _V\\
\ipa{F(',\super h)} \change\ \ipa{p}\\
\ipa{\:s} \change\ \ipa{S} / _\ipa{a}\\
\ipa{\:s} \change\ \ipa{h} / _\{\ipa{i,u}\}\\
\ipa{\textltailn} \change\ \ipa{n}\\
\ipa{aw aj} \{\ipa{uj,ij}\} \change\ \ipa{u: e: i:}

\subsection{Proto-Quechumaran to Putamayo}{\it Pogostick Man}, from Orr, Carolyn, and Robert E. Longacre (1968), ``Proto-Quechumaran". {\it Language} 44(3):528 -- 555

\ipa{p' t' k' q'} \change\ \ipa{p t k q}\\
\ipa{p} \change\ \ipa{b} / \ipa{m}_\\
\ipa{t(\super h)} \change\ \ipa{d} / \ipa{n}_\\
\ipa{t\super h} \change\ \ipa{t}\\
\ipa{k} \change\ \ipa{g} / \ipa{n}_\\
\ipa{k} \change\ \ipa{g} / _\{L,\ipa{j}\}\\
\ipa{h} \change\ \O\ / \ipa{ts}_\\
\ipa{tS'} \change\ \ipa{tS}\\
\ipa{\:t\:s} \change\ \ipa{tS} / _V\\
\ipa{F(',\super h)} \change\ \ipa{p}\\
\ipa{\:s} \change\ \ipa{s}\\
\ipa{h} \change\ \O\ / \#_

\subsection{Proto-Quechumaran to Quito}{\it Pogostick Man}, from Orr, Carolyn, and Robert E. Longacre (1968), ``Proto-Quechumaran". {\it Language} 44(3):528 -- 555

\ipa{q} \change\ \ipa{h} / \{\ipa{r,s}\}_\\
\ipa{p' k' q'} \change\ \ipa{p k q}\\
\ipa{p} \change\ \ipa{b} / \ipa{m}_\\
\ipa{t(\super h)} \change\ \ipa{d} / \ipa{n}_\\
\ipa{t'} \change\ \ipa{t\super h} / \#_\ipa{i}\\
\ipa{t'} \change\ \ipa{t}\\
\ipa{k} \change\ \ipa{g} / _\#\\
\ipa{k} \change\ \ipa{g} / \ipa{n}_\\
\ipa{k} \change\ \ipa{g} / _\{L,\ipa{j}\}\\
\ipa{k'} \change\ \ipa{h} / \ipa{j}_\\
\ipa{ts} \change\ \ipa{dz} / ! \#_\\
\ipa{h} \change\ \O\ / \ipa{ts}_\\
\ipa{tS} \change\ \ipa{S} / _\ipa{q}\\
\ipa{tSh} \change\ \ipa{S}\\
\ipa{tS'} \change\ \ipa{tS}\\
\ipa{\:t\:s} \change\ \ipa{S} / _K\\
\ipa{\:t\:s} \change\ \ipa{tS} / _V\\
\ipa{F} \change\ \ipa{p} / \ipa{r}_\\
\{\ipa{F',Fh}\} \change\ \ipa{F}\\
\ipa{\:s} \change\ \ipa{S}\\
\ipa{X} \change\ \ipa{h} \#_

\subsection{Proto-Quechumaran to Riobamba}{\it Pogostick Man}, from Orr, Carolyn, and Robert E. Longacre (1968), ``Proto-Quechumaran". {\it Language} 44(3):528 -- 555

\ipa{q} \change\ \ipa{k}
\ipa{qh} \change\ \ipa{k\super h} / _\ipa{i} ! _\ipa{i}\{\ipa{S,tS}\}\\
\ipa{qh} \change\ \ipa{k}\\
\ipa{p' t' k' q'} \change\ \ipa{p t k q}\\
\ipa{p} \change\ \ipa{b} / \ipa{m}_\\
\ipa{t(\super h)} \change\ \ipa{d} / \ipa{n}_\\
\ipa{t'} \change\ \ipa{t\super h} / \#_\ipa{i}\\
\ipa{k} \change\ \ipa{h} / _\#\\
\ipa{k} \change\ \ipa{g} / _\{L,\ipa{j}\}\\
\ipa{kh} \change\ \ipa{k\super h} / \#_\{\ipa{i,u}\}\\
\ipa{kh} \change\ \ipa{k}\\
\ipa{k ts} \change\ \ipa{g dz} / \ipa{n}_\\
\ipa{h} \change\ \O\ / \ipa{ts}_\\
\ipa{tSh} \change\ \ipa{S}\\
\ipa{tS'} \change\ \ipa{tS}\\
\ipa{\:t\:s} \change\ \ipa{s} / _K\\
\ipa{\:t\:s} \change\ \ipa{tS} / _V\\
\ipa{\:t\:s'} \change\ \ipa{ts}\\
\ipa{F} \change\ \ipa{p\super h} / _V\\
\ipa{F} \change\ \ipa{b} / _\ipa{j}\\
\ipa{s} \change\ \ipa{S} / _C[+alveolar]\\
\ipa{\:s} \change\ \ipa{S}\\
\ipa{X} \change\ \ipa{k\super h} \#_\\
\ipa{\textltailn} \change\ \ipa{n} / ! \ipa{h}_\ipa{i}\\
\ipa{L} \change\ \ipa{Z} / _\{\ipa{a,u}\}

\subsection{Proto-Quechumaran to Santiago}{\it Pogostick Man}, from Orr, Carolyn, and Robert E. Longacre (1968), ``Proto-Quechumaran". {\it Language} 44(3):528 -- 555

\ipa{p' t' tS' k' q'} \change\ \ipa{p t tS k q}\\
\ipa{h} \change\ \O\ / \{\ipa{p,t,k}\}_\\
\{\ipa{k,q}\} \change\ \ipa{h} / _\{C,\#\}\\
\ipa{k} \change\ \ipa{c} / \ipa{j}_ (?)\\
\ipa{ts(h)} \change\ \ipa{tS}\\
\ipa{tS} \change\ \ipa{S} / _\ipa{q}\\
\ipa{\:t\:s} \change\ \ipa{S} / _K\\
\ipa{\:t\:s} \change\ \ipa{tS} / _V\\
\ipa{F(',\super h)} \change\ \ipa{p}\\
\ipa{S} \change\ \ipa{s} / ! \ipa{i}_\ipa{i} or _S\\
\ipa{\:s} \change\ \O\ / _\{\ipa{a,i}\}\\
\ipa{\:s} \change\ \ipa{h} / _\ipa{u}\\
\ipa{h} \change\ \O\ / \#_\\
\ipa{L} \change\ \ipa{Z} / _\{\ipa{a,u}\}\\
\ipa{w} \change\ \O\ / V_V\\
\ipa{w} \change\ \ipa{m} / _\%N\\
\ipa{j} \change\ \O\ / \ipa{i}_\{\ipa{a,u}\}\\
\ipa{j} \change\ \O\ / \ipa{u}_\ipa{L}

\subsection{Proto-Quechumaran to Tena}{\it Pogostick Man}, from Orr, Carolyn, and Robert E. Longacre (1968), ``Proto-Quechumaran". {\it Language} 44(3):528 -- 555

\ipa{q(h)} \change\ \ipa{k}\\
\ipa{p' t' k' q'} \change\ \ipa{p t k q}\\
\ipa{p} \change\ \ipa{b} / \ipa{m}_\\
\ipa{t(\super h)} \change\ \ipa{d} / \ipa{n}_\\
\ipa{t\super h} \change\ \ipa{t}\\
\ipa{k} \change\ \ipa{g} / \ipa{n}_\\
\ipa{k} \change\ \ipa{g} / _\{L,\ipa{j}\}\\
\ipa{h} \change\ \O\ / \ipa{ts}_\\
\ipa{tS} \change\ \ipa{S} / _C\\
\ipa{tSh} \change\ \ipa{S}\\
\ipa{tS'} \change\ \ipa{tS}\\
\ipa{\:t\:s} \change\ \ipa{S} / _K\\
\ipa{\:t\:s} \change\ \ipa{tS} / _V\\
\ipa{\:t\:s'} \change\ \ipa{tS}\\
\ipa{F(',\super h)} \change\ \ipa{p}\\
\ipa{\:s} \change\ \ipa{S}\\
\ipa{h} \change\ \O\ / \#_\\
\ipa{X} \change\ \ipa{k} / \#_\\
\ipa{\textltailn} \change\ \ipa{n} / _\ipa{i}\\
\ipa{w} \change\ \O\ / \#_\ipa{i}

\clearpage

\section{Salishan}\tab Kuipers (1981) gives the following reconstruction for the Proto-Salish phoneme inventory (converted into IPA):

\begin{center}\begin{tabular}{c | c c c c c c}
& Labial & Coronal & Palatal & Velar & Postvelar & Glottal \\ \hline
Nasal & \ipa{m m\super P} & \ipa{n n\super P}\\
Stop & \ipa{p p'} & \ipa{t t'} & & \ipa{k k\super w k' k\super w'} & \ipa{q q\super w q' q\super w'} & \ipa{P}\\
Fricative & & \ipa{s \textbeltl} & & \ipa{x x\super w} & \ipa{X X\super w} & \ipa{h}\\
Affricate & & \ipa{ts ts' t\textbeltl'}\\
Resonant & & \ipa{r r\super P l l\super P} & \ipa{j j\super P} & \ipa{\textturnmrleg\ \textturnmrleg\super P w w\super P} & \ipa{Q Q\super w Q\super P Q\super w\super P}\end{tabular}

\begin{tabular}{c | c c c}
& Front & Central & Back\\ \hline
High & \ipa{i} & & \ipa{u}\\
Mid & & \ipa{@}\\
Low & & \ipa{a}\end{tabular}\end{center}

\tab For the following changes, the superscript numerals $^1$, $^2$, and $^3$ refer to low, mid, and high tones, respectively. Not all Salishan languages have all three tones; for most, there is no tone $^2$ (mid). Vowel pairs in between curly braces $\langle$\{ \}$\rangle$ and with a tilde between are pairs which apparently existed in some sort of ablaut-like alternation.

\tab (From Kuipers, Aert H. (1981), ``On Reconstructing the Proto-Salish Sound System". {\it International Journal of American Linguistics} 47(4):323 -- 335; and Galloway, Brent (1982), ``Proto-Central Salish Phonology and Sound Correspondences". From the {\it 17th International Conference on Salish and Neighboring Languages})

\subsection{Central Salish}

\subsubsection{Proto-Central Salish to Comox}{\it Pogostick Man}, from Galloway, Brent (1982), ``Proto-Central Salish Phonology and Sound Correspondences". From the {\it 17th International Conference on Salish and Neighboring Languages}

\ipa{ts(')} \change\ \ipa{T(')}\\
\ipa{l(\super j)} \change\ \ipa{w} / _\ipa{u}\\
\ipa{l(\super j)} \change\ \ipa{w} / \ipa{u}_\\
\ipa{l(\super j)} \change\ \ipa{j} / else\\
\ipa{s} \change\ \O\ / \#_C\\
\ipa{s} \change\ \O\ / \#_\{\ipa{wa,wi}\}\\
\ipa{w j} \change\ \ipa{g dZ} / _V\\
V$^3$\ipa{P} \change\ V$^3$(\ipa{:})\ipa{P} / _\#\\
\ipa{P} \change\ \O\ V$^3$_RV\\
\ipa{P} \change\ \O\ V$^3$R_V\\
\{\ipa{a}$^3$\raisebox{-0.6ex}{\textasciitilde}\ipa{@}$^3$\} \change\ \{\ipa{@}$^3$,\ipa{a}$^3$\}\\
\{\ipa{a}$^3$\raisebox{-0.6ex}{\textasciitilde}\ipa{i}$^3$\} \change\ \ipa{i}$^3$\\
\{\ipa{a}$^1$\raisebox{-0.6ex}{\textasciitilde}\ipa{i}$^1$\} \change\ \{\ipa{a}$^1$,\ipa{@}$^1$\}\\
\{\ipa{i}$^3$\raisebox{-0.6ex}{\textasciitilde}\ipa{@}$^3$\} \change\ \ipa{@}$^3$\\
\{\ipa{i}$^3$\raisebox{-0.6ex}{\textasciitilde}\ipa{@}$^1$\} \change\ \{\ipa{@}$^1$,\ipa{i}$^1$,\ipa{i}$^3$\}\\
\{\ipa{i}$^1$\raisebox{-0.6ex}{\textasciitilde}\ipa{@}$^1$\} \change\ \ipa{i}$^1$

\subsubsection{Proto-Central Salish to Chilliwack Halkomelem}{\it Pogostick Man}, from Galloway, Brent (1982), ``Proto-Central Salish Phonology and Sound Correspondences". From the {\it 17th International Conference on Salish and Neighboring Languages}

\ipa{n} \change\ \ipa{l}\\
\ipa{Pn} \change\ \O\ / \ipa{m}_\\
\ipa{ts(') tS(')} \change\ \ipa{T(') ts(')}\\
\ipa{l\super j} \change\ \ipa{l}\\
V$^3$\ipa{h} \change\ V$^3$\ipa{:} / _C\\
V$^3$\ipa{P} \change\ V$^3$(\ipa{:}) / _\#\\
\ipa{P} \change\ \O\ / V$^1$_\#\\
V$^3$\ipa{P} \change\ V$^3$\ipa{:} / _O\\
\O\ \change\ V$_0$ / "V$^3_0$_\\
V$^3$\ipa{P}R \change\ V$^3$\ipa{:}R\\
\ipa{P} \change\ \O\ / VR_V$^3$\\
\ipa{lPn} \change\ \ipa{l:} / V$^3$_V\\
\ipa{P} \change\ \O\ / V$^1$_\#\\
\ipa{u}$^3$ \{\ipa{u}$^1$,\ipa{a}$^1$\} \ipa{a}$^3$ \ipa{i}$^1$ \change\ \ipa{a}$^3$ \ipa{@}$^1$ \ipa{E}$^3$ \{\ipa{i}$^1$,\ipa{@}$^1$\}\\
\{\ipa{u}$^3$\raisebox{-0.6ex}{\textasciitilde}\ipa{@}$^3$\} \change\ \{\ipa{o}$^3$,\ipa{@}$^3$,\ipa{a}$^3$\}\\
\{\ipa{a}$^3$\raisebox{-0.6ex}{\textasciitilde}\ipa{@}$^3$\} \change\ \{\ipa{E}$^3$,\ipa{@}$^3$\}\\
\{\ipa{a}$^3$\raisebox{-0.6ex}{\textasciitilde}\ipa{i}$^3$\} \change\ \ipa{E}$^3$\\
\{\ipa{i}$^3$\raisebox{-0.6ex}{\textasciitilde}\ipa{e}$^3$\} \change\ \{\ipa{@}$^3$,\ipa{i}$^3$\}\\
\{\ipa{i}$^3$\raisebox{-0.6ex}{\textasciitilde}\ipa{@}$^1$\} \change\ \ipa{@}$^1$\\
\{\ipa{i}$^1$\raisebox{-0.6ex}{\textasciitilde}\ipa{@}$^1$\} \change\ \ipa{@}$^1$

\subsubsection{Proto-Central Salish to Cowichan Halkomelem}{\it Pogostick Man}, from Galloway, Brent (1982), ``Proto-Central Salish Phonology and Sound Correspondences". From the {\it 17th International Conference on Salish and Neighboring Languages}

\ipa{ts(') tS(')} \change\ \ipa{T(') ts(')}\\
\ipa{s} \change\ \ipa{S} / _\ipa{x\super w}\\
\ipa{x\super j} \change\ \ipa{S}\\
V$^3$\ipa{h} \change\ V$^3$\ipa{:} / _C\\
V$^3_0$\ipa{P}(V$_0$) \change\ \{V$^3_0$\ipa{:},V$^3_0$\ipa{P}V$_0$\}\\
V$^3$\ipa{P}R \change\ \{V$^3$\ipa{P}R,V$^3$\ipa{:}R\ipa{P}\} / _V\\
\ipa{a}$^3$ \ipa{u}$^3$ \{\ipa{a}$^1$,\ipa{u}$^1$\} \ipa{i}$^1$ \change\ \ipa{E}$^3$ \ipa{a}$^3$ \ipa{@}$^1$ \{\ipa{i}$^1$,\ipa{@}$^1$\}\\
\{\ipa{a}$^3$\raisebox{-0.6ex}{\textasciitilde}\ipa{@}$^3$\} \change\ \ipa{@}$^3$\\
\{\ipa{a}$^3$\raisebox{-0.6ex}{\textasciitilde}\ipa{i}$^3$\} \change\ \ipa{a}$^3$\\
\{\ipa{a}$^1$\raisebox{-0.6ex}{\textasciitilde}\ipa{i}$^1$\} \change\ \ipa{@}$^1$\\
\{\ipa{i}$^3$\raisebox{-0.6ex}{\textasciitilde}\ipa{e}$^3$\} \change\ \{\ipa{@}$^3$,\ipa{i}$^3$\}\\
\{\ipa{i}$^3$\raisebox{-0.6ex}{\textasciitilde}\ipa{@}$^1$\} \change\ \ipa{@}$^1$\\
\{\ipa{i}$^1$\raisebox{-0.6ex}{\textasciitilde}\ipa{@}$^1$\} \change\ \ipa{@}$^1$

\subsubsection{Proto-Central Salish to Musqueam Halkomelem}{\it Pogostick Man}, from Galloway, Brent (1982), ``Proto-Central Salish Phonology and Sound Correspondences". From the {\it 17th International Conference on Salish and Neighboring Languages}

\ipa{Pn} \change\ \O\ / \ipa{m}_\\
\ipa{ts(') tS(')} \change\ \ipa{T(') ts(')}\\
\ipa{l\super j} \change\ \ipa{l}\\
V$^3$\ipa{h} \change\ V$^3$\ipa{:} / _C\\
V$^3$\ipa{P} \change\ V$^3$\{\ipa{:,P}\} / _O\\
V$^3_0$\ipa{P}V$_0$ \change\ \{V$^3$\ipa{:},V$^3_0$\ipa{P}V$_0$\}\\
V$^3$\ipa{P}R \change\ \{V$^3$\ipa{P}R,V$^3$\ipa{:}R\ipa{P}\} / _V\\
\ipa{u}$^3$ \{\ipa{u}$^1$,\ipa{a}$^1$\} \ipa{a}$^3$ \ipa{i}$^1$ \change\ \ipa{a}$^3$ \ipa{@}$^1$ \ipa{E}$^3$ \{\ipa{i}$^1$,\ipa{@}$^1$\}\\
\{\ipa{u}$^3$\raisebox{-0.6ex}{\textasciitilde}\ipa{@}$^3$\} \change\ \ipa{@}$^3$\\
\{\ipa{a}$^3$\raisebox{-0.6ex}{\textasciitilde}\ipa{@}$^3$\} \change\ \{\ipa{E}$^3$,\ipa{@}$^3$\}\\
\{\ipa{a}$^3$\raisebox{-0.6ex}{\textasciitilde}\ipa{i}$^3$\} \change\ \{\ipa{a}$^3$,\ipa{E}$^3$\}\\
\{\ipa{a}$^1$\raisebox{-0.6ex}{\textasciitilde}\ipa{i}$^1$\} \change\ \{\ipa{@}$^1$,\ipa{E}$^1$\}\\
\{\ipa{i}$^3$\raisebox{-0.6ex}{\textasciitilde}\ipa{e}$^3$\} \change\ \{\ipa{@}$^3$,\ipa{i}$^3$\}\\
\{\ipa{i}$^3$\raisebox{-0.6ex}{\textasciitilde}\ipa{@}$^1$\} \change\ \ipa{@}$^1$\\
\{\ipa{i}$^1$\raisebox{-0.6ex}{\textasciitilde}\ipa{@}$^1$\} \change\ \ipa{@}$^1$

\subsubsection{Proto-Central Salish to Klallam}{\it Pogostick Man}, from Galloway, Brent (1982), ``Proto-Central Salish Phonology and Sound Correspondences". From the {\it 17th International Conference on Salish and Neighboring Languages}

\ipa{mPn} \change\ \ipa{nP}\\
\ipa{p(') m} \change\ \ipa{ts(') N} / ! _\ipa{u}\\
\ipa{l(\super j)} \change\ \ipa{j}\\
\ipa{x\super j} \change\ \{\ipa{s,S}\} (the latter mainly from borrowings?)\\
\ipa{tS} \change\ \ipa{ts}\\
\ipa{tS'} \change\ \ipa{ts'} / medially\\
\ipa{w j} \change\ \ipa{k\super w tS} / _V\\
V$_0^3$\ipa{P}(V$_0$) \change\ V$_0^3$\ipa{P}V$_0$\\
\ipa{u}$^1$ \change\ \ipa{@}$^1$\\
\ipa{a}$^3$ \change\ \ipa{u}$^3$ / \{C\super w[+uvular],K\super w,\ipa{w}\}\\
\ipa{a}$^3$ \change\ \ipa{u}$^3$ / _\{C\ipa{u},C\super w[+uvular],\ipa{w}\}\\
\ipa{a}$^1$ \change\ \ipa{@}$^1$\\
\{\ipa{u}$^3$\raisebox{-0.6ex}{\textasciitilde}\ipa{@}$^3$\} \change\ \ipa{@}$^3$\\
\{\ipa{a}$^3$\raisebox{-0.6ex}{\textasciitilde}\ipa{@}$^3$\} \change\ \ipa{@}$^3$\\
\{\ipa{a}$^3$\raisebox{-0.6ex}{\textasciitilde}\ipa{i}$^3$\} \change\ \ipa{@}$^3$\\
\{\ipa{a}$^1$\raisebox{-0.6ex}{\textasciitilde}\ipa{i}$^1$\} \change\ \ipa{@}$^1$\\
\{\ipa{i}$^3$\raisebox{-0.6ex}{\textasciitilde}\ipa{@}$^3$\} \change\ \ipa{@}$^3$\\
\{\ipa{i}$^3$\raisebox{-0.6ex}{\textasciitilde}\ipa{@}$^1$\} \change\ \ipa{@}$^1$\\
\{\ipa{i}$^1$\raisebox{-0.6ex}{\textasciitilde}\ipa{@}$^1$\} \change\ \ipa{@}$^1$

\subsubsection{Proto-Central Salish to Lushootseed}{\it Pogostick Man}, from Galloway, Brent (1982), ``Proto-Central Salish Phonology and Sound Correspondences". From the {\it 17th International Conference on Salish and Neighboring Languages}

\ipa{mPn} \change\ \ipa{d}\\
\ipa{m n} \change\ \ipa{b d}\\
\ipa{l\super j} \change\ \ipa{l}\\
\ipa{s} \change\ \{\ipa{S,s}\} / _\ipa{x\super w}\\
\ipa{x\super j} \change\ \ipa{S}\\
\ipa{w j} \change\ \ipa{g\super w dz} / _V\\
V$^3$\ipa{h} \change\ V$^3$\{\ipa{:,P}\} / _C\\
V$^3$\ipa{P} \change\ V$^3$(\ipa{P}) / _O\\
R\ipa{P} \change\ \ipa{P}R / V$^3$C\\
\ipa{i}$^1$ \change\ \{\ipa{i}$^1$,\ipa{@}$^1$\}\\
\{\ipa{u}$^3$\raisebox{-0.6ex}{\textasciitilde}\ipa{@}$^3$\} \change\ \ipa{a}$^3$\\
\{\ipa{a}$^3$\raisebox{-0.6ex}{\textasciitilde}\ipa{@}$^3$\} \change\ \{\ipa{@}$^3$,\ipa{a}$^3$\}\\
\{\ipa{a}$^1$\raisebox{-0.6ex}{\textasciitilde}\ipa{i}$^1$\} \change\ \ipa{i}$^1$\\
\{\ipa{i}$^3$\raisebox{-0.6ex}{\textasciitilde}\ipa{@}$^3$\} \change\ \{\ipa{i}$^1$,\ipa{@}$^1$\}\\
\{\ipa{i}$^3$\raisebox{-0.6ex}{\textasciitilde}\ipa{@}$^1$\} \change\ \{\ipa{i}$^3$,\ipa{i}$^1$\}\\
\{\ipa{i}$^1$\raisebox{-0.6ex}{\textasciitilde}\ipa{@}$^1$\} \change\ \{\ipa{@}$^1$,\ipa{i}$^1$\}

\subsubsection{Proto-Central Salish to Nooksack}\textit{Pogostick Man}, from Galloway, Brent (1982), \textquotedblleft Proto-Central Salish Phonology and Sound Correspondences". From the \textit{17th International Conference on Salish and Neighboring Languages}

n \textrightarrow\hspace{0pt} \O\hspace{0pt} / m\textipa{P}_ \\
l\super j \textrightarrow\hspace{0pt} l \\
s \textrightarrow\hspace{0pt} \{s,\textipa{S}\} / _x\super w \\
s \textrightarrow\hspace{0pt} \textipa{S} / \#_\{x\super j,w\{i,a\},q\super wa\} \\
x\super j \textrightarrow\hspace{0pt} \textipa{S} \\
\O\hspace{0pt} \textrightarrow\hspace{0pt} V$_0$ / "V$^3_0$\textipa{P}_ \\
\textipa{P}R \textrightarrow\hspace{0pt} \{\textipa{P}S,R\} / V$^3$_V \\
\textipa{P} \textrightarrow\hspace{0pt} \O\hspace{0pt} / VR_V$^3$ \\
\textipa{P} \textrightarrow\hspace{0pt} \O\hspace{0pt} / V$^3$R_\{C,\#\} \\
\textipa{P} \textrightarrow\hspace{0pt} \O\hspace{0pt} / V$^1$R_\# \\
\textipa{P} \textrightarrow\hspace{0pt} \O\hspace{0pt} / V$^1$_\# \\
a$^1$ u$^3$ u$^1$ i$^1$ \textrightarrow\hspace{0pt} \ae$^2$ o$^3$ o$^1$ i$^2$ \\
\textipa{@}$^1$ \textrightarrow\hspace{0pt} \ae$^2$ / a$^3$C(C)_ \\
\textipa{@}$^1$ \textrightarrow\hspace{0pt} \ae$^2$ / _C(C)a$^3$ \\
\textipa{@}$^1$ \textrightarrow\hspace{0pt} \ae$^2$ / in some other unspecified circumstances \\
\{u$^3$\raisebox{-0.7ex}{\textasciitilde}\textipa{@}$^3$\} \textrightarrow\hspace{0pt} o$^3$ \\
\{a$^3$\raisebox{-0.7ex}{\textasciitilde}\textipa{@}$^3$\} \textrightarrow\hspace{0pt} \ae$^3$ \\
\{a$^3$\raisebox{-0.7ex}{\textasciitilde}i$^3$\} \textrightarrow\hspace{0pt} \ae$^3$ \\
\{a$^1$\raisebox{-0.7ex}{\textasciitilde}i$^1$\} \textrightarrow\hspace{0pt} \ae$^2$ \\
\{i$^3$\raisebox{-0.7ex}{\textasciitilde}e$^3$\} \textrightarrow\hspace{0pt} \{i$^3$.\textipa{@}$^3$\} \\
\{i$^3$\raisebox{-0.7ex}{\textasciitilde}\textipa{@}$^1$\} \textrightarrow\hspace{0pt} i$^3$ \\
\{i$^1$\raisebox{-0.7ex}{\textasciitilde}\textipa{@}$^1$\} \textrightarrow\hspace{0pt} \{i$^2$,\textipa{@}$^1$\}

\subsubsection{Proto-Central Salish to Lummi Northern Straits}\textit{Pogostick Man}, from Galloway, Brent (1982), \textquotedblleft Proto-Central Salish Phonology and Sound Correspondences". From the \textit{17th International Conference on Salish and Neighboring Languages}

p(') m \textrightarrow\hspace{0pt} \textipa{tS}(') \textipa{N} / ! _u \\
m\textipa{P} \textrightarrow\hspace{0pt} \O\hspace{0pt} / _n \\
ts \textrightarrow\hspace{0pt} s \\
l\super j \textrightarrow\hspace{0pt} l \\
s \textrightarrow\hspace{0pt} \textipa{S} / _x\super w \\
\textipa{tS} \textrightarrow\hspace{0pt} s \\
\textipa{tS}' \textrightarrow\hspace{0pt} ts' / medially \\
j \textrightarrow\hspace{0pt} \textipa{tS} / _V \\
\textipa{P} \textrightarrow\hspace{0pt} \O\hspace{0pt} / V$^3$l_nV \\
\textipa{P} \textrightarrow\hspace{0pt} \{\O,\textipa{P}\} / V$^3$R_\{C,\#\} \\
u$^3$ u$^1$ \textrightarrow\hspace{0pt} o$^3$ \textipa{@}$^1$ \\
a$^3$ \textrightarrow\hspace{0pt} o$^3$ / \{\{C[+ uvular],K\}\super w,w\}_ \\
a$^3$ \textrightarrow\hspace{0pt} o$^3$ / _\{Cu,C[+ uvular]\super w,w\} \\
a$^3$ \textrightarrow\hspace{0pt} e$^3$ / else \\
a$^1$ \textrightarrow\hspace{0pt} \textipa{@}$^1$ \\
\{u$^3$\raisebox{-0.7ex}{\textasciitilde}\textipa{@}$^3$\} \textrightarrow\hspace{0pt} \textipa{@}$^1$ \\
\{a$^3$\raisebox{-0.7ex}{\textasciitilde}\textipa{@}$^3$\} \textrightarrow\hspace{0pt} \textipa{@}$^3$ \\
\{a$^3$\raisebox{-0.7ex}{\textasciitilde}i$^3$\} \textrightarrow\hspace{0pt} \textipa{@}$^3$ \\
\{a$^1$\raisebox{-0.7ex}{\textasciitilde}i$^1$\} \textrightarrow\hspace{0pt} \textipa{@}$^1$ \\
\{i$^3$\raisebox{-0.7ex}{\textasciitilde}e$^3$\} \textrightarrow\hspace{0pt} \textipa{@}$^3$ \\
\{i$^3$\raisebox{-0.7ex}{\textasciitilde}\textipa{@}$^1$\} \textrightarrow\hspace{0pt} \textipa{@}$^1$ \\
\{i$^1$\raisebox{-0.7ex}{\textasciitilde}\textipa{@}$^1$\} \textrightarrow\hspace{0pt} \textipa{@}$^1$

\subsubsection{Proto-Central Salish to Saanich Northern Straits}\textit{Pogostick Man}, from Galloway, Brent (1982), \textquotedblleft Proto-Central Salish Phonology and Sound Correspondences". From the \textit{17th International Conference on Salish and Neighboring Languages}

p(') m \textrightarrow\hspace{0pt} \textipa{tS}(') \textipa{N} / ! _u \\
m\textipa{P}n \textrightarrow\hspace{0pt} n\textipa{P} \\
ts ts' \textrightarrow\hspace{0pt} \{\textipa{T,s}\} \textipa{T}' \\
l\super j \textrightarrow\hspace{0pt} l \\
s \textrightarrow\hspace{0pt} \{\textipa{S,s}\} / _x\super w \\
x\super j \textrightarrow\hspace{0pt} s \\
\textipa{tS} \textrightarrow\hspace{0pt} s \\
\textipa{tS}' \textrightarrow\hspace{0pt} \textipa{T}' / medially \\
j w \textrightarrow\hspace{0pt} \textipa{tS k\super w} / _V \\
V$^3$h \textrightarrow\hspace{0pt} V$^3$(\textipa{:}) / _C \\
\textipa{P}R \textrightarrow\hspace{0pt} R\textipa{P} / V$^3$_V \\
u$^3$ u$^1$ \textrightarrow\hspace{0pt} a$^3$ \textipa{@}$^1$ \\
a$^3$ \textrightarrow\hspace{0pt} e$^3$ / ! \{\{C[+ uvular],K\}\super w,w\}_ or when _\{Cu,C[+ uvular]\super w,w\} \\
a$^1$ \textrightarrow\hspace{0pt} \textipa{@}$^1$ \\
\{a$^3$\raisebox{-0.7ex}{\textasciitilde}\textipa{@}$^3$\} \textrightarrow\hspace{0pt} \{\textipa{@}$^3$,e$^3$\} \\
\{a$^3$\raisebox{-0.7ex}{\textasciitilde}i$^3$\} \textrightarrow\hspace{0pt} \textipa{@}$^3$ \\
\{a$^1$\raisebox{-0.7ex}{\textasciitilde}i$^1$\} \textrightarrow\hspace{0pt} \textipa{@}$^1$ \\
\{i$^3$\raisebox{-0.7ex}{\textasciitilde}e$^3$\} \textrightarrow\hspace{0pt} \{\textipa{@}$^3$,i$^3$\} \\
\{i$^3$\raisebox{-0.7ex}{\textasciitilde}\textipa{@}$^1$\} \textrightarrow\hspace{0pt} \textipa{@}$^1$ \\
\{i$^1$\raisebox{-0.7ex}{\textasciitilde}\textipa{@}$^1$\} \textrightarrow\hspace{0pt} \{\textipa{@}$^1$,i$^1$\}

\subsubsection{Proto-Central Salish to Songish Northern Straits}\textit{Pogostick Man}, from Galloway, Brent (1982), \textquotedblleft Proto-Central Salish Phonology and Sound Correspondences". From the \textit{17th International Conference on Salish and Neighboring Languages}

p(') m \textrightarrow\hspace{0pt} \textipa{tS}(') \textipa{N} / ! _u \\
m\textipa{P}n \textrightarrow\hspace{0pt} n\textipa{P} \\
ts \textrightarrow\hspace{0pt} s \\
l\super j \textrightarrow\hspace{0pt} l \\
s \textrightarrow\hspace{0pt} \{\textipa{S,s}\} / _x\super w \\
x\super j \textrightarrow\hspace{0pt} \{s,\textipa{S}\} (the latter mainly from borrowings?) \\
\textipa{tS}' \textrightarrow\hspace{0pt} ts' / medially \\
j w \textrightarrow\hspace{0pt} \textipa{tS k\super w} / _V \\
V$^3$h \textrightarrow\hspace{0pt} V$^3$\textipa{:} / _C \\
u$^3$ u$^1$ \textrightarrow\hspace{0pt} a$^3$ \textipa{@}$^1$ \\
a$^3$ \textrightarrow\hspace{0pt} a$^3$ / \{\{C[+ uvular],K\}\super w,w\}_ \\
a$^3$ \textrightarrow\hspace{0pt} a$^3$ / _\{Cu,C[+ uvular]\super w,w\} \\
a$^1$ \textrightarrow\hspace{0pt} \textipa{@}$^1$ \\
\{u$^3$\raisebox{-0.7ex}{\textasciitilde}\textipa{@}$^3$\} \textrightarrow\hspace{0pt} \textipa{@}$^3$ \\
\{a$^3$\raisebox{-0.7ex}{\textasciitilde}\textipa{@}$^3$\} \textrightarrow\hspace{0pt} \textipa{@}$^3$ \\
\{a$^3$\raisebox{-0.7ex}{\textasciitilde}i$^3$\} \textrightarrow\hspace{0pt} \textipa{@}$^3$ \\
\{a$^1$\raisebox{-0.7ex}{\textasciitilde}i$^1$\} \textrightarrow\hspace{0pt} \{\textipa{@}$^1$,e$^1$\} \\
\{i$^3$\raisebox{-0.7ex}{\textasciitilde}e$^3$\} \textrightarrow\hspace{0pt} \textipa{@}$^3$ \\
\{i$^3$\raisebox{-0.7ex}{\textasciitilde}\textipa{@}$^1$\} \textrightarrow\hspace{0pt} \textipa{@}$^1$ \\
\{i$^1$\raisebox{-0.7ex}{\textasciitilde}\textipa{@}$^1$\} \textrightarrow\hspace{0pt} \textipa{@}$^1$

\subsubsection{Proto-Central Salish to Sooke Northern Straits}\textit{Pogostick Man}, from Galloway, Brent (1982), \textquotedblleft Proto-Central Salish Phonology and Sound Correspondences". From the \textit{17th International Conference on Salish and Neighboring Languages}

p(') m \textrightarrow\hspace{0pt} \textipa{tS}(') \textipa{N} / ! _u \\
m\textipa{P}n \textrightarrow\hspace{0pt} n\textipa{P} \\
ts \textrightarrow\hspace{0pt} s \\
l(\super j) \textrightarrow\hspace{0pt} j \\
x\super j \textrightarrow\hspace{0pt} \{s,\textipa{S}\} (the latter mainly from borrowings?) \\
\textipa{tS} \textrightarrow\hspace{0pt} s \\
\textipa{tS}' \textrightarrow\hspace{0pt} ts' / medially \\
j w \textrightarrow\hspace{0pt} \textipa{tS k\super w} / _V \\
u$^3$ u$^1$ \textrightarrow\hspace{0pt} a$^3$ \textipa{@}$^1$ \\
a$^3$ \textrightarrow\hspace{0pt} \{a$^3$,o$^3$\} / \{\{C[+ uvular],K\}\super w,w\}_ \\
a$^3$ \textrightarrow\hspace{0pt} e$^3$ \\
a$^1$ \textrightarrow\hspace{0pt} \textipa{@}$^1$ \\
\{u$^3$\raisebox{-0.7ex}{\textasciitilde}\textipa{@}$^3$\} \textrightarrow\hspace{0pt} \textipa{@}$^3$ \\
\{a$^3$\raisebox{-0.7ex}{\textasciitilde}\textipa{@}$^3$\} \textrightarrow\hspace{0pt} \textipa{@}$^3$ \\
\{a$^3$\raisebox{-0.7ex}{\textasciitilde}i$^3$\} \textrightarrow\hspace{0pt} \textipa{@}$^3$ \\
\{a$^1$\raisebox{-0.7ex}{\textasciitilde}i$^1$\} \textrightarrow\hspace{0pt} \textipa{@}$^1$ \\
\{i$^3$\raisebox{-0.7ex}{\textasciitilde}e$^3$\} \textrightarrow\hspace{0pt} \textipa{@}$^3$ \\
\{i$^3$\raisebox{-0.7ex}{\textasciitilde}\textipa{@}$^1$\} \textrightarrow\hspace{0pt} \textipa{@}$^1$ \\
\{i$^1$\raisebox{-0.7ex}{\textasciitilde}\textipa{@}$^1$\} \textrightarrow\hspace{0pt} \textipa{@}$^1$

\subsubsection{Proto-Central Salish to Pentlatch}\textit{Pogostick Man}, from Galloway, Brent (1982), \textquotedblleft Proto-Central Salish Phonology and Sound Correspondences". From the \textit{17th International Conference on Salish and Neighboring Languages}

ts ts' \textrightarrow\hspace{0pt} s ts' \\
l\super j \textrightarrow\hspace{0pt} l \\
x\super j \textrightarrow\hspace{0pt} \textipa{S} \\
\textipa{P} \textrightarrow\hspace{0pt} \O\hspace{0pt} / V$^3$_O \\
\textipa{P} \textrightarrow\hspace{0pt} \O\hspace{0pt} / V$^3$R_\{V,\#\} \\
i$^1$ \textrightarrow\hspace{0pt} \textipa{@}$^1$ \\
\{a$^3$\raisebox{-0.7ex}{\textasciitilde}\textipa{@}$^3$\} \textrightarrow\hspace{0pt} \textipa{@}$^3$ \\
\{a$^1$\raisebox{-0.7ex}{\textasciitilde}i$^1$\} \textrightarrow\hspace{0pt} \{i$^1$,\textipa{@}$^1$\} \\
\{i$^3$\raisebox{-0.7ex}{\textasciitilde}e$^3$\} \textrightarrow\hspace{0pt} \{\textipa{@}$^3$,i$^3$\} \\
\{i$^3$\raisebox{-0.7ex}{\textasciitilde}\textipa{@}$^1$\} \textrightarrow\hspace{0pt} i$^3$ \\
\{i$^1$\raisebox{-0.7ex}{\textasciitilde}\textipa{@}$^1$\} \textrightarrow\hspace{0pt} \{i$^1$,\O\} \\

\subsubsection{Proto-Central Salish to Sechelt}\textit{Pogostick Man}, from Galloway, Brent (1982), \textquotedblleft Proto-Central Salish Phonology and Sound Correspondences". From the \textit{17th International Conference on Salish and Neighboring Languages}

\textipa{P} \textrightarrow\hspace{0pt} \O\hspace{0pt} / m_n \\
l\super j \textrightarrow\hspace{0pt} l \\
x\super j \textrightarrow\hspace{0pt} \textipa{S} \\
\textipa{P} \textrightarrow\hspace{0pt} \O\hspace{0pt} / V_V$^3$ \\
\textipa{P} \textrightarrow\hspace{0pt} \O\hspace{0pt} / V$^1$_\# \\
i$^1$ \textrightarrow\hspace{0pt} \{i$^1$,\textipa{@}$^1$j\} \\
\{u$^3$\raisebox{-0.7ex}{\textasciitilde}\textipa{@}$^3$\} \textrightarrow\hspace{0pt} u$^3$ \\
\{a$^3$\raisebox{-0.7ex}{\textasciitilde}\textipa{@}$^3$\} \textrightarrow\hspace{0pt} \textipa{@}$^3$ \\
\{a$^3$\raisebox{-0.7ex}{\textasciitilde}i$^3$\} \textrightarrow\hspace{0pt} i$^3$ \\
\{a$^1$\raisebox{-0.7ex}{\textasciitilde}i$^1$\} \textrightarrow\hspace{0pt} \{i$^1$,\textipa{@}$^1$\} \\
\{i$^3$\raisebox{-0.7ex}{\textasciitilde}e$^3$\} \textrightarrow\hspace{0pt} \{\textipa{@}$^3$,i$^3$\} \\
\{i$^3$\raisebox{-0.7ex}{\textasciitilde}\textipa{@}$^1$\} \textrightarrow\hspace{0pt} i$^3$ \\
\{i$^1$\raisebox{-0.7ex}{\textasciitilde}\textipa{@}$^1$\} \textrightarrow\hspace{0pt} \{i$^1$,\O\}

\subsubsection{Proto-Central Salish to Sqamish}\textit{Pogostick Man}, from Galloway, Brent (1982), \textquotedblleft Proto-Central Salish Phonology and Sound Correspondences". From the \textit{17th International Conference on Salish and Neighboring Languages}

n \textrightarrow\hspace{0pt} \O\hspace{0pt} / m\textipa{P}_ \\
l\super j \textrightarrow\hspace{0pt} j \\
x\super j \textrightarrow\hspace{0pt} \textipa{S} \\
\textipa{P}R \textrightarrow\hspace{0pt} R\textipa{P} / V$^3$_V \\
u$^1$ \textrightarrow\hspace{0pt} \{u$^1$,\textipa{@}$^1$\} \\
\{u$^3$\raisebox{-0.7ex}{\textasciitilde}\textipa{@}$^3$\} \textrightarrow\hspace{0pt} \textipa{@}$^3$ \\
\{a$^3$\raisebox{-0.7ex}{\textasciitilde}\textipa{@}$^3$\} \textrightarrow\hspace{0pt} \{a$^3$,\textipa{@}$^3$\} \\
\{a$^3$\raisebox{-0.7ex}{\textasciitilde}i$^3$\} \textrightarrow\hspace{0pt} i$^3$ \\
\{a$^1$\raisebox{-0.7ex}{\textasciitilde}i$^1$\} \textrightarrow\hspace{0pt} i$^1$ \\
\{i$^3$\raisebox{-0.7ex}{\textasciitilde}e$^3$\} \textrightarrow\hspace{0pt} \{\textipa{@}$^3$,i$^3$\} \\
\{i$^3$\raisebox{-0.7ex}{\textasciitilde}\textipa{@}$^1$\} \textrightarrow\hspace{0pt} i$^3$ \\
\{i$^1$\raisebox{-0.7ex}{\textasciitilde}\textipa{@}$^1$\} \textrightarrow\hspace{0pt} \{i$^1$,\O\}

\subsubsection{Proto-Central Salish to Twana}\textit{Pogostick Man}, from Galloway, Brent (1982), \textquotedblleft Proto-Central Salish Phonology and Sound Correspondences". From the \textit{17th International Conference on Salish and Neighboring Languages}

m n \textrightarrow\hspace{0pt} b d \\
l\super j \textrightarrow\hspace{0pt} l \\
s \textrightarrow\hspace{0pt} \{\textipa{S,s}\} / _x\super w \\
x\super j \textrightarrow\hspace{0pt} \textipa{S} \\
\textipa{P}R \textrightarrow\hspace{0pt} \textipa{P}\{R,b\} / V$^3$_V \\
R\textipa{P} \textrightarrow\hspace{0pt} \textipa{P}R / V$^3$_\# \\
\textipa{P} \textrightarrow\hspace{0pt} \O\hspace{0pt} / V$^1$R_\# \\
\textipa{P} \textrightarrow\hspace{0pt} \O\hspace{0pt} / V$^3$R_C \\
u$^3$ u$^1$ \textrightarrow\hspace{0pt} o$^3$ \textipa{@}$^1$ \\
\{a$^3$\raisebox{-0.7ex}{\textasciitilde}\textipa{@}$^3$\} \textrightarrow\hspace{0pt} \{\textipa{@}$^3$,a$^3$\} \\
\{a$^3$\raisebox{-0.7ex}{\textasciitilde}i$^3$\} \textrightarrow\hspace{0pt} a$^3$ \\
\{a$^1$\raisebox{-0.7ex}{\textasciitilde}i$^1$\} \textrightarrow\hspace{0pt} \{i$^1$,\textipa{@}$^1$\} \\
\{i$^3$\raisebox{-0.7ex}{\textasciitilde}e$^3$\} \textrightarrow\hspace{0pt} \{i$^3$,\textipa{@}$^3$\} \\
\{i$^3$\raisebox{-0.7ex}{\textasciitilde}\textipa{@}$^1$\} \textrightarrow\hspace{0pt} \{i$^3$,i$^1$\} \\
\{i$^1$\raisebox{-0.7ex}{\textasciitilde}\textipa{@}$^1$\} \textrightarrow\hspace{0pt} i$^1$

\subsection{Interior Salish}

\subsubsection{Proto-Interior Salish to Columbian and Okanagan Nasal-to-Vowel Shifts}\textit{Pogostick Man}, from Kinkade, Dale M. ``Shifts of Nasals to Vowels in Interior Salish"

n\textipa{\super P} \textrightarrow\hspace{0pt} a\textipa{P} / _\# (all other Interior Salishan languages have /e\textipa{P}/ in this position) \\

\subsubsection{Proto-Interior Salish to Thompson Nasal-to-Vowel Shifts}\textit{Pogostick Man}, from Kinkade, Dale M. ``Shifts of Nasals to Vowels in Interior Salish"

N[- glottalized] \textrightarrow\hspace{0pt} e / _O[+ same POA] \textquotedblleft \textquoteleft in primary forms'" \\
\textipa{\s{n}} \textrightarrow\hspace{0pt} e / n_ (this is admittedly a bit conjectural; the paper is not being very clear here)

\subsection{Shuswap to Eastern Shuswap Nasal-to-Vowel Shifts}\textit{Pogostick Man}, from Kinkade, Dale M. ``Shifts of Nasals to Vowels in Interior Salish"

em em\textipa{\super P} \textrightarrow\hspace{0pt} u u\textipa{P} / w_ in U[- stressed] \\
em em\textipa{\super P} \textrightarrow\hspace{0pt} a a\textipa{P} / in U[- stressed] ! \{p('),m(\textipa{\super P})\}_ \\
en en\textipa{\super P} \textrightarrow\hspace{0pt} i i\textipa{P} / \{ts('),s,j(\textipa{\super P})\}_ in U[- stressed] \\
en en\textipa{\super P} \textrightarrow\hspace{0pt} a \textipa{P} / in U[- stressed] ! \{t('),l\super j,\{n,l\}(\textipa{\super P})\}_

\subsection{Shuswap to Spokane-Kalispel Shuswap Nasal-to-Vowel Shifts}\textit{Pogostick Man}, from Kinkade, Dale M. ``Shifts of Nasals to Vowels in Interior Salish"

n n\textipa{\super P} \textrightarrow\hspace{0pt} i i\textipa{P} / C_s \\
n n\textipa{\super P} \textrightarrow\hspace{0pt} i i\textipa{P} / _\{i,\textipa{S}\} (sporadic)

\clearpage

\section{Sino-Tibetan}

\subsection{Proto-Sino-Tibetan to Middle Chinese}{\it Ran \& thedukeofnuke}, from Handel, Z. (1998), {\it The Medial Systems of Old Chinese and Proto-Sino-Tibetan}

\tab {\it NB: ``P T \d{T} K represent labial, dental, retroflex, and velar obstruents respectively. \textbf{*r} is reconstructed as being an approximant \textbf{\ipa{\*r}}. . . .PST and OC lacked any initial/medial clusters of the form Tr- and Tl-. . . .The initials and medials for Old Chinese are the same as those for PST. Note that initial consonants separated by a hyphen (e.g., C-r-) are derived from prefixes and are not true consonant clusters.}

(C-)\ipa{r}- \change\ \ipa{l}-\\
\ipa{s}(-)\ipa{r} \change\ \ipa{\:s}-\\
\ipa{r} \change\ \O\ / C_-\\
\ipa{r}-T- \change\ \d{T}\\
(C-)\ipa{l}- \change\ \{\ipa{d,ji}\}-\\
\ipa{s}-\ipa{l}- \change\ \{\ipa{t\super h,z}\}-\\
\ipa{l} \change\ \O\ / \ipa{s}_-\\
\{\ipa{m,N}\}\ipa{l}- \change\ \ipa{d\textctz}-\\
C\ipa{l}- \change\ T(\ipa{\textctz})-\\
K\ipa{w} \change\ K\ipa{\super w}- / _\ipa{a}-\\
K\ipa{w@}- \change\ K\ipa{\super w1}-\\
\ipa{w} \change\ \ipa{r} \change\ \O\ / P_-

\subsubsection{Late Middle Chinese to Old Mandarin}{\it Ran}, from Hsueh, F.S. (1975), {\it Phonology of Old Mandarin}

\tab {\it NB: Ran says, ``The author uses V1, V2, V3, V4, Vn, Vch, and Vta to represent Late Middle Chinese vowels. I am going to \textbf{very tentatively assign the values of o, a, ia, e,?, a(ch) and a(ta) to these vowels} [emphasis added]. These should be taken as orthographical convenience rather than actual speculation." It should be noted that the vowel represented by \textless ?\textgreater\ could be palatalized. Ran adds, ``I am also going to number tones according to their traditional order, i.e. F1, F2, F3, F4, F5. Since tones change rapidly, it is impossible to accurately reconstruct their values; we can only know how many there were, and agree on an order to renumerate them. Middle Chinese starts out with no F2." For the purposes of this list of sound changes, tones are superscript numbers following vowel markers.}

\O\ \change\ \ipa{w} / P_V\\
\ipa{xH} \change\ \O\ / _\ipa{j(w)}\{?,\ipa{ia,a(ta)}\}\\
C\super j \change\ C \{\d{A},\d{F}\}_\\
C\super j \change\ \d{C}\\
\ipa{\:n} \change\ \ipa{\:r}\\
V$^1$ \change\ V$^2$ / in syllables with /\ipa{H}/, a nasal, or a liquid\\
V$^3$ \change\ V$^4$ / in syllables with /\ipa{H}/\\
? \change\ \O\\
V$^5$ \change\ V$^2$ / in syllables with /\ipa{H}/\\
V$^5$ \change\ V$^4$ / in syllables with a nasal or liquid\\
V$^5$ \change\ V$^3$ / else\\
\ipa{H} \change\ \ipa{h} / S_V$^2$\\
\ipa{H} \change\ \O\ / else\\
C\super j \change\ C / \{\ipa{f,v}\}_\\
\ipa{w} \change\ \ipa{o} / C\super j_?\ipa{w}\\
\ipa{i} \change\ \O\ / C(\ipa{w})_\ipa{a}\\
\ipa{ia} \change\ \ipa{e} / else\\
C \change\ C\super j / \{K,C[+pharyngeal]\}_\{\ipa{a,a(ch)}\}\\
\ipa{o} \change\ \ipa{a} / C_\ipa{w} ! C = \{K,C[+pharyngeal]\}\\
\ipa{N} \change\ \ipa{n} / \ipa{n}_C\super j\ipa{a(ta)}\\
\ipa{N} \change\ \O\ / \#_ ! \#_\ipa{o(w)}\\
\ipa{k} \change\ \ipa{j} / \{\ipa{e,a,o}\}_\\
\ipa{k} \change\ \ipa{w} / V_\\
\{\ipa{o,a}\} \{\ipa{a(ta),a(ch)}\} \change\ \ipa{e a} / _\ipa{N}\\
\{\ipa{a(ta),a(ch)}\} \change\ \ipa{o} / else\\
? \change\ \ipa{o} / _\ipa{N}\\
\{A\super j,F\super j\}[+alveolar] \change \{A,F\} / _?\super j\\
\ipa{\:t} \change\ \ipa{\:t\:s} / _\ipa{r}\\
C \change\ C\super j / ?\super j\{\ipa{p,t}\}_\\
\{\ipa{p,t}\} \change\ \O\ / V_\\
\ipa{o} \change\ \ipa{a} / C\super j_\ipa{w}\\
\O\ \change\ \ipa{w} / C_\ipa{o}\# ! C = \{K,C[+pharyngeal]\}\\
\ipa{w} \change\ \O\ / \d{C}C\super j_?\ipa{w} (``optional")\\
\ipa{w} \change\ \O\ / \ipa{w}?_\\
? \change\ \ipa{a} / C\ipa{w}_C\super j\\
\{\ipa{o,e}\} \change\ ? / _C\super j\\
\ipa{w\super j} \change\ \ipa{w} / _?C\super j\\
\O\ \change\ \ipa{w} / \d{C}_\ipa{aN}\\
\ipa{e} \change\ \ipa{o} / \ipa{w}_\ipa{N}\\
\ipa{w\super j} \change\ \ipa{w} / _\ipa{aN}\\
\ipa{w\super j} \change\ \ipa{w} / \d{C}_\ipa{oN}

\paragraph{Old Mandarin to Modern Pekingese}{\it Ran}, from Hsueh, F.S. (1975), {\it Phonology of Old Mandarin}

\ipa{N} \change\ \O\ / \#_\\
\{\ipa{e,o}\} \change\ \ipa{a} / _\ipa{w}\\
\ipa{m} \change\ \ipa{n} / V_\\
\ipa{o} \change\ \ipa{e} / _\ipa{N}\\
\ipa{i} \change\ \ipa{e} / C_\ipa{P}\\
\ipa{P} \change\ \O\\
\ipa{j} \change\ \O\ / \ipa{j\^{i}}_\\
\#\ipa{r}\textellipsis\ipa{\^{i}}\# \change\ \#\ipa{\^{i}}\textellipsis\ipa{r}\\
\ipa{v} \change\ \O\\
\ipa{j} \change\ \O\ / \d{C}_\\
\ipa{o} \change\ \ipa{e} / _\#\\
\ipa{\^{i}} \change\ \ipa{e} / _C\# ! C = \ipa{r}\\
\ipa{j} \change\ \O\ / C_\ipa{weN}\\
\ipa{k(\super h) h} \change\ \ipa{\t C(\super h) C} / _\ipa{j}

\subsection{Sin Sukchu to Gu\={a}nhu\`{a}}{\it Pogostick Man}, from Coblin, W. South (2000), ``A Diachronic Study of M\'{i}ng {\it Gu\={a}nhu\`{a}} Phonology". {\it Monumenta Serica} 48:267 -- 335

\textbf{Initials:}\\
\ipa{b d dz g} \change\ \{\ipa{p,p'}\} \{\ipa{t,t'}\} \{\ipa{ts,ts'}\} \{\ipa{k,k'}\}\\
\ipa{N} \change\ \O\ / _\{\ipa{i,j,w,y}\}\\
\ipa{N} \change\ \O\ / _\ipa{u} / _V\\
\ipa{w} \change\ \ipa{u} / _V\\
\ipa{v V} \change\ \ipa{f v}\\
\ipa{z} \change\ \{\ipa{s,ts'}\} (the former ``without exception'' ``in oblique tone words")\\
\ipa{\:d\:z} \change \ipa{\:t\:s} / ``[i]n oblique tone syllables"\\
\ipa{\:d\:z} \change\ \ipa{\:t\:s'} / ``in {\it p\'{i}ng}-tone syllables"\\
\ipa{\:z} \change\ \ipa{\:s}\\
\ipa{r} \change\ \ipa{\:z} (Apparently there was some situation where this went to \O, and then something happened with the output syllable being [\ipa{@\textrhoticity}])\\
\ipa{P} \change\ \ipa{N} / _V[-high]\\
\ipa{P} \change\ \O\ / _\{\ipa{j,i,y}\}\\
\ipa{P} \change\ \O\ / _\ipa{u} (not always? Perhaps some variation with [\ipa{G}] here?)\\
\ipa{G} \change\ \ipa{x}\\
\ipa{j} \change\ \ipa{i}\\
There seems to ahve been some stuff going on with palatalized [\ipa{N}] \change\ \{\ipa{\textltailn,n}\} but it seems highly dialectal and I'm not entirely sure just what exactly was going on here\\
\ipa{w} \change\ \{\ipa{v,u}\}\\
\ipa{uj} \change\ \ipa{(G)u}

\textbf{Finals:}\\
\ipa{m} \change\ \ipa{n}\\
long-tail vowel thing \change\ \ipa{i} (or [truncated vowel-thing] ``where GH sibilant initial variants occur")\\
\{long-tail vowel thing\ipa{P,@P}\} \change\ \ipa{EP}\\
\ipa{i} \change\ [long-tail vowel thing] / \d{C}_ (``sometimes", in ``variant readings"); when following /\ipa{\:s \:z}/, sometimes yields [\ipa{\:si}], other times [s truncated vowel-thing]\\
\ipa{iP iw} \change\ \ipa{eP ew} / \d{C}_ (the latter with variant \ipa{iEu}?)\\
\ipa{iP} \change\ \ipa{ieP} (\change\ \ipa{i}[truncated vowel-thing]\ipa{P}?)\\
\ipa{uP} \change\ \ipa{oP} (occasionally \change\ \ipa{uEP}?)\\
\ipa{uj} \change\ \ipa{u(E)i} / \ipa{m}_\\
\ipa{uj} \change\ \ipa{uEi} / \{P,C[+guttural],\O\}_\\
\ipa{uj} \change\ \ipa{ui} / \{C[+dental],C[+sibilant]\}_\\
\ipa{ujP} \change\ \ipa{uEP} (dialectally \change\ \ipa{uOP}?)\\
\ipa{un} \change\ \ipa{uEn} / ! \{C[+dental],C[+sibilant]\}_\\
\ipa{uEn} \change\ \ipa{En} / \ipa{V}_\\
\ipa{uN} \change\ \ipa{oN} (in one source?)\\
\ipa{juN} \change\ \ipa{iuN} / _\{\O,\ipa{x,V,P}\}\\
\ipa{juN} \change\ \ipa{iuN} / _\ipa{g}[+{\it p\'{i}ng} tone]\\
\ipa{juN} \change\ \ipa{uN} (\change\ \ipa{oN} dialectally?)\\
\ipa{jujN ujN} \change\ \ipa{iuN uN}\\
\ipa{y} \change\ \ipa{4} (\change\ \ipa{y}\raisebox{-0.6ex}{\textasciitilde}\ipa{u} dialectally?) / \d{C}_\\
\ipa{yP} \change\ \ipa{yEP} / \{\ipa{l},C[+dental +sibilant]\}_ in ``QYS {\it -ju\ipa{@}t}-type" finals, dialectally?\\
\ipa{yP} \change\ \ipa{oP} (eventually \change\ \{\ipa{UP,(i)uP}?) / \{\ipa{l},C[+dental +sibilant]\}_, in ``QYS {\it-k}-types"\\
\ipa{yP} \change\ \ipa{yP}(\raisebox{-0.6ex}{\textasciitilde}\ipa{yEP}?) / \{\O,C[+guttural]\}, in {\it -ju\ipa{@}t}-types\\
\ipa{yP} \change\ \ipa{ioP} (\change\ \ipa{iUP} dialectally?) / \{\O,C[+guttural]\}_, in {\it-k}-types\\
\ipa{yP} \change\ \{\ipa{4P,yP,uP}\} / \d{C}, in {\it -ju\ipa{@}t}-types\\
\ipa{yP} \change\ \ipa{oP} (\change\ \ipa{UP}?) / \d{C}_, in {\it -k}-types\\
\ipa{yjP} \change\ \ipa{yP}\\
\ipa{yn} \change\ \ipa{un} / \d{C}_ (may have stayed \ipa{yn} or \change\ \ipa{4n} in at least one area?)\\
\ipa{je jeP yeP} \change\ \ipa{E EP uEP} / \d{C}_\\
\ipa{je jeP} \change\ \ipa{iE iEP}\\
\ipa{yeP} \change\ \ipa{yEP} (\change\ \{\ipa{uOP,yOP}\} in southern speech?)\\
\ipa{ye jej} \change\ \ipa{yE i}\\
\ipa{jew} \change\ \ipa{au} / \d{C}(C?)_\\
\ipa{jew} \change\ \ipa{iau} / else\\
\{\ipa{jem,jen}\} \change\ \ipa{iEn} / sometimes after \d{C}_ (but ! \ipa{\:z}_) (only in one variety?)\\
\ipa{yen} \change\ \ipa{uEn} / \d{C}_\\
\ipa{yen} \change\ \ipa{yEn}\\
\ipa{O} \change\ \ipa{a} / in two cases cited; extremely rare change\\
\ipa{wO} \change\ \ipa{O} (occasionally \change\ \ipa{uO} after a guttural?)\\
\ipa{wOP} \change\ \ipa{uOP} / C[+guttural]_\\
\ipa{wOP} \change\ \ipa{OP} / else\\
\ipa{On} \change\ \ipa{an}; ``[t]his final occurs exclusively after SR gutturals"\\
\ipa{wOn ja wa} \change\ \ipa{uOn ia ua}\\
\ipa{aP} \change\ \ipa{OP} / C[+guttural]_\\
\ipa{jaP} \change\ \ipa{iaP}\\
\ipa{waP} \change\ \ipa{aP} / C[+labiodental]_\\
\ipa{waP} \change\ \ipa{uaP}\\
\ipa{aj jaj waj aw jaw} \change\ \ipa{ai iai uai au iau}\\
\ipa{awP} \change\ \ipa{OP} (``Trigault gives a variant in {\it -\ipa{EP}}, which becomes general in the later GH varieties")\\
\{\ipa{jawP,wawP}\} \change\ \ipa{OP} / \d{C}_\\
\ipa{jawP wawP} \change\ \ipa{iOP uOP}\\
\ipa{am} \change\ \ipa{an}\\
\{\ipa{jam,jan}\} \change\ \ipa{iEn}\\
\ipa{wan} \change\ \ipa{an} / C[+labiodental]_\\
\ipa{wan} \change\ \ipa{uan} / else\\
\ipa{aN jaN} \change\ \ipa{uaN aN} / \d{C}_\\
\ipa{waN} \change\ \ipa{uaN}\\
\ipa{@jP @w} \{\ipa{@m,@n}\} \change\ \ipa{eP Eu En}\\
\ipa{@jN} \change\ \ipa{En} (varies with \ipa{En}?)

\textbf{Tones:}\\
q\={\i}ng p\'{i}ng \change\ y\={\i}n p\'{i}ng\\
zhu\'{o} p\'{i}ng \change\ y\'{a}ng p\'{i}ng\\
q\={\i}ng sh\u{a}ng \change\ sh\u{a}ng\\
zhu\'{o} sh\u{a}ng \change\ q\`{u}\\
(There seems to be some conflict between {\it sh\u{a}ng} tones and {\it q\`{u}} tones, the latter noted as being the spoken forms)

\subsection{Tibeto-Burman}

\subsubsection{Qiangic}

\paragraph{Proto-Naish to Laze}{\it Pogostick Man}, from Jacques, Guillaume, and Alexis Michaud (2011), ``Approaching the historical phonology of three highly eroded Sino-Tibetan languages: Naxi, Na and Laze". {\it Diachronica} 28:4 (2011), 468 -- 498

\ipa{a u i i}N \change\ \ipa{e y W i} / T_\%\\
\ipa{a} \change\ \ipa{i} / \{\ipa{N,w}\}_\%\\
\ipa{a} \change\ \ipa{w7} / \{K,\ipa{N}\}\ipa{w}_\%\\
\{\ipa{a,i}\} \change\ \ipa{W} / R_\%\\
\ipa{a} \change\ \ipa{ie} / ! K_\%\\
\ipa{a a}S \change\ \ipa{A} \{\ipa{A,u}\}\\
\ipa{i}N \change\ \ipa{\ae} / \{P,C\}\ipa{r}_\%\\
\ipa{i} \change\ \ipa{\s{v}} / \ipa{m}_\%\\
\ipa{u o} \change\ \ipa{\s{v} u}\\
B \change\ \ipa{o} / \{\ipa{q\super h},(N)\ipa{q}\}_\\
V\% \change\ low tone\\
\{N\ipa{p,mb}\} \change\ \ipa{b} / _V (the paper implies similar developments occurred at other POAs)\\
\{\ipa{r,s}\}\ipa{p(\super h)} \{\ipa{r,s}\}\{N\ipa{p,(m)b}\} \change\ \ipa{f v} / _V (the paper implies similar developments occurred at other POAs)\\
\{\ipa{r,s}\}\ipa{k} \{\ipa{r,s}\}N\ipa{k} \change\ \ipa{f w} / _V\\
S\{\ipa{b,g}\} S\ipa{k} \change\ \ipa{v h} / _V\\
\{\ipa{r,s}\}\ipa{l} \{\ipa{r,s}\}\ipa{n} \change\ \ipa{\r*{l}} \textsubring{N} / \%_V\\
\ipa{\r*{n}} \change\ \ipa{\r*{l}} / \%_V\\
\textsubring{N}V \change\ \ipa{h}\~{V}\\
\ipa{\r*{l}} \change\ \ipa{\textbeltl} / \%_V

\paragraph{Proto-Naish to Mosuo (Na)}{\it Pogostick Man}, from Jacques, Guillaume, and Alexis Michaud (2011), ``Approaching the historical phonology of three highly eroded Sino-Tibetan languages: Naxi, Na and Laze". {\it Diachronica} 28:4 (2011), 468 -- 498

\ipa{a} \change\ \ipa{e} / \{R,T\}_\%\\
\ipa{a} \change\ \ipa{w7} / \{K,\ipa{N}\}\ipa{w}_\%\\
\ipa{a} \change\ \ipa{i} / ! K_\%\\
\ipa{i}N \change\ \ipa{\ae} / \{P,C\}\ipa{r}_\\
\{\ipa{i}N,\ipa{u}\} \ipa{i} \change\ \ipa{i W} / T_\\
\ipa{i} \change\ \ipa{W} / \{R,K\ipa{r}\}_\\
\ipa{i} \change\ \ipa{\s{v}} / \ipa{m}_\\
\ipa{u o} \change\ \ipa{\s{v} u}\\
B \change\ \ipa{O} / \{\ipa{q\super h},(N)\ipa{q}\}_\\
V\% \change\ high rising\\
\{N\ipa{p,mb}\} \change\ \ipa{b} / _V\\
\{\ipa{r,s}\}\ipa{p\super h} \{\ipa{r,s}\}\{(N)\ipa{p,(m)b}\} \change\ \ipa{p\super h p b} / _V (the paper implies similar developments occurred with stops at other POAs)\\
\{\ipa{r,s}\}\ipa{k} \{\ipa{r,s}\}N\ipa{k} \change\ \ipa{k K} / _V\\
S\{\ipa{b,g}\} S\ipa{k} \change\ \O\ \ipa{h}\\
\ipa{\r*{n}} \change\ \ipa{\r*{l}} / \%_V\\
\textsubring{N}V \change\ \ipa{h}\~{V}\\
\ipa{\r*{l}} \change\ \ipa{\textbeltl} / \%_V

\paragraph{Proto-Naish to Naxi}{\it Pogostick Man}, from Jacques, Guillaume, and Alexis Michaud (2011), ``Approaching the historical phonology of three highly eroded Sino-Tibetan languages: Naxi, Na and Laze". {\it Diachronica} 28:4 (2011), 468 -- 498

\ipa{a i}N \{\ipa{i,u}\} \change\ \ipa{e @\textrhoticity W} / T_\%\\
\ipa{a} \change\ \ipa{i} / \ipa{N}_\%\\
\ipa{a} \change\ \ipa{W} / \{R,\ipa{w}\}_\%\\
\ipa{a} \change\ \ipa{wa} / \{K,\ipa{N}\}\ipa{w}_\%\\
\ipa{a} \change\ \ipa{e} / ! K_\%\\
\ipa{a a}S \change\ \ipa{A} \{\ipa{A,o}\} / _\%\\
\ipa{i}N \change\ \ipa{@\textrhoticity} / \{P,C\}\ipa{r}_\%\\
\ipa{i} \change\ \ipa{W} / \{R,\ipa{kr}\}_\%\\
\ipa{u} \change\ \ipa{@\textrhoticity} / P\ipa{r}_\%\\
\ipa{u o} \change\ \ipa{\s{v} u}\\
B \change\ \ipa{\s{v}} / \{\ipa{q\super h,(N)q}\}_\\
V\% \change\ mid tone / C_\ipa{ru}\\
V\% \change\ high tone / else\\
N \change\ \O\ / _\ipa{p}V\\
\{\ipa{r,s}\}\ipa{p\super h} \{\ipa{r,s}\}(N)\ipa{p} \{\ipa{r,s}\}\ipa{b} \{\ipa{r,s}\}\ipa{mb} \change\ \ipa{p\super h p b mb} / _V (the paper implies similar developments occurred with stops at other POAs)\\
\{\ipa{r,s}\}(N)\ipa{k} \change\ \ipa{k} / _V\\
S\ipa{b} S\ipa{k} S\ipa{g} \change\ \ipa{b} ? \ipa{g} / _V\\
\ipa{\r*{n}} \change\ \ipa{\r*{l}} \change\ \ipa{h} / \%_V\\
\{\ipa{r,s}\} \change\ \O\ / \%_V\\
\{\ipa{r,s}\}N \change\ \textsubring{N} \change\ \ipa{h}\~{V} \change\ \ipa{h}V

\subsubsection{rGyalrongic}

\paragraph{Proto-rGyalrongic to bTshan La}{\it Pogostick Man}, from Nagano, Yashuiki (1979), ``A Historical Study of rGyarong Initials and Prefixes''. {\it Linguistics of the Tibeto-Burman Area} 4(2):44 -- 68; and Nagano, Yashuiki (1979), ``A Historical Study of rGyarong Rhymes''. {\it Linguistics of the Tibeto-Burman Area} 5(1):37 -- 47

{\bf Initials:}\\
\ipa{p b \c{c}p \c{c}b s}P \ipa{r}P N\ipa{p} N\ipa{b pr \{br,pj\} bj} \change\ \ipa{p\super h p Jb \c{c}p sp rp\super h mp\super h mb br p\super hj pj}\\
P \change\ \ipa{p} / \ipa{l}_\\
\ipa{t d} N\ipa{t} N\ipa{d st sd \c{c}}T \ipa{tr (\c{c})dr} KT \change\ \ipa{t\super h t} N\ipa{t\super h md zd st \c{c}tj trh (\c{c})tr kt}\\
\ipa{k sk kr g} P\ipa{g sg} N\ipa{g (s)gr \c{c}}K \ipa{r}K K\ipa{\c{c}} \change\ \ipa{k\super h zgw dr} N\ipa{g pk sk mk (s)kr \c{c}k rgj g\c{c}kr}\\
\ipa{(P)kj (s)gj} N\ipa{kj} \change\ \ipa{(k)t\c{c} (s)kj} N\ipa{dZ}\\
\ipa{\c{c}r} N\ipa{\c{c}} \change\ \ipa{dr mk\super hj}\\
K \change\ \ipa{P} / _\ipa{s}\\
K \change\ \ipa{k} / _\ipa{\c{c}}\\
C[+ sibilant] \change\ \ipa{s} / _\ipa{w}\\
P\ipa{\c{c}} \change\ \ipa{mS}\\
TS \change\ \ipa{dZ} / \ipa{l}_\\
NTS PTS KTS \ipa{s}TS \ipa{\c{c}}TS \change\ \{N\ipa{dz,mts\} kts(\super h) pts\super h sts \{sts,rj\}}\\
\ipa{(r)ts (r)dz} \change\ \ipa{(r)ts\super h (r)ts}\\
\O\ \change\ \ipa{j} / \ipa{\c{c}m}_\\
N \change\ \ipa{m} / _\{\ipa{N.nj}\}\\
\ipa{rnj} \change\ \ipa{r\textltailn} ?\\
\ipa{w} \change\ \ipa{P} / something to do with either back vowels or prefixes

{\bf Rhymes:}\\
\ipa{n} \change\ \O\ / \ipa{a}_\\
\ipa{N} \change\ \O\ / \ipa{i}_\\
\ipa{u} \change\ \ipa{a} / _\ipa{k}\\
\ipa{u} \change\ \ipa{i} / _\{\ipa{r,s}\}\\
\ipa{o} \change\ \ipa{e} / _\ipa{s}\\
\ipa{a} \change\ \ipa{e} / _\ipa{m}\\
\ipa{aj} \change\ \ipa{i}\\
\ipa{ew iw} \change\ \ipa{im ju}\\
\ipa{i} \change\ \ipa{e} / _\ipa{s}

\subparagraph{Proto-rGyalrongic to Chos Kia}{\it Pogostick Man}, from Nagano, Yashuiki (1979), ``A Historical Study of rGyarong Initials and Prefixes''. {\it Linguistics of the Tibeto-Burman Area} 4(2):44 -- 68; and Nagano, Yashuiki (1979), ``A Historical Study of rGyarong Rhymes''. {\it Linguistics of the Tibeto-Burman Area} 5(1):37 -- 47

{\bf Initials:}\\
\ipa{b} \change\ \ipa{m} / _\ipa{j}\\
\ipa{p \c{c}p (\c{c})b r}P N\ipa{p} N\ipa{b pr \{br,pj\}} \change\ \ipa{p. \c{c}.w (\c{c})p rp. kj m.p n.br p.j}\\
P \change\ \ipa{p} / \ipa{l}_\\
\ipa{t st tr d sd (\c{c})dr} N\ipa{t \c{c}}T KT \change\ \ipa{t. s.d tr. t s.t (\c{c}.)tr m.t. n.dr g.t}\\
\ipa{r} \change\ \O\ / _\ipa{t}\\
\ipa{\c{c}}K \change\ \ipa{\c{c}.k}\\
N \change\ \O\ / _\ipa{g}\\
\ipa{sk r}K \change\ \ipa{j sg}\\
\ipa{kj} N\ipa{kj Pkj gj sgj} \change\ \ipa{kj. nj g.ts\super h kj skj.}\\
\ipa{kr (s)gr} \change\ \ipa{n.br (s)kr}\\
C[+ sibilant] \change\ \ipa{s} / _\ipa{w}\\
K \change\ \ipa{g.} / _\ipa{\c{c}}\\
N\ipa{\c{c}} \change\ \ipa{n.pj.}\\
\ipa{\c{c}l} \change\ \ipa{sj}\\
K\ipa{ts} \change\ \ipa{g.ts\super h.}\\
\ipa{l}TS \change\ \ipa{l.dz}\\
\ipa{ts (r)dz dZ} PTS KTS \ipa{s}TS \ipa{\c{c}}TS \change\ \ipa{\{ts\super h,s\} (r)ts ts\super h g.ts s.\{ts,pj\} \{s.ts,br\}}\\
\ipa{rm rn} \change\ \ipa{mj rw}\\
\ipa{\c{c}m \c{c}nj} \change\ \ipa{\c{c}.n}\\
N \change\ \ipa{m.} / _\ipa{N}\\
N\ipa{n} \change\ \ipa{m} / _\ipa{j}\\
\ipa{w} \change\ \O\ / \ipa{N}_\\
\ipa{rnj} \change\ \ipa{r.mj}\\
\ipa{j} \change\ \O\ / \ipa{n}_\\

{\bf Rhymes:}\\
\ipa{ut uk} \change\ \ipa{ud og}\\
\ipa{s} \change\ \O\ / \ipa{u}_\\
\ipa{uj} \change\ \ipa{ui}\\
\ipa{ok} \change\ \ipa{ig}\\
\ipa{o} \change\ \ipa{e} / _\ipa{l}\\
\ipa{oj} \change\ \ipa{oi}\\
\ipa{a ap at ar am aj} \change\ \ipa{e eb e(d) er om e.i}\\
\ipa{\{k,j\}} \change\ \O\ / \ipa{i}_\\
\ipa{e} \change\ \ipa{i} / _\#\\
\ipa{et ej} \change\ \ipa{o e.i}\\
\ipa{iw} \change\ \ipa{jo}\\
\ipa{i} \change\ \ipa{e} / _\ipa{m}\\
\ipa{ip it} \change\ \ipa{ib o}

\subparagraph{Proto-rGyalrongic to Hanniu}{\it Pogostick Man}, from Nagano, Yashuiki (1979), ``A Historical Study of rGyarong Initials and Prefixes''. {\it Linguistics of the Tibeto-Burman Area} 4(2):44 -- 68; and Nagano, Yashuiki (1979), ``A Historical Study of rGyarong Rhymes''. {\it Linguistics of the Tibeto-Burman Area} 5(1):37 -- 47

\tab {\it NB: These changes are likely very incomplete. The source did not have much to say about this language.}

{\bf Initials:}\\
\ipa{p k ts} \change\ \ipa{p. k. tS.}\\
NTS \change\ \ipa{mnj}\\
\ipa{m} \change\ \ipa{mt.}\\
C[+ sibilant]\ipa{w} \change\ \ipa{s}\\
\ipa{\c{c}l} \change\ \ipa{rts}

{\bf Finals:}\\
\ipa{k} \change\ \O\ / \ipa{\{u,a\}}_\\
\ipa{or} \change\ \ipa{ro}\\
\ipa{an \{at.is\}} \change\ \ipa{o ie}\\
\ipa{t} \change\ \O\ / \ipa{i}_

\paragraph{Proto-rGyalrongic to Japhug}{\it Pogostick Man}, from Jacques, Guillaume (2004), ``Phonologie et Morphologie du Japhug (rGyalrong)''. Universit� Paris-Diderot - Paris VII \textless tel-00138568\textgreater. \textless\url{https://tel.archives-ouvertes.fr/file/index/docid/138568/filename/these-japhug.pdf}\textgreater

\tab{\it NB: This source is in French and looks to at least sometimes use a transcription that isn't IPA.}

\ipa{u o} \change\ \ipa{W u} / _\#\\
\ipa{aN} \change\ \ipa{o} / _\#\\
\ipa{Ok} \change\ \ipa{7G} / _\#\\
\ipa{O} \change\ \ipa{7} / _\{\ipa{t,r}\}\# (possibly also _\ipa{s}\# dialectally)\\
\ipa{a} \change\ \ipa{o} / _\ipa{m}\#\\
\ipa{\textctz} \ipa{j} \change\ \ipa{nd\textctz\ \textctz}\\
\ipa{b} \change\ \ipa{w} / \#\{\ipa{z,r}\}_\\
\ipa{N} \change\ \ipa{m} / \#_\ipa{k\super h}

\subparagraph{Proto-rGyalrongic to Kham To}{\it Pogostick Man}, from Nagano, Yashuiki (1979), ``A Historical Study of rGyarong Initials and Prefixes''. {\it Linguistics of the Tibeto-Burman Area} 4(2):44 -- 68; and Nagano, Yashuiki (1979), ``A Historical Study of rGyarong Rhymes''. {\it Linguistics of the Tibeto-Burman Area} 5(1):37 -- 47

{\bf Initials:}\\
\ipa{p b pr br \c{c}p} \change\ \ipa{p.j p br pj Sb}\\
\ipa{d(r)} \change\ \ipa{t}\\
\ipa{\c{c}}T \change\ \ipa{St}\\
\ipa{r} \change\ \ipa{s} / _\ipa{k}\\
\ipa{k} (N)\ipa{g} \change\ \ipa{k. (m)k}\\
N\ipa{kj gj sgj} \change\ \ipa{mj ts stSj}\\
\ipa{sgr} \change\ \ipa{skr}\\
C[+ sibilant]\ipa{w \c{c}} P\ipa{\c{c}} N\ipa{\c{c} \c{c}l} \change\ \ipa{s \:s S mS Sl}\\
\ipa{ts \{dz,}KTS\} NTS \ipa{rdz dzl dZ} \change\ \ipa{s ts \{dz,mtS.\} rts tsl tS.}\\
\ipa{\c{c}} \change\ \ipa{S} / _N\\
\ipa{N(w)} N\{\ipa{N,nj}\} \change\ \ipa{\textctn\ m\:n}\\
\ipa{j} \change\ \O\ / \ipa{\{S,r\}n}_

{\bf Rhymes:}\\
\ipa{u} \change\ \ipa{o} / _\ipa{k}\\
\ipa{uj} \change\ \ipa{os}\\
\ipa{op} \change\ \ipa{u}\\
\ipa{at} \change\ \{\ipa{at,E,ed}\}\\
\ipa{an} \change\ \ipa{ia}\\
\ipa{N} \change\ \ipa{\:n} / \ipa{\{o,a\}}_\\
\ipa{aj} \change\ \ipa{oi}\\
\ipa{e} \change\ \ipa{i} / _\{\ipa{s},\#\}\\
\ipa{et} \change\ \ipa{o}\\
\ipa{ej} \change\ \ipa{ai}
\ipa{i} \change\ \ipa{o} / _\ipa{m}\\
\ipa{\{N,j\}} \change\ \O\ / \ipa{i}_

\subparagraph{Proto-rGyalrongic to lCog Rtse}{\it Pogostick Man}, from Nagano, Yashuiki (1979), ``A Historical Study of rGyarong Initials and Prefixes''. {\it Linguistics of the Tibeto-Burman Area} 4(2):44 -- 68; and Nagano, Yashuiki (1979), ``A Historical Study of rGyarong Rhymes''. {\it Linguistics of the Tibeto-Burman Area} 5(1):37 -- 47

{\bf Initials:}\\
(N)\ipa{p} N\ipa{b} \ipa{s}P \ipa{\c{c}p \c{c}b pr br \{p,b\}j} \change\ \ipa{(m)p\super h (m)p sp Jb \c{c}p br p\super hj p\super hj}\\
(N)\ipa{t} (N)\ipa{d st sd} KT \ipa{\c{c}}T \ipa{tr dr \c{c}tr} \change\ (N)\ipa{t\super h mt zd st kt \c{c}t(j) trh tr \c{c}tr}\\
(P)\ipa{g sg \c{c}}K N\ipa{g k sk rk (s)gj (P)kj} \change\ \ipa{(p)k sk \c{c}k mk k\super h zg} N\ipa{g (s)kj (P)k\super hj}\\
N\ipa{kj (s)gr} K\ipa{\c{c} kr skr gr} \change\ \ipa{mgj (s)kr g\c{c}kr k\super hr zgr} N\ipa{gr}\\
K \change\ \ipa{P} / _\ipa{s}\\
C[+ sibilant] \change\ \ipa{s} / _\ipa{w}\\
N P K \change\ \ipa{m p k} / _\ipa{\c{c}}\\
\ipa{l} \change\ \O\ / \#\ipa{\c{c}}_\\
\ipa{ts (r)dz dzl dZ} NTS PTS K\ipa{ts} KTS \ipa{s}TS \ipa{\c{c}}TS \change\ \ipa{ts\super h (r)ts} N\ipa{dz tsl ts pt\c{c} k\ipa{t\c{c}} kts sts \{s,\c{c}\}ts}\\
LTS \change\ \ipa{ldZ}\\
N \change\ \ipa{m} / _\{\ipa{N,\textltailn}\}\\
*\ipa{w} \change\ \ipa{P} / something to do either with back vowels or prefixes

{\bf Rhymes:}\\
\ipa{uN} \change\ \ipa{ak}\\
\ipa{om} \change\ \{\ipa{o,a}\}\ipa{m}\\
\ipa{ew iw} \change\ \ipa{i jo}

\subparagraph{Proto-rGyalrongic to Pati}{\it Pogostick Man}, from Nagano, Yashuiki (1979), ``A Historical Study of rGyarong Initials and Prefixes''. {\it Linguistics of the Tibeto-Burman Area} 4(2):44 -- 68; and Nagano, Yashuiki (1979), ``A Historical Study of rGyarong Rhymes''. {\it Linguistics of the Tibeto-Burman Area} 5(1):37 -- 47

\tab {\it NB: These changes are likely very incomplete. The source did not have much to say about this language.}

{\bf Initials:}\\
\ipa{d} \change\ \ipa{l}\\
\ipa{ts dZ} NTS \change\ \ipa{\{s,tS.\} tS. m}\\
\ipa{s} \change\ \ipa{S} / _\ipa{n}

{\bf Rhymes:}\\
\ipa{n} \change\ \O\ / \ipa{a}_\\
\ipa{im it ik is ij} \change\ \ipa{em u e es e}\\
\ipa{uk} \change\ \ipa{o}

\subparagraph{Proto-rGyalrongic to Suo Mo}{\it Pogostick Man}, from Nagano, Yashuiki (1979), ``A Historical Study of rGyarong Initials and Prefixes''. {\it Linguistics of the Tibeto-Burman Area} 4(2):44 -- 68; and Nagano, Yashuiki (1979), ``A Historical Study of rGyarong Rhymes''. {\it Linguistics of the Tibeto-Burman Area} 5(1):37 -- 47

{\bf Initials:}\\
\ipa{\{p,b\}} N\ipa{p pr \{s,r\}}P \ipa{\c{c}b} N\ipa{b} \change\ \ipa{p mp mbr sp Sp p.s}\\
N\ipa{t tr d} KT \ipa{\c{c}}T \change\ \ipa{mt. t\:s t kt St}\\
\ipa{kr rk \{r}K,P\ipa{g\} g} P\ipa{g sg} \change\ \ipa{k.r rk. pk k sk}\\
\ipa{gj} \change\ \ipa{c\c{c}} ? (might be \change\ \ipa{ts\c{c}}?)\\
\ipa{Pkj} \change\ \ipa{ktS.}\\
\ipa{\c{c}} \change\ \ipa{zg} / _\ipa{r}\\
\ipa{g} \change\ \O\ / \#_\ipa{r}\\
\ipa{rs} \change\ \ipa{sr}\\
C[+ sibilant] \change\ \ipa{s} / _\#\\
\ipa{lts rts} NTS PTS STS \change\ \ipa{l\:d\:z rts. \{mdzr,mtS\} nt. sts}\\
\ipa{ts dZ} \change\ \ipa{tS. tS}\\
\ipa{N} \change\ \ipa{\textctn}\\
\ipa{nj} \change\ \ipa{\textltailn} / \#(\ipa{r})_\\
\ipa{j} \change\ \ipa{dz} / \#_

{\bf Rhymes:}\\
\ipa{u} \change\ \ipa{o} / _\ipa{p}\\
\ipa{a} \change\ \ipa{o} / _\ipa{r}\\
\ipa{es er} \change\ \ipa{or @r}\\
\ipa{e} \change\ \ipa{iE}\\
\ipa{i} \change\ \ipa{a} / _\ipa{t}\\
\ipa{i} \change\ \ipa{iE} / _\ipa{s}\\
\ipa{i} \change\ \ipa{\textturna} / _\ipa{m}

\subparagraph{Proto-rGyalrongic to Trung}{\it Pogostick Man}, from Nagano, Yashuiki (1979), ``A Historical Study of rGyarong Initials and Prefixes''. {\it Linguistics of the Tibeto-Burman Area} 4(2):44 -- 68; and Nagano, Yashuiki (1979), ``A Historical Study of rGyarong Rhymes''. {\it Linguistics of the Tibeto-Burman Area} 5(1):37 -- 47

\tab {\it NB: These changes are likely very incomplete. The source did not have much to say about this language.}

{\bf Initials:}\\
\ipa{kr} \change\ \ipa{dz}\\
\ipa{z} \change\ \ipa{k}\\
C[+ sibilant]\ipa{w} \change\ \ipa{s}\\
\ipa{dZ} \change\ \ipa{tsh}

{\bf Rhymes:}\\
\ipa{un} \change\ \ipa{ial}\\
\ipa{an} \change\ \ipa{a(i)}\\
\ipa{i} \change\ \ipa{@i} / _\#\\
\ipa{i} \change\ \ipa{i@} / _\ipa{m}\\
\ipa{it} \change\ \ipa{u}

\subparagraph{Proto-rGyalrongic to Tsa Ku Nao}{\it Pogostick Man}, from Nagano, Yashuiki (1979), ``A Historical Study of rGyarong Initials and Prefixes''. {\it Linguistics of the Tibeto-Burman Area} 4(2):44 -- 68; and Nagano, Yashuiki (1979), ``A Historical Study of rGyarong Rhymes''. {\it Linguistics of the Tibeto-Burman Area} 5(1):37 -- 47

{\bf Initials:}\\
\O\ \change\ \ipa{n} / _\ipa{pr}\\
P \change\ \ipa{p} / \ipa{r}_\\
\ipa{\c{c}b} \change\ \ipa{\:sp}\\
N\ipa{b} \change\ \ipa{p.}\\
\ipa{d(r) \c{c}}T \change\ \ipa{t St}\\
\ipa{k} \ipa{\{sg,\c{c}}K\} N\ipa{g r}K \change\ \ipa{k. sk mg nk}\\
\ipa{kj} \change\ \ipa{tS.}\\
\ipa{z} \change\ \ipa{ts} / \ipa{r}_\\
C[+ sibilant]\ipa{w} \change\ \ipa{sj}\\
(P)\ipa{\c{c}} N\ipa{\c{c} \c{c}l} \change\ \ipa{(b)C np.j \:sl}\\
\ipa{ts dz} NTS KTS \ipa{s}TS \change\ \ipa{tS ts \{ts,m\} gts sp}\\
\ipa{s} \change\ \ipa{\:s} / _\ipa{n}\\
N\ipa{nj} \change\ \ipa{m\textltailn}\\
\ipa{nj rnj} \change\ \ipa{nj r\textltailn}\\
\O\ \change\ \ipa{dz} / \#_\ipa{j}

{\bf Finals:}\\
\ipa{ut u\{k,n\} ur} \change\ \ipa{ud uo uE}\\
\ipa{uj} \change\ \ipa{ue}\\
\ipa{om} \change\ \{\ipa{on,am}\}\\
\ipa{a(t)} \change\ \ipa{E} / _\#\\
\ipa{ap} \change\ \ipa{Ek}\\
\ipa{an} \change\ \ipa{\textlhtlongi E}\\
\ipa{ew e ej} \change\ \ipa{@ i ei}\\
\ipa{i} \change\ \ipa{e} / _\ipa{m}\\
\ipa{i i\{t,k\}} \change\ \{\ipa{@,iE}\} \ipa{@}\\
\ipa{iN} \change\ \ipa{\textraisevibyi}\\
\ipa{j} \change\ \O\ / \ipa{i}_

\subparagraph{Proto-rGyalrongic to Tzu Ta}{\it Pogostick Man}, from Nagano, Yashuiki (1979), ``A Historical Study of rGyarong Initials and Prefixes''. {\it Linguistics of the Tibeto-Burman Area} 4(2):44 -- 68; and Nagano, Yashuiki (1979), ``A Historical Study of rGyarong Rhymes''. {\it Linguistics of the Tibeto-Burman Area} 5(1):37 -- 47

{\bf Initials:}\\
P \change\ \ipa{p} / \ipa{s}_\\
P \ipa{t} \change\ \ipa{pdz tS} / \ipa{r}_\\
N\ipa{p} N\ipa{b} N\ipa{t} N\ipa{g} N\ipa{\c{c}} \change\ \ipa{sts mp mt\super h mk nts\super hdz}\\
\ipa{p} \change\ \ipa{b} / _\ipa{r}\\
\{\ipa{d},KT\} \ipa{st} \change\ \ipa{tS zdZ}\\
\ipa{dr \c{c}dr} \change\ \ipa{\:t \:s\:tdz}\\
\ipa{\c{c}}K \change\ \ipa{\:sk}\\
\ipa{k rk kr kj} N\ipa{kj gr} \change\ \ipa{k\super hdz ng k\super hr tSh dZ nk\super hr}\\
\ipa{g} \change\ \O\ / _\ipa{r}\\
P\ipa{\c{c}} \change\ \ipa{b.\c{c}}\\
\ipa{\c{c}l} \change\ \ipa{s\textbeltl}\\
C[+ sibilant]\ipa{w \c{c}} \change\ \ipa{swdz sdz}\\
NTS KTS \ipa{l}TS \change\ \ipa{m\:t\:s \{kts,\:t\:s\super h\} b\:d\:z}\\
\ipa{ts dz rdz dZ} \change\ \ipa{\{\:t\:s,Z\} tsj rts\super h \:t\:s}\\
\O\ \change\ \ipa{j} / \ipa{r}N_\\
\ipa{\c{c}} \change\ \ipa{\:s} / _\ipa{m}\\
\ipa{N(w)} \change\ \ipa{\textctn(wj)}\\
N\ipa{nj} \change\ \ipa{m\textltailn} ?\\
\ipa{j} \change\ \ipa{s} / \#_

{\bf Rhymes:}\\
\ipa{uk uN} \change\ \{\ipa{u,o}\} \ipa{e}\\
\ipa{o} \change\ \ipa{e} / _\ipa{s}
\ipa{s} \change\ \O\ / \ipa{u}_\\
\ipa{a} \change\ \ipa{e} / _(\ipa{p})\\
\ipa{t} \change\ \ipa{N} / \ipa{a}_\\
\ipa{aw aj} \change\ \ipa{au ai}\\
\ipa{e i} \change\ \ipa{i a} / _\ipa{m}\\
\ipa{it ik} \change\ \ipa{o ek}\\
\ipa{iw ij} \change\ \ipa{iu ei}

\subparagraph{Proto-rGyalrongic to Wassu}{\it Pogostick Man}, from Nagano, Yashuiki (1979), ``A Historical Study of rGyarong Initials and Prefixes''. {\it Linguistics of the Tibeto-Burman Area} 4(2):44 -- 68; and Nagano, Yashuiki (1979), ``A Historical Study of rGyarong Rhymes''. {\it Linguistics of the Tibeto-Burman Area} 5(1):37 -- 47

{\bf Initials:}\\
\ipa{pr b} \change\ \ipa{br p}\\
\ipa{d} KT \change\ \ipa{l kt}\\
\ipa{k kr kj g} \change\ \ipa{k. k.r ts. g}\\
\ipa{g \c{c}} \change\ \O\ \ipa{zg} / _\ipa{r}\\
P\ipa{\c{c} \c{c}l} \change\ \ipa{S Sn}\\
\ipa{ts dzl} NTS \change\ \ipa{\{tS.,j\} tsl m}\\
\ipa{s} \change\ \ipa{S} / _\ipa{n}\\
\ipa{N} \change\ \ipa{j}\\
N\ipa{nj \c{c}nj} \change\ \ipa{mn Sn}

{\bf Rhymes:}\\
\ipa{uk} \change\ \ipa{o}\\
\ipa{o} \change\ \ipa{@} / _\ipa{n}\\
\ipa{t} \change\ \O\ / \ipa{a}_\\
\ipa{an ap} \change\ \ipa{ai ie}\\
\ipa{aj} \change\ \ipa{ui}\\
\ipa{ew} \change\ \ipa{i}\\
\ipa{it ik} \change\ \ipa{o i}\\
\ipa{is} \change\ \ipa{eu}\\
\ipa{im iN} \change\ \ipa{wa ie}\\
\ipa{ij} \change\ \ipa{e}

\subsubsection{Tibetic}

\paragraph{Old Tibetan to Amdo dialects}{\it Pogostick Man}, from Jacques, Guillaume (2004), ``Phonologie et Morphologie du Japhug (rGyalrong)''. Universit\'{e} Paris-Diderot -- Paris VII \textless tel-00138568\textgreater. \textless\url{https://tel.archives-ouvertes.fr/file/index/docid/138568/filename/these-japhug.pdf}\textgreater

\tab{\it NB: This source is in French and looks to at least sometimes use a transcription that isn't IPA.}

\{\ipa{d,g,s,l,r}\} \change\ \{\ipa{h,r}\} / \#_\\
\ipa{S} \change\ \ipa{x} / \#_ (some dialects, never when following preinitials)\\
\ipa{k(\super h)\{r,j\} g\{r,j\}} \change\ \ipa{tC(\super h) d\textctz}\\
\ipa{p\super h b} \change\ \ipa{h w}\\
\ipa{i} \change\ \ipa{@} / _\#\\
\ipa{sr} \change\ \ipa{\:s} / \#_\\
\ipa{s} \change\ either \ipa{i} or a diphthong ending in \ipa{i}? / _\#\\
\ipa{d} \change\ \ipa{l} / _\# (some dialects)\\
\ipa{t} \change\ \ipa{l} / _\# (further development in bLa-brang)

\clearpage

\section{Siouan-Iroquoian}\tab Based upon Julian and Chafe, Proto-Siouan-Iroquoian, if it existed, appears to have had the following phonetic inventory:

\begin{center}\begin{tabular}{c | c c c c c c c}
& Bilabial & Dental & Alveolar & Postalveolar & Palatal & Velar & Glottal \\ \hline
Nasal & \ipa{m} & & \ipa{n}\\
Plosive & \ipa{p p\super h} & & \ipa{t t\super h} & & & \ipa{k k\super h} & \ipa{P}\\
Fricative & & \ipa{T} & \ipa{s} & \ipa{S} & & \ipa{x} & \ipa{h}\\
Liquid & & & \ipa{r} & & \ipa{j} & \ipa{w}\end{tabular}

\begin{tabular}{c | c c c}
& Front & Central & Back\\ \hline
High & \ipa{i \~{\i}} & & \ipa{u \~{u}}\\
Mid & \ipa{e \~{e}} & & \ipa{o \~{o}}\\
Low & & \ipa{a \~{a}}\end{tabular}
\end{center}

\tab For this following section, the sound transcribed here as $\langle$\ipa{r}$\rangle$ may in actuality represent something akin to /\textipa{\*r}/. % On liquids in Siouan languages

\tab Siouan-Iroquoian, and for that matter the inclusion of Yuchian and Caddoan within the former and the latter, respectively, is far from universally accepted; their inclusion here is in large part due to the available sources giving correspondences for each. It was unknown whether Proto-Caddoan was the same as the Proto-Iroquois-Caddoan indicated in Cafe's paper, so the Caddoan changes have been presented after the main Iroquoian changes.

\tab Per KneeQuickie, Whimemsz wishes to ``[n]ote that Siouan-Iroquoian is a proposed, rather than firmly-demonstrated, language grouping".

\tab (From Chafe, Wallace L. (1964), ``Another Look at Siouan and Iroquoian". {\it American Anthropologist New Series}, 66:852 -- 862; Julian, Charles (2010), ``A History of the Iroquoian Languages", University of Manitoba, Winnipeg; and from cedh aumdmanh's Iroquoian changes)

\subsection{Proto-Siouan-Iroquoian to Proto-Iroquoian}{\it Pogostick Man}, from Chafe, Wallace L. (1964), ``Another Look at Siouan and Iroquoian". {\it American Anthropologist New Series}, 66:852 -- 862; and from cedh aumdmanh's Iroquoian changes

\ipa{w} \change\ \O\ / _\{\ipa{o,\~{o},\~{\i}}\}\\
\ipa{m} \change\ \ipa{w} / _\ipa{\~{a}}\\
\ipa{t} \change\ \ipa{ts} / _\{\ipa{i,\~{\i}}\}\\
\ipa{t\super h} \change\ \ipa{ts} / _\ipa{i}\\
\ipa{t\super h} \change\ \ipa{n} / else\\
\ipa{\~{a}} \change\ \ipa{\~{e}}\\
\ipa{e} \change\ \ipa{i} / \ipa{r}_\ipa{P}\\
\ipa{r} \change\ \ipa{ts} / _\ipa{i}\\
\ipa{\~{\i}} \change\ \ipa{i}\\
\ipa{k} \change\ \O\ / \ipa{t}_\\
\ipa{k\super h} \change\ \ipa{r}\\
\ipa{m} \change\ \ipa{n}\\
\ipa{p} \change\ \O\ / C_ ! \ipa{s}_\\
\ipa{p} \change\ \ipa{k\super w} / else\\
\ipa{p\super h} \change\ \ipa{\textturnw} (this is a bit of a guess; the paper proper has $\langle$hw$\rangle$ here)\\
\ipa{S T} \change\ \ipa{s t}\\
The paper is unclear about what happened to /\ipa{u}/.\\
\ipa{x} \change\ \O\ / _\ipa{k}\\
\ipa{x} \change\ \ipa{h} / _C ! C_C\\
\ipa{x} \change\ \ipa{k}\\
\ipa{P} \change\ \O\ / C_

\subsubsection{Proto-Iroquoian to Cherokee}{\it cedh audmanh}, from Julian, Charles (2010), ``A History of the Iroquoian Languages", University of Manitoba, Winnipeg

\ipa{k k\super w} \change\ \ipa{ts k} / _\ipa{i}\\
\ipa{w} \change\ \O\ / \ipa{h}_\ipa{i}\\
\ipa{w} \change\ \O\ / \ipa{t(h)}_\\
\{\ipa{w}V,\ipa{j}V\} \change\ V\ipa{:}[+low falling tone]\\
V \change\ \O\ / C_\ipa{h}C\\
V\ipa{P} \change\ V\ipa{:}[+low falling tone] / _C\\
\{V\ipa{h},V\ipa{P}\} \change\ V\ipa{:} / _\#\\
\ipa{a(:)w\~{e}(:)} \change\ \ipa{a(:)ma(:)}\\
V[+nas] \change\ \ipa{a:}[+high rising tone]\\
\O\ \change\ V\ipa{:}[+high rising tone] / C_\# (``usually one of [/\ipa{a: i: \~{2}:}/ with this tone], the conditions are unclear")\\
\ipa{iji} \change\ \ipa{i:}\\
\ipa{tsn} \change\ \ipa{hst}\\
\ipa{n} \change\ \ipa{h} / _\ipa{st}\\
\{\ipa{n,r}\} \change\ \O\ / _\ipa{j}\\
\ipa{t} \change\ \O\ / _\{\ipa{k,n}\}\\
\ipa{t} \change\ \O\ / \ipa{n}_\\
\ipa{j} \change\ \O\ / \ipa{ts}_\\
\O\ \change\ \ipa{i} / C_R\\
\ipa{s} \change\ \O\ / \#\ipa{h}_V\\
\ipa{ts} \change\ \ipa{s} / \ipa{h}_\\
\ipa{ks} \change\ \ipa{ts} / _V\\
\ipa{nh} \change\ \ipa{hn}\\
\ipa{\~{e}(:) \~{o}(:)} \change\ \ipa{o(:) \~{2}(:)}\\
\ipa{r} \change\ \ipa{l}\\
``Some additional changes seem to have taken place in one or more Cherokee dialects, affecting consonant clusters whose reconstructed identity is in most cases uncertain. Example correspondences include /hs \raisebox{-0.6ex}{\textasciitilde} lh \raisebox{-0.6ex}{\textasciitilde} thl/ (probably \textless\ */hs\ipa{\*r}/) or /ts \raisebox{-0.6ex}{\textasciitilde} tl \raisebox{-0.6ex}{\textasciitilde} thl/ (maybe \textless\ */ts\ipa{\*r}/?)"\\
\tab ``[A] synchronic allophonic rule:"
\ipa{t ts k k\super w} \change\ \ipa{d dz g g\super w} / _V

\subsubsection{Proto-Iroquoian to Proto-Northern Iroquoian}{\it cedh audmanh}, from Julian, Charles (2010), ``A History of the Iroquoian Languages", University of Manitoba, Winnipeg

\ipa{o(:) u(:)} \change\ \ipa{a(:) o(:)}\\
\ipa{iji(:)} \change\ \ipa{hi(:)} / \{\ipa{k,s}\}_ ``(possibly after all non-glottal obstruents)"\\
\ipa{i} \change\ \ipa{e} / ! _\ipa{h}CC (``short only")\\
\O\ \change\ \ipa{i(:)} / \#_(C)(C)CVC(C)(C)\#\\
V \change\ "V / ``in antepenultimate syllables, if the vowel of the penultimate syllable was short */a/ followed by a single non-glottal consonant"\\
V \change\ "V / ``in penultimate syllables not preceded by an accented antepenult"\\
V\ipa{:} \change\ V[-long] / ! in U\#\\
"V \change\ "V\ipa{:} / ``in open penultimate syllables followed by a non-glottal consonant"\\
\ipa{h} \change\ \O\ / \#_\ipa{s}\\
\ipa{n} \change\ \O\ / _\ipa{ti}\\
\ipa{t} \change\ \O\ / \ipa{n}_V

\paragraph{Proto-Northern Iroquoian to Cayuga}{\it cedh audmanh}, from Julian, Charles (2010), ``A History of the Iroquoian Languages", University of Manitoba, Winnipeg

\O\ \change\ \ipa{a} / \ipa{w}_\ipa{j}\\
"V(C)(C)C\ipa{a}CV \change\ V(C)"(C)C\ipa{a}CV / _\#\\
\ipa{h} \change\ \O\ / _\ipa{nh}\\
\ipa{P} \change\ \O\ / _\ipa{nk(\super w)}\\
\ipa{n} \change\ \ipa{t} / _\ipa{k(\super w)}\\
\ipa{ts} \change\ \ipa{hs} / V_ ! _\{\ipa{h,i,j,r}\}\\
\ipa{ts} \change\ \ipa{s} / ! _\{\ipa{h,i,j,r}\}\\
\ipa{ns} \change\ \ipa{ts} / _\ipa{k(\super w)}\\
V \change\ V\ipa{:} / _C[-glottal] ``in even-numbered syllables when accented or immediately before the accent"\\
``[A]ccented short vowels in odd-numbered penults lose their accent"\\
``[W]ords with no accent acquire a new accent on the vowel of the last non-final even syllable of the word"\\
V\ipa{P} \change\ \ipa{P}V / ``in odd-numbered unaccented non-final syllables;" ! \{\ipa{P,h}\}_\\
\ipa{j} \change\ \O\ / \ipa{ts}_\\
\ipa{h} \change\ \O\ / \ipa{s}_\ipa{w}\\
\ipa{r} \change\ \ipa{n} / _\ipa{(h)j}\\
\ipa{r} \change\ \ipa{w} / \{\ipa{o(:),\~{o}(:)}\}_\{\ipa{a(:),e(:),\~{e}(:),i(:)}\}\\
\ipa{r} \change\ \ipa{j} / \{\ipa{e(:),\~{e}(:),i(:)}\}_\{\ipa{a(:),o(:),\~{o}(:)}\}\\
\ipa{r} \change\ \O\ / VH_\\
\ipa{r} \change\ \O\ / _H\\
\ipa{r} \change\ \O\ / \ipa{w}_\\
\ipa{r} \change\ \O\ / V_V\\
V$_1$"V$_2$ "V$_1$\ipa{:}V$_2$ \change\ "V$_1$V$_2$ "V$_1$[-long]V$_2$\\
\ipa{e(:)} \change\ \ipa{\~{e}} / _\ipa{\~{e}(:)}\\
\ipa{o(:)} \change\ \ipa{\~{o}} / _\ipa{\~{o}(:)}\\
"V$_0$V$_0$ \change\ V$_0$\ipa{:}[-accent]\\
R \change\ \O\ / \ipa{P}_\#\\
C \change\ \O\ / \{\ipa{s,k}\}_\#\\
C\ipa{h} \change\ \O\ / _\ipa{s}\#\\
\ipa{h} \change\ \O\ / V\ipa{:}_\#\\
\ipa{t} \change\ \ipa{h} / _\ipa{t}\\
\ipa{ths} \change\ \ipa{tsh}\\
\O\ \change\ \ipa{h} / \{\ipa{t,k}\}_\ipa{n}\\

\subparagraph{Cayuga to Upper Cayuga}{\it cedh audmanh}, from Julian, Charles (2010), ``A History of the Iroquoian Languages", University of Manitoba, Winnipeg

\ipa{s} \change\ \ipa{f} / \ipa{h}_\ipa{r}\\
\ipa{ts} \change\ \ipa{s} / _\ipa{(h)r}

``Allophonic changes:"\\
\ipa{s} \change\ \ipa{S} / _\{\ipa{r,j}\}\\
\ipa{t k k\super w} \change\ \ipa{d g g\super w} / _\{V,R\}

\subparagraph{Cayuga to Lower Cayuga}{\it cedh audmanh}, from Julian, Charles (2010), ``A History of the Iroquoian Languages", University of Manitoba, Winnipeg

\ipa{ts} \change\ \ipa{t} / _\ipa{(h)r}\\
\ipa{t} \change\ \ipa{k} / _\ipa{j}\\
V \change\ V[-voiced] / _\ipa{h} ``(odd syllables only)"\\
\ipa{tP tsP kP k\super wP} \change\ \ipa{t' ts' k' k\super w'}

``Allophonic changes:"\\
\ipa{s} \change\ \ipa{S} / _\{\ipa{r,j}\}\\
\ipa{t k k\super w} \change\ \ipa{d g g\super w} / _\{V,R\} ! _V[-voiced]

\paragraph{Proto-Northern Iroquoian to Huron}{\it cedh audmanh}, from Julian, Charles (2010), ``A History of the Iroquoian Languages", University of Manitoba, Winnipeg

\ipa{s} \change\ \ipa{S} / ! _\{\ipa{n,t,k(\super w),w}\} ``or when part of the affricate /ts/"\\
\ipa{ts} \change\ \ipa{S} / _\ipa{r}\\
\ipa{ts} \change\ \ipa{s} / ! _\{\ipa{i,j}\}\\
\ipa{n} \change\ \O\ / \ipa{t(h)}_\\
\ipa{n} \change\ \O\ / _\ipa{s}\\
\ipa{n} \change\ \O\ / _\ipa{i} ``(in pronominal prefixes only)"\\
\ipa{k} \change\ \ipa{i} / \#_\ipa{n}\\
\ipa{k} \change\ \O\ / _\ipa{n}\\
\ipa{n hn s}C \change\ \ipa{t th} C\ipa{h} / \ipa{s}_\\
\ipa{k} \change\ \ipa{h} / _\{\ipa{t,ts,s,S}\}\\
\ipa{k} \change\ \ipa{x} / \{\#,R,\ipa{P},V\}_\{V,\ipa{P},R,\#\}\\
\ipa{k\super w} \change\ \ipa{x\super w} / V_V\\
\ipa{t} \change\ \ipa{k} / _\ipa{(h)w}\\
\ipa{t} \change\ \O\ / _\ipa{k(\super w)}\\
\ipa{j} \change\ \O\ / \ipa{ts}_\\
\ipa{j} \change\ \O\ / \#_V\\
\ipa{j} \change\ \O\ / V_\{V,\#\}\\
\{\ipa{r,w}\} \change\ \O\ / _\ipa{j}\\
\ipa{w} \change\ \O\ / \#_\\
\ipa{w} \change\ \O\ / _\{\ipa{r},\#\}\\
\ipa{h} \change\ \O\ / \#_\ipa{w}\\
\O\ \change\ \ipa{k} / \ipa{s}_\ipa{(h)w}\\
\O\ \change\ \ipa{a} / CC_\ipa{P}

``Some known changes in dialects other than pre-Wyandot:"\\
\ipa{r} \change\ \ipa{h} / \ipa{S}_\\
\ipa{t} \change\ \ipa{k} / _\ipa{r}\\

\subparagraph{Huron to Wyandot}{\it cedh audmanh}, from Julian, Charles (2010), ``A History of the Iroquoian Languages", University of Manitoba, Winnipeg

V\ipa{:} \change\ "V / in U\#; ``this change may have been present in Huron already"\\
\ipa{t} \change\ \ipa{k} / _\ipa{j}\\
\ipa{x\super w} \change\ \ipa{w}\\
\{\ipa{P,h}\} \change\ \O\ / _\ipa{nh}\\
\ipa{n} \change\ \ipa{t} / _\ipa{h}\\
\ipa{h} \change\ \O\ / \{\ipa{t,ts,s,k}\}_\\
V\ipa{h} \change\ V\ipa{:} / _R\\
\O\ \change\ \ipa{w} / \{\ipa{o(:),\~{o}(:)}\}_V\\
\O\ \change\ \ipa{j} / \{\ipa{e(:),\~{e}(:),i(:)}\}_V\\
\ipa{x} \change\ \O\ / _\{\ipa{i,j}\}\\
\ipa{x} \change\ \ipa{e} / \#_\ipa{r}\\
\ipa{w j} \change\ \ipa{m \textltailn} / between two vowels of unlike nasality\\
\ipa{j} \change\ \ipa{Z} / \{\#,\ipa{P},V\}_V\\
\ipa{x} \change\ \ipa{j} / _V\\
\ipa{x} \change\ \O\\
\ipa{k} \change\ \O\ / _\#\\
\ipa{\~{e}(:) \~{o}(:) o(:)} \change\ \ipa{\~{E}(:) \~{O}(:) u(:)}\\
\ipa{n} \change\ \ipa{\super nd} / _\{V[-nas],\ipa{r}\}\\
\ipa{n} \change\ \ipa{N} / _\{\ipa{j,w}\}\\
\ipa{\~{E}(:)} \change\ \ipa{\~{a}(:)} / \ipa{w}_

\paragraph{Proto-Northern Iroquoian to Onondaga}{\it cedh audmanh}, from Julian, Charles (2010), ``A History of the Iroquoian Languages", University of Manitoba, Winnipeg

\ipa{s} \change\ \ipa{S} / ! \ipa{n}_ ``or when part of the unit affricate /ts/"\\
\ipa{ts} \change\ \ipa{hs} / V_V ! _\ipa{i}\\
\ipa{ts} \change\ \ipa{s} / ! _\{\ipa{h,i,j}\}\\
\ipa{ns} \change\ \ipa{ts} / _\ipa{k(\super w)}\\
\ipa{n} \change\ \O\ / _\ipa{s}\\
"V(\ipa{:})(C)(C)V\ipa{:} \change\ V[-long](C)"(C)CV[-long] / _\#\\
\ipa{ara} \change\ \ipa{a:} / ``unaccented syllables only"\\
\ipa{jh} \change\ \ipa{hj}\\
\ipa{n} \change\ \ipa{t} / _\ipa{k(\super w)}\\
"V(C)(C)C\ipa{a}CV \change\ V(C)"(C)C\ipa{a}CV / _\#\\
\ipa{S} \change\ \ipa{s}\\
\ipa{n} \change\ \O\ / \ipa{h}_\ipa{r}\\
\ipa{h} \change\ \O\ / \ipa{w}_\ipa{j}\\
V\ipa{w} \change\ V\ipa{:} / _\{\ipa{r,j}\}\\
"V \change\ "V\ipa{:} / _C(R)V\\
\ipa{hs} \change\ \ipa{sh} / C_\\
\ipa{hts} \change\ \ipa{tsh} / C_V\\
V \change\ V\ipa{:} / _"C[-glottal](R)V\{\ipa{:},H\} in ``even numbered syllables only"\\
V \change\ V\ipa{:} / _KRV ``in the second syllable of a word"\\
V \change\ V[+high tone] / _\$"V\\
\ipa{a(:) o(:) \~{o}(:)} \change\ \ipa{\ae(:) e(:) \~{e}(:)} / \ipa{r}_\\
V\ipa{r} \change\ V\ipa{:} / _C\\
\ipa{r}V \change\ V\ipa{:} / C_\\
\ipa{r} \change\ \ipa{j} / \{\ipa{e(:),\~{e}(:),i(:)}\}_V\\
\ipa{r} \change\ \ipa{w} / \{\ipa{o(:),\~{o}(:)}\}_V\\
\ipa{r} \change\ \O\\
\ipa{h} \change\ \O\ / _\ipa{sn}\\
\ipa{h} \change\ \O\ / _\{\ipa{k,t,ts,s}\}\#\\
\ipa{k} \change\ \ipa{h} / _\ipa{k}\\
\ipa{\~{o}(:)} \change\ \ipa{\~{u}(:)}

``Allophonic changes:"\\
\ipa{ts} \change\ \ipa{tS} / _\{\ipa{(h)i,(h)j}\}\\
\ipa{s} \change\ \ipa{S} / _\{\ipa{hi,hj}\}\\
\ipa{t tS k(\super w)} \change\ \ipa{d dZ g(\super w)} / _\{V,R\}

\paragraph{Proto-Northern Iroquoian to Proto-Mohawk-Oneida}{\it cedh audmanh}, from Julian, Charles (2010), ``A History of the Iroquoian Languages", University of Manitoba, Winnipeg

\ipa{\~{e}(:) \~{o}(:)} \change\ \ipa{\~{2}(:) \~{u}(:)}\\
\ipa{ts} \change\ \ipa{hs} / V_\{\ipa{t,k(\super w)}\}\\
\ipa{ts} \change\ \ipa{s} / ! _\{\ipa{h,i,j}\}\\
\ipa{ns} \change\ \ipa{ts} / _\{\ipa{t,k(\super w)}\}\\
\ipa{n} \change\ \O\ / _\ipa{s}\\
"V \change\ "V\ipa{:}[+falling tone] / _\{\ipa{P,h}R\}\\
\ipa{P} \change\ \O\ / "V\ipa{:}[+falling tone]_C\\
\ipa{h} \change\ \O\ / "V\ipa{:}[+falling tone]_R\\
\ipa{h} \change\ \O\ / _\#\\
\ipa{h} \change\ \O\ / \#_\ipa{w}\\
V\ipa{:} \change\ V / _(C)(C)(C)\#\\
C\ipa{P}V$_0$ \change\ CV$_0$\ipa{P}V$_0$\\
\O\ \change\ \ipa{e} / \ipa{w}_\ipa{r} (and ``probably. . .in other environments")

\subparagraph{Proto-Mohawk-Oneida to Mohawk}{\it cedh audmanh}, from Julian, Charles (2010), ``A History of the Iroquoian Languages", University of Manitoba, Winnipeg

V$_0$ \change\ \O\ / "VC(C)(C)V$_0$\ipa{P}_C(C)(C)\#\\
\O\ \change\ \ipa{e} / \{\ipa{tsh,s,n}\}_\ipa{r}\\
\ipa{w} \change\ \O\ / _\ipa{jh}\\
\ipa{h} \change\ \O\ / \ipa{w}_\ipa{j}\\
\ipa{jh} \change\ \ipa{hj}\\
\O\ \change\ \ipa{e} / \ipa{w}_\ipa{j}\\
\O\ \change\ \ipa{e} / \ipa{n}_\ipa{k(\super w)}\\
\O\ \change\ \ipa{e} / \{\ipa{t,k}\}_\{\ipa{r,n}\}\\
\O\ \change\ \ipa{e} / \{\#,V\}\ipa{s}_\ipa{n}\\
\O\ \change\ \ipa{e} / \ipa{t}_\ipa{w}\\
\O\ \change\ \ipa{e} / \{\#,V\}\ipa{s}_\ipa{w}

``Dialectal changes include:"\\
--- \ipa{r} \change\ \ipa{l}\\
--- \ipa{t} \change\ \ipa{k} / _\ipa{j}\\
--- \ipa{k} \change\ \ipa{t} / _\ipa{j}\\
--- \ipa{w\~{@}} \change\ \ipa{\~{u}} / \{\ipa{h,s}\}_\\
--- \ipa{j} \change\ \O\ / \ipa{ts}_\\
--- \ipa{t} \change\ \ipa{tS} / _\ipa{(h)j}\\
--- \ipa{wh} \change\ \ipa{f}

``Allophonic changes:"\\
--- \ipa{ts} \change\ \ipa{tS} / _\{\ipa{(h)i,(h)j}\}\\
--- \ipa{t tS k(\super w)} \change\ \ipa{d dZ g(\super w)} / _\{V,R\}\\
--- \ipa{s} \change\ \ipa{S} / _\ipa{(h)j}\\
--- \ipa{s} \change\ \ipa{z} / \{\#,V\}_\{V,R\} ! R = \ipa{j}\\

\subparagraph{Proto-Mohawk-Oneida to Oneida}{\it cedh audmanh}, from Julian, Charles (2010), ``A History of the Iroquoian Languages", University of Manitoba, Winnipeg

"V\ipa{:}[-falling tone]CV \change\ V\ipa{:}"CV \\
"V \change\ "V\ipa{:} / _\ipa{P}\\
\ipa{P} \change\ \O\ / "V\ipa{:}_\\
\O\ \change\ \ipa{i} / \{V,\ipa{t}\}\ipa{n}_\ipa{k(\super w)}V\\
\ipa{ths} \change\ \ipa{tsh}\\
\ipa{hs} \change\ \ipa{sh} / _\{\ipa{n,w}\}\\
\ipa{h} \change\ \O\ / _C\ipa{h}\\
\ipa{h} \change\ \O\ / \ipa{k}_\{\ipa{s,ts}\}\\
\ipa{h} \change\ \O\ / _\{\ipa{sk,st}\}\\
\ipa{h} \change\ \O\ / \{\ipa{st,tst}\}_\\
\ipa{h} \change\ \O\ / \ipa{ts}_\ipa{r} ! ``in pre-pausal forms, see also below"\\
\{\ipa{h,P}\} \change\ \O\ / _R ``in post-tonic syllables"\\
\ipa{P} \change\ \ipa{h} / _C ``in post-tonic syllables"\\
\ipa{r} \change\ \ipa{l}\\
"V\ipa{:}[+falling tone] \change\ "V[-long -falling tone]\\
V\ipa{:} \change\ V[-long] _"C(C)(C)V (``this change happens only in the Ontario dialect")\\
``In addition, a number of sound changes have applied to words {\it only in the pre-pausal position}. Most of these changes are characterized by the devoicing of one or more segments at the end of a word''\\
--- \O\ \change\ \ipa{e} / C_\{\ipa{n,l}\}V(H)\#\\
--- \O\ \change\ \ipa{o} / C_\ipa{w}V(H)\#\\
--- \O\ \change\ \ipa{i} / C_\ipa{j}V(H)\#\\
--- V\ipa{:}[+falling tone]C(C)V(H) \change\ \r{V}\ipa{:}[+falling tone]\r{C}(\r{C})\r{V}(\r{H})\ / _\#\\
--- CV\ipa{:}[+falling tone] \change\ \r{C}\r{V}\ipa{:}[+falling tone] / _\#\\
--- CV\ipa{P} \change\ \r{C}\r{V}\ipa{\r{P}} / _\#\\
--- V[-long] \change\ \r{V} / R_\#\\
--- OV[-long] \change\ \r{O}\r{V} / _\#\\
--- \ipa{j}V \change\ \O\ / C\ipa{i}_(H)\#\\
--- \ipa{P} \change\ \ipa{h} / _C\#\\
--- R \change\ \r{R} / _\#

``Allophonic changes:"\\
 \ipa{ts} \change\ \ipa{tS} / _\{\ipa{(h)i,(h)j}\}\\
\ipa{t tS k(\super w)} \change\ \ipa{d dZ g(\super w)} / _\{V,R\}\\
\ipa{s} \change\ \ipa{S} / _\ipa{(h)j}\\
\ipa{s} \change\ \ipa{z} / \{\#,V\}_\{V,R\}

\paragraph{Proto-Northern Iroquoian to Seneca}{\it cedh audmanh}, from Julian, Charles (2010), ``A History of the Iroquoian Languages", University of Manitoba, Winnipeg

\ipa{ts} \change\ \ipa{s} / ! _\{\ipa{i,j}\}\\
\ipa{n} \change\ \O\ / _\ipa{s}\\
\ipa{j} \change\ \O\ / _\ipa{ts}\\
V \change\ V\ipa{:} / _\{\ipa{t,k(\super w),s,n,r,j,w}\} ``in even penultimate syllables"\\
``The inherited accent system is replaced by a new one, by which...\\
--- ``the accent falls on the last nonfinal even short syllable of a word if this vowel is followed directly\\
------ ``by a single glottal consonant,\\
------ ``by /sn/ or /sw/,\\
------ ``by any two-consonant cluster that does not end in a resonant,\\
------ ``or by any three consonant cluster;\\
--- ``failing that, the accent falls on the last non-final even short syllable that is followed by a non-final syllable such as that just described;\\
--- ``failing that, a word has no accent."\\
\ipa{a} \change\ \ipa{\ae} / _\ipa{ra(:)}\\
\ipa{a} \change\ \ipa{e} / _\ipa{ro(:)}\\
\ipa{\~{a}} \change\ \ipa{\~{e}} / _\ipa{r\~{o}:}\\
\ipa{a(:)} \change\ \ipa{\ae(:)} / \ipa{r}_\\
\ipa{h} \change\ \O\ / _\{\ipa{tk,nh,s}C,C\#\}\\
\ipa{h} \change\ \O\ / \#_\ipa{w}\\
\ipa{h} \change\ \O\ / \ipa{w}_\ipa{j}\\
\ipa{hw} \change\ \O\ / \ipa{\~{o}}_\\
\ipa{w} \change\ \O\ / _\{\ipa{r,j}\}\\
\ipa{r} \change\ \ipa{n} / _\ipa{(h)j}\\
\ipa{r} \change\ \O\ / V\ipa{h}_\\
V\ipa{h} \change\ V\ipa{:} / _\{\ipa{n,w,j}\}\\
\ipa{h} \change\ \O\ / V_V\\
\ipa{r} \change\ \ipa{j} / C[-glottal](\ipa{h})_\{\ipa{o(:),\~{o}(:)}\\
\ipa{r} \change\ \ipa{h} / \ipa{s}_\\
\ipa{r} \change\ \ipa{j} / \ipa{i(:)}_V\\
\ipa{r} \change\ \ipa{w} / \{\ipa{o(:),\~{o}(:)}\}_V\\
\ipa{r} \change\ \O\\
\ipa{o} \change\ \ipa{o:} / _\{\ipa{a:,\ae:}\}\\
V\ipa{:} \change\ V / V_\\
V$_1$"V$_2$ \change\ "V$_1$V$_2$\\
\ipa{a(:)} \change\ \ipa{\~{e}(:)} / adjacent to a nasal vowel\\
\ipa{\~{e}(:)} \change\ \ipa{e(:)} / _\{\ipa{e(:),o(:)}\\
\ipa{P} \change\ \O\ / _\ipa{nk(\super w)}\\
\ipa{n} \change\ \ipa{t} / _\ipa{k(\super w)}\\
\ipa{n} \change\ \ipa{t} / \ipa{P}_\#\\
R \change\ \O\ / _\ipa{h}\\
C \change\ \O\ / \ipa{s}_\#\\
C[-glottal] \change\ \O\ / _\ipa{s}\#\\
\ipa{k} \change\ \O\ / _\ipa{hts}\\
\ipa{t} \change\ \ipa{h} / _\{\ipa{n,t}\}\\
\ipa{t} \change\ \O\ / \ipa{k}_\#\\
\{\ipa{ths,tts}\} \change\ \ipa{tsh}\\
\ipa{a(:) \ae(:)} \change\ \ipa{\~{o}(:) \~{e}(:)} / \ipa{n}_\\
\ipa{a(:)} \change\ \ipa{\~{o}(:)} / V[+nas]H_\\
\ipa{a(:)} \change\ \ipa{\~{o}(:)} / V[+nas](\{\ipa{P,s}\})\ipa{w}_\\
\ipa{\~{e}(:) \~{o}(:)} \change\ \ipa{\~{E}(:) \~{O}(:)}\\
\O\ \change\ \ipa{h} / \ipa{k}_\ipa{n}\\

``Allophonic changes:"\\
--- \ipa{t k(\super w)} \change\ \ipa{d g(\super w)} / _\{V,R\}\\
--- \ipa{s} \change\ \ipa{S} / _\ipa{j}\\
--- \ipa{a e o} \change\ \ipa{@ I U} / C_\{C,\ipa{i}[-long]\}

\paragraph{Proto-Northern Iroquoian to Tuscarora}{\it cedh audmanh}, from Julian, Charles (2010), ``A History of the Iroquoian Languages", University of Manitoba, Winnipeg

\ipa{ts} \change\ \ipa{tS} / _\{\ipa{h,i,j}\}\\
\ipa{ths} \change\ \ipa{tS}\\
\ipa{j} \change\ \O\ / \ipa{tS}_\\
\ipa{t} \change\ \ipa{\super Pt}\\
"V \change\ "V\ipa{:} / _\ipa{n} in ``penultimate syllables only"\\
\ipa{n} \change\ \ipa{t} / ! _\{\ipa{h,k\super w},V[+nas]\}\\
"V \change\ "V\ipa{:} / _\{\ipa{k(\super w),(\super P)t}\}\{\ipa{s},R,H\} (``penultimate syllables only")\\
"V \change\ "V\ipa{:} / _RR\\
\{\ipa{\~{e}(:),\~{o}(:)}\} \change\ \ipa{\~{@}(:)}\\
\ipa{ts} \change\ \ipa{T}\\
\ipa{\super Pt} \change\ \ipa{tP} / _\ipa{\super Pt}\\
\ipa{\super Pt} \change\ \O\ / _\ipa{t}\\
\ipa{\super Pt} \change\ \ipa{P} / _\ipa{r}\\
\ipa{\super Pt} \change\ \ipa{Pn} / V_\{V,\ipa{w,j}\}\\
\ipa{\super Pt} \change\ \ipa{n} / \{\#,C\}_\{V,\ipa{w,j}\}\\
\ipa{\super Pt} \change\ \ipa{t}\\
\ipa{h} \change\ \O\ / _\ipa{nh}\\
\ipa{h} \change\ \O\ / \#_\ipa{w}\\
\ipa{h} \change\ \O\ / \ipa{k}_\{\ipa{s,ts,tS}\}\\
\ipa{h} \change\ \O\ / V\ipa{:}_\#\\
\ipa{w} \change\ \O\ / _\ipa{jh}\\
\ipa{n} \change\ \ipa{t} / _\ipa{k\super w}\\
\O\ \change\ \ipa{t} / \ipa{hs}_\ipa{r}\\
\O\ \change\ \ipa{j} / \ipa{k}_\ipa{e(:)}\\
\ipa{r} becomes a trill\\
\ipa{e(:)} \change\ \ipa{E(:)}\\
\ipa{\~{@}(:)} \change\ \ipa{\~{1}(:)} / ! ``when both short and stressed"\\
\ipa{a(:) o(:)} \change\ \ipa{O(:) u(:)} / ! ``when both short and unstressed"\\
\ipa{t k(\super w)} \change\ \ipa{d g(\super w)} / _\{V,R\}\\
\ipa{n r w j} \change\ \ipa{\r*{n} \r*{r} \r*{w} \r{\j}} / _\{\#,O\}

\subparagraph{Tuscarora to Western Tuscarora}{\it cedh audmanh}, from Julian, Charles (2010), ``A History of the Iroquoian Languages", University of Manitoba, Winnipeg

\ipa{T} \change\ \ipa{s}\\
\ipa{w} \change\ \O\ / _\ipa{j}\\
\ipa{jP} \change\ \ipa{Pj}\\
\ipa{\r*{r} \r*{w} \r{\j}} \change\ \ipa{s f S}

\subparagraph{Tuscarora to Eastern Tuscarora}{\it cedh audmanh}, from Julian, Charles (2010), ``A History of the Iroquoian Languages", University of Manitoba, Winnipeg

\ipa{r} \change\ \O\ / \ipa{st}_\\
\ipa{\~{@}(:) a(:) o(:)} \change\ \ipa{\~{1}(:) O(:) u(:)} (``in all positions")

%%%%%%%% SIOUAN GOES BELOW THIS LINE %%%%%%%%

\subsection{Proto-Siouan-Iroquoian to Proto-Siouan}{\it Pogostick Man}, from Chafe, Wallace L. (1964), ``Another Look at Siouan and Iroquoian". {\it American Anthropologist New Series}, 66:852 -- 862; and from cedh aumdmanh's Iroquoian changes

\ipa{\~{e} \~{o}} \change\ \ipa{\~{\i} \~{u}}\\
\{\ipa{t,h}\} \change\ \O\ / \ipa{s}_\\
\ipa{h} \change\ \O\ / V_C\\
\ipa{s} \change\ \O\ / \ipa{h}_\\
\ipa{T} \change\ \ipa{r}\\
\ipa{P} \change\ \O\ / V_\\
\ipa{x} \change\ \ipa{\c{c}} / _\{\ipa{i,u}\}\\
Also, apparently /\ipa{t\super j}/ got picked up and added to the phonology somewhere along the line, but the circumstances are unclear

\subsubsection{Proto-Siouan to Catawba}{\it Pogostick Man}, from Wolff, Hans (1950), ``Comparative Siouan I". {\it International Journal of American Linguistics} 16(2):61 -- 66; Wolff, Hans (1950), ``Comparative Siouan II" {\it International Journal of American Linguistics} 16(3):113 -- 121; Wolff, Hans (1950), ``Comparative Siouan III". {\it International Journal of American Linguistics} 16(4):168 -- 178; and cedh audmanh's changes above, which assisted me in deciphering the vintage phonetic transcription scheme

\tab {\it NB: Does not include developments in unstressed non-nasal vowels. L was apparently either /\ipa{\*r}/ or /l/; L\super j (L\super y in the text) was apparently /j/, or maybe /\ipa{L}/. Also, the changes of /p/ before a consonant are unclear, as described within the text. Changes appended with an asterisk are putative; there was a seeming lack of material for this language, so I've attempted to do some tracking work from the examples given in the text.}

\ipa{\c{c} x} \change\ \ipa{x S}\\
\ipa{p} \change\ \{\ipa{p,b,m,w}\} / _C\\
\ipa{p} \change\ \ipa{b} / V_V (*)\\
\ipa{t\super j} \change\ \ipa{S} / in "U\\
\ipa{t\super j} \change\ \ipa{Z} / else\\
\ipa{t} \change\ \O\ / _\ipa{k}, when medial\\
"V[+nas] \change\ V\ipa{n} (*)\\
\O\ \change\ \ipa{P} / C_\# (only sometimes?) (*)

\subsubsection{Proto-Siouan to Dakota}{\it Pogostick Man}, from Wolff, Hans (1950), ``Comparative Siouan I". {\it International Journal of American Linguistics} 16(2):61 -- 66; Wolff, Hans (1950), ``Comparative Siouan II" {\it International Journal of American Linguistics} 16(3):113 -- 121; Wolff, Hans (1950), ``Comparative Siouan III". {\it International Journal of American Linguistics} 16(4):168 -- 178; and cedh audmanh's changes above, which assisted me in deciphering the vintage phonetic transcription scheme

\tab {\it NB: Does not include developments in unstressed non-nasal vowels. L was apparently either /\ipa{\*r}/ or /l/; L\super j (L\super y in the text) was apparently /j/, or maybe /\ipa{L}/. Also, the changes of /p/ before a consonant are unclear, as described within the text.}

\ipa{\c{c} x} \change\ \ipa{x} \{\ipa{S,Z}\}\\
\ipa{p} \change\ \{\ipa{p,b,m,w}\}\\
\ipa{t\super j} \change\ \ipa{S} / in "U\\
\ipa{t\super j} \change\ \ipa{Z} / else\\
\ipa{s} \change\ \ipa{z} / in U[-stress]\\
\ipa{s} \change\ \ipa{z} / V_V\\
\ipa{r} \change\ \ipa{d} / \ipa{s}_\\
\ipa{t\super jr} \change\ \{\ipa{st,St}\}\\
\ipa{r} \change\ \ipa{d} / \ipa{x}_\\
\ipa{mn} \change\ \ipa{m}V$_0$\ipa{n}V$_0$ / \#_\\
\ipa{km} \change\ \ipa{k}V$_0$\ipa{m}V$_0$\\
\ipa{w} \change\ \ipa{p} / _\ipa{t}\\
\ipa{t} \change\ \O\ / _\ipa{k}, when medial\\
\ipa{hk} \change \ipa{tS}\\
\ipa{k} \change\ \O\ / _\ipa{x}"V\\
\ipa{x} \change\ \ipa{P} / "V\ipa{k}_

\paragraph{Dakota to Santee Dakota}{\it Pogostick Man}, from Wolff, Hans (1950), ``Comparative Siouan I". {\it International Journal of American Linguistics} 16(2):61 -- 66; Wolff, Hans (1950), ``Comparative Siouan II" {\it International Journal of American Linguistics} 16(3):113 -- 121; Wolff, Hans (1950), ``Comparative Siouan III". {\it International Journal of American Linguistics} 16(4):168 -- 178; and cedh audmanh's changes above, which assisted me in deciphering the vintage phonetic transcription scheme

\tab {\it NB: Does not include developments in unstressed non-nasal vowels.}

\{\ipa{pr,mt}\} \change\ \ipa{md}\\
\ipa{kr} \change\ \ipa{hd} / \#_\\
\ipa{kr} \change\ \ipa{gj} / medial

\paragraph{Dakota to Teton Dakota}{\it Pogostick Man}, from Wolff, Hans (1950), ``Comparative Siouan I". {\it International Journal of American Linguistics} 16(2):61 -- 66; Wolff, Hans (1950), ``Comparative Siouan II" {\it International Journal of American Linguistics} 16(3):113 -- 121; Wolff, Hans (1950), ``Comparative Siouan III". {\it International Journal of American Linguistics} 16(4):168 -- 178; and cedh audmanh's changes above, which assisted me in deciphering the vintage phonetic transcription scheme

\tab {\it NB: Does not include developments in unstressed non-nasal vowels.}

\{\ipa{pr,mt}\} \ipa{kr} \change\ \ipa{bl gl}

\paragraph{Dakota to Yankton Dakota}{\it Pogostick Man}, from Wolff, Hans (1950), ``Comparative Siouan I". {\it International Journal of American Linguistics} 16(2):61 -- 66; Wolff, Hans (1950), ``Comparative Siouan II" {\it International Journal of American Linguistics} 16(3):113 -- 121; Wolff, Hans (1950), ``Comparative Siouan III". {\it International Journal of American Linguistics} 16(4):168 -- 178; and cedh audmanh's changes above, which assisted me in deciphering the vintage phonetic transcription scheme

\tab {\it NB: Does not include developments in unstressed non-nasal vowels.}

\ipa{r} \change\ \ipa{d} / \ipa{k}_

\subsubsection{Proto-Siouan to Mandan}{\it Pogostick Man}, from Wolff, Hans (1950), ``Comparative Siouan I". {\it International Journal of American Linguistics} 16(2):61 -- 66; Wolff, Hans (1950), ``Comparative Siouan II" {\it International Journal of American Linguistics} 16(3):113 -- 121; Wolff, Hans (1950), ``Comparative Siouan III". {\it International Journal of American Linguistics} 16(4):168 -- 178; and cedh audmanh's changes above, which assisted me in deciphering the vintage phonetic transcription scheme

\tab {\it NB: Does not include developments in unstressed non-nasal vowels. L was apparently either /\ipa{\*r}/ or /l/; L\super j (L\super y in the text) was apparently /j/, or maybe /\ipa{L}/.}

\ipa{s} \change\ \ipa{S}\\
\ipa{t} \change\ \O\ / _\ipa{s}\\
\ipa{\c{c}} \change\ \ipa{x}\\
\ipa{w} \change\ \ipa{m}\\
L\super j \change\ \ipa{n} / _V[+nas]\\
L \change\ \ipa{r} / _V[-nas]\\
\ipa{\~{a}} \change\ \ipa{a} / in "U (sporadic)\\
``Phonemic vowel length was gained somehow."\\
\ipa{t\super j} \change\ \ipa{s} / _\ipa{P}V\\
C\ipa{P}V$_0$ \change\ CV$_0$\ipa{P}V$_0$\\
\ipa{t\super jr} \change\ \ipa{s}V$_0$\ipa{r}V$_0$\\
\ipa{r} \change\ \O\ / \ipa{k}_\\
\ipa{mn} \change\ \ipa{m}V$_0$\ipa{n}V$_0$ / \#_\\
\ipa{mn} \change\ \ipa{m}V$_0$\ipa{n}V$_0$ / \{C,V\}_\{C,V\}\\
\ipa{sn} \change\ \{\ipa{S}V$_0$\ipa{n}V$_0$,\ipa{s}V$_0$\ipa{r}V$_0$\}\\
\ipa{km} \change\ \ipa{k}V$_0$\ipa{p}V$_0$\\
\ipa{t} \change\ \O\ / _\ipa{k}, when medial\\
\ipa{sk} \change\ \ipa{S} / _"V

\subsubsection{Proto-Siouan to Proto-\v{C}iwere-Winnebago}{\it Pogostick Man}, from Wolff, Hans (1950), ``Comparative Siouan I". {\it International Journal of American Linguistics} 16(2):61 -- 66; Wolff, Hans (1950), ``Comparative Siouan II" {\it International Journal of American Linguistics} 16(3):113 -- 121; Wolff, Hans (1950), ``Comparative Siouan III". {\it International Journal of American Linguistics} 16(4):168 -- 178; and cedh audmanh's changes above, which assisted me in deciphering the vintage phonetic transcription scheme

\tab {\it NB: Does not include developments in unstressed non-nasal vowels. L was apparently either /\ipa{\*r}/ or /l/; L\super j (L\super y in the text) was apparently /j/, or maybe /\ipa{L}/. Also, the changes of /p/ before a consonant are unclear, as described within the text.}

\ipa{\c{c} x} \change\ \ipa{x} \{\ipa{S,Z}\}\\
\ipa{p} \change\ \{\ipa{p,b,m,w}\}\\
\ipa{w} \change\ \O\ / _\ipa{t}\\
\ipa{t} \change\ \ipa{tS} / _"E\\
\ipa{t} \change\ \ipa{dZ} / _E[-stress]\\
\ipa{t\super j} \change\ \ipa{S} / in "U\\
\ipa{t\super j s} \change\ \ipa{Z z} / in U[-stress]\\
\ipa{k} \change\ \ipa{g} / V[+nas]_ ! _\#\\
\ipa{k} \change\ \ipa{g} / _\ipa{P}\\
\ipa{kr} \change\ \ipa{k}V$_0$\ipa{r}V$_0$\\
\ipa{m} \change\ \O\ / _\ipa{n} ! _\ipa{n}\#\\

\paragraph{Proto-\v{C}iwere-Winnebago to \v{C}iwere}{\it Pogostick Man}, from Wolff, Hans (1950), ``Comparative Siouan I". {\it International Journal of American Linguistics} 16(2):61 -- 66; Wolff, Hans (1950), ``Comparative Siouan II" {\it International Journal of American Linguistics} 16(3):113 -- 121; Wolff, Hans (1950), ``Comparative Siouan III". {\it International Journal of American Linguistics} 16(4):168 -- 178; and cedh audmanh's changes above, which assisted me in deciphering the vintage phonetic transcription scheme

\tab {\it NB: Does not include developments in unstressed non-nasal vowels. L was apparently either /\ipa{\*r}/ or /l/; L\super j (L\super y in the text) was apparently /j/, or maybe /\ipa{L}/. Also, the changes of /p/ before a consonant are unclear, as described within the text.}

\ipa{p} \change\ \{\ipa{p,b,m,w}\} / _C\\
\ipa{p} \change\ \ipa{w} / V_V\\
\ipa{g} \change\ \ipa{N}\\
\ipa{k} \change\ \ipa{g} / "V_\\
\ipa{s z} \change\ \ipa{T D} (sporadic)\\
L \change\ \ipa{l}\\
L\super j \change\ \ipa{n} / _V[+nas]\\
L\super j \change\ \ipa{r} / _V[-nas]\\
\ipa{t} \change\ \ipa{tS} / _\ipa{P}\\
\ipa{pr} \change\ \ipa{bl}\\
\ipa{sr} \change\ \{\ipa{Tl,Sl}\}\\
\ipa{t\super jr} \change\ \ipa{S}V$_0$\ipa{r}V$_0$\\
\ipa{kr} \change\ \ipa{gl}\\
\ipa{r} \change\ \ipa{l} / \ipa{x}_\\
\ipa{k} \change\ \ipa{h} / _\ipa{m}\\
\ipa{k} \change\ \O\ / \#\ipa{t}_\\
\ipa{tk} \change\ \ipa{g} / when medial\\
\ipa{x} \change\ \O\ / _\ipa{k}\\
\ipa{x} \change\ \O\ / \ipa{k}_"V

\paragraph{Proto-\v{C}iwere-Winnebago to Winnebago}{\it Pogostick Man}, from Wolff, Hans (1950), ``Comparative Siouan I". {\it International Journal of American Linguistics} 16(2):61 -- 66; Wolff, Hans (1950), ``Comparative Siouan II" {\it International Journal of American Linguistics} 16(3):113 -- 121; Wolff, Hans (1950), ``Comparative Siouan III". {\it International Journal of American Linguistics} 16(4):168 -- 178; and cedh audmanh's changes above, which assisted me in deciphering the vintage phonetic transcription scheme

\tab {\it NB: Does not include developments in unstressed non-nasal vowels. L was apparently either /\ipa{\*r}/ or /l/; L\super j (L\super y in the text) was apparently /j/, or maybe /\ipa{L}/. Also, the changes of /p/ before a consonant are unclear, as described within the text.}

\ipa{p} \change\ \{\ipa{p,b,m,w}\} / _C\\
V \change\ \O\ / _\#\\
\ipa{p} \change\ \ipa{b} / V_V\\
\ipa{t} \change\ \{\ipa{tS,dZ}\}\\
Wolff says that ``Winnebago preserved the intermediate stages of *k reflexes"\\
L \change\ \ipa{r}\\
L\super j \change\ \ipa{n} / _V[+nas]\\
L\super j \change\ \ipa{r} / _V[-nas]\\
\ipa{t\super j} \change\ \ipa{x} / _\ipa{P}\\
\ipa{pr} \{\ipa{sr,xr}\} \ipa{sn km} \change\ \ipa{p}V$_0$\ipa{r}V$_0$ \ipa{S}V$_0$\ipa{r}V$_0$ \ipa{s}V$_0$\ipa{n}V$_0$ \ipa{k}V$_0$\ipa{w}V$_0$\\
\ipa{mt} \change\ \ipa{r}\\
\ipa{tk} \change\ \{\ipa{tSk,tSg}\} / \#_\\
\ipa{t} \change\ \O\ / _\ipa{k}, when medial\\
\ipa{xk} \change\ \ipa{g}

\subsubsection{Proto-Siouan to Proto-Crow-Hidatsa}{\it Pogostick Man}, from Wolff, Hans (1950), ``Comparative Siouan I". {\it International Journal of American Linguistics} 16(2):61 -- 66; Wolff, Hans (1950), ``Comparative Siouan II" {\it International Journal of American Linguistics} 16(3):113 -- 121; Wolff, Hans (1950), ``Comparative Siouan III". {\it International Journal of American Linguistics} 16(4):168 -- 178; and cedh audmanh's changes above, which assisted me in deciphering the vintage phonetic transcription scheme

\tab {\it NB: Does not include developments in unstressed non-nasal vowels. L was apparently either /\ipa{\*r}/ or /l/; L\super j (L\super y in the text) was apparently /j/, or maybe /\ipa{L}/.}

\ipa{s \c{c}} \{\ipa{t\super j,x}\} \change\ \ipa{ts x S}\\
\ipa{m} \change\ \ipa{w} (Crow seems to have gained a phonemic /\ipa{m}/ after this, however)\\
L(\super j) \change\ \ipa{r}\\
\ipa{\~{a} \~{\i} \~{u}} \change\ \ipa{a i u}\\
Phonemic vowel length was gained somehow.\\
\ipa{wt} \change\ \ipa{w}V$_0$\ipa{t}V$_0$\\
\ipa{t} \change\ \O\ / _\ipa{k}, when medial

\paragraph{Proto-Crow-Hidatsa to Crow}{\it Pogostick Man}, from Wolff, Hans (1950), ``Comparative Siouan I". {\it International Journal of American Linguistics} 16(2):61 -- 66; Wolff, Hans (1950), ``Comparative Siouan II" {\it International Journal of American Linguistics} 16(3):113 -- 121; Wolff, Hans (1950), ``Comparative Siouan III". {\it International Journal of American Linguistics} 16(4):168 -- 178; and cedh audmanh's changes above, which assisted me in deciphering the vintage phonetic transcription scheme

\tab {\it NB: Does not include developments in unstressed non-nasal vowels.}

\ipa{t} \change\ \ipa{S} / _E\\
\ipa{t} \change\ \ipa{s} / _V\\
\ipa{k} \change\ \ipa{ts} / _\ipa{i}\\
\ipa{n} \change\ \ipa{r} / ! at word boundaries\\
\ipa{t} \change\ \ipa{S} / _\ipa{P}\\
\ipa{P} \change\ \O\ / _C\\
\ipa{sk} \change\ \ipa{tsk} / _"V\\
\ipa{x} \change\ \O\ / \ipa{k}_"V

\paragraph{Proto-Crow-Hidatsa to Hidatsa}{\it Pogostick Man}, from Wolff, Hans (1950), ``Comparative Siouan I". {\it International Journal of American Linguistics} 16(2):61 -- 66; Wolff, Hans (1950), ``Comparative Siouan II" {\it International Journal of American Linguistics} 16(3):113 -- 121; Wolff, Hans (1950), ``Comparative Siouan III". {\it International Journal of American Linguistics} 16(4):168 -- 178; and cedh audmanh's changes above, which assisted me in deciphering the vintage phonetic transcription scheme

\tab {\it NB: Does not include developments in unstressed non-nasal vowels.}

\ipa{n} \change\ \ipa{r}\\
\ipa{P} \change\ \O\ / C_\\
\ipa{kr} \change\ \ipa{k}V$_0$\ipa{r}V$_0$ / \#_\\
\ipa{r} \change\ \O\ / \{C,V\}\ipa{k}_\{C,V\}\\
\ipa{mn} \change\ \ipa{w} / \{C,V\}_\{C,V\}\\
\ipa{sn} \change\ \ipa{ts}V$_0$\ipa{r}V$_0$\\
\ipa{km} \change\ \ipa{hp}\\
\ipa{sk} \change\ \ipa{tsuk} / _"\ipa{u}\\
\ipa{sk kx} \change\ \ipa{tsk hk} / _"V\\
\ipa{sk} \change\ \ipa{hts} / "V_\\
\O\ \change\ V / \ipa{x}_\ipa{k}

\subsubsection{Proto-Siouan to Proto-Dhegiha}{\it Pogostick Man}, from Wolff, Hans (1950), ``Comparative Siouan I". {\it International Journal of American Linguistics} 16(2):61 -- 66; Wolff, Hans (1950), ``Comparative Siouan II" {\it International Journal of American Linguistics} 16(3):113 -- 121; Wolff, Hans (1950), ``Comparative Siouan III". {\it International Journal of American Linguistics} 16(4):168 -- 178; and cedh audmanh's changes above, which assisted me in deciphering the vintage phonetic transcription scheme

\tab {\it NB: Does not include developments in unstressed non-nasal vowels. L was apparently either /\ipa{\*r}/ or /l/; L\super j (L\super y in the text) was apparently /j/, or maybe /\ipa{L}/. Also, the changes of /p/ before a consonant are unclear, as described within the text.}

\ipa{p} \change\ \{\ipa{p,b,m,w}\} / _C\\
\ipa{w} \change\ \O\ / _\ipa{t}\\
\ipa{t t\super j s} \change\ \ipa{d Z z} / in U[-stress]\\
\ipa{t\super j} \change\ \ipa{S} / in "U\\
\ipa{k} \change\ \ipa{g} / "V_\\
\ipa{s} \change\ \ipa{z} / V_V\\
L L\super j \change\ \{\ipa{D,j}\} \ipa{Z} / _"V\\
L \change\ \ipa{d} / "V_\\
\ipa{w} \change\ \ipa{B} (sporadic, allophonic)\\
\ipa{u} \change\ \ipa{i} / in "U (sporadic)\\
\ipa{k} \change\ \O\ \#\ipa{t}_ (in morphemes)\\
\ipa{tk} \change\ \ipa{g} / when medial

\paragraph{Proto-Dhegiha to Kansa}{\it Pogostick Man}, from Wolff, Hans (1950), ``Comparative Siouan I". {\it International Journal of American Linguistics} 16(2):61 -- 66; Wolff, Hans (1950), ``Comparative Siouan II" {\it International Journal of American Linguistics} 16(3):113 -- 121; Wolff, Hans (1950), ``Comparative Siouan III". {\it International Journal of American Linguistics} 16(4):168 -- 178; and cedh audmanh's changes above, which assisted me in deciphering the vintage phonetic transcription scheme

\tab {\it NB: Does not include developments in unstressed non-nasal vowels. L was apparently either /\ipa{\*r}/ or /l/; L\super j (L\super y in the text) was apparently /j/, or maybe /\ipa{L}/. Also, the changes of /p/ before a consonant are unclear, as described within the text. Changes appended with an asterisk are putative; there was a seeming lack of material for this language, so I've attempted to do some tracking work from the examples given in the text.}

V[+nas] \change\ V[-nas]\ipa{n}\\
\ipa{p} \change\ \{\ipa{p,b,m,w}\} / _C\\
\ipa{w} \change\ \ipa{b} / \#_\\
\ipa{r} L(\super j) \change\ \ipa{d j} / \#_ (*)\\
L \change\ \ipa{bl} / \#_\ipa{e} (*)\\
V \change\ V[+nas] / N_\\
N \change\ S / V_V\\
\ipa{\~{u}} \change\ \ipa{a}N / stressed\\
\ipa{t} \change\ \ipa{ts} / _\ipa{P}\\
\ipa{t\super jr kr} \change\ \ipa{St l}\\
\ipa{mn} \change\ \ipa{bl} / \{C,V\}_\{C,V\}\\
\ipa{mt} \change\ \ipa{d}

\paragraph{Proto-Dhegiha to Omaha-Ponca}{\it Pogostick Man}, from Wolff, Hans (1950), ``Comparative Siouan I". {\it International Journal of American Linguistics} 16(2):61 -- 66; Wolff, Hans (1950), ``Comparative Siouan II" {\it International Journal of American Linguistics} 16(3):113 -- 121; Wolff, Hans (1950), ``Comparative Siouan III". {\it International Journal of American Linguistics} 16(4):168 -- 178; and cedh audmanh's changes above, which assisted me in deciphering the vintage phonetic transcription scheme

\tab {\it NB: Does not include developments in unstressed non-nasal vowels. L was apparently either /\ipa{\*r}/ or /l/; L\super j (L\super y in the text) was apparently /j/, or maybe /\ipa{L}/.}

\ipa{p} \change\ \ipa{b} / V_V\\
\ipa{n} \change\ \ipa{T} / _\{\ipa{\~{a},\~{e},\~{o}}\}\\
\ipa{w} \change\ \ipa{m} / \#_\\
L \change\ \ipa{T} / _"V\\
\ipa{d} \change\ \ipa{n} / "V_\\
L\super j \change\ \{\ipa{T,n}\} / _V[+nas -stress]\\
\ipa{\~{u}} \change\ \ipa{\~{a}} / stressed\\
\ipa{k} \change\ \O\ / _\ipa{P}\\
\ipa{pr sr t\super jr kr} \change\ \ipa{bT sn Sn gT}\\
\ipa{mn} \change\ \ipa{m}V$_0$\ipa{n}V$_0$ / \#_\\
\ipa{mn} \change\ \ipa{bT} / \{C,V\}_\{C,V\}\\
\ipa{mt} \change\ \ipa{n}

\paragraph{Proto-Dhegiha to Osage}{\it Pogostick Man}, from Wolff, Hans (1950), ``Comparative Siouan I". {\it International Journal of American Linguistics} 16(2):61 -- 66; Wolff, Hans (1950), ``Comparative Siouan II" {\it International Journal of American Linguistics} 16(3):113 -- 121; Wolff, Hans (1950), ``Comparative Siouan III". {\it International Journal of American Linguistics} 16(4):168 -- 178; and cedh audmanh's changes above, which assisted me in deciphering the vintage phonetic transcription scheme

\tab {\it NB: Does not include developments in unstressed non-nasal vowels. L was apparently either /\ipa{\*r}/ or /l/; L\super j (L\super y in the text) was apparently /j/, or maybe /\ipa{L}/.}

\ipa{p} \change\ \{\ipa{p,b,m,w}\} / _C\\
\ipa{t} \change\ \ipa{ts} / _E\\
\ipa{s} \change\ \ipa{T}\\
\{\ipa{\c{c},x}\} \change\ \ipa{z} (sporadic)\\
\ipa{n} \change\ \ipa{D} / _\{\ipa{\~{a},\~{e},\~{o}}\}\\
\ipa{w} \change\ \ipa{b} / \#_\\
L\super j \change\ \ipa{D} / _V[+nas -stress]\\
\ipa{P} \change\ \O\ / \ipa{p}_\\
\ipa{t} \change\ \ipa{ts} / _\ipa{P}\\
\ipa{pr sr t\super jr kr} \change\ \ipa{bD sts Sd gD}\\
\ipa{r} \change\ \ipa{D} / \ipa{x}_\\
\ipa{m} \change\ \O\ / \#_\ipa{n}\\
\ipa{mn} \change\ \ipa{bD} / \{C,V\}_\{C,V\}\\
\ipa{sn mt} \change\ \ipa{hn d}\\
\ipa{S} \change\ \ipa{s} / "V_\ipa{k}\\
\ipa{xk} \change\ \ipa{(S)k} / _"V\\
\ipa{xk} \change\ \ipa{g} / "V_\\
\ipa{x} \change\ \ipa{P} / \ipa{k}_"V

\paragraph{Proto-Dhegiha to Quapaw}{\it Pogostick Man}, from Wolff, Hans (1950), ``Comparative Siouan I". {\it International Journal of American Linguistics} 16(2):61 -- 66; Wolff, Hans (1950), ``Comparative Siouan II" {\it International Journal of American Linguistics} 16(3):113 -- 121; Wolff, Hans (1950), ``Comparative Siouan III". {\it International Journal of American Linguistics} 16(4):168 -- 178; and cedh audmanh's changes above, which assisted me in deciphering the vintage phonetic transcription scheme

\tab {\it NB: Does not include developments in unstressed non-nasal vowels. L was apparently either /\ipa{\*r}/ or /l/; L\super j (L\super y in the text) was apparently /j/, or maybe /\ipa{L}/. Also, the changes of /p/ before a consonant are unclear, as described within the text. Changes appended with an asterisk are putative; there was a seeming lack of material for this language, so I've attempted to do some tracking work from the examples given in the text.}

\ipa{p} \change\ \{\ipa{p,b,m,w}\} / _C\\
\ipa{t\super j} \change\ \ipa{S}\\
L \change\ \ipa{d} / \#_\\
\ipa{x} \change\ \ipa{Z} (*)\\
L\super j \change\ \ipa{j} / \#_\ipa{\~{o}} (*)\\
L\super j \change\ \ipa{t} / \#_\ipa{\~{a}} (*)\\
\ipa{\~{a}} \change\ \ipa{\~{o}} (*)\\
\ipa{\~{u}} \change\ \ipa{\~{a}} / stressed (*)\\
\ipa{m} \change\ \O\ / \#_\ipa{n} (in morphemes) (*)

\subsubsection{Proto-Siouan to Proto-Ohio-Valley}{\it Pogostick Man}, from Wolff, Hans (1950), ``Comparative Siouan I". {\it International Journal of American Linguistics} 16(2):61 -- 66; Wolff, Hans (1950), ``Comparative Siouan II" {\it International Journal of American Linguistics} 16(3):113 -- 121; Wolff, Hans (1950), ``Comparative Siouan III". {\it International Journal of American Linguistics} 16(4):168 -- 178; and cedh audmanh's changes above, which assisted me in deciphering the vintage phonetic transcription scheme

\tab {\it NB: Does not include developments in unstressed non-nasal vowels. L was apparently either /\ipa{\*r}/ or /l/; L\super j (L\super y in the text) was apparently /j/, or maybe /\ipa{L}/. Also, the changes of /p/ before a consonant are unclear, as described within the text.}

\ipa{p} \change\ \{\ipa{p,b,m,w}\} / _C\\
\ipa{t\super j} L\super j \change\ \ipa{tS j}\\
\ipa{m} \change\ \O\ / \{C,V\}_\ipa{n}\{C,V\}\\
\{\ipa{w,m}\} \change\ \O\ / _\ipa{t}

\paragraph{Proto-Ohio-Valley to Biloxi}{\it Pogostick Man}, from Wolff, Hans (1950), ``Comparative Siouan I". {\it International Journal of American Linguistics} 16(2):61 -- 66; Wolff, Hans (1950), ``Comparative Siouan II" {\it International Journal of American Linguistics} 16(3):113 -- 121; Wolff, Hans (1950), ``Comparative Siouan III". {\it International Journal of American Linguistics} 16(4):168 -- 178; and cedh audmanh's changes above, which assisted me in deciphering the vintage phonetic transcription scheme

\tab {\it NB: Does not include developments in unstressed non-nasal vowels. L was apparently either /\ipa{\*r}/ or /l/; L\super j (L\super y in the text) was apparently /j/, or maybe /\ipa{L}/. Also, the changes of /p/ before a consonant are unclear, as described within the text.}

\ipa{p} \change\ \{\ipa{p,b,m,w}\} / _C\\
\ipa{p} \change\ \ipa{w} / V_V, apparently as a result of some dissimilation, as this appears to be an allophone of /\ipa{p}/ here, IIUC\\
\ipa{w} \change\ \O\ / \#_ (sporadic)\\
\ipa{m} \change\ \ipa{w} / \#_\\
L \change\ \ipa{d}\\
\ipa{\~{a}} \change\ \ipa{an} (sporadic)\\
\ipa{P} \change\ \O\ / C_\\
\ipa{r} \change\ \{\ipa{d,n}\}\\
\ipa{m} \change\ \O\ / \#_\ipa{n}\\
\ipa{k} \change\ \O\ / \#\ipa{t}_\\
\ipa{k} \change\ \O\ / "V\ipa{s}_\\
\ipa{k} \change\ \O\ / _\ipa{x}"V\\
\ipa{kx} \change\ \ipa{xk} / "V_

\paragraph{Proto-Ohio-Valley to Ofo}{\it Pogostick Man}, from Wolff, Hans (1950), ``Comparative Siouan I". {\it International Journal of American Linguistics} 16(2):61 -- 66; Wolff, Hans (1950), ``Comparative Siouan II" {\it International Journal of American Linguistics} 16(3):113 -- 121; Wolff, Hans (1950), ``Comparative Siouan III". {\it International Journal of American Linguistics} 16(4):168 -- 178; and cedh audmanh's changes above, which assisted me in deciphering the vintage phonetic transcription scheme

\tab {\it NB: Does not include developments in unstressed non-nasal vowels. L was apparently either /\ipa{\*r}/ or /l/; L\super j (L\super y in the text) was apparently /j/, or maybe /\ipa{L}/. Also, the changes of /p/ before a consonant are unclear, as described within the text.}

\ipa{p} \change\ \{\ipa{p,b,m,w}\} / _C\\
\ipa{s} \{\ipa{\c{c},x}\} \change\ \ipa{f s}\\
\ipa{w} \change\ \O\ / \#_ (sporadic)\\
\ipa{m} \change\ \ipa{w} / \#_\\
L \ipa{j} \change\ \ipa{t tS}\\
\ipa{\~{a}} \change\ \{\ipa{an,\~{o}}\}\\
\ipa{P} \change\ \O\ / C_\\
\ipa{sr kr} \change\ \ipa{ft k}V$_0$\ipa{l}V$_0$\\
\ipa{m} \change\ \O\ / \#_\ipa{n}\\
\O\ \change\ V / \ipa{k}_\ipa{m}\\
\ipa{k} \change\ \O\ / \#\ipa{t}_\\
\ipa{sk} \change\ \ipa{f} / "V_\\
\ipa{x} \change\ \ipa{s} / _\ipa{k}\\
\ipa{kx} \change\ \ipa{sk} / "V_\\
\ipa{kx} \change\ \ipa{s}

\paragraph{Proto-Ohio-Valley to Tutelo}{\it Pogostick Man}, from Wolff, Hans (1950), ``Comparative Siouan I". {\it International Journal of American Linguistics} 16(2):61 -- 66; Wolff, Hans (1950), ``Comparative Siouan II" {\it International Journal of American Linguistics} 16(3):113 -- 121; Wolff, Hans (1950), ``Comparative Siouan III". {\it International Journal of American Linguistics} 16(4):168 -- 178; and cedh audmanh's changes above, which assisted me in deciphering the vintage phonetic transcription scheme

\tab {\it NB: Does not include developments in unstressed non-nasal vowels. L was apparently either /\ipa{\*r}/ or /l/; L\super j (L\super y in the text) was apparently /j/, or maybe /\ipa{L}/. Also, the changes of /p/ before a consonant are unclear, as described within the text. Changes appended with an asterisk are putative; there was a seeming lack of material for this language, so I've attempted to do some tracking work from the examples given in the text.}

\ipa{p} \change\ \{\ipa{p,b,m,w}\} / _C\\
L \change\ \ipa{l}\\
\ipa{k} \change\ \ipa{\super Nk} / _\ipa{P}\\
\ipa{P} \change\ \O\ / C_\\
\ipa{mn} \change\ \ipa{m}V$_0$\ipa{n}V$_0$ / \#_ (in morphemes)\\
\ipa{sn} \change\ \ipa{s}V$_0$\ipa{n}V$_0$\\
\O\ \change\ V / \ipa{k}_\ipa{m}\\
\ipa{s} \change\ \ipa{S} / "V_\ipa{k}

\clearpage

\section{Tai-Kadai}

\subsection{Kam-Tai}

\subsubsection{Tai}

\paragraph{Proto-Tai to Ahom}{\it Pogostick Man}, from Li, Fang Kuei (1977). ``A Handbook of Comparative Tai''. {\it Oceanic Linguistics Special Publications} (15), i -- 389

\ipa{w} \change\ \O\ / \ipa{m}_\\
\ipa{\{f,v\} \{m,b\}} \change\ \ipa{p\super h b}\\
\ipa{pl b\{l,r\}} \change\ \ipa{v pj}\\
\ipa{\{l,r\}} \change\ \O\ / \ipa{\{d,\!d\}}_\\
\ipa{d \!d} \change\ \ipa{t d}\\
\ipa{\r{N}} \change\ \ipa{h}\\
N[- voice] \ipa{\r*{l}} \change\ N[+ voice] \ipa{l}\\
\ipa{t\super h} \change\ \O\ / _\ipa{r}\\
\ipa{G} \change\ \ipa{k\super h}\\
\ipa{gl} \change\ \ipa{k(w)}\\
\ipa{x} \change\ \O\ / _\ipa{r}\\
\ipa{r} \change\ \ipa{l} / K_\\
\ipa{k} \change\ \ipa{k\super h} / _\ipa{r}\\
\ipa{x\super w} K\ipa{\super w} \change\ \ipa{\{k\super h,x\}} K(\ipa{\super w})\\
V \change\ V\ipa{:} / _\%

\paragraph{Proto-Tai to Saek}{\it Pogostick Man}, from Li, Fang Kuei (1977). ``A Handbook of Comparative Tai''. {\it Oceanic Linguistics Special Publications} (15), i -- 389

\ipa{\r{\textltailn}} \change\ \ipa{j}\\
N[- voice] \change\ N[+ voice]\\
\ipa{w} \change\ \O\ / \ipa{m}_\\
\ipa{pr b\{l,r\} vr} \change\ \ipa{v bj d}\\
\ipa{\{l,r\}} \change\ \O\ / \ipa{\{n,\!d\}}_\\
\ipa{d \!d} \change\ \ipa{t d}\\
\ipa{t} \change\ \ipa{\{p,t\}} / _\ipa{r}\\
\ipa{l} \change\ \O\ / \ipa{t\super h}_\\
\ipa{d} \change\ \ipa{t} / _\{\ipa{l,r}\}\\
\ipa{\{g,x\}} \change\ \ipa{k\super h}\\
\ipa{k} \change\ \ipa{t} / _\ipa{l}\\
V \change\ V\ipa{:} / _\%

\paragraph{Proto-Tai to Central Tai}{\it Pogostick Man}, from Li, Fang Kuei (1977). ``A Handbook of Comparative Tai''. {\it Oceanic Linguistics Special Publications} (15), i -- 389

\ipa{\!b \!d} \change\ \ipa{b d} / ! _\{\ipa{l,r}\}\\
\ipa{\{l,r\}} \change\ \O\ / \ipa{n,\!d}_\\
\ipa{f} \change\ \ipa{p\super h}\\
\ipa{\{m,w\}} \change\ \ipa{v}\\
\ipa{l} \change\ \O\ / \ipa{p}_\{\ipa{W,e,i}\}\\
\ipa{l} \change\ \ipa{j} / \ipa{p}_\\
\ipa{b\{l,r\}} \change\ \ipa{pj}\\
\ipa{v} \change\ \ipa{b} / _\ipa{r}\\
\ipa{l} \change\ \O\ / \ipa{t}_\\
\ipa{tr} \change\ \ipa{t\super h(r)}_\\
\ipa{d} \change\ \O\ / _\{\ipa{l,r}\}\\
N[- voice] \change\ N[+ voice]\\
\ipa{g x} \change\ \ipa{k k\super h}\\
\ipa{kl kr} \change\ \ipa{\{kj,tS\} k\super hj}\\
\ipa{\{l,r\}} \change\ \O\ / \ipa{N}_\\
\ipa{xr} \change\ \ipa{k\super h\{l,r\}}\\
\ipa{x\super w G\super w} \change\ \ipa{k\super w\super h \{v,w\}}\\
V \change\ V\ipa{:} / _\%\\
\ipa{i@} \change\ \ipa{\u{\i}} / _C\%\\
\ipa{\textsubarch{1}a} \change\ \ipa{a:}\\
\ipa{Ei ei} \change\ \ipa{ai i:}

\subparagraph{Central Tai to Lungchow}{\it Pogostick Man}, from Li, Fang Kuei (1977). ``A Handbook of Comparative Tai''. {\it Oceanic Linguistics Special Publications} (15), i -- 389

\ipa{v} \change\ \ipa{f}\\
\ipa{pr} \change\ \ipa{p\super h} / _\{\ipa{W,e,i}\}\\
\ipa{pr} \change\ \ipa{p\super hj}\\
\ipa{\{r,s,z\} tS\super h dZ} \change\ \ipa{\textbeltl\ S tS}\\
\ipa{\textltailn} \change\ \ipa{j}\\
\ipa{gl} \change\ \ipa{kj}\\
\ipa{1} \change\ \O\ / _\ipa{u}\\
\ipa{1@} \change\ \ipa{\u{1}} / _C\%\\
\ipa{\u{1} \textsubarch{i}o} \change\ \ipa{u @}\\
\ipa{u@} \change\ \ipa{\u{u}}\\
\ipa{\textsubarch{u}1 \textsubarch{u}o} \change\ \ipa{\{1,@\} u}\\
\ipa{uo \textsubarch{1}u} \change\ \ipa{u(:)} u / _C\%\\
\ipa{\textsubarch{1}u} \change\ \ipa{u:}\\
\ipa{o \{\textsubarch{u}O,1O,O\}} \change\ \ipa{u o:}\\
\ipa{e} \change\ \ipa{i} / _C\%\\
\ipa{\textsubarch{i}e} \change\ \ipa{i}\\
\ipa{\{(i)E,\textsubarch{i}E\}} \change\ \ipa{e:}\\
\ipa{uO \textsubarch{1}O} \change\ \ipa{o: 1}\\
\ipa{\{1\textsubarch{o},1\textsubarch{a},1\textsubarch{e}\} i\textsubarch{e}} \change\ \ipa{1: i:}\\
\ipa{\{u\textsubarch{1},u\textsubarch{a},u\textsubarch{e}\}} \change\ \ipa{u:}\\
\ipa{ai} \change\ \ipa{a:i}\\
V \change\ V\ipa{:} / _V\\
\ipa{\textsubarch{u}@i u@i uai \textsubarch{i}@u iau} \change\ \ipa{o:i u:i u:iau o:u}\\
\ipa{a} \change\ \ipa{a:} / _\ipa{u,i}\\
\ipa{@i} \change\ \ipa{ai}\\
\ipa{\textsubarch{1}ai} \change\ \ipa{a:i}\\
\ipa{O 1} \change\ \ipa{o: 1:} / _\ipa{i}\\
\ipa{\textsubarch{u}@i} \change\ \ipa{o:i}\\
\ipa{\{o,@\}} \change\ \ipa{a} / _\ipa{1}\\
\ipa{\{E1,e1\}} \change\ \ipa{a1}\\
\ipa{eu} \change\ \ipa{u:}\\
\ipa{o} \change\ \ipa{a} / _\ipa{u}\\
\ipa{\textsubarch{i}@u} \change\ \ipa{au}\\
\ipa{E i} \change\ \ipa{e: i:} / _\ipa{u}\\
\ipa{\{u@i,uai\} iau} \change\ \ipa{u:i e:u}

\subparagraph{Central Tai to Nung}{\it Pogostick Man}, from Li, Fang Kuei (1977). ``A Handbook of Comparative Tai''. {\it Oceanic Linguistics Special Publications} (15), i -- 389

\ipa{v} \change\ \ipa{f}\\
\ipa{pr} \change\ \ipa{p\super h} / _\{\ipa{W,e,i}\}\\
\ipa{pr} \change\ \ipa{p\super hj}\\
\ipa{tS tS\super h dZ} \change\ \ipa{S ts\super h \{S,tS\}}

\subparagraph{Central Tai to Tay}{\it Pogostick Man}, from Li, Fang Kuei (1977). ``A Handbook of Comparative Tai''. {\it Oceanic Linguistics Special Publications} (15), i -- 389

\ipa{v} \change\ \ipa{f}\\
\ipa{pr} \change\ \ipa{t\super h}\\
\{\ipa{s,z}\} \change\ \{\ipa{x,t\super h}\}\\
\ipa{tS dZ} \change\ \{\ipa{x,t}\} \ipa{tS}\\
\ipa{(P)j} \change\ \ipa{Z}\\
\ipa{g} \change\ \O\ / _\ipa{l}\\
\ipa{G} \change\ \ipa{k\super h}

\subparagraph{Central Tai to Tho}{\it Pogostick Man}, from Li, Fang Kuei (1977). ``A Handbook of Comparative Tai''. {\it Oceanic Linguistics Special Publications} (15), i -- 389

\ipa{pr} \change\ \ipa{t\super h}\\
\ipa{z} \change\ \ipa{\{r,s\}}\\
\ipa{dZ} \change\ \ipa{tS}\\
\ipa{(P)j} \change\ \ipa{Z}

\subparagraph{Central Tai to T'ien-Pao}{\it Pogostick Man}, from Li, Fang Kuei (1977). ``A Handbook of Comparative Tai''. {\it Oceanic Linguistics Special Publications} (15), i -- 389

\ipa{v} \change\ \ipa{f}\\
\ipa{pr} \change\ \ipa{t\super h}\\
\ipa{r} \change\ \ipa{\r*{r}}\\
\ipa{\{s,z\}} \change\ \ipa{t}\\
\ipa{dZ} \change\ \ipa{tS}\\
\ipa{G} \change\ \ipa{w} / _V[+ round]\\
\ipa{G} \change\ \ipa{j}\\
\ipa{i i: u} \change\ \ipa{@ ei o\textsubarch{U}}

\paragraph{Proto-Tai to North Tai}{\it Pogostick Man}, from Li, Fang Kuei (1977). ``A Handbook of Comparative Tai''. {\it Oceanic Linguistics Special Publications} (15), i -- 389

\ipa{\{k\super h,g\}\{l,r\}} \change\ \ipa{tS}\\
\ipa{k\{l,r\}} \change\ \{\ipa{kj,tS}\}\\
\ipa{l} \change\ \O\ / \ipa{p(\super h),b}_E\\
\ipa{l} \change\ \ipa{j} / \ipa{p(\super h),b}_V\\
\ipa{\{l,r\}} \change\ \O\ / \ipa{\!d}_\\
\ipa{\{p\super h,b\} \!b \{t\super h,d\} \!d \{k\super h,g\} \{k\super w\super h,g\}} \change\ \ipa{p b t d k k\super w}\\
\ipa{\r*{m} \r*{n}} \change\ \ipa{m n}\\
\ipa{mw f} \change\ \ipa{f \{f,v,w,h\}}\\
\ipa{t\{l,r\}} \change\ \ipa{\r*{r}}\\
\ipa{n} \change\ \O\ / _\ipa{r}\\
\ipa{z} \change\ \ipa{s}\\
\ipa{\{w,m\}} \change\ \ipa{v}\\
\ipa{\{tS(\super h),dZ\}} \change\ \ipa{S}\\
\ipa{xr x \{x,G\}\super w} \change\ \ipa{\r*{r} h \{w,v,h\}}\\
V \change\ V\ipa{:} / _\%\\
\ipa{i@} \change\ \ipa{\u{\i}} / _C\%\\
\ipa{\u{1}} \change\ \ipa{a} / _K\\
\ipa{uo} \change\ \ipa{O:}\\
\ipa{\{\textsubarch{1}u,\textsubarch{i}E\}} \change\ \ipa{1\textsubarch{@}}\raisebox{-0.6ex}{\textasciitilde}\ipa{1\textsubarch{a}} / _\%\\
\ipa{\textsubarch{u}o \textsubarch{i}o} \change\ \ipa{u o}\\
\ipa{\textsubarch{1}a} \change\ \ipa{1\textsubarch{a}}\raisebox{-0.6ex}{\textasciitilde}\ipa{1\textsubarch{@}}\\
\ipa{\{\textsubarch{u}O,\textsubarch{u}a\}} \change\ \ipa{ua}\raisebox{-0.6ex}{\textasciitilde}\ipa{u@}\\
\ipa{uo} \change\ \ipa{uu} \change\ \ipa{u} / _\%\\
\ipa{\textsubarch{1}O} \change\ \ipa{1a}\raisebox{-0.6ex}{\textasciitilde}\ipa{1@}\\
\ipa{u\textsubarch{1} u\textsubarch{a}} \change\ \ipa{1@ u\textsubarch{O}} \change\ \ipa{1: O:}\\
\ipa{Ei ei} \change\ \ipa{ai @i}

\subparagraph{North Tai to Dioi}{\it Pogostick Man}, from Li, Fang Kuei (1977). ``A Handbook of Comparative Tai''. {\it Oceanic Linguistics Special Publications} (15), i -- 389

\ipa{k} \change\ \ipa{tS} / _E\\
\ipa{kl} \change\ \ipa{D}\\
\ipa{\{r,\r*r}\} \change\ \ipa{D}\\
\ipa{v} \change\ \ipa{w} (possibly a conservation with other languages changing *\ipa{w} to \ipa{v}?)\\
\ipa{N} \change\ \ipa{g}\\
T\v{S} \change\ TS

\subparagraph{North Tai to Po-Ai}{\it Pogostick Man}, from Li, Fang Kuei (1977). ``A Handbook of Comparative Tai''. {\it Oceanic Linguistics Special Publications} (15), i -- 389

\ipa{k} \change\ \ipa{tS} / _E\\
\ipa{\{r,\r*{r}\}} \change\ \ipa{l}\\
\ipa{\!d} \change\ \ipa{n}\\
\ipa{s} \change\ \ipa{\textbeltl}\\
\ipa{1} \change\ \O\ / _\ipa{u}\\
\ipa{\u{1}@} \change\ \ipa{\u{1}} / _C\%\\
\ipa{\u{1}} \change\ \ipa{a} / _K\\
\ipa{\u{1}} \change\ \ipa{@}\\
\ipa{u@} \change\ \ipa{\u{u}}\\
\ipa{O:} \change\ \ipa{o:}\\
\ipa{1\textsubarch{@}}\raisebox{-0.6ex}{\textasciitilde}\ipa{1\textsubarch{a}} \change\ \ipa{1:} / _\%\\
\ipa{1\textsubarch{@}}\raisebox{-0.6ex}{\textasciitilde}\ipa{1\textsubarch{a}} \change\ \ipa{1} / _C\%\\
\ipa{\textsubarch{i}o} \change\ \ipa{o}\\
\ipa{e} \change\ \ipa{o} / _\{\ipa{m,p}\}\%\\
\ipa{e} \change\ \ipa{E} / _C\%\\
\ipa{E iE} \change\ \ipa{e: \u{\i}}\\
\ipa{\textsubarch{i}e} \change\ \ipa{i}\\
\ipa{o} \change\ \ipa{O}\\
C\ipa{\super w@} \change\ C\ipa{O}\\
\ipa{@} \change\ \{\ipa{a,6,A,2}\} ?\\
\ipa{\textsubarch{u}1} \change\ \ipa{O} / \ipa{m}_\\
\ipa{\textsubarch{u}1} \change\ \ipa{1}\\
\ipa{1\textsubarch{a}}\raisebox{-0.6ex}{\textasciitilde}\ipa{1\textsubarch{@}} \change\ \ipa{i:} / \ipa{j}_\\
\ipa{1\textsubarch{a}}\raisebox{-0.6ex}{\textasciitilde}\ipa{1\textsubarch{@}} \change\ \ipa{1:}\\
\ipa{ua}\raisebox{-0.6ex}{\textasciitilde}\ipa{u@} \change\ \ipa{u:}\\
\ipa{O} \change\ \ipa{o:}\\
\ipa{ua}\raisebox{-0.6ex}{\textasciitilde}\textsubarch{u@} \change\ \ipa{u:}\\
\ipa{1O} \change\ \ipa{1} / _C\%\\
\ipa{\textsubarch{1}a}\raisebox{-0.6ex}{\textasciitilde}\ipa{1@} \change\ \ipa{1:}\\
\ipa{u\textsubarch{e}} \change\ \ipa{u} / _C\%\\
\ipa{u\textsubarch{e}} \change\ \ipa{u:}\\
\ipa{1\textsubarch{o} 1\textsubarch{a} i\textsubarch{e} 1\textsubarch{e}} \change\ \ipa{u: a: e: i:}\\
\ipa{1 a} \change\ \ipa{1: a:} / _\ipa{i}\\
\ipa{\{@i,Ei,ei\}} \change\ \ipa{ai}\\
\ipa{\textsubarch{1}ai} \change\ \ipa{1:i}\\
\ipa{\textsubarch{u}@i Oi} \change\ \ipa{(w)i: o:i}\\
\ipa{@} \change\ \ipa{a} / _\ipa{\{u,1\}}\\
\ipa{\{Ei,ei\}} \change\ \ipa{1:}\\
\ipa{\{ou,o1\}} \change\ \ipa{o:}\\
\ipa{E e i} \change\ \ipa{e: a i:} / _\ipa{u}\\
\ipa{\textsubarch{i}@u} \change\ \ipa{u:}\\
\ipa{u@i uai iau} \change\ \ipa{i: o:i e:u}

\subparagraph{North Tai to Wu-Ming}{\it Pogostick Man}, from Li, Fang Kuei (1977). ``A Handbook of Comparative Tai''. {\it Oceanic Linguistics Special Publications} (15), i -- 389

\ipa{\r*{r}} \change\ \ipa{r}\\
\ipa{s} \change\ \ipa{T}\\
\ipa{i} \change\ \ipa{\{i,oi\}} / _\%\\
\ipa{u} \change\ \ipa{a\textsubarch{U}}\\
\ipa{@i} \change\ \ipa{ai}

\paragraph{Proto-Tai to Southwest Tai}{\it Pogostick Man}, from Li, Fang Kuei (1977). ``A Handbook of Comparative Tai''. {\it Oceanic Linguistics Special Publications} (15), i -- 389

\ipa{p} \change\ \ipa{p(\super h)}\\
\ipa{v} \change\ \ipa{f}\\
\ipa{m} \change\ \ipa{w} / ! _\ipa{w}\\
\ipa{w} \change\ \O\ / \ipa{m}_\\
\ipa{l} \change\ \O\ / \{\ipa{p,k(\super h),N}\}_\\
\ipa{pr} \change\ \ipa{t}\\
\ipa{v} \change\ \ipa{b} / _\ipa{r}\\
\ipa{\{l,r\}} \change\ \O\ / \ipa{t,\!d}_\\
\ipa{t\super hl t\super hr} \change\ \ipa{t\super h \r*{r}}\\
\ipa{d} \change\ \O\ / _\{\ipa{l,r}\}\\
\ipa{z r} \change\ \ipa{s \r*{r}}\\
\ipa{xr} \change\ \ipa{h}\\
\ipa{x} \change\ \ipa{k\super h}\\
\ipa{g\super w} \change\ \ipa{k\super w\super h}\\
V \change\ V\ipa{:} / _\%\\
\ipa{1@} \change\ \ipa{1:}\\
\ipa{i@} \change\ \ipa{\u{\i}} / _C\% (not in all languages)\\
\ipa{\textsubarch{1}a} \change\ \ipa{a:}\\
\ipa{Ei ei} \change\ \ipa{ai i:}\\
\ipa{o} \change\ \ipa{O:} / _\ipa{i}

\subparagraph{Southwest Tai to Lao}{\it Pogostick Man}, from Li, Fang Kuei (1977). ``A Handbook of Comparative Tai''. {\it Oceanic Linguistics Special Publications} (15), i -- 389

\ipa{\!b \!d} \change\ \ipa{b d}\\
\ipa{bl br} \change\ \ipa{p p\super h}\\
\ipa{\r*{r}} \change\ \ipa{h}\\
\ipa{dZ} \change\ \ipa{s}\\
\ipa{g\{l,r\}} \change\ \ipa{k\super h}\\
\ipa{g G} \change\ \ipa{k\super h g}\\
\ipa{x\super w G\super w} \change\ \ipa{k\super w g\super w}

\subparagraph{Southwest Tai to L\"{u}}{\it Pogostick Man}, from Li, Fang Kuei (1977). ``A Handbook of Comparative Tai''. {\it Oceanic Linguistics Special Publications} (15), i -- 389

\ipa{\!b \!d} \change\ \ipa{b d}\\
\ipa{bl br} \change\ \ipa{p p\super h}\\
\ipa{\r*{r}} \change\ \ipa{\r*{r}} / ``literary''\\
\ipa{\r*{r}} \change\ \ipa{h}\\
\ipa{dZ} \change\ \ipa{s}\\
\ipa{\textltailn} \change\ \ipa{j}\\
\ipa{P} \change\ \O\ / _\ipa{j}\\
\ipa{gl gr} \change\ \ipa{k k\super h}\\
\ipa{\{k\super h,G\} g} \change\ \ipa{x k}\\
\ipa{k\super w\super h G\super w} \change\ \ipa{x\super w x(\super w)}\\
\ipa{o} \change\ \ipa{u} / _N\\
\ipa{e} \change\ \ipa{i} / _N\%\\
\ipa{\{u\textsubarch{1},u\textsubarch{a},u\textsubarch{e}\} \{1\textsubarch{a},1\textsubarch{e}\} i\textsubarch{e}} \change\ \ipa{o @ e}\\
\ipa{@i} \change\ \ipa{ai}

\subparagraph{Southwest Tai to Shan}{\it Pogostick Man}, from Li, Fang Kuei (1977). ``A Handbook of Comparative Tai''. {\it Oceanic Linguistics Special Publications} (15), i -- 389

\ipa{\!b \!d} \change\ \ipa{b l}\\
\ipa{bl br} \change\ \ipa{p p\super h}\\
\ipa{\r*{r}} \change\ \ipa{h}\\
\ipa{ts} \change\ \ipa{s\super j}\\
\ipa{dZ} \change\ \ipa{s}\\
\ipa{gl gr} \change\ \ipa{k k\super h}\\
\ipa{g G} \change\ \ipa{k k\super h}\\
\ipa{x\super w} \change\ \ipa{k\super w}\\
\ipa{\{u\textsubarch{1},u\textsubarch{a},u\textsubarch{e}\} \{1\textsubarch{a},1\textsubarch{e}\} i\textsubarch{e}} \change\ \ipa{o @ e}\\
\ipa{@i} \change\ \ipa{ai}

\subparagraph{Southwest Tai to Siamese}{\it Pogostick Man}, from Li, Fang Kuei (1977). ``A Handbook of Comparative Tai''. {\it Oceanic Linguistics Special Publications} (15), i -- 389

\ipa{b} \change\ \ipa{p\super h} / _\{\ipa{l,r}\}\\
\ipa{\!b \!d} \change\ \ipa{m d}\\
\ipa{\r*{r}} \change\ \ipa{h}\\
\ipa{N g G} \change\ \ipa{\{h,S\} k\super h g}\\
\ipa{x\super w G\super w} \change\ \ipa{k\super w g\super w}\\
\ipa{u\{o,@\} 1u} \change\ \ipa{u: 1:}\\
\ipa{\{\textsubarch{u}o,\textsubarch{u}1\} \textsubarch{u}O} \change\ \ipa{o O:}\\
\ipa{\textsubarch{u}a} \change\ \ipa{a:}\\
\ipa{\textsubarch{1}u \textsubarch{i}o} \change\ \ipa{u: u}\\
\ipa{e} \change\ \ipa{o} / _\{\ipa{m,p}\}\%\\
\ipa{\textsubarch{i}e} \change\ \ipa{e}\\
\ipa{\{(i)E,\textsubarch{i}E\}} \change\ \ipa{E:}\\
\ipa{\{1O,\textsubarch{1}O\}} \change\ \ipa{O:}\\
\ipa{a} \change\ \ipa{a:} / _\ipa{i}\\
\ipa{\textsubarch{1}ai} \change\ \ipa{a:i}\\
\ipa{\{\textsubarch{u}@i,Oi\}} \change\ \ipa{O:i}\\
\ipa{@1} \change\ \ipa{ai}\\
\ipa{\{E1,e1\}} \change\ \ipa{ai}\\
\ipa{o1} \change\ \ipa{ai}\\
\ipa{\{o,@\} E} \change\ \ipa{a E:} / _\ipa{u}\\
\ipa{eu} \change\ \ipa{u:}\\
\ipa{\textsubarch{1}@u au} \change\ \ipa{au a:u}\\
\ipa{1} \change\ \ipa{@} / _\ipa{i}\\
\ipa{u@i} \change\ \ipa{uai}

\subparagraph{Southwest Tai to Black Tai}{\it Pogostick Man}, from Li, Fang Kuei (1977). ``A Handbook of Comparative Tai''. {\it Oceanic Linguistics Special Publications} (15), i -- 389

\ipa{\!b \!d} \change\ \ipa{b l}\\
\ipa{bl br} \change\ \ipa{p p\super h}\\
\ipa{\r*{r}} \change\ \ipa{h}\\
\ipa{g gl} \change\ \ipa{k tS}\\
\ipa{x\super w G\super w} \change\ \ipa{k\super w g\super w}

\subparagraph{Southwest Tai to White Tai}{\it Pogostick Man}, from Li, Fang Kuei (1977). ``A Handbook of Comparative Tai''. {\it Oceanic Linguistics Special Publications} (15), i -- 389

\ipa{\!b \!d} \change\ \ipa{b d}\\
\ipa{bl br} \change\ \ipa{p p\super h}\\
\ipa{\r*{r}} \change\ \ipa{h}\\
\ipa{\{k\super h,G\} g g\{l,r\}} \change\ \ipa{x k tS}\\
\ipa{k\super w\super h G\super w} \change\ \ipa{x\super w x(\super w)}\\
\ipa{o} \change\ \ipa{u} / _N\\
\ipa{e} \change\ \ipa{i} / _N\%\\
\ipa{\{u\textsubarch{1},u\textsubarch{a},u\textsubarch{e}\} \{1\textsubarch{a},1\textsubarch{e}\} i\textsubarch{e}} \change\ \ipa{o @ e}\\
\ipa{@i} \change\ \ipa{ai}

\section{Tanoan}\tab Proto-Tanoan is reconstructed as having had the following consonantal phonology, at least for phones in initial position:

\begin{center}\begin{tabular}{c | c c c c}
& Bilabial & Alveolar & Velar & Glottal \\ \hline
Nasal & \ipa{m} & \ipa{n} & & \\
Stop & \textipa{p p\super h p' b} & \textipa{t ts t\super h ts\super h t' ts' d dz} & \textipa{k k\super w k\super h k\super w\super h k' k\super w' g g\super w} & \textipa{P} \\
Fricative & & \ipa{s} & & \ipa{h} \\
Glide & \ipa{w} & & & \\ \end{tabular}\end{center}

\tab Only initials are reconstructed here. Vowels are believed to have had nasality and possibly length, though no correspondences are given here for sure. The affricates, as per Hale (1967), appear to have patterned as stops.

\tab (From Hale, Kenneth (1967), ``Toward a Reconstruction of Kiowa-Tanoan Phonology". \textit{International Journal of American Linguistics}, 33.2:112 -- 120; and Wikipedia contributors (2012), ``Tanoan languages". \textit{Wikipedia, the Free Encyclopedia}. \textless\url{http://en.wikipedia.org/w/index.php?title=Tanoan_languages&oldid=496916321}\textgreater)

\subsection{Proto-Tanoan to Jimez}{\it Pogostick Man}, from Wikipedia contributors (2012), ``Tanoan languages". \textit{Wikipedia, the Free Encyclopedia}. \textless\url{http://en.wikipedia.org/w/index.php?title=Tanoan_languages&oldid=496916321}\textgreater, citing Hale, Kenneth (1967), ``Toward a Reconstruction of Kiowa-Tanoan Phonology". \textit{International Journal of American Linguistics}, 33.2:112 -- 120

\ipa{h} \change\ \O\\
\ipa{p b} \change\ \ipa{F M}\\
\ipa{ts dz} \change\ \ipa{s z}\\
\{\ipa{t\super h,ts\super h}\} \change\ \ipa{S}\\
\ipa{s} \change\ \ipa{c}\\
\ipa{ts'} \change\ \ipa{t'}\\
\ipa{d} \change\ \ipa{n} / _V[+nas]\\
\ipa{k(\super w)\super h k\super w(') g g\super w} \change\ \ipa{h g k k\super w}

\subsection{Proto-Tanoan to Kiowa}{\it Pogostick Man}, from Wikipedia contributors (2012), ``Tanoan languages". \textit{Wikipedia, the Free Encyclopedia}. \textless\url{http://en.wikipedia.org/w/index.php?title=Tanoan_languages&oldid=496916321}\textgreater, citing Hale, Kenneth (1967), ``Toward a Reconstruction of Kiowa-Tanoan Phonology". \textit{International Journal of American Linguistics}, 33.2:112 -- 120

\ipa{P} \change\ \O\\
\ipa{ts ts\super h ts' dz} \change\ \ipa{t t\super h t' d}\\
\ipa{w} \change\ \ipa{j}\\
\ipa{k\super w k\super w\super h k\super w' g\super w} \change\ \ipa{k k\super h k' g}

\subsection{Proto-Tanoan to Taos}{\it Pogostick Man}, from Wikipedia contributors (2012), ``Tanoan languages". \textit{Wikipedia, the Free Encyclopedia}. \textless\url{http://en.wikipedia.org/w/index.php?title=Tanoan_languages&oldid=496916321}\textgreater, citing Hale, Kenneth (1967), ``Toward a Reconstruction of Kiowa-Tanoan Phonology". \textit{International Journal of American Linguistics}, 33.2:112 -- 120

\ipa{b} \change\ \ipa{m}\\
\ipa{s} \change\ \ipa{\textbeltl}\\
\ipa{ts ts\super h ts' dz} \change\ \ipa{tS s tS' j}\\
\ipa{d} \change\ \ipa{l} / _V[-nas]\\
\ipa{d} \change\ \ipa{n} / _V[+nas]\\
\ipa{k\super h k\super h\super w g g\super w} \change\ \ipa{x x\super w k w}

\subsection{Proto-Tanoan to Tewa}{\it Pogostick Man}, from Wikipedia contributors (2012), ``Tanoan languages". \textit{Wikipedia, the Free Encyclopedia}. \textless\url{http://en.wikipedia.org/w/index.php?title=Tanoan_languages&oldid=496916321}\textgreater, citing Hale, Kenneth (1967), ``Toward a Reconstruction of Kiowa-Tanoan Phonology". \textit{International Journal of American Linguistics}, 33.2:112 -- 120

\ipa{p\super h t\super h ts\super h} \change\ \ipa{f T s}\\
\ipa{b} \change\ \ipa{m}\\
\ipa{dz} \change\ \{\ipa{j,dZ}\}\\
\ipa{k\super h k\super w\super h g\super w} \change\ \ipa{x x\super w w}

\clearpage

\section{Totozoquean}

The following inventory is from Brown, Beck, Kondrak, Watters, and Wichmann, with laryngeal modality on the vowels assumed to be distinctive (the authors consider it an option but do not explicitly propose it).

\begin{center}\begin{tabular}{c | c c c c c c}
& Bilabial & Coronal & Palatal & Velar & Uvular & Glottal \\ \hline
Nasal & \ipa{m} & \ipa{n} & \ipa{n\super j}\\
Stop & \ipa{p} & \ipa{t} & \ipa{t\super j} & \ipa{k k\super j} & \ipa{q} & \ipa{P}\\
Affricate & & \ipa{ts}\\
Lateral Affricate & & \ipa{t\textbeltl}\\
Fricative & & \ipa{s} & \ipa{S} & \ipa{x} & & \ipa{h}\\
Lateral Fricative & & \ipa{\textbeltl}\\
Resonant & \ipa{w} & \ipa{l} & \ipa{j}
\end{tabular}

\begin{tabular}{c | c c c}
& Front & Central & Back\\ \hline
High & \ipa{i i: \~*i \~*i:} & \ipa{1 1: \~*1 \~*1:} & \ipa{u u: \~*u \~*u:}\\
Mid & \ipa{e e: \~*e \~*e:} & \ipa{@ @: \~*@ \~*@:} & \ipa{o o: \~*o \~*o:}\\
Mid-Low & & & \ipa{O O: \~*O \~*O:}\\
Low & & \ipa{a a: \~*a \~*a:}
\end{tabular}
\end{center}

(From Brown, Cecil H., David Beck, Grzegorz Kondrak, James K. Watters, and S�ren Wichmann, ``Linking proto-Totonacan and proto-Mixe-Zoquean". \textless\url{http://www.ualberta.ca/~dbeck/TzEILNXI.pdf}\textgreater)

\subsection{Proto-Totozoquean to Proto-Mixe-Zoquean}{\it Pogostick Man}, from Brown, Cecil H., David Beck, Grzegorz Kondrak, James K. Watters, and S�ren Wichmann, ``Linking proto-Totonacan and proto-Mixe-Zoquean". \textless\url{http://www.ualberta.ca/~dbeck/TzEILNXI.pdf}\textgreater

\ipa{l} \change\ \ipa{j}\\
\ipa{q} \change\ \ipa{P}\\
\ipa{n\super j t\super j tS k\super j} \change\ \ipa{n t s ts k}\\
\ipa{x \{\textbeltl,t\textbeltl}\} \change\ \ipa{h j}\\
V[+ creaky voice] \change\ V[- creaky voice]\\
\ipa{1 O} \change\ \ipa{@ o}

\subsection{Proto-Totozoquean to Proto-Totonacan}{\it Pogostick Man}, from Brown, Cecil H., David Beck, Grzegorz Kondrak, James K. Watters, and S�ren Wichmann, ``Linking proto-Totonacan and proto-Mixe-Zoquean". \textless\url{http://www.ualberta.ca/~dbeck/TzEILNXI.pdf}\textgreater

\ipa{P} \change\ \O\\
\ipa{n\super j t\super j k\super j k} \change\ \ipa{l tS k q}\\
\ipa{h} \change\ \O\ / ! \#_\\
\ipa{j} \change\ \ipa{t}\\
\ipa{o \~*o} \change\ \ipa{u \~*u}\\
\ipa{\{@,O\} \{\~*@,\~*O\}} \change\ \ipa{a \~*a}\\
\ipa{\{e,1\} \{\~*e \~*1\}} \change\ \ipa{i \~*i}

\clearpage

\section{Trans-New Guinea}\tab Pawley (2012) reconstructs the following inventory for Proto-Trans New Guinea. The use of the terms ``apical" and ``laminal" is his, but the table has been restructured somewhat.

\begin{center}\begin{tabular}{c | c c c c}
& Labial & Apical & Laminal & Velar \\ \hline
Stop & \ipa{p \super mb} & \ipa{t \super nd} & \ipa{c \super\textltailn\textbardotlessj} & \ipa{k \super Ng}\\
Nasal & \ipa{m} & \ipa{n} & & \ipa{N}\\
Fricative & & & \ipa{s}\\
Approximant & \ipa{w} & \ipa{l} & \ipa{j}\end{tabular}\end{center}

\begin{center}\begin{tabular}{c | c c c}
& Front & Central & Back\\ \hline
High & \ipa{i} & & \ipa{u}\\
Mid & \ipa{e} & & \ipa{o}\\
Low & & \ipa{a}\end{tabular}\end{center}

\tab (From Pawley, Andrew (2012). ``How Reconstructible is Proto Trans New Guinea? Problems, Progress, Prospects''. In {\it Languages \& Linguistics in Melanesia} Special Issue I:89 -- 164)

\subsection{Proto-Trans New Guinea to Apal\ipa{1}}{\it Pogostick Man}, from Pawley, Andrew (2012). ``How Reconstructible is Proto Trans New Guinea? Problems, Progress, Prospects''. In {\it Languages \& Linguistics in Melanesia} Special Issue I:89 -- 164

\ipa{t k N} \change\ \{\ipa{l,t}\} \{\ipa{h,k}\} \ipa{n} / \#_\\
\ipa{p k \super nd} \change\ \ipa{B \{h,k\} nj} / V_V\\
\{\ipa{p,t}\} \change\ \O\ / _\#\\
\ipa{e u i} \change\ \ipa{a \{u,1\} \{i,1\}}

\subsection{Proto-Trans New Guinea to Asmat}{\it Pogostick Man}, from Pawley, Andrew (2012). ``How Reconstructible is Proto Trans New Guinea? Problems, Progress, Prospects''. In {\it Languages \& Linguistics in Melanesia} Special Issue I:89 -- 164

\ipa{p} \change\ \ipa{f} / \#_ (?)\\
\ipa{t} \change\ \ipa{s} / \#_\ipa{i}\\
\ipa{k s} \change\ \O\ \{\ipa{t,s}\} / \#_\\
\ipa{\super mb \super Ng} \change\ \ipa{p k} / V_V\\
\ipa{p t nj} \change\ \ipa{\{t,r\} \{r,s,t\} s} / _\#

\subsection{Proto-Trans New Guinea to Binandere}{\it Pogostick Man}, from Pawley, Andrew (2012). ``How Reconstructible is Proto Trans New Guinea? Problems, Progress, Prospects''. In {\it Languages \& Linguistics in Melanesia} Special Issue I:89 -- 164

\ipa{t} \change\ \ipa{j} / \#_\ipa{i}\\
\ipa{N \super Ng} \change\ \O\ \ipa{g} / \#_\\
\ipa{nj} \change\ \ipa{s} / \#_ (?)\\
\ipa{t} \change\ \{\ipa{r,s}\} / V_\ipa{i}\\
\ipa{\super mb \super nd \super Ng nj} \change\ \ipa{\{p,\super mb\} \{\super nd,z\} k z} / V_V\\
\ipa{a} \change\ \{\ipa{a,o}\}

\subsection{Proto-Trans New Guinea to Kaeti}{\it Pogostick Man}, from Pawley, Andrew (2012). ``How Reconstructible is Proto Trans New Guinea? Problems, Progress, Prospects''. In {\it Languages \& Linguistics in Melanesia} Special Issue I:89 -- 164

\{\ipa{p,\super mb}\} \ipa{Ng} \change\ \ipa{b g} / \#_\\
\ipa{\super Ng} \change\ \ipa{g}\\
\ipa{\super nd \super Ng} \change\ \ipa{d k}\\
\ipa{u a} \change\ \ipa{\{u,o,y\} \{a,o\}}

\subsection{Kainantu-Goroka}

\subsubsection{Gorokan}

\paragraph{Proto-Gorokan to Asaro}{\it Pogostick Man}, from Haiman, John (1987), ``Proto-Gorokan Syllable Structure''. {\it Language and Linguistics in Melanesia} 16(1 -- 2):1 -- 22 (Pogostick Man is not sure if it's supposed to be 1985; the Web site says ``1987, for 1985''); Ford, Kevin (1993), ``A Preliminary Comparison of Kamano-Yagaria''. {\it Language and Linguistics in Melanesia} 24(2):191 -- 202; and Lewis, M. Paul, Gary F. Simons, and Charles D. Fennig (eds.) (2014). ``Gorokan''. {\it Ethnologue: Languages of the World}, Seventeenth edition. Dallas, Texas: SIL International. Online version: \textless\url{http://www.ethnologue.com/17/subgroups/gorokan/}\textgreater

N\ipa{l} \change\ \ipa{nd}\\
N \change\ N[+ same POA] / _S\\
\ipa{P}\{\ipa{l,d}\} \ipa{Pg} \change\ \ipa{t k}

\paragraph{Proto-Gorokan to North Fore}{\it Pogostick Man}, from Haiman, John (1987), ``Proto-Gorokan Syllable Structure''. {\it Language and Linguistics in Melanesia} 16(1 -- 2):1 -- 22 (Pogostick Man is not sure if it's supposed to be 1985; the Web site says ``1987, for 1985''); Ford, Kevin (1993), ``A Preliminary Comparison of Kamano-Yagaria''. {\it Language and Linguistics in Melanesia} 24(2):191 -- 202; and Lewis, M. Paul, Gary F. Simons, and Charles D. Fennig (eds.) (2014). ``Gorokan''. {\it Ethnologue: Languages of the World}, Seventeenth edition. Dallas, Texas: SIL International. Online version: \textless\url{http://www.ethnologue.com/17/subgroups/gorokan/}\textgreater

N \change\ \ipa{Nk} / _V\\
N \change\ \ipa{P} / _S[- voice]\\
\ipa{P} \change\ \ipa{n} / before modal suffixes\\
N\ipa{w} N\ipa{m} N\{\ipa{n,j}\} \change\ \ipa{Nk mp nt}

\paragraph{Proto-Gorokan to South Fore}{\it Pogostick Man}, from Haiman, John (1987), ``Proto-Gorokan Syllable Structure''. {\it Language and Linguistics in Melanesia} 16(1 -- 2):1 -- 22 (Pogostick Man is not sure if it's supposed to be 1985; the Web site says ``1987, for 1985''); Ford, Kevin (1993), ``A Preliminary Comparison of Kamano-Yagaria''. {\it Language and Linguistics in Melanesia} 24(2):191 -- 202; and Lewis, M. Paul, Gary F. Simons, and Charles D. Fennig (eds.) (2014). ``Gorokan''. {\it Ethnologue: Languages of the World}, Seventeenth edition. Dallas, Texas: SIL International. Online version: \textless\url{http://www.ethnologue.com/17/subgroups/gorokan/}\textgreater

N \change\ \ipa{P} / _\{V,S[- voice]\}\\
\ipa{P} \change\ \ipa{n} / before modal suffixes\\
N\ipa{w} N\ipa{m} N\{\ipa{n,j}\} \change\ \ipa{Nk mp nt}\\
\ipa{m n} \change\ \ipa{mb nd} / \#_\\
C \change\ \O\ / VN_V

\paragraph{Proto-Gorokan to Gende}{\it Pogostick Man}, from Haiman, John (1987), ``Proto-Gorokan Syllable Structure''. {\it Language and Linguistics in Melanesia} 16(1 -- 2):1 -- 22 (Pogostick Man is not sure if it's supposed to be 1985; the Web site says ``1987, for 1985''); Ford, Kevin (1993), ``A Preliminary Comparison of Kamano-Yagaria''. {\it Language and Linguistics in Melanesia} 24(2):191 -- 202; and Lewis, M. Paul, Gary F. Simons, and Charles D. Fennig (eds.) (2014). ``Gorokan''. {\it Ethnologue: Languages of the World}, Seventeenth edition. Dallas, Texas: SIL International. Online version: \textless\url{http://www.ethnologue.com/17/subgroups/gorokan/}\textgreater

\ipa{r} \change\ \ipa{P} / _O

\paragraph{Proto-Gorokan to Gimi}{\it Pogostick Man}, from Haiman, John (1987), ``Proto-Gorokan Syllable Structure''. {\it Language and Linguistics in Melanesia} 16(1 -- 2):1 -- 22 (Pogostick Man is not sure if it's supposed to be 1985; the Web site says ``1987, for 1985''); Ford, Kevin (1993), ``A Preliminary Comparison of Kamano-Yagaria''. {\it Language and Linguistics in Melanesia} 24(2):191 -- 202; and Lewis, M. Paul, Gary F. Simons, and Charles D. Fennig (eds.) (2014). ``Gorokan''. {\it Ethnologue: Languages of the World}, Seventeenth edition. Dallas, Texas: SIL International. Online version: \textless\url{http://www.ethnologue.com/17/subgroups/gorokan/}\textgreater

\ipa{P} \change\ \O\ / _\#\\
\ipa{Pv Pm Pg Pr} \change\ \ipa{t p k v}

\paragraph{Proto-Gorokan to Hua}{\it Pogostick Man}, from Haiman, John (1987), ``Proto-Gorokan Syllable Structure''. {\it Language and Linguistics in Melanesia} 16(1 -- 2):1 -- 22 (Pogostick Man is not sure if it's supposed to be 1985; the Web site says ``1987, for 1985''); Ford, Kevin (1993), ``A Preliminary Comparison of Kamano-Yagaria''. {\it Language and Linguistics in Melanesia} 24(2):191 -- 202; and Lewis, M. Paul, Gary F. Simons, and Charles D. Fennig (eds.) (2014). ``Gorokan''. {\it Ethnologue: Languages of the World}, Seventeenth edition. Dallas, Texas: SIL International. Online version: \textless\url{http://www.ethnologue.com/17/subgroups/gorokan/}\textgreater

\{N,\ipa{r}\} \change\ \ipa{P} / _\{\#,C\}

\paragraph{Proto-Gorokan to Kamano}{\it Pogostick Man}, from Haiman, John (1987), ``Proto-Gorokan Syllable Structure''. {\it Language and Linguistics in Melanesia} 16(1 -- 2):1 -- 22 (Pogostick Man is not sure if it's supposed to be 1985; the Web site says ``1987, for 1985''); Ford, Kevin (1993), ``A Preliminary Comparison of Kamano-Yagaria''. {\it Language and Linguistics in Melanesia} 24(2):191 -- 202; and Lewis, M. Paul, Gary F. Simons, and Charles D. Fennig (eds.) (2014). ``Gorokan''. {\it Ethnologue: Languages of the World}, Seventeenth edition. Dallas, Texas: SIL International. Online version: \textless\url{http://www.ethnologue.com/17/subgroups/gorokan/}\textgreater

\ipa{r} \change\ \ipa{P} / _\#

\paragraph{Proto-Gorokan to Move}{\it Pogostick Man}, from Haiman, John (1987), ``Proto-Gorokan Syllable Structure''. {\it Language and Linguistics in Melanesia} 16(1 -- 2):1 -- 22 (Pogostick Man is not sure if it's supposed to be 1985; the Web site says ``1987, for 1985''); Ford, Kevin (1993), ``A Preliminary Comparison of Kamano-Yagaria''. {\it Language and Linguistics in Melanesia} 24(2):191 -- 202; and Lewis, M. Paul, Gary F. Simons, and Charles D. Fennig (eds.) (2014). ``Gorokan''. {\it Ethnologue: Languages of the World}, Seventeenth edition. Dallas, Texas: SIL International. Online version: \textless\url{http://www.ethnologue.com/17/subgroups/gorokan/}\textgreater

N \change\ \ipa{P} / _\{\#,C\}\\
\ipa{Pv Pm Pg Pr Ph} \change\ \ipa{p b k t \{s,f\}}\\
V[+ low tone] \change\ \O\ / C_\ipa{h}V[+ high tone]\\
V[+ low tone] \change\ \O\ / C_CV[+ high tone] if both vowels are the same

\paragraph{Proto-Gorokan to Siane}{\it Pogostick Man}, from Haiman, John (1987), ``Proto-Gorokan Syllable Structure''. {\it Language and Linguistics in Melanesia} 16(1 -- 2):1 -- 22 (Pogostick Man is not sure if it's supposed to be 1985; the Web site says ``1987, for 1985''); Ford, Kevin (1993), ``A Preliminary Comparison of Kamano-Yagaria''. {\it Language and Linguistics in Melanesia} 24(2):191 -- 202; and Lewis, M. Paul, Gary F. Simons, and Charles D. Fennig (eds.) (2014). ``Gorokan''. {\it Ethnologue: Languages of the World}, Seventeenth edition. Dallas, Texas: SIL International. Online version: \textless\url{http://www.ethnologue.com/17/subgroups/gorokan/}\textgreater

C \change\ \O\ / _\#\\
\{N,\ipa{r}\} \change\ \ipa{P} / _C\\
\ipa{PNg Pd} \change\ \ipa{Nk t}

\paragraph{Proto-Gorokan to Yagaria}{\it Pogostick Man}, from Haiman, John (1987), ``Proto-Gorokan Syllable Structure''. {\it Language and Linguistics in Melanesia} 16(1 -- 2):1 -- 22 (Pogostick Man is not sure if it's supposed to be 1985; the Web site says ``1987, for 1985''); Ford, Kevin (1993), ``A Preliminary Comparison of Kamano-Yagaria''. {\it Language and Linguistics in Melanesia} 24(2):191 -- 202; and Lewis, M. Paul, Gary F. Simons, and Charles D. Fennig (eds.) (2014). ``Gorokan''. {\it Ethnologue: Languages of the World}, Seventeenth edition. Dallas, Texas: SIL International. Online version: \textless\url{http://www.ethnologue.com/17/subgroups/gorokan/}\textgreater

\ipa{Pv Pm Pg Pr Ph} \change\ \ipa{p b k t \{s,f\}}

\subsubsection{Kainantu}

\paragraph{Proto-Kainantu to Auyana}{\it Pogostick Man}, from Haiman, John (1987), ``Proto-Gorokan Syllable Structure''. {\it Language and Linguistics in Melanesia} 16(1 -- 2):1 -- 22 (Pogostick Man is not sure if it's supposed to be 1985; the Web site says ``1987, for 1985''); Ford, Kevin (1993), ``A Preliminary Comparison of Kamano-Yagaria''. {\it Language and Linguistics in Melanesia} 24(2):191 -- 202; and Lewis, M. Paul, Gary F. Simons, and Charles D. Fennig (eds.) (2014). ``Gorokan''. {\it Ethnologue: Languages of the World}, Seventeenth edition. Dallas, Texas: SIL International. Online version: \textless\url{http://www.ethnologue.com/17/subgroups/gorokan/}\textgreater

N \change\ [+ same POA] / _C\\
N\{\ipa{w,d,r}\} nj \change\ \ipa{Nk nt} (not sure if *\ipa{nj} is supposed to be *\ipa{\textltailn\textbardotlessj})\\
N \change\ \O\ / _\#\\
\ipa{r} \change\ \ipa{P} / _\{N,\#\}\\
\ipa{r} \change\ \O\ / _S[- voice]

\paragraph{Proto-Kainantu to Awa}{\it Pogostick Man}, from Haiman, John (1987), ``Proto-Gorokan Syllable Structure''. {\it Language and Linguistics in Melanesia} 16(1 -- 2):1 -- 22 (Pogostick Man is not sure if it's supposed to be 1985; the Web site says ``1987, for 1985''); Ford, Kevin (1993), ``A Preliminary Comparison of Kamano-Yagaria''. {\it Language and Linguistics in Melanesia} 24(2):191 -- 202; and Lewis, M. Paul, Gary F. Simons, and Charles D. Fennig (eds.) (2014). ``Gorokan''. {\it Ethnologue: Languages of the World}, Seventeenth edition. Dallas, Texas: SIL International. Online version: \textless\url{http://www.ethnologue.com/17/subgroups/gorokan/}\textgreater

N \change\ \ipa{n} / _\{\ipa{v,k,s}\}\\
N \change\ \O\ / _\{\ipa{p,t,}\#\}\\
N\ipa{d} \change\ \ipa{n}\\
\ipa{P} \change\ \O\ / _\{\ipa{p,t}\}\\
\ipa{Pw Pb Pd Pg} \change\ \ipa{m p t k}

\paragraph{Proto-Kainantu to Gadsup}{\it Pogostick Man}, from Haiman, John (1987), ``Proto-Gorokan Syllable Structure''. {\it Language and Linguistics in Melanesia} 16(1 -- 2):1 -- 22 (Pogostick Man is not sure if it's supposed to be 1985; the Web site says ``1987, for 1985''); Ford, Kevin (1993), ``A Preliminary Comparison of Kamano-Yagaria''. {\it Language and Linguistics in Melanesia} 24(2):191 -- 202; and Lewis, M. Paul, Gary F. Simons, and Charles D. Fennig (eds.) (2014). ``Gorokan''. {\it Ethnologue: Languages of the World}, Seventeenth edition. Dallas, Texas: SIL International. Online version: \textless\url{http://www.ethnologue.com/17/subgroups/gorokan/}\textgreater

N \change\ \O\ / _N\\
N \change\ [+ same POA] / _C\\
\ipa{nw nr} \change\ \ipa{mb nd}\\
\{D,Y\} \change\ \O\ / _\ipa{n}\\
YO[+ voice] YO[- voice] \change\ Y t\\
DO[+ voice] DO[- voice] \change\ \ipa{nd nt}

\paragraph{Proto-Kainantu to Usarufa}{\it Pogostick Man}, from Haiman, John (1987), ``Proto-Gorokan Syllable Structure''. {\it Language and Linguistics in Melanesia} 16(1 -- 2):1 -- 22 (Pogostick Man is not sure if it's supposed to be 1985; the Web site says ``1987, for 1985''); Ford, Kevin (1993), ``A Preliminary Comparison of Kamano-Yagaria''. {\it Language and Linguistics in Melanesia} 24(2):191 -- 202; and Lewis, M. Paul, Gary F. Simons, and Charles D. Fennig (eds.) (2014). ``Gorokan''. {\it Ethnologue: Languages of the World}, Seventeenth edition. Dallas, Texas: SIL International. Online version: \textless\url{http://www.ethnologue.com/17/subgroups/gorokan/}\textgreater

N$_1$N$_2$ \change\ N$_2$\ipa{:} ?\\
N \change\ \ipa{P} / _O\\
N \change\ \ipa{n} / _V\\
N\{\ipa{w,r}\} N\ipa{j} \change\ \ipa{Pk Pt}\\
\ipa{r} \change\ \ipa{P} / _C

\subsection{Proto-Trans New Guinea to Kalam}{\it Pogostick Man}, from Pawley, Andrew (2012). ``How Reconstructible is Proto Trans New Guinea? Problems, Progress, Prospects''. In {\it Languages \& Linguistics in Melanesia} Special Issue I:89 -- 164

\ipa{t} \change\ \{\ipa{t},\O\} / _\#\\
\ipa{l} \change\ \ipa{\:r}\\
Frequent insertion of ``epenthetic vowels, often realized as very short [\ipa{1}], but in some contexts as a copy of a neighboring full vowel. In some cases the epenthetic vowels appear to be, historically, reductions of full vowels"

\subsection{Proto-Trans New Guinea to K\^{a}te}{\it Pogostick Man}, from Pawley, Andrew (2012). ``How Reconstructible is Proto Trans New Guinea? Problems, Progress, Prospects''. In {\it Languages \& Linguistics in Melanesia} Special Issue I:89 -- 164

\ipa{k \super mb \super nd} \change\ \ipa{\{k,h\} b \{s,t\}} / \#_\\
\ipa{\super mb \super nd} \change\ \ipa{\{\super mb,p\} s} / V_V\\
\ipa{p k} \change\ \ipa{t P} / _\#\\
\ipa{p} \change\ \ipa{f}\\
\ipa{u a} \change\ \ipa{\{u,O\} \{O,a\}}

\subsection{Proto-Trans New Guinea to Kiwai}{\it Pogostick Man}, from Pawley, Andrew (2012). ``How Reconstructible is Proto Trans New Guinea? Problems, Progress, Prospects''. In {\it Languages \& Linguistics in Melanesia} Special Issue I:89 -- 164

\ipa{t k \super mb} \change\ \ipa{\{s,t\} \{g},\O\ipa{\} \{b,p\}} / \#_\\
\ipa{t \super mb \super nd \{k,\super Ng\} nj} \change\ \ipa{\{r,t\} p \{d,t\} g r} / V_V\\
\ipa{s} \change\ \ipa{\{s,t\}} / \#_ (?)\\
\ipa{u i} \change\ \ipa{\{u,o\} \{i,e\}}

\subsection{Proto-Trans New Guinea to Selepet}{\it Pogostick Man}, from Pawley, Andrew (2012). ``How Reconstructible is Proto Trans New Guinea? Problems, Progress, Prospects''. In {\it Languages \& Linguistics in Melanesia} Special Issue I:89 -- 164

\ipa{\super mb \super nd s} \change\ \ipa{b \{s,t\} \{t,s\}} / \#_\\
\ipa{t \super mb nj s} \change\ \ipa{r \{b,p\} \super nd \{s,d\}} / V_V\\
\ipa{t} \change\ \ipa{t} / _\# (?)\\
\ipa{N} \change\ \{\ipa{m,N}\} / _\#\\
\ipa{\super Ng} \change\ \ipa{g}\\
\ipa{u o a e} \change\ \ipa{\{u,O\} \{o,O\} \{a,O\} \{e,o\}}

\subsection{Proto-Trans New Guinea to Telefol}{\it Pogostick Man}, from Pawley, Andrew (2012). ``How Reconstructible is Proto Trans New Guinea? Problems, Progress, Prospects''. In {\it Languages \& Linguistics in Melanesia} Special Issue I:89 -- 164

\ipa{\{p,\super mb\}} \change\ \ipa{f} / \#_\\
\ipa{s} \change\ \ipa{s} / \#_ (?)\\
\ipa{\super mb \super nd \super Ng} \change\ \ipa{b n k} / V_V

\subsection{Proto-Trans New Guinea to Middle Wahgi}{\it Pogostick Man}, from Pawley, Andrew (2012). ``How Reconstructible is Proto Trans New Guinea? Problems, Progress, Prospects''. In {\it Languages \& Linguistics in Melanesia} Special Issue I:89 -- 164

\ipa{n N} \change\ \ipa{m n} / \#_\\
\ipa{\super Ng} \change\ \ipa{\{\super Ng,N\}} / V_V\\
\ipa{i} \change\ \ipa{\{i,e\}}

\clearpage

\section{Uralic}\tab The following reconstructed phonology for Proto-Uralic is adapted from the Wikipedia:

\begin{center}\begin{tabular}{c | c c c c c c c c}
& Bilabial & Dental & Alv. & Alv.-pal. & Palatal & Postalv. & Velar & (Unk.) \\ \hline
Nasal & \ipa{m} & & \ipa{n n\super j} & & & & \textipa{N} & \\
Plosive & \ipa{p} & & \ipa{t} & & & & \ipa{k} & \\
Fricative & & \textipa{D D\super j} & s & \textipa{C} & \textipa{S} & & & \\
Trill & & & \ipa{r} & & && & \\
Approximant & \ipa{w} & & \ipa{l l\super j} & & & & & \\
Unknown & & & & & & & & x \end{tabular}\end{center}

\begin{center}\begin{tabular}{c | c c}
& Front & Back \\ \hline\
Close & \ipa{i y} & \textipa{W u} \\
Mid & \ipa{e} & \ipa{o} \\
Open & \ipa{\ae} & \textipa{A} \end{tabular}\end{center}

\tab For the series of changes starting with Proto-Uralic to Pre-Finnic and ending with Proto-Finnic to Livonian and in several other of Tropylium's contributions, the following alterations to the stand-in variable list apply.
\begin{itemize}
\item /@/ means that a vowel assimilates to the one that comes before it.
\item /A O U/ assimilate to [\textipa{A o u}] or to [\textipa{\ae\ \o\ y}], with the specific phone dependent upon the vowel harmony.
\item \^{e} \^{o} may have�been semi-rounded forms of /\textipa{I U}/.
\item \super x is a final which geminates the next word's first consonant; if no such consonant begins the word, [\textipa{P:}] or hiatus occurs here.
\item X stands for the second mora of a syllable, be it a consonant, as part of a diphthong, or as a segment of a long vowel.
\end{itemize}
\tab Changes marked with an asterisk are somewhat contentious.

\tab Tropylium wishes to note that his sound changes are subject to change. (Note 2014/06/21: As per a Tumblr post of his, \url{http://tropylium.tumblr.com/post/81916666722/index-diachronica-4-2}, many of the compilations presented here are out of date or erroneous, and he still is updating his page on Finnic, \url{http://www.frathwiki.com/Finnish}.)

\tab (From Wikipedia contributors (2011), \textquotedblleft Proto-Uralic language". \textit{Wikipedia, The Free Encyclopedia}. \textless\url{http://en.wikipedia.org/w/index.php?title=Proto-Uralic_language\&oldid=442512196}\textgreater; the TCL thread and KQ pages proper; and Tropylium.)

\subsection{Proto-Uralic to Pre-Finnic}{\it Tropylium}, from Hakulinen, Lauri (1979), ``Suomen kielen rakenne ja kehitys". {\it Otava}; H\"{a}kkinen, Kaisa (2004), ``Nykysuomen etymologinen sanakirja". {\it WSOY}; Kallio, Petri (2007), ``Kantasuomen konsonanttihistoriaa". {\it M\'{e}moires de la Soci\'{e}t\'{e} Finno-Ougrienne} 253. \textless\url{http://www.sgr.fi/sust/sust253/sust253_kallio.pdf}\textgreater; and from Janhunen, Juha (2007), ``The primary laryngeal in Uralic and beyond". {\it M\'{e}moires de la Soci\'{e}t\'{e} Finno-Ougrienne} 253. \textless\url{http://www.sgr.fi/sust/sust253/sust253_janhunen.pdf}\textgreater

\ipa{N} \change\ \ipa{k} / _\#, in latives\\
\ipa{N} \change\ \ipa{n} / _\# else\\
\ipa{iw ow} \change\ \ipa{y u:} / _(C) (*)\\
x \change\ @ / _C\\
\ipa{m} \change\ \ipa{n} / _\{\ipa{t,ts\super j},\#\}\\
\ipa{7(:?) A: \ae:} \change\ \ipa{A(:?) o: e:} / stressed\\
\ipa{a} \change\ \ipa{\ae} / \{\ipa{a,e,}\^{e}\ipa{,i,y}\}(X)(C)(C)_, when unstressed\\
A\ipa{w} \change\ \ipa{o} / unstressed (possibly analogical)\\
\ipa{i} \change\ \ipa{e} / _C, when unstressed\\
\ipa{iw} \change\ \ipa{u} / unstressed

\subsubsection{Pre-Finnic to Proto-Finnic}{\it Tropylium}, from Hakulinen, Lauri (1979), ``Suomen kielen rakenne ja kehitys". {\it Otava}; H\"{a}kkinen, Kaisa (2004), ``Nykysuomen etymologinen sanakirja". {\it WSOY}; Kallio, Petri (2007), ``Kantasuomen konsonanttihistoriaa". {\it M\'{e}moires de la Soci\'{e}t\'{e} Finno-Ougrienne} 253. \textless\url{http://www.sgr.fi/sust/sust253/sust253_kallio.pdf}\textgreater; and from Janhunen, Juha (2007), ``The primary laryngeal in Uralic and beyond". {\it M\'{e}moires de la Soci\'{e}t\'{e} Finno-Ougrienne} 253. \textless\url{http://www.sgr.fi/sust/sust253/sust253_janhunen.pdf}\textgreater

V \change\ V\ipa{:} / _\#\\
\^{e} \^{o} \change\ \ipa{e o} / _(X)C\ipa{i}\\
\^{e} \change\ \ipa{y} / _(X)CA\\
\^{o} \change\ \ipa{W} \change\ \ipa{i} / _(X)CA\\
\ipa{ej} \change\ \ipa{i} / unstressed\\
\ipa{\ae} \change\ \ipa{e} / _\ipa{j}, unstressed\\
\ipa{A} \change\ \ipa{e} / "\{\ipa{o,u}\}(X)C_\ipa{j}\\
\ipa{A} \change\ \ipa{o} / "\{\ipa{a,e,i}\}(X)C_\ipa{j}\\
\ipa{i}x\ipa{i} \ipa{u}x\ipa{u} \change\ \ipa{\o: o:}\\
x\ipa{i} \change\ @ / else\\
x \change\ \ipa{w} / \{U,O\}_C\\
x \change\ \ipa{j} / \{I,E\}_C\\
U\ipa{N}A \ipa{eNi} \change\ O\ipa{: \o:}\\
\ipa{Ni} \change\ @ / V_\\
\ipa{N} \change\ \ipa{n} / _\ipa{t} (?)\\
\ipa{N} \change\ \ipa{j} / _C\ipa{\super j} (possibly _F instead?)\\
\ipa{N} \change\ \ipa{w} / _\{A,O,U\}\\
\ipa{N} \change\ \ipa{w} / \{O,U\}_\\
\ipa{N} \change\ \ipa{w} / _C ! _\ipa{k}\\
\ipa{N} \change\ \ipa{w} / C_\\
\ipa{uwa} \change\ \ipa{o:} (*)\\
U\ipa{wi ewi} \change\ \ipa{o: \o:}\\
\ipa{i} \change\ \O\ / \ipa{\ae w}_\\ %DOUBLE-CHECK?
\ipa{w}I \change\ \ipa{i}\\
\ipa{ji} \change\ O / \{\ipa{i,e,y}\}\\
\ipa{i} \change\ \O\ / A\ipa{j}_\#\\
\ipa{i} \change\ \O\ / \{\ipa{o,u}\}\ipa{j}_\\
\ipa{j} \change\ \O\ / C_\ipa{i}\{C,\#\}\\
\ipa{yje} \change\ \ipa{\o:} \change\ \ipa{j\o} (?)\\
\ipa{uw ij} \change\ \ipa{ow ej} / _C\\
\ipa{tS ts\super j} \change\ \ipa{S s\super j} / \#_\\
\ipa{D\super j s\super j ts\super j(:) l\super j} \change\ \ipa{D s ts(:) l}\\
\ipa{n\super j} \change\ \ipa{ni} / \#(C)\ipa{i}_V\\
\ipa{n\super j} \change\ \ipa{in} / V_V\\
\ipa{n\super j} \change\ \ipa{n} / else\\
\ipa{n} \change\ \O\ / _\ipa{t:}\\
\ipa{w} \change\ \O\ / \ipa{o}_\ipa{st} (*)\\
\ipa{D} \change\ \ipa{t}\\
\ipa{tS} \change\ \ipa{ts} / _\ipa{k}, in South Estonian\\
\ipa{tk} \change\ \ipa{k} / in Pre-Livonian (?)\\
\ipa{tS tS:} \change\ \ipa{t tS}\\
\ipa{t} \change\ \ipa{ts} / _\ipa{i} ! following a coronal obstruent or ``before a derivational suffix''\\
\ipa{tj} \change\ \ipa{ts} / ! following a coronal obstruent or ``before a derivational suffix"\\
\ipa{S} \change\ \ipa{\:s} \change\ \ipa{x}\\
\ipa{s} \change\ \ipa{x} / _\ipa{l}\\
\ipa{n} \change\ \O\ / _\{\ipa{s,ts}\}\\
\ipa{w} \change\ \ipa{V}

\paragraph{Proto-Finnic to Proto-Finnish}{\it Tropylium}, from Hakulinen, Lauri (1979), ``Suomen kielen rakenne ja kehitys". {\it Otava}; H\"{a}kkinen, Kaisa (2004), ``Nykysuomen etymologinen sanakirja". {\it WSOY}; Kallio, Petri (2007), ``Kantasuomen konsonanttihistoriaa". {\it M\'{e}moires de la Soci\'{e}t\'{e} Finno-Ougrienne} 253. \textless\url{http://www.sgr.fi/sust/sust253/sust253_kallio.pdf}\textgreater; and from Janhunen, Juha (2007), ``The primary laryngeal in Uralic and beyond". {\it M\'{e}moires de la Soci\'{e}t\'{e} Finno-Ougrienne} 253. \textless\url{http://www.sgr.fi/sust/sust253/sust253_janhunen.pdf}\textgreater

\ipa{j w} \change\ \ipa{i} U / V_\{C,\#\}\\
\ipa{i} \change\ \O\ / \{VC,\ipa{ks}\}_ at the ends of a suffix\\
\ipa{e} \change\ \O\ / C[+coronal]_\%\\
\ipa{p: t: ts: k:} \change\ \ipa{p; t; ts; k;}\\
\ipa{p t ts s k} \change\ \ipa{b d s z g}\\
\ipa{b d g} \change\ \ipa{B D G} / ! N_\\
\ipa{G} \change\ \ipa{j}\raisebox{-0.6ex}{\textasciitilde}\ipa{V}\\
\ipa{Bi} \change\ U / _\#\\
\ipa{f} \change\ \ipa{V} / \#_\\
\ipa{V} \change\ \O\ / \#_\{\ipa{o,u,y}\}\\
\ipa{j} \change\ \O\ / _\ipa{i} (*)\\
\O\ \change\ \ipa{V} / \#_\{\ipa{y:,\o:,o:}\}\\
\ipa{oi} \change\ \ipa{o} / unstressed\\
V\ipa{:} \change\ V[-long] / _\ipa{i}\\
\{\ipa{kt,pt}\} \{\ipa{kts,pts}\} \change\ \ipa{t: t:s}\\
\ipa{xk} \change\ \ipa{k:} (even across word boundaries)\\
\ipa{(t(:))sn kx(tx) rn ln} \change\ \ipa{s: x: r: l:}\\
\{\ipa{p,t,k}\}(\{\ipa{p,t,k}\})\ipa{n} \{\ipa{p,t,k}\}(\{\ipa{p,t,k}\})\ipa{m} \change\ \ipa{n: m:}\\
\{\ipa{p,t,k}\} \change\ \O\ / _\ipa{st} \\ %double-check?
\{\ipa{ks,nts}\} \ipa{nt} \change\ \ipa{s t} / _\#\\
\ipa{ts ts; ts:} \change\ \ipa{s T; T:}\\
\{\ipa{z,x(:)}\} \change\ \ipa{h}\\
\ipa{e} \change\ @ / \ipa{h}_ (suffixal)\\
\{\ipa{p,k}\} \change\ \ipa{h} / _\ipa{t}

\subparagraph{Proto-Finnish to Standard Finnish}{\it Tropylium}, from Hakulinen, Lauri (1979), ``Suomen kielen rakenne ja kehitys". {\it Otava}; H\"{a}kkinen, Kaisa (2004), ``Nykysuomen etymologinen sanakirja". {\it WSOY}; Kallio, Petri (2007), ``Kantasuomen konsonanttihistoriaa". {\it M\'{e}moires de la Soci\'{e}t\'{e} Finno-Ougrienne} 253. \textless\url{http://www.sgr.fi/sust/sust253/sust253_kallio.pdf}\textgreater; and from Janhunen, Juha (2007), ``The primary laryngeal in Uralic and beyond". {\it M\'{e}moires de la Soci\'{e}t\'{e} Finno-Ougrienne} 253. \textless\url{http://www.sgr.fi/sust/sust253/sust253_janhunen.pdf}\textgreater

\ipa{mb nd Ng} \change\ \ipa{m: n: N:}\\
\ipa{e: \o: o:} \change\ \ipa{ie y\o\ uo}\\
\ipa{p; t; T; k;} \change\ \ipa{p t T k}\\
\ipa{j} \change\ \ipa{i} / C_, when initial in a suffix\\
V\ipa{h} \change\ \ipa{h}V / \{\ipa{j,v,n,r,l}\}_\# (also some double-metathesis triggered by the condition of being \ipa{m}_ ?)\\
\ipa{sn} \change\ \ipa{ns}\\
V \change\ V\ipa{:} / \ipa{h}_\ipa{h}C\\
V \change\ V\ipa{:} / _\ipa{h}C (sporadic)\\
\{\ipa{k,x}\} \change\ \ipa{\super x} / _\#\\
\ipa{t} \change\ \O\ / \ipa{s}_\ipa{r}\\
\ipa{p} \change\ \ipa{B} \change\ U / _R\\
\ipa{t} \change\ \ipa{z} \change\ U / _\ipa{r}\{A,O\}\\
\ipa{t} \change\ \ipa{z} \change\ @ / _\ipa{r}\{\ipa{i,e}\}\\
\ipa{k} \change\ \ipa{z} \change\ @ / _\ipa{j}\\
\ipa{k} \change\ \ipa{G} \ipa{i} / \{\ipa{i,e}\}_R\{\ipa{i,e}\}\\
\ipa{k} \change\ \ipa{G} \change\ U / \{A,O,U\}_R ! R = \ipa{j}\\
\ipa{B} \change\ \ipa{V}\\
\ipa{V} \change\ \O\ / _UC\\
\ipa{iD} \change\ \ipa{j} / "V_V\\
\ipa{lD rD} \change\ \ipa{l: r:}\\
\ipa{D} \change\ \O\ / ! "V(X)_\\
\ipa{G} \change\ \ipa{j} / C_\ipa{e}\\
\ipa{G} \change\ \ipa{V} / U_U\\
\ipa{G} \change\ \ipa{P} / VV$_0$_V$_0$ ! V$_0$ = U\\
\ipa{G} \change\ \O\ / else\\
\ipa{h} \change\ \O\ / V[-stress](X)_V\\
AO \change\ \{A,O,U\}\ipa{:} / unstressed\\
\ipa{e} \change\ \ipa{i} / A_, when unstressed\\
U\ipa{e} \change\ \ipa{e:} / unstressed\\
VU \change\ V\ipa{:} / _\#\\
\ipa{i}U OU \change\ U\ipa{:} O\ipa{:}\\
\ipa{\ae} \change\ \ipa{a} / \ipa{e}(C\textellipsis)_(C\textellipsis)\ipa{o}\\
\ipa{e} \change\ \ipa{\o} / \#(C\textellipsis)_\ipa{y}\\
\ipa{i} \change\ \ipa{y} / \#(C\textellipsis)\ipa{l}_\ipa{y}\\
\ipa{i} \change\ \ipa{y} / \#(C\textellipsis)_\ipa{v\ae}\\
\ipa{T(:) D} \change\ \ipa{ts d} (this latter does have some {\it highly} sporadic exceptions; additionally, in some dialects these may become \{\ipa{t(:),h}\raisebox{-0.6ex}{\textasciitilde}\ipa{t}\} and \{\ipa{r},\O\}, respectively)

\subparagraph{Standard Finnish to Modern Standard Finnish}{\it Tropylium}, from Hakulinen, Lauri (1979), ``Suomen kielen rakenne ja kehitys". {\it Otava}; H\"{a}kkinen, Kaisa (2004), ``Nykysuomen etymologinen sanakirja". {\it WSOY}; Kallio, Petri (2007), ``Kantasuomen konsonanttihistoriaa". {\it M\'{e}moires de la Soci\'{e}t\'{e} Finno-Ougrienne} 253. \textless\url{http://www.sgr.fi/sust/sust253/sust253_kallio.pdf}\textgreater; and from Janhunen, Juha (2007), ``The primary laryngeal in Uralic and beyond". {\it M\'{e}moires de la Soci\'{e}t\'{e} Finno-Ougrienne} 253. \textless\url{http://www.sgr.fi/sust/sust253/sust253_janhunen.pdf}\textgreater

\ipa{n} \change\ \O\ / _\#\\
\ipa{d} \change\ \O\ / _\ipa{r} ``in inherited vocabulary"\\
V\ipa{a} \change\ V\ipa{:} / unstressed\\
\ipa{ie y7 uo} \change\ \ipa{i: y: u:} / _A

\subsubsection{Proto-Finnic to Livonian}{\it Tropylium}, from Hakulinen, Lauri (1979), ``Suomen kielen rakenne ja kehitys". {\it Otava}; H\"{a}kkinen, Kaisa (2004), ``Nykysuomen etymologinen sanakirja". {\it WSOY}; Kallio, Petri (2007), ``Kantasuomen konsonanttihistoriaa". {\it M\'{e}moires de la Soci\'{e}t\'{e} Finno-Ougrienne} 253. \textless\url{http://www.sgr.fi/sust/sust253/sust253_kallio.pdf}\textgreater; and from Janhunen, Juha (2007), ``The primary laryngeal in Uralic and beyond". {\it M\'{e}moires de la Soci\'{e}t\'{e} Finno-Ougrienne} 253. \textless\url{http://www.sgr.fi/sust/sust253/sust253_janhunen.pdf}\textgreater

\ipa{t ts s}(C) \ipa{n l r} \change\ \ipa{t\super j ts\super j s\super j}(C) \ipa{n\super j l\super j r\super j} / _\ipa{i}\\
\ipa{ts(\super j)} \change\ \ipa{s(\super j)} / ! \ipa{n}_\\
\ipa{e} \change\ \ipa{7} / _C(C)\{\ipa{a,o,u}\}\\
\ipa{\ae} \change\ \ipa{A} / unstressed\\
\ipa{h} \change\ \O\ / \{\#,C\}_\\
V\ipa{n} \change\ V\ipa{:} / _\ipa{s}\\
\ipa{A \ae} \change\ \ipa{\ae e} / _(C\textellipsis)\ipa{i}\\
V \change\ \ipa{@} / unstressed ! V = \ipa{A}\\
\ipa{A} \change\ \ipa{@} / VC(C)\ipa{A}C(C)_\# when unstressed\\
V\ipa{h} \change\ V\ipa{:H} / _C, except maybe ! _\ipa{j} and/or _\ipa{V}\\
LV \change\ VL / \{\#,V,O\}_\\
\ipa{p t(\super j) s(\super j) k} \change\ \ipa{b d(\super j) z(\super j) g} / ! \#_ or adjacent to C[-voice]\\
\O\ \change\ \ipa{P} / (C)V_CV\\
\ipa{@} \change\ \O\ / _\#\\
\ipa{@} \change\ \O\ / VC_CV\\
C \change\ C\ipa{:} / \ipa{P}_V\\
\ipa{dj lj rj gj} \change\ \ipa{d\super j l:\super j r:\super j jg}\\
\ipa{V} \change\ \O\ / \{\ipa{d,z}\}_\\
\ipa{lV rV jV} \change\ \ipa{l: r: j:}\\
\ipa{V} \change\ \O\ / C_\\
VC\ipa{:A} \change\ V\ipa{:}C\ipa{A}\\
C\ipa{:} \change\ C[- long] / ! in verbal forms when V_\ipa{@}\\
\ipa{A: au} \change\ \ipa{O: Ou} (though sometimes \ipa{A:} develops, apparently at least partially due to metathesis?)\\
\ipa{e: \o: o: 7(:)} \change\ \ipa{i:e y:\o\ u:o 1(:)}\\
\ipa{H} \change\ \ipa{P}\\
\ipa{s\super j ts\super j z\super j dz\super j} \change\ \ipa{S\super j tS\super j Z\super j dZ\super j}\\
\ipa{\ae y ey} \change\ \ipa{\ae u eu}\\
\ipa{y \o} \change\ \ipa{i e} / else\\
V \change\ V\ipa{:} / _RC(C)\ipa{A} (includes diphthongs)\\
\ipa{a} \change\ \ipa{a:} / VC_\\
\ipa{e o} \change\ \ipa{e: o:} / _C\ipa{A}\\
\ipa{o} \change\ \ipa{o:} / _\{RC\#,\ipa{i}\}\\
\ipa{e: o:} \change\ \ipa{je wo}\\
\ipa{w} \change\ \ipa{V} / \#_\ipa{o}\\
\ipa{wo} \change\ \ipa{U} / P_\\
\ipa{O(:)} \change\ \ipa{o(:)}

\clearpage

\section{Uto-Aztecan}\tab The Wikipedia provides the following reconstruction for the phonology of Proto-Uto-Aztecan, which here is adapted with slight modifications as to the layout:

\begin{center}\begin{tabular}{c | c c c c c}
& Bilabial & Coronal & Palatal & Velar & Glottal \\ \hline
Nasal & \ipa{m} & \ipa{n} & & \textipa{N} & \\
Plosive & \ipa{p} & \ipa{t} & & \ipa{k k\super w} & \textipa{P} \\
Fricative & & \ipa{s} & & & \\
Affricate & & \ipa{ts} & & & \\
Rhotic & & \ipa{r} & & & \\
Approximant & & & \ipa{j} & \ipa{w} & \end{tabular}\end{center}

\begin{center}\begin{tabular}{c | c c c}
& Front & Central & Back \\ \hline
Close & \ipa{i} & \textipa{1} & \ipa{u} \\
Mid & & & \ipa{o} \\
Open & & \ipa{a} & \end{tabular}\end{center}

\tab Quoth the Wiki, \textquotedblleft *n and *\textipa{N} may have actually been *l and *n, respectively." It should be noted that there exists some discrepancy between this given reconstruction and in that set up for the studies deriving the reconstructions below. Radius Solis includes *h and *l as distinct phonemes as per the source he cited.

\tab For the following Uto-Aztecan changes, V$_u$, V$_s$, and V$_n$ refer to normal (\textquotedblleft unaffecting"), \textquotedblleft suspending", and \textquotedblleft nasalizing" vowels, respectively. According to Radius Solis, \textquotedblleft Reconstructed PUA had three sets of vowels; this book calls them \textquoteleft suspending', \textquoteleft unaltering', and \textquoteleft nasalizing'. The nasalizing vowels likely were actually nasal, but it's uncertain; their existence was deduced only by the sound changes that revolved around them. There's few good guesses yet about the nature of the `suspending' vowels, but their existence is likewise deducible from the sound changes that have been affected by them across a majority of the UA family - more changes than from the nasalizing series, occurring in all UA branches, enough to be pretty certain that it was a reality in PUA."

\tab (From Wikipedia contributors (2011), ``Proto-Uto-Aztecan language''. {\it Wikipedia, the Free Encyclopedia}. \textless\url{http://en.wikipedia.org/w/index.php?title=Proto-Uto-Aztecan_language&oldid=406159488}\textgreater; and from Radius Solis' changes listed on KneeQuickie and in the TCL thread proper)

\subsection{Proto-Uto-Aztecan to Comanche}{\it Radius Solis}, from from Voegelin, Charles F., Florence M. Voegelin, \& Kenneth L. Hale (1962), ``Typological and Comparative Grammar of Uto-Aztecan: I (Phonology)''. {\it International Journal of American Linguistics} 28: Memoir 17

\ipa{p t ts s} \change\ \ipa{v r} \O\ \ipa{h}\\
\O\ \change\ \ipa{h} / V$_u$_\ipa{k}\\
\ipa{s} \change\ \O\ / V$_n$_ \\ %double-check?
\{\ipa{N,l}\} \change\ \ipa{n}\\
\{\ipa{w,j}\} \change\ \O\ / medial

\subsection{Proto-Uto-Aztecan to Hopi}{\it Radius Solis}, from from Voegelin, Charles F., Florence M. Voegelin, \& Kenneth L. Hale (1962), ``Typological and Comparative Grammar of Uto-Aztecan: I (Phonology)''. {\it International Journal of American Linguistics} 28: Memoir 17

\ipa{p} \change\ \ipa{v} / V$_n$_\\
\ipa{k} \change\ \ipa{q} / _V[+low]\\
\ipa{i} \change\ \ipa{j} / \ipa{h}_ ! _\#\\
\ipa{l h} \change\ \ipa{n} \O\ / medially\\
\ipa{w} \change\ \ipa{l} / \{\#,V[+low]\}_V[+low]\\
\ipa{w} \change\ \ipa{N\super w} / \ipa{1}$_n$_\\
\ipa{o} \change\ \ipa{\o}

\subsection{Proto-Uto-Aztecan to Luise\~{n}o}{\it Radius Solis}, from from Voegelin, Charles F., Florence M. Voegelin, \& Kenneth L. Hale (1962), ``Typological and Comparative Grammar of Uto-Aztecan: I (Phonology)''. {\it International Journal of American Linguistics} 28: Memoir 17

\ipa{p} \change\ \ipa{v} / \{V$_n$,\ipa{1}\}_ (the latter ``sometimes'')\\
\ipa{p} \change\ \ipa{v} / ``other conditions not known''\\
\ipa{t} \change\ \ipa{l} / medially\\
\ipa{ts} \change\ \ipa{tS}\\
\ipa{k} \change\ \ipa{q} / \#_V[+low]\\
\ipa{k} \change\ \ipa{q} / \ipa{a}$_n$_\\
``[I]solated other instances of \ipa{k} \change\ \ipa{q} occur with uncertain conditions''\\
\ipa{k} \change\ \ipa{x} / \ipa{a}_\\
\ipa{P} \change\ \O\ / \#_\\
\ipa{s} \change\ \ipa{S} \\
\ipa{l} \change\ \ipa{n} / medially\\
\ipa{o 1} \change\ \ipa{e o}\\
V \change\ \O\ / ``in some final syllables (conditions are unknown and it varies by dialect)''

\subsection{Proto-Uto-Aztecan to Nahuatl}{\it Radius Solis}, from from Voegelin, Charles F., Florence M. Voegelin, \& Kenneth L. Hale (1962), ``Typological and Comparative Grammar of Uto-Aztecan: I (Phonology)''. {\it International Journal of American Linguistics} 28: Memoir 17

\ipa{t} \change\ \ipa{t\textbeltl} / _\{\ipa{a,u}\}\\
\ipa{p} \change\ \O\ / \{\#,V$_s$\}_\\
\ipa{s ts} \change\ \ipa{S tS} / _\ipa{i}\\
\{\ipa{P,h}\} \change\ \O\\
\ipa{N} \change\ \ipa{n}\\
\ipa{m} \change\ \ipa{n} / _\#\\
\ipa{l} \change\ \ipa{n} / \#_\\
\ipa{w} \change\ \O\ / _\ipa{o}\\
\ipa{1 u} \change\ \ipa{e} \{\ipa{i,e}\} ``(all */u/ affected, but conditions for when it became /i/ or /e/ are not known)''\\
``(What happened to PUA */r/ is not known. Nahuatl has no cognates that would have a reflex.)''

\subsection{Numic}

\subsubsection{Mono-Kawaiisu}

\paragraph{Proto-Mono-Kawaiisu to Kawaiisu}{\it Pogostick Man}, from Klein, Sheldon (1959), ``Comparative Mono-Kawaiisu''. {\it International Journal of American Linguistics} 25(4):233 -- 238

Possible development of vowel harmony\\
\ipa{hk\super w hP (h)}S S\ipa{:} \change\ \ipa{w P} S[+ voiced] S[- voiced - long] / V_V\\
\ipa{ts} \change\ \ipa{z} / V_V\\
\ipa{hts} \change\ \ipa{z} / V_\ipa{i}\\
\ipa{h} \change\ \O\ / V_\ipa{ts}V\\
\ipa{h} \change\ \O\ / _\{\ipa{n,s,P}\}\\
\ipa{p} \change\ \ipa{b} / \ipa{m}_\\
*\ipa{n:} became ``an apical nasal with devoiced release''\\
\ipa{j:} \change\ \ipa{j}\\
\ipa{a} \change\ \ipa{o} / P_\\
\ipa{u}V V\ipa{:} \change\ \ipa{u:} V\ipa{:} (not sure if this occurs before or after the previous change)\\
\ipa{k} \change\ \O\ / V_\ipa{w}V

\paragraph{Proto-Mono-Kawaiisu to Mono}{\it Pogostick Man}, from Klein, Sheldon (1959), ``Comparative Mono-Kawaiisu''. {\it International Journal of American Linguistics} 25(4):233 -- 238

\ipa{(h)k} \change\ \ipa{(h)q} / _\ipa{\{o,a\}} ! \ipa{1}_\\
\ipa{kw} \change\ \ipa{q} / _\ipa{a}\\
\ipa{(h)k\super w} \change\ \ipa{hq(\super w)} / _\{\ipa{o,a}\}\\
\ipa{m} \change\ \ipa{h} / _\ipa{p}\\
\ipa{n:} \change\ \ipa{h}\\
\ipa{1} \change\ \ipa{i} / _\ipa{h}\\
\ipa{u}V \change\ \ipa{u(i)}

\subsection{Proto-Uto-Aztecan to Tohono O'odham}{\it Radius Solis}, from from Voegelin, Charles F., Florence M. Voegelin, \& Kenneth L. Hale (1962), ``Typological and Comparative Grammar of Uto-Aztecan: I (Phonology)''. {\it International Journal of American Linguistics} 28: Memoir 17

\ipa{p} \change\ \ipa{w} / \{\#,V$_s$\}_\\
\ipa{t} \change\ \ipa{tS} / _V[+high]\\
\ipa{ts} \change\ \ipa{s} / _\ipa{i}\\
\ipa{k\super w} \change\ \ipa{b}\\
\ipa{h} \change\ \ipa{P} / \#_\\
\ipa{s N} \change\ \ipa{h n}\\
\ipa{n} \change\ \ipa{\textltailn} / _V[+high]\\
\ipa{l} \change\ \ipa{\textltailn} / \#_ ``(in doubt; initial *l occurs in too few cognates to be sure. Apparently PUA initial *l was rare and is of questionable certainty whether it existed at all.)''\\
\ipa{l} \change\ \ipa{\:l}\\
\ipa{\:l} \change\ \ipa{\:d} / _\ipa{a}\\
\ipa{w} \change\ \ipa{g}\\
\ipa{j} \change\ \ipa{dZ} / _V[+high]\\
\ipa{j} \change\ \ipa{d} / _V[+low]\\
V \change\ \O\ / ``when in the first syllable of a bisyllabic morpheme, if after a morpheme boundary in the word (all other first-syllable vowels have non-zero reflexes)''\\
\ipa{i} \change\ \O\ / \ipa{ts}_\#\\
\ipa{l} \change\ \ipa{i} / \{\ipa{p,m,k(\super w),w}\}_\# ``in all dialects, and varies by dialect after other consonants''\\
``What happened to PUA */r/ in O'odham is difficult to say. There are only two known cognates, each showing a different reflex: /\ipa{\:l}/ and /\ipa{\:d}/''

\clearpage

\section{Vasconic}\tab The following phonology for Proto-Basque ({\it not} Proto-Vasconic) is adapted from Egurtzegi (2013), citing Martinet (1974 [1950]: 533), but differs from that given in Tables 4.3 -- 4.6 when accounting for other data in the paper. Capital letters indicate fortis phonemes, and the affricates were fortis as well.

\begin{center}\begin{tabular}{c | c c c c c c}
& Labial & Dental & Alveolar & Palatal & Velar & Glottal\\ \hline
Nasal & & \ipa{n} N\\
Stop & \ipa{p} (P) & \ipa{t} T & & & \ipa{k} K\\
Fricative & \ipa{f} ? & & \ipa{\|]s \textsubsquare{s}} & & & \ipa{h}\\
Affricate & & & \ipa{t\|]s t\textsubsquare{s}}\\
Liquid & & & \ipa{r} R \ipa{l} L\\
Glide & & & & \ipa{j} & \ipa{w}
\end{tabular}\end{center}

\begin{center}\begin{tabular}{c | c c c}
& Front & Central & Back\\ \hline
High & \ipa{i} & & \ipa{u}\\
Mid & \ipa{e} & & \ipa{o}\\
Low & & \ipa{a}\end{tabular}\end{center}

\tab (From Egurtzegi, Ander (2013), ``Phonetics and Phonology", in {\it Basque and Proto-Basque}. \textless\url{https://www.academia.edu/3570162/2013a_-_Basque_and_Proto-Basque_Phonetics_and_Phonology}\textgreater)

\subsection{Proto-Vasconic to Aquitanian}{\it Pogostick Man}, from Egurtzegi, Ander (2013), ``Phonetics and Phonology", in {\it Basque and Proto-Basque}. \textless\url{https://www.academia.edu/3570162/2013a_-_Basque_and_Proto-Basque_Phonetics_and_Phonology}\textgreater; Owstrowski, Manfred, ``History of the Basque Language". \textless\url{http://hisp462.tamu.edu/Classes/603/Lects/BasqueHist.pdf}\textgreater; Wikipedia contributors (2014), ``Aquitanian langauge". {\it Wikipedia, the Free Encyclopedia}. \textless\url{https://en.wikipedia.org/w/index.php?title=Aquitanian_language&oldid=609638407}\textgreater; Wikipedia contributors (2014), "Basque language". {\it Wikipedia, the Free Encyclopedia}. \textless\url{https://en.wikipedia.org/w/index.php?title=Basque_language&oldid=610796497}\textgreater; and Wikipedia contributors (2014), ``Vasconic languages". {\it Wikipedia, the Free Encyclopedia}. \textless\url{http://en.wikipedia.org/w/index.php?title=Vasconic_languages&oldid=607530415}\textgreater

\ipa{\'s} \change\ \{\ipa{s(:),S}\} / _\#\\
\ipa{\'s} \change\ \ipa{s}\\
\ipa{s} \change\ \ipa{S} / \ipa{i}_\#\\
\ipa{ts} may become \ipa{Ss} or \ipa{s:}? The written forms are $\langle$xs$\rangle$ and $\langle$ss$\rangle$\\
S[+ fortis] \change\ S[- voice]\ipa{:} (specifically, the source lists \ipa{t}[+ fortis] \ipa{k}[+ fortis] \change\ \ipa{t(:) k(:)}, both of the tokens with optional length suffixes and *\ipa{a}T\ipa{a} \change\ $\langle$atta$\rangle$, so I'm extrapolating)\\
\ipa{n}[+ fortis] \change\ \ipa{n(:)} / V_V\\
\ipa{n}[- fortis] \ipa{n}[+ fortis] \change\ \{\ipa{n,r}\}(?) \ipa{n}\\
N \change\ [+ same POA] / _S\\
\ipa{r}[+ fortis] \change\ \ipa{R} / _\#\\
\ipa{r}[+ fortis] \change\ \ipa{r}\\
Fortis *L is of uncertain outcome, being written as $\langle$l$\rangle$ or $\langle$ll$\rangle$\\
\ipa{g} \change\ \ipa{k} / \#_ (sometimes?)\\
There seem to have been a few (variant?) forms which possibly show height assimilation in vowels

\subsection{Proto-Basque to Basque}{\it Pogostick Man}, from Egurtzegi, Ander (2013), "Phonetics and Phonology", in {\it Basque and Proto-Basque}. \textless\url{https://www.academia.edu/3570162/2013a_-_Basque_and_Proto-Basque_Phonetics_and_Phonology}\textgreater; Owstrowski, Manfred, ``History of the Basque Language" \textless\url{http://hisp462.tamu.edu/Classes/603/Lects/BasqueHist.pdf}\textgreater; Wikipedia contributors (2014), ``Proto-Basque language". {\it Wikipedia, the Free Encyclopedia}. \textless\url{https://en.wikipedia.org/w/index.php?title=Proto-Basque_language&oldid=605488703}\textgreater; Wikipedia contributors (2014), ``Aquitanian langauge". {\it Wikipedia, the Free Encyclopedia}. \textless\url{https://en.wikipedia.org/w/index.php?title=Aquitanian_language&oldid=609638407}\textgreater; Wikipedia contributors (2014), ``Basque language". {\it Wikipedia, the Free Encyclopedia}. \textless\url{https://en.wikipedia.org/w/index.php?title=Basque_language&oldid=610796497}\textgreater; Wikipedia contributors (2014), ``Iberian language". {\it Wikipedia, the Free Encyclopedia}. \textless\url{https://en.wikipedia.org/w/index.php?title=Iberian_language&oldid=601317949}\textgreater; Wikipedia contributors (2014), ``Basque dialects". {\it Wikipedia, the Free Encyclopedia}. \textless\url{https://en.wikipedia.org/w/index.php?title=Basque_dialects&oldid=595514648}\textgreater; Wikipedia contributors (2014), ``Biscayan dialect". {\it Wikipedia, the Free Encyclopedia}. \textless\url{http://en.wikipedia.org/w/index.php?title=Biscayan_dialect&oldid=613190357}\textgreater; Wikipedia contributors (2014), ``Gipuzkoan dialect". {\it Wikipedia, the Free Encyclopedia}. \textless\url{https://en.wikipedia.org/w/index.php?title=Gipuzkoan_dialect&oldid=606871281}\textgreater; Wikipedia contributors (2014), ``Vasconic languages". {\it Wikipedia, the Free Encyclopedia}. \textless\url{http://en.wikipedia.org/w/index.php?title=Vasconic_languages&oldid=607530415}\textgreater; Wikipedia contributors (2014), ``Navarro-Lapurdian dialect". {\it Wikipedia, the Free Encyclopedia}. \textless\url{http://en.wikipedia.org/w/index.php?title=Navarro-Lapurdian_dialect&oldid=601150726}\textgreater; and Campbell, Lyle, ``Language Isolates and Their History, or, What's Weird, Anyway?". \textless\url{http://www2.hawaii.edu/~lylecamp/CAMPBELL%20BLS%20isolates.pdf}\textgreater

Pre-Proto-Basque may have had some stuff involving reduplication that ended up dropping the first consonant\\
fortis \change\ aspirated / ``in a prominent position" ({\it i.e.}, word-initially?)\\
fortis \change\ [- voice] / else\\
lenis \change\ devoiced / ``in a prominent word-initial position"\\
lenis \change\ voiced fricative (\change\ approximant, at least by the 12th Century?) / unstressed\\
lenis (voiced) \change\ fricative / \{\ipa{l,r,\|]s,\textsubsquare{s}},V\}_\{\ipa{l,r,\|]s,\textsubsquare{s}},V\}\\
--- At least one reconstruction seems to indicate *\ipa{s} and *\ipa{\'s}, which may have been an affricate and /s/. Pretty reliably, *-\ipa{s} tends to turn into -\ipa{t\|]s}, and *-\ipa{\'s} \change\ -\ipa{t\textsubsquare{s}}, probably after the below-mentioned affrication. Beyond that, it's messy. *-\ipa{tso} seems to have become -\ipa{tSo}/-\ipa{tSu}, though.\\
S\ipa{\super h} \change\ F \change\ \ipa{h} (\change\ \O) / \#_\\
S \change\ S[+ voiced] / \#_\\
``[T]wo similar vocalic segments" usually contract, though some dialects (especially Biscayan) seem not to exhibit this\\
V\ipa{n} \change\ \~{V} / _\# (seems to have been reverted in most dialects, except for Souletin)\\
V\ipa{n} \change\ \~{V} / _V (?)\\
V \change\ \~{V} / _N (Souletin, perhaps in other dialects?)\\
\ipa{d} \change\ \ipa{l} / \#_ (except verbs)\\
\ipa{n} \change \ipa{m} / \ipa{u}_V\\
\ipa{n} \change\ \ipa{\textltailn} / \{\ipa{i,I}\}_V\\
\ipa{n} \change\ \ipa{\~{h}} / V_V\\
\ipa{nb} \change\ \ipa{m:} \change\ \ipa{m}\\
N \change\ [+ same POA] / _C\\
\ipa{b} \change\ \ipa{m} / _VN\\
\{\ipa{R,r}\} \change\ \O\ / \#_\\
\ipa{l} \change\ \ipa{R} / V_V\\
\ipa{R} \change\ \ipa{r} / _C
\ipa{r} \change\ \ipa{R} / _\#\\
C \change\ \O\ / \ipa{r}_\\
C\ipa{r} \change\ C\ipa{R} \change\ CV\ipa{R} (perhaps not a sound change {\it per se}, just a historical tendency)\\
\ipa{R} \change\ \O\ / V_V (Souletin)\\
*L (fortis) \change\ \ipa{l} (or *\ipa{lh} \change\ \ipa{l:}, which then lost gemination?)\\
\ipa{D} \change\ some sort of tap distinct from \ipa{R} (Biscayan, Guipuscoan, High Navarrese)\\
\ipa{b} \change\ \O\ / \#_B (a few exceptions, mostly before _\ipa{u})\\
F[+ voiced] \change\ \O\ / V_V (sometimes, usually involving ``compound surnames"?)\\
S[+ voice] \change\ S[- voice] / F[+ sibilant]_\\
\~V \change\ V\ipa{\textltailn} / _V (not Souletin)\\
\~V \change\ V\ipa{n} or a diphthong (not Souletin)\\
\ipa{\~h} \change\ \ipa{h} (not Souletin)\\
\ipa{u \~u} \change\ \ipa{y \~y} / _\ipa{r(p(\super h),B,k(\super h),G,l,\|]s,\textsubsquare{s},S,h}) (Souletin)\\
\ipa{u \~u} \change\ \ipa{y \~y} / _\{\ipa{\textsubsquare{s},t\|]s,t\textsubsquare{s}}\} (but not _\ipa{\|]s}) (Souletin)\\
\ipa{\~o} \change\ \ipa{\~u} (Souletin)\\
\O\ \change\ \ipa{a} / \#_\{\ipa{ra,ro}\} (sporadic)\\
\O\ \change\ \ipa{e} / \#_\ipa{r}\\
\O\ \change\ \ipa{e} / \#_\{\ipa{\|]s,\textsubsquare{s}}\}C\\
\ipa{i} \change\ \ipa{u} / _(C\textellipsis)\ipa{u} (Roncalese)\\
\ipa{i} \change\ \ipa{y} / _(C\textellipsis)\ipa{y} (Souletin)\\
\ipa{e} \change\ \ipa{o} / _(C\textellipsis)\ipa{o} (eastern dialects, Bermeo Biscayan)\\
\ipa{e} \change\ \ipa{o} / \ipa{o}(C\textellipsis)_ (eastern dialects)\\
\ipa{a o e} \change\ \ipa{E u i} / \{\ipa{i,u}\}(C\textellipsis)_ (this [\ipa{E}] is tentatively marked as such; Egurtzegi transcribes it as /\ipa{e}/ but says it's not as close as /\ipa{e}/)\\
\ipa{o} \change\ \ipa{u} / _\ipa{n}\{C,\#\} (Souletin; some raising occurred elsewhere)\\
\ipa{a} \change\ \ipa{e} / _\$\ipa{a} (Biscayan, Alavese, some Guipuscoan)\\
\ipa{o e} \change\ \ipa{u i} / _\$\ipa{a} (raising of *\ipa{o} is less common)\\
\ipa{e} \change\ \ipa{i} / _\{\ipa{n},C[+ sibilant]\} (sporadic)\\
\ipa{e} \change\ \ipa{a} / \{V,C\}_\ipa{r} (``mainly in the western dialects")\\
\ipa{u i} \change\ \ipa{o e} / _\ipa{r}\{C,\#\}\\
``[S]ome variations between /\ipa{a}/ and /\ipa{e}/ or /\ipa{e}/ and /\ipa{i}/" / _\ipa{l}\{C,\#\}\\
\O\ \change\ \ipa{j} / V_\{N,\ipa{\|]s,\textsubsquare{s}}\}S\\
\O\ \change\ \ipa{j} / \ipa{u}_V (eastern dialects)\\
\{\ipa{w,y}\} \change\ \O\ / _\ipa{ja}\\
\O\ \change\ \ipa{m} / \ipa{o}\$_V (Orozko Biscayan)\\
\O\ \change\ V / V\ipa{k}_\# (Zeberio Biscayan)\\
\ipa{e} \change\ \O\ / \#_ (Navarrese, rare)\\
\ipa{e} \change\ \ipa{j} / \#_V (at least a few times?)\\
\ipa{a} \change\ \O\ / _V\\
V \change\ \O\ / V\ipa{j}_\\
\ipa{h} \change\ \O\ (western dialects)\\
\{\ipa{w,B}\}\ipa{h} \change\ \ipa{f}\\
*\ipa{h} may have metathesized given that it's only found in the first two syllables of proto-forms\\
\ipa{h}\textellipsis\ipa{h} \change\ \O\textellipsis\ipa{h} (``affect[s] both the oral /\ipa{h}/ and the nasalized aspiration")\\
*-\ipa{R} \change\ -\ipa{h} stuff in compounds\\
\ipa{l n} \change \ipa{L \textltailn} / E_\\
\{\ipa{R,r}\} \change\ \ipa{L} / \{\ipa{i,j}\}_ (eastern dialects)\\
\ipa{\textsubsquare{s} t\textsubsquare{s}} \change\ \ipa{S tS} / \{E,\ipa{j}\}_ (mostly Biscayan)\\
\ipa{t} \change\ \ipa{c} / \{E,\ipa{j}\}_ (``some areas")\\
\ipa{t} \change\ \ipa{tS} / \{E,\ipa{j}\}_ (partially spread amongst Biscayan and Guipuscoan)\\
\ipa{d D} \change\ \ipa{\textbardotlessj\ J} / \{E,\ipa{j}\}_ ? (``some dialects")\\
\ipa{d D} \change\ \ipa{\textbardotlessj\ J} / \{\ipa{L,\textltailn}\}_ (Guipuscoan, High Navarrese)\\
\{\ipa{g,G}\} \change\ \{\ipa{\textbardotlessj,J}\} / \{E,\ipa{j}\}_\\
\ipa{g} \change\ \ipa{dZ} / \{E,\ipa{j}\}_ (``in some Biscayan areas")\\
\ipa{j} \change\ \ipa{J} \change\ \ipa{j} (northern High Navarrese, most Labourd, some Biscayan)\\
\ipa{j} \change\ \ipa{J} \change\ \ipa{Z} (Souletin, sporadic in northwestern Biscayan)\\
\ipa{j} \change\ \ipa{J} (some Biscayan and Navarrese)\\
\ipa{j} \change\ \ipa{J} \change\ \ipa{\textbardotlessj} (typical of Low Navarrese)\\
\ipa{j} \change\ \ipa{J} \change\ \ipa{Z} \change\ \ipa{dZ} (northwestern Biscayan)\\
\ipa{j} \change\ \ipa{J} \change\ \ipa{Z} \change\ \ipa{S} (Aescoan, Salazarese, Roncalese, most southern High Navarrese)\\
\ipa{j} \change\ \ipa{J} \change\ \ipa{Z} \change\ \ipa{S} \change\ \ipa{x} (Guipuscoan, northwestern High Navarrese, eastern Biscayan)\\
\ipa{j} \change\ \ipa{X} (probably through intermediates like above, Wikipedia doesn't go into particulars of how and where)\\
\ipa{L \textltailn} \change\ \ipa{jl jn} (``common in Low Navarrese, Labourdin, and is even regular in the High Navarrese of Sakana")\\
Vowel syncope:\\
--- V \change\ \O\ / S_\{\ipa{R,l}\} (more common in Roncalese and Salazarese, but also in Navarrese and Aescoan?)\\
--- V \change\ \O\ / C[+ sibilant]_\ipa{R} (Roncalese and Salazarese)\\
--- V \change\ \O\ / \{O,\ipa{R,r}\}_O (Roncalese, Salazarese, Navarrese, Aescoan)\\
\ipa{n}[+ fortis] \change\ \ipa{n}\\
Something about final devoicing of stops and initial stops losing voicing as a result of vowel deletion\\
\ipa{e} \change\ \O\ / \#U\ipa{r}_\\
\ipa{a} \change\ \O\ / V_\# (Guipuscoan; happens because of reanalysis of the definite article)\\
V \change\ \O\ / _\#, in trisyllables\\
\ipa{i} \change\ \O\ / _\#, in disyllables\\
\ipa{u} \{\ipa{o,e}\} \change\ \O\ \ipa{a} / _\#, in disyllables (eastern dialects)\\
``-\ipa{a} or -\ipa{e} from the definite article" is dropped Markina Biscayan and Getxo Biscayan\\
Some vowel metathesis only when vowels are matched in height\\
\ipa{hu hi} \change\ \ipa{U I} / \{\ipa{o,e}\}_ (also happened with /\ipa{a}/ sometimes, but usually such sequences just dropped one vowel)\\
Something about diphthongs occurring where intervocalic /\ipa{n}/ was lost\\
V\ipa{I}C \change\ VC\ipa{\super j}\\
Glide dissimilation if the homorganic vowel was in the following syllable, but usually the glide just deleted\\
\ipa{aI} \change\ \ipa{eI} \change\ \ipa{e} (rare)\\
\ipa{aU} \change\ \ipa{aI} / !_\{\ipa{R,r,\|]s,\textsubsquare{s}}\} (Souletin, Roncalese)\\
\ipa{eU} \change\ \{\ipa{e,egu}\}\\
\ipa{eI} \change\ \ipa{e} / \#_\\
\ipa{oI} \change\ \ipa{uI} (rare)\\
\ipa{eD} (\change\ \ipa{e} ?) \change\ \ipa{j} / \#_V\\
\ipa{e} \change\ \ipa{j} / \#_\ipa{a}\\
\ipa{e} \change\ \O\ / \#_\ipa{e}\\
\ipa{Ua} \change\ \ipa{o} ``especially after a velar stop"\\
\ipa{Ue} \change\ \ipa{e}\\
C[- voice] \change\ C[+ voice] / \{\ipa{l},N\}_ (not Roncalese or Souletin)\\
Some speakers (Labourd and Low Navarrese?) have \ipa{\;R} for \ipa{r}, and a few have \ipa{\u{\;G}} for \ipa{R}\\
\ipa{l} gets a velar(ized?) articulation in Souletin (possibly only in the coda?)\\
Souletin preserves something involving historical aspiration in pretonic position, apparently\\
Souletin keeps initial \ipa{S}- and \ipa{tS}- distinct; Labourdin only has \ipa{S}-, and the rest apparently only have \ipa{tS}-?\\
C[+ sibilant] \change\ C[+ affricate] / _\#\\
\ipa{\textsubsquare{s} t\|]s} \change\ \ipa{\|]s t\textsubsquare{s}} (Biscayan, partially in Guipuscoan, Donostia, San Sebasti\'{a}n, though these latter two may be varieties of Guipuscoan)\\
\ipa{\textsubsquare{s}} \change\ \ipa{\|]s} / _\{C,\#\} (sometimes)\\
From the Wikipedia article on Biscayan: ``Convergence of sibilants: z, x and s \textgreater\ x, s and tz, tx and ts \textgreater\ tz." I'm not sure what this means. $\langle$s z$\rangle$ are apparently \ipa{\textsubsquare{s} \|]s}, and $\langle$x$\rangle$ is \ipa{S}.\\
\ipa{it\|]s} \change\ \ipa{tS} / _\# (Biscayan)\\
\ipa{oa ea} \change\ \ipa{u(e) i(e)} / _\#\\
Beterri Guipuscoan has V\ipa{j}V\# where Biscayan has V\ipa{\textltailn}V\# and regular Basque has VV\#\\
\ipa{\|]s} \change\ \ipa{tS} / \#_ (Guipuscoan)\\
\ipa{\|]s} \change\ \ipa{\|]S} ``for most French Basque speakers (Trask 1997:84), due to French influence" according to Campbell\\
Accentual changes:\\
--- Navarrese and Labourdin seem to have gotten rid of phonemic accent; High Navarrese typically stresses the penult, while Low Navarrese and Labourdin are claimed to lack stress on the word level.\\
--- Guipuscona, southeastern Biscayan, and western varieties of Navarrese stress the second syllable (unless it is a disyllable, in which case the first syllable gets the accent, though a few varieties don't do this).\\
--- North Biscayan does something with roots and affixes marked for prosody; "[m]ost native roots and almost all singular affixes are unaccented"; loans, "compounds and plural affixes" tend to be accented. Stress is typically assigned to the syllable immediately before the accent, but a few areas accent the penult or the antepenult.\\
--- Souletin does its own thing with accent. Stress usually falls on the penult, but contractions can mess with this (one of the examples given in the paper is ``{\it alh\'{a}ba} `daughter' + abs. sg. {\it -a} \textgreater\ {\it alhab\'{a}} `the daughter'''). Something similar is posited for ``older\textellipsis\ Salazarese". Roncalese was much the same, but the stress was stem-oriented as opposed to word-oriented unless contraction occurred, and there's some stuff about proparoxytones that Souletin didn't have.

\clearpage

\section{Yuman-Cochim\'{\i}}

\subsection{Pai}

\subsubsection{Proto-Pai to Chapai}{\it Pogostick Man}, from Wares, Alan C. (1954?), ``Three Pai Dialects of Lower California''. Summer Institute of Linguistics Bartholomew Collection of Unpublished Materials

\ipa{tS} \change\ \ipa{S} / _\{\ipa{w,i}\}\\
\ipa{tS} \change\ \O\ / _\ipa{x\super w}\\
\ipa{tS} \change\ \ipa{s}\\
\ipa{t} \change\ \ipa{tS} / ! \ipa{n}_\\
\ipa{k\super w} \change\ \ipa{k} / _\#\\
\ipa{b} \change\ \ipa{p}\\
\ipa{o} \change\ \ipa{u}\\
\ipa{s} \change\ \ipa{\:s}\\
\ipa{P} \change\ \O\ / _\{\ipa{\textltailn,j}\}\\
\{\ipa{w,j}\} \change\ \O\ / \ipa{a}_\\
V\ipa{:} \change\ V ?\\
Stress lost?

\subsubsection{Proto-Pai to Paipai}{\it Pogostick Man}, from Wares, Alan C. (1954?), ``Three Pai Dialects of Lower California''. Summer Institute of Linguistics Bartholomew Collection of Unpublished Materials

\ipa{b} \change\ \ipa{B}\\
\ipa{x\super w} \change\ \ipa{w} / \ipa{tS}_\\
\ipa{k\super w x\super w} \change\ \ipa{k x} / _\#\\
\ipa{S \textbeltl} \change\ \ipa{\:s l}\\
\ipa{i} \change\ \ipa{@} / unstressed\\
\ipa{n} \change\ \O\ / _\ipa{t}\\
\ipa{P} \change\ \O\ / _\ipa{\textltailn}\\
\ipa{aw aj} \change\ \ipa{o e}

\subsubsection{Proto-Pai to Tipai}{\it Pogostick Man}, from Wares, Alan C. (1954?), ``Three Pai Dialects of Lower California''. Summer Institute of Linguistics Bartholomew Collection of Unpublished Materials

\ipa{k\super w x\super w} \change\ \ipa{q X} / _\# (the paper calls these ``back velars'')\\
\ipa{b} \change\ \ipa{p}\\
\ipa{i} \change\ \ipa{@} / unstressed\\
\ipa{u} \change\ \ipa{o} / _K\\
\ipa{t} \change\ \O\ / \ipa{n}_\\
\ipa{tS} \change\ \O\ / _\ipa{x\super w}\\
\ipa{\textltailn j} \change\ \ipa{n} \O\ / \ipa{P}_\\
V\ipa{:} \change\ V (sporadic? conditioned?)\\
Contrastive stress lost?

\clearpage

\section{Vowel Shifts}\tab A miscellaneous collection of vowel shifts.

\subsection{7-to-5 Vowel Merger (Bantu)}\textit{Pogostick Man}, from Schadeberg, Theo C. (1995), \textquotedblleft Spirantization and the 7-to-5 Vowel Merger in Bantu". In \textit{Sound Change}, M. Dominicy and D. Demolin (Eds.), Amsterdam: John Benjamins, 1995.

S \textrightarrow\hspace{0pt} F / _\{\ipa{i,u}\} (Do not necessarily have to be fricatives at the same POA; in some cases, the phones go to null or to /\ipa{l}/) \\
\textipa{I U} \textrightarrow\hspace{0pt} \ipa{i u}

\subsection{California Vowel Shift (English)}{\it Pogostick Man}, from Wikipedia contributors (2013), ``California English". {\it Wikipedia, the Free Encyclopedia}. \textless\url{https://en.wikipedia.org/w/index.php?title=California_English&oldid=584388388}\textgreater; and Eckert, Penelope, ``Vowel Shifts in Northern California and the Detroit Suburbs". \textless\url{http://www.stanford.edu/~eckert/vowels.html}\textgreater

\ipa{\ae\ I} \change\ \ipa{e i} / _\ipa{N}; some speakers (esp. in southern regions) may also have {\sc pin-pen} and ``a single phoneme in contrast to the nasal diphthong [\ipa{\~{a}\~{I}}] of the U.S. Northeast" (though the article doesn't specify what this is; maybe it's just plain \ipa{\~{a}})\\
/\ipa{I}/ otherwise has a highly variable pronunciation\\
\ipa{\ae} \change\ \{\ipa{e\textsubarch{@},I\textsubarch{@}}\} / _N\\
\{\ipa{\ae,e}\} \change\ \ipa{E} / _\ipa{\*r}\\
\ipa{\ae} \change\ \ipa{a}\\
\ipa{U 2 E} \change\ \ipa{2 E \ae}\\
\ipa{A} \change\ \ipa{O} (does not occur in Sacramento)\\
\ipa{u} \change\ \{\ipa{i\textsubarch{U},0,W}\}\\
\ipa{o\textsubarch{U}} \change\ \ipa{e\textsubarch{U}} (``common only within certain social groups")

\subsection{Belgian and Netherlandish Dutch Monophthongization}{\it Pogostick Man}, from Wikipedia contributors (2014), ``Dutch Phonology". {\it Wikipedia, the Free Encyclopedia}. \textless\url{https://en.wikipedia.org/w/index.php?title=Dutch_phonology&oldid=602553868}\textgreater

\ipa{Ei \oe y Ou} \change\ \ipa{E: \oe: O:}

\subsection{Polder Dutch Vowel Shift}{\it Pogostick Man}, from Wikipedia contributors (2014), ``Dutch Phonology". {\it Wikipedia, the Free Encyclopedia}. \textless\url{https://en.wikipedia.org/w/index.php?title=Dutch_phonology&oldid=602553868}\textgreater

\ipa{Ei \oe y 2u} \change\ \ipa{ai ay au}\\
\ipa{e: \o: o:} \change\ \ipa{Ei \oe y  Ou}

\subsection{Old English-to-Scots Vowel Shifts}{\it Pogostick Man}, from Wikipedia contributors (2014), ``Phonological history of Scots". {\it Wikipedia, the Free Encyclopedia}. \textless\url{https://en.wikipedia.org/w/index.php?title=Phonological_history_of_Scots&oldid=582962563}\textgreater; and Wikipedia contributors (2014), ``Scottish Vowel Length Rule". {\it Wikipedia, the Free Encyclopedia}. \textless\url{https://en.wikipedia.org/w/index.php?title=Scottish_vowel_length_rule&oldid=589349104}\textgreater

\ipa{ai} \change\ \ipa{Ei} \change\ \ipa{@i} / when stem-final\\
\ipa{u:} \change\ \ipa{2u} / when-stem final, in northern varieties\\
\ipa{\o:} \change\ \ipa{wi} / \{\ipa{k,g}\}_ (in Mid Northern dialects)\\
\ipa{\o:} \change\ \ipa{i} (in northern dialects)\\
\ipa{\o:} \change\ (\ipa{j})\{\ipa{u,2}\} / _\{\ipa{k,x}\} (outcome varies depending upon dialect)\\
\ipa{a} \change\ \ipa{i} / _\ipa{n} (in northern varieties)\\
\ipa{a} \change\ \ipa{e} / _\ipa{n} (otherwise)\\
\ipa{a} \change\ \{\ipa{E,e}\} / _\ipa{r}C\\
\ipa{ai oi ui ei au ou iu E(o)u} \change\ \ipa{e: oe @i i:} \{\ipa{A:,O:}\} \ipa{2u ju j(2)u}\\
\ipa{E:} \change\ \ipa{Ei} (\change\ \ipa{@i}?) / in some northern varieties\\
\ipa{i: e: E: a: o: u:} \{\ipa{\o:,y:}\} \change\ \ipa{@i i} \{\ipa{i,e}\} \ipa{e o u \o}\\
\ipa{\ae} \change\ \ipa{E} / _C[+alveolar]\\
\ipa{a O u} \change\ \{\ipa{a,A}\} \ipa{O 2}\\
Application of the Scottish vowel-length rule:\\
--- V \change\ V\ipa{:} / _\{\ipa{r},F[+voiced],\$,\#\}\\
--- \ipa{@i} \change\ \ipa{aI} / _\{\ipa{r},F[+voiced],\$,\#\} (pursuant to the above)

\subsection{Great Ngamo Tone Shift}\textit{Pogostick Man}, from Schuh, Russel (2005), \textquotedblleft The Great Ngamo Tone Shift"

\tab In the Gudi dialect, the tone on a given domain (which can be more than one syllable/mora, as long as said syllables/morae are consecutive and share the same tone) shift to the following domain, with a low tone cropping up on the first domain. The original tone of the word-final domain floats or tacks itself onto the next domain, depending upon the surrounding conditions. When utterance-final, these tones remain on that domain. This can cause a falling tone, but not a rising tone, which Ngamo does not permit; where such would occur, tone goes to high.

\subsection{Great Vowel Shift (English)}\textit{Jaaaaaa and Ran}, citing \url{http://www.peak.org/~jeremy/dictionary/chapters/history.php}

\textipa{i: u:} \textrightarrow\hspace{0pt} \textipa{@j @w} \textrightarrow\hspace{0pt} \textipa{Aj Aw} \\
\textipa{e: o:} \textrightarrow\hspace{0pt} \textipa{i: u:} \\
\textipa{E:} \textrightarrow\hspace{0pt} \textipa{e:} \textrightarrow\hspace{0pt} \textipa{i:} \\
\textipa{a: O:} \textrightarrow\hspace{0pt} \textipa{E: o:} \textrightarrow\hspace{0pt} \textipa{e:} \ipa{ow} \textrightarrow\hspace{0pt} \ipa{ej} (\textipa{@}w)

\subsection{Greek Vowel Shift}{\it Chris Zoller}, from Trask, R.L. (1996), {\it Historical Linguistics}

\ipa{u(:)} \change\ \ipa{y(:)}\\
\ipa{o:} \change\ \ipa{u:}\\
\ipa{e: E:} \change\ \ipa{i: e:}\\
\ipa{ai Oi} \change\ \ipa{E y:}\\
\ipa{e:} \change\ \ipa{i:}\\
\ipa{y(:) O:} \change\ \ipa{i(:) O}\\
\ipa{Eu au} \change\ \ipa{Ev av}

\subsection{Kikuyu Tone Shift}\textit{Pogostick Man}, from Schuh, Russel (2007), \textquotedblleft The Great Ngamo Tone Shift (GNTS)"

\tab Tones move to the following vowel with the initial syllable acquiring a low tone. Original final tones are lost.

\subsection{Late Proto-Finnic to Savonian Vowel Shift}{\it PM_Vanhanen}

\tab ``Long close-mid vowels have become diphthongs:"\\
\ipa{e: \o: o:} \change\ \ipa{ie y\o\ uo}\\
\tab``In some dialects, they have shifted further to /\ipa{uA}/, /\ipa{y\ae}/ and /\ipa{i\ae}/ or /\ipa{iA}/ depending on front-back vowel harmony: /tieto/ to /ti\ipa{A}to/ but /tiet\ae/ to /ti\ae t\ae/."\\

\tab ``These shifts have occurred in some eastern dialects."\\
\ipa{A: \ae: Ai \ae i} \change\ \ipa{uA i\ae\ Ae \ae e}\\
\ipa{ei oi \o i} \change\ \ipa{e: oe \o e}\\
\ipa{Au ou} \change\ \ipa{A: o:}\\
\ipa{\ae y \o y} \change\ \ipa{\ae: \o:}\\
\ipa{eu} \change\ \ipa{eo}\\
\ipa{li ni ri si} \change\ \ipa{l\super ji n\super ji r\super ji s\super ji}

\subsection{Middle Chinese to Cantonese Vowel Shift (``The Inner-Outer Flip'')}\textit{LoneWolf}, from Newman, J. (1983). \textit{Cahiers de Linguistique Asie Orentale} XII.1:65 -- 79.

\tab Relevant changes occurring before the shift:

\ipa{a} \textrightarrow\hspace{0pt} \textipa{O} / _\{\textipa{N},k\} \\
\textipa{u@} \textipa{y@} \textrightarrow\hspace{0pt} \textipa{O \oe} / _\{n,t\} \\
\textipa{@} \textrightarrow\hspace{0pt} \O\hspace{0pt} / i_\{\textipa{N},k\} \\
\O\hspace{0pt} \textrightarrow\hspace{0pt} \textipa{@} / C[+ labiovelar]_i \\
\O\hspace{0pt} \textrightarrow\hspace{0pt} \textipa{@} / _u \\
V \textrightarrow\hspace{0pt} V\textipa{:} / _\#

\tab The actual vowel shift:

\textipa{@} \textrightarrow\hspace{0pt} \ipa{a} \\
\ipa{a} \textrightarrow\hspace{0pt} \{\textipa{a:},\textipa{@}\} \textquotedblleft(the environments for these respective changes are somewhat unclear)" \\
\textipa{@} \textrightarrow\hspace{0pt} \ipa{a} / _\{\ipa{i,u}\} \\
\textipa{i@u} \textrightarrow\hspace{0pt} \ipa{au} \\
\ipa{a} \textrightarrow\hspace{0pt} \textipa{a:} / _\{\ipa{i,u}\} \\
\ipa{a} \textrightarrow\hspace{0pt} \textipa{@} \textrightarrow\hspace{0pt} \O\hspace{0pt} / \ipa{i}_\ipa{u}

\tab Other relevant changes occurring at the same time:

\textipa{@} \textrightarrow\hspace{0pt} \O\hspace{0pt} / W_ \\
W \textrightarrow\hspace{0pt} \O\hspace{0pt} / C_ \\
(Apparently, /\ipa{i u}/ either were or became glides in the appropriate positions)

\subsection{Northern Cities Vowel Shift (English)}{\it Pogostick Man}, from Wikipedia contributors (2013), ``Northern Cities Vowel Shift". {\it Wikipedia, the Free Encyclopedia}. \textless\url{https://en.wikipedia.org/w/index.php?title=Northern_Cities_Vowel_Shift&oldid=581062574}\textgreater

\ipa{\ae} raises and diphthongizes, typically becoming one of \{\ipa{E@ e@ I@}\}\\
\ipa{A O} \change\ \ipa{a A}\\
\ipa{E} \change\ \ipa{5}\\
\ipa{2} \change\ \ipa{O}\\
\ipa{I} \change\ \ipa{\|`I}

\subsection{Old Norse to Faroese Vowel Shift}{\it johanpeturdam}

\tab {\it NB: ``The reflexes of the vowels are given in the order of their reflex when stressed/long, and unstressed/short"}

\{\ipa{a,\ae:}\} \change\ \ipa{Ea / a}\\
\ipa{a:} \change\ \ipa{Oa} ``(except NE of the Faroes \change\ a\ipa{:})" / \ipa{O}\\
\ipa{e} \change\ \ipa{e: / E}\\
\ipa{e:} \change\ \ipa{Ea} ``(except Su\ipa{D}uroy \change\ \ipa{e:})" 	/ \ipa{a} ``(except Su\ipa{D}uroy \change\ \ipa{E})"\\
\{\ipa{i,y}\} \change\ \ipa{i: / I}\\
\{\ipa{i:,y:}\} \change\ \ipa{Ui / U(i)}\\
\ipa{o} \change\ \ipa{o: / O}\\
\ipa{o:} \change\ \{\ipa{Ou,Eu,\oe u}\} / \ipa{\oe} ``(except Su\ipa{D}uroy \change\ \ipa{O})"\\
\ipa{u} \change\ \ipa{u:} / short: \ipa{U} / unstressed: \{\ipa{o,O}\}\\
\ipa{u:} \change\ \ipa{0u / Y}\\
\{\ipa{\oe,O}\} \change\ \ipa{\o:/\oe} ``(except Su\ipa{D}uroy \change\ \ipa{Y})"

\subsection{Pre-Slavic Vowel Changes}{\it Macska}

``PIE *a and *o with their variants have merged in the Balto-Slavic period; below they're written both as *a."

\ipa{e:} \change\ \ipa{\ae}\\
\ipa{en an} \change\ \ipa{\~{e} \~{a}}\\
\ipa{ej} \change\ \ipa{i}\\
\ipa{ew} \change\ \ipa{ju}\\
\ipa{i} \change\ \ipa{\u{\i}} [\ipa{@}?] \change\ \{\ipa{e,a}\} (strong)/\O\ (weak) ``in modern languages"\\
\ipa{i:} \change\ \ipa{i}\\
\ipa{a a:} \change\ \ipa{o a}\\
\ipa{aj} \change\ \{\ipa{\ae}$_2$,\ipa{i}$_2$\} ``(reduced)"\\
\ipa{aw} \change\ \ipa{u}\\
\ipa{u} \change\ \ipa{\u{u}} [\ipa{7}?] \change\ \{\ipa{e,o,7,a}\} (strong)/\O\ (weak) ``in modern languages"\\
\ipa{u:} \change\ \ipa{1}

\subsection{Proto-Japanese to Old Japanese Vowel Shift}\textit{Pogostick Man}, from Frellesvig, Bjarke and John Witman (2005), \textquotedblleft The Japanese-Korean vowel correspondences"

\ipa{e o} \textrightarrow\hspace{0pt} \ipa{je wo} / _\# \\
\ipa{e o} \textrightarrow\hspace{0pt} \ipa{i u} / else \\
\{\textipa{1},\textipa{@}\} \textrightarrow\hspace{0pt} \ipa{o} \\
\{\ipa{u},\textipa{1}\}\ipa{i} \{,\textipa{a,i} \{\textipa{1}i,i\{a,\textipa{@}\}\} \ipa{u}\{\textipa{1,a,@}\} \textrightarrow\hspace{0pt} \ipa{wi e je wo}

\subsection{Development of Proto-Lolo-Burmese -i(C)\# and -u(C)\# to Lahu}{\it Pogostick Man}, from Jacques, Guillaume, and Alexis Michaud (2011), ``Approaching the historical phonology of three highly eroded Sino-Tibetan languages: Naxi, Na and Laze". {\it Diachronica} 28:4 (2011), 468 -- 498; citing Matisoff 2003:186, 248 -- 249, 314

\ipa{-i -i}\{\ipa{p,k}\} \ipa{-it -i}\{\ipa{m,N}\} \ipa{-in} \change\ \ipa{-i -1P -iP -E -1}\\
\ipa{-u -up -ut -uk -um -un -uN} \change\ \ipa{-u -OP -@P -uP -O -@ -E}

\subsection{Proto-Maidun to Nisenian Vowel Shift}\textit{Pogostick Man}, from Ultan, Russell (1964), \textquotedblleft Proto-Maidun Phonology". \textit{International Journal of American Linguistics}, Vol. 30, No. 4 (Oct., 1964), 355 -- 370.

\ipa{u i e a} \textrightarrow\hspace{0pt} \textipa{y e a o}

\subsection{South African Chain Shift (English)}\textit{Pogostick Man}, from Mesthrie, Rajend (2002), \textit{Language in South Africa}

\tab \textit{NB: The author gives \textipa{\u{\i}} as the shifted vowel but calls it \textquotedblleft centralized"; based on this description, I'm calling it /\textipa{1}/.}

\textipa{\ae} \textipa{E I} \textrightarrow\hspace{0pt} \textipa{E e 1}

\subsection{Southern [United States] Shift (English)}\textit{Pogostick Man}, from Wikipedia contributors (2012), ``Southern American English".

\textipa{E} \textrightarrow\hspace{0pt} \textipa{I} / _N \\
``Lax and tense vowels often neutralize before /\ipa{l}/" \\
\textipa{aI} \textrightarrow\hspace{0pt} \textipa{a:} / _\# \\
\textipa{aI} \textrightarrow\hspace{0pt} \textipa{a:} / _C[+ voiced] \\
\textipa{aI} \textrightarrow\hspace{0pt} \textipa{Ae:} / else (only for some speakers) \\
\textipa{aI} \textrightarrow\hspace{0pt} \textipa{a:} / else (only for some speakers) \\
\textipa{\ae} \textipa{E I} \textrightarrow\hspace{0pt} \textipa{\ae}j(\textipa{@}) \textipa{E}j(\textipa{@}) \textipa{I}j(\textipa{@}) \\
\textipa{E}j(\textipa{@}) \textipa{I}j(\textipa{@}) i e\textipa{I} \textrightarrow\hspace{0pt} \ipa{ej}(\textipa{@}) \ipa{ij}(\textipa{@}) \textipa{Ii} \textipa{Ei} \\
\textipa{uU oU} \textrightarrow\hspace{0pt} \textipa{0\"{U} @\"{U}} (a bit of a guesstimate based upon the prose description in the article and the mean-formant-value chart cited from Labov, Ash \&\hspace{0pt} Bobert (2006))\\
\textipa{O} \textrightarrow\hspace{0pt} \textipa{A6} (for some speakers) \\
\textipa{A\*r} \textrightarrow\hspace{0pt} \textipa{6\*r} (``often") \\
\ipa{z} \textrightarrow\hspace{0pt} \ipa{d} / _\ipa{n} (not strictly a vowel shift but included here anyway because it's cool, and also because it doesn't occur in $\langle$hasn't$\rangle$ because of the influence of $\langle$hadn't$\rangle$) \\
Stress reassignment to the initial syllable (again, not strictly a vowel shift) \\
Merger of \textipa{O\*r} and \textipa{A\*r} (``in some regions") \\
Loss of distinction between \textipa{I\*r} and \textipa{I@\*r}, and between \textipa{U@\*r} and \textipa{O\*r} \\
Pronunciation of the $\langle$l$\rangle$ in words like $\langle$walk$\rangle$ and $\langle$talk$\rangle$ (again, not really a vowel development)\\
\textipa{\ae}/\textipa{A:} \textrightarrow\hspace{0pt} \textipa{\ae I}

\clearpage

\section{Most-Wanted Sound Changes}\tab This section replicates the ``Most wanted sound changes" article from Knee Quickie. It is presented mostly as it was found with the following modifications:
\begin{itemize}
\item The formatting rules are not adhered to any longer due to the format, and the preamble (and table of contents) are omitted.
\item Some liberty has been taken with the presentation of the *\ipa{f} \change\ \ipa{p} change in Afro-Asiatic.
\item Bullets are no longer present.
\item Some corrections have been made (e.g., to the conditioning of the \={O}kami change of *\ipa{t} \change\ \ipa{k}).
\item Instead of footnotes, sources (where cited) are listed next to the relevant entries.
\item Some URLs have been shortened using \url{http://is.gd} due to potential conflict with the way \LaTeX\ handles the $\langle$\%$\rangle$ character.
\item Subsection 4 is specifically listed as being ``empty as of yet" for aesthetic purposes.
\item Wikipedia URLs have been changed to \texttt{https://}.
\end{itemize}

\subsection{List 1: Simple Consonant Changes}
\ipa{w} \change\ \ipa{p} (Navajo, some Polynesian languages)\\
\ipa{k\super j g\super j} \change\ \ipa{k g} (Danish)\\
\ipa{\'{s}} \change\ \ipa{k} (Possibly unconditional; some Samoyedic langs)\\
\ipa{p\super j} \change\ \ipa{k\super j} (some Romanian dialects, Tsakonian)\\
\ipa{ts} \change\ \ipa{t} (unconditional; some Samoyedic langs)\\
\ipa{t} \change\ \ipa{k} (general Polynesian)\\
\ipa{n} \change\ \ipa{N} (Samoan, but only in colloquial speech)\\
\ipa{j} \change\ \ipa{p} (some Polynesian languages, such as Levei and Drehet) (\url{https://en.wikipedia.org/wiki/Proto-Austronesian_language})\\
\ipa{b} \change\ \#\ipa{c, -nc-} (Sundanese)\\
\ipa{N} \change\ \{\ipa{x,h}\} (various Mayan languages)\\
\ipa{h} \change\ \ipa{N} (Nyole)\\
\ipa{Q} \change\ \ipa{N} (allegedly in European Hebrew, both Sephardic and Ashkenazi, but possibly not a {\it sound change} so much as a substitution) (\url{https://en.wikipedia.org/wiki/Hebrew_language#Varieties_of_ayin}, \url{http://sites.google.com/site/londonsephardiminhag/pronunciation}, \url{http://www.forward.com/articles/105938/})\\
\ipa{f} \change\ \ipa{p} (? claimed to have occurred independently in Proto-Semitic and Proto-Omotic, and to a limited extent in Egyptian (but this may be related to the Semitic change); note that the Wikipedia article is cited to a single source and that source is admittedly theoretical, and acknowledges on page 77 another reconstruction that doesn't believe Proto-Afro-Asiatic had /\ipa{f}/ at all) (Page 77 of \url{http://is.gd/WNyXdn}, \url{https://en.wikipedia.org/wiki/Proto-Afro-Asiatic_language})\\
\ipa{t\v{s}} \change\ \ipa{t} (general Baltic-Finnic; may not be unconditional but was certainly the most common outcome)\\
\ipa{r} \change\ \ipa{g\;L} \change\ \ipa{G} (Hiw) (\url{http://is.gd/jCDLO1})\\
\ipa{mb nd} \change\ \ipa{\;B dr} (Nias)

\subsection{List 2: Conditional or complex consonant changes}
Western and Eastern Armenian often have swapped voicing in stop consonants: e.g. {\it vardapet} vs. {\it vartabed}. This is a result of changes related to aspiration.\\
\ipa{w} \change\ \ipa{f} (Common Celtic; I'm not sure of the conditions)\\
\ipa{m} \change\ \ipa{n} / _\ipa{i} (Tsakonian)\\
\{\ipa{t,k}\} \change\ \O\ / V_V (Marathi) {\it probably with voiced stops as intermediates, since they also became silent}\\
\ipa{p} \change\ \ipa{w} / V_V (Marathi)\\
\ipa{b d g} \change\ \ipa{b: d: g:} / V_V (some dialects of Italian; there may be more to it than this, since words like ``repubblica" are in standard Italian and not just dialects)\\
\ipa{t} \change\ \ipa{k} / _\ipa{\s{s}} (\={O}gami) {\it (NB: The article doesn't have an underscore indicating whether this occurs before or after the /\ipa{\s{s}}/, but the linked page indicates where this change occurred)} (\url{http://amritas.com/101023.htm#10192359})\\
\ipa{n} \change\ \ipa{i} / _\ipa{s} and sometimes other fricatives (Montana Salish)

\subsection{List 3: Vowels}

\ipa{y} \change\ \ipa{u} (some mainland Greek dialects, and Tsakonian; this particular sound change has been said in some places to be impossible) (\url{https://en.wikipedia.org/wiki/Tsakonian_language#Consonants})\\
\ipa{i u} \change\ \ipa{\s{s} \s{f}} (\={O}gami) (\url{http://amritas.com/101023.htm#10192359})

\subsection{List 4: Other}

This section is empty as of yet.

\end{document}  